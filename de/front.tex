% !TeX root = surprises.tex

\author{Mordechai Ben-Ari}
\title{Surprises mathématiques}
\date{Traduction : Nicolas Bacaër}

\pagestyle{empty}
\begin{center}
\textbf{\huge Mathematische Überraschungen}\\
\vspace{2cm}
\textbf{\Large Mordechai Ben-Ari}\\
\vspace{1cm}
\textbf{\Large Übersetzung: ......}
\end{center}



\chapter*{Vorwort}

\begin{flushright}
\parbox{7cm}{
\begin{footnotesize}
\begin{flushright}
Wenn jeder die Mathematik in ihrem natürlichen Zustand kennen lernen würde, mit all den Herausforderungen, dem Spaß und den Überraschungen, die das mit sich bringt, würden wir meiner Meinung nach eine dramatische Veränderung sowohl in der Einstellung der Schüler zur Mathematik als auch in unserer Vorstellung davon, was es bedeutet, ``gut in Mathe zu sein'', erleben.
Paul Lockhart\mbox{}\\\mbox{}\\
Ich bin wirklich hungrig nach Überraschungen, denn jede macht uns ein klein wenig, aber wesentlich schlauer.\\
Tadashi Tokieda
\end{flushright}
\end{footnotesize}
}
\end{flushright}

\bigskip

Die Mathematik kann uns, wenn wir sie richtig angehen, viele angenehme Überraschungen bescheren. Dies wird durch eine Google-Suche nach ``mathematischen Überraschungen'' bestätigt, die überraschenderweise fast eine halbe Milliarde Einträge liefert. Was ist eine Überraschung? Die Ursprünge des Wortes gehen auf das Altfranzösische zurück und haben ihre Wurzeln im Lateinischen: ``ur'' (über) und ``prendre'' (ergreifen, ergreifen, ergreifen). Wörtlich bedeutet "überraschen" "überholen". Als Substantiv bezeichnet Überraschung sowohl ein unerwartetes oder verwirrendes Ereignis oder einen Umstand als auch das Gefühl, das dadurch ausgelöst wird.

Betrachten wir zum Beispiel einen Auszug aus einer Vorlesung von Maxim Bruckheimer\footnote{Maxim Bruckheimer war ein Mathematiker, der zu den Gründern der Open University UK gehörte und Dekan ihrer Fakultät für Mathematik war. Er war Leiter der Abteilung für naturwissenschaftliche Lehre am Weizmann Institute of Science.} über den Feuerbach-Kreis: ``Zwei Punkte liegen auf einer und nur einer Geraden, das ist keine Überraschung. Drei Punkte liegen jedoch nicht notwendigerweise auf einer Geraden, und wenn bei einer geometrischen Untersuchung drei Punkte in eine Gerade "hineinfallen", ist das eine Überraschung, und häufig müssen wir diese Tatsache als ein zu beweisendes Theorem bezeichnen. Alle drei Punkte, die nicht auf einer Geraden liegen, liegen auf einem Kreis. Wenn jedoch vier Punkte auf demselben Kreis liegen, ist dies eine Überraschung, die als Theorem formuliert werden sollte. \ldots{} Insofern die Anzahl der Punkte auf einer Geraden größer als $3$ ist, ist der Satz umso überraschender. Ebenso ist der Satz umso überraschender, je größer die Anzahl der Punkte ist, die auf einem Kreis liegen, der größer als $4$ ist. So ist die Aussage, dass es für jedes Dreieck neun zusammenhängende Punkte auf demselben Kreis gibt ... sehr überraschend. Und trotz des Ausmaßes der Überraschung ist der Beweis elegant und einfach.''

In diesem Buch bietet Mordechai Ben-Ari eine reichhaltige Sammlung mathematischer Überraschungen, von denen die meisten weniger bekannt sind als der Feuerbach-Kreis und für die es gute Gründe gibt, sie aufzunehmen. Erstens sind die mathematischen Perlen dieses Buches, obwohl sie nicht in Lehrbüchern zu finden sind, mit nur einem High-School-Hintergrund zugänglich (und mit Geduld und Papier und Bleistift, denn Spaß gibt es nicht umsonst). Zweitens: Wenn ein mathematisches Ergebnis das in Frage stellt, was wir für selbstverständlich halten, sind wir tatsächlich überrascht (Kap. ~\ref{c.collapse}, \ref{c.compass}). In ähnlicher Weise überraschen uns: die Cleverness eines Arguments (Kap.~\ref{c.trisect}, \ref{c.square}), die Rechtfertigung der Möglichkeit einer geometrischen Konstruktion mit algebraischen Mitteln (Kap.~\ref{c.heptadecagon}), ein Beweis, der sich auf ein scheinbar nicht verwandtes Thema stützt (Kap.~\ref{c.five}, \ref{c.museum}), ein seltsamer Beweis durch Induktion (Kap.~\ref{c.induction}), neue Sichtweisen auf ein bekanntes Ergebnis (Kap.~\ref{c.quadratic}), ein scheinbar unbedeutender Satz, der zur Grundlage eines ganzen Bereichs der Mathematik wird (Kap.~\ref{c.ramsey}), unerwartete Inspirationsquellen (Kap.~\ref{c.langford}), reichhaltige Formalisierungen, die aus reinen Freizeitaktivitäten wie Origami entstehen (Kap.~\ref{c.origami-axioms}--\ref{c.origami-constructions}). Dies alles sind verschiedene Gründe für die Aufnahme der angenehmen, schönen und denkwürdigen mathematischen Überraschungen in dieses schöne Buch.
   
Bisher habe ich mich damit befasst, wie sich das Buch auf den ersten Teil der Definition von Überraschung, die kognitiv-rationalen Gründe für das Unerwartete, bezieht. Was den zweiten, den emotionalen Aspekt betrifft, so ist dieses Buch eine lebendige Umsetzung dessen, was viele Mathematiker über den Hauptgrund für die Beschäftigung mit der Mathematik behaupten: Sie ist faszinierend! Außerdem behaupten sie, dass die Mathematik sowohl unsere intellektuelle Neugier als auch unser ästhetisches Empfinden anregt und dass das Lösen eines Problems oder das Verstehen eines Konzepts eine geistige Belohnung darstellt, die uns dazu verleitet, weiter an weiteren Problemen und Konzepten zu arbeiten. 

Es wurde gesagt, dass die Funktion eines Vorworts darin besteht, dem Leser zu sagen, warum er das Buch lesen sollte. Ich habe versucht, dies zu tun, aber ich glaube, dass die umfassendere Antwort von Ihnen, dem Leser, kommen wird, nachdem Sie es gelesen und erlebt haben, was die Etymologie des Wortes Überraschung nahelegt: von ihr überholt zu werden!

\vspace{\baselineskip}
\begin{flushright}
\textit{Abraham Arcavi}
\end{flushright}

\chapter*{Einleitung}

Godfried Toussaints Artikel über den ,,kollabierenden Kompass'' hat einen tiefen Eindruck auf mich gemacht. Es wäre mir nie in den Sinn gekommen, dass der moderne Kompass mit einem Reibungsgelenk nicht derjenige ist, der zu Euklids Zeiten verwendet wurde. In diesem Buch präsentiere ich eine Auswahl von mathematischen Ergebnissen, die nicht nur interessant sind, sondern mich auch überrascht haben, als ich ihnen zum ersten Mal begegnete.

Die Mathematik, die für die Lektüre des Buches erforderlich ist, ist Mathematik für die Sekundarstufe, aber das bedeutet nicht, dass das Material einfach ist. Einige der Beweise sind recht lang und erfordern die Bereitschaft des Lesers, sich mit der Materie zu beschäftigen. Die Belohnung ist das Verständnis einiger der schönsten Ergebnisse der Mathematik. Das Buch ist kein Lehrbuch, denn das breite Spektrum der behandelten Themen lässt sich nicht in einen Lehrplan einordnen. Es eignet sich zur Vertiefung für Schüler der Sekundarstufe, für Seminare auf Hochschulniveau und für Mathematiklehrer.

 Die Kapitel können unabhängig voneinander gelesen werden. (Eine Ausnahme ist, dass das Kapitel~\ref{c.origami-axioms} über die Axiome des Origami eine Voraussetzung für die Kapitel~\ref{c.origami-cube}, \ref{c.origami-constructions}, die anderen Kapitel über Origami ist). Hinweise, die für alle Kapitel relevant sind, finden Sie unten in der Liste mit dem Titel Stil.

\subsection*{Was ist eine Überraschung?}

Es gab drei Kriterien für die Aufnahme eines Themas in das Buch:
\begin{itemize}
\item Der Satz hat mich überrascht. Besonders überraschend waren die Theoreme zur Konstruierbarkeit mit Lineal und Zirkel. Die äußerst reichhaltige Mathematik des Origami war fast schockierend: Als eine Mathematiklehrerin ein Projekt über Origami vorschlug, lehnte ich zunächst ab, weil ich bezweifelte, dass mit dieser Kunstform ernsthafte Mathematik verbunden sein könnte.
Andere Themen wurden aufgenommen, weil ich zwar die Ergebnisse kannte, ihre Beweise aber durch ihre Eleganz und Zugänglichkeit überraschten, insbesondere der rein algebraische Beweis von Gauß, dass ein regelmäßiges Heptadeck konstruiert werden kann.

\item Das Material kommt in Lehrbüchern für die Sekundarstufe und die Hochschule nicht vor, und ich habe diese Theoreme und Beweise nur in Lehrbüchern für Fortgeschrittene und in der Forschungsliteratur gefunden. Zu den meisten Themen gibt es Wikipedia-Artikel, aber man muss wissen, wo man suchen muss, und die Artikel sind oft sehr knapp gehalten.

\item Die Theoreme und Beweise sind mit guten Kenntnissen der Sekundarschulmathematik zugänglich.
\end{itemize}
Jedes Kapitel schließt mit einem Absatz \textit{Was ist die Überraschung?} der meine Wahl des Themas erklärt.

\subsection*{Ein Überblick über den Inhalt}

Kapitel~\ref{c.collapse} präsentiert Euklids Beweis, dass jede Konstruktion, die mit einem festen Zirkel möglich ist, auch mit einem kollabierenden Zirkel möglich ist. Es gibt viele Beweise, aber wie Toussaint zeigt, sind die meisten falsch, weil sie von Diagrammen abhängen, die die Geometrie nicht immer korrekt darstellen. Um zu verdeutlichen, dass man Diagrammen nicht trauen darf, stelle ich den berühmten angeblichen Beweis vor, dass jedes Dreieck gleichschenkliges ist. 

Im Laufe der Jahrhunderte versuchten Mathematiker erfolglos, einen beliebigen Winkel nur mit Hilfe eines Lineals und eines Zirkels zu Dreiteilung (in drei gleiche Teile zu zerlegen). Underwood Dudley hat eine umfassende Studie über Dreiteilung durchgeführt und dabei fehlerhafte Konstruktionen gefunden; die meisten Konstruktionen sind Annäherungen, die als genau bezeichnet werden. In Kapitel~\ref{c.trisect} werden zunächst zwei dieser Konstruktionen vorgestellt und die trigonometrischen Formeln entwickelt, die zeigen, dass es sich nur um Näherungen handelt. Um zu zeigen, dass eine Dreiteilung nur mit Lineal und Zirkel keine praktische Bedeutung hat, werden Dreiteilungen mit komplexeren Hilfsmitteln vorgestellt: Archimedes' \emph{neusis} und Hippias' \emph{quadratrix}. Das Kapitel endet mit dem Beweis, dass es unmöglich ist, einen beliebigen Winkel mit einem Lineal und einem Zirkel zu Dreiteilung. 

Die Quadratur eines Kreises (wenn ein Kreis ein Quadrat mit demselben Flächeninhalt bildet) kann nicht mit Lineal und Zirkel durchgeführt werden, weil der Wert von $\pi$ nicht konstruiert werden kann. In Kapitel~\ref{c.square} werden drei elegante Konstruktionen von Näherungen an $\pi$ vorgestellt, eine von Kocha\'{n}ski und zwei von Ramanujan. Das Kapitel schließt mit dem Nachweis, dass eine Quadratrix zur Quadratur eines Kreises verwendet werden kann.

Das Vier-Farben-Theorem besagt, dass es möglich ist, jede ebene Karte mit vier Farben zu färben, so dass keine Länder mit einer gemeinsamen Grenze mit der gleichen Farbe gefärbt werden. Der Beweis dieses Satzes ist äußerst kompliziert, aber der Beweis des Fünf-Farben-Satzes ist einfach und elegant, wie in Kapitel~\ref{c.five} gezeigt wird. In diesem Kapitel wird auch Percy Heawoods Beweis vorgestellt, dass Alfred Kempes "Beweis" des Vierfarbensatzes falsch ist.

Wie viele Wächter muss ein Kunstmuseum beschäftigen, damit alle Wände ständig von mindestens einem Wächter beobachtet werden? Der Beweis in Kapitel~\ref{c.museum} ist ziemlich raffiniert, denn er verwendet Graphenfärbung, um ein auf den ersten Blick rein geometrisches Problem zu lösen.

Kapitel~\ref{c.induction} stellt einige weniger bekannte Ergebnisse und ihre Beweise durch Induktion vor: Theoreme über Fibonacci- und Fermat-Zahlen, McCarthys $91$-Funktion und das Josephus-Problem.

Kapitel~\ref{c.quadratic} behandelt Po-Shen Lohs Methode zum Lösen quadratischer Gleichungen. Die Methode ist ein entscheidendes Element von Gauß' algebraischem Beweis, dass ein Heptadeck konstruiert werden kann (Kapitel~\ref{c.heptadecagon}). Das Kapitel enthält al-Khwarizmis geometrische Konstruktion für die Suche nach Wurzeln von quadratischen Gleichungen und eine geometrische Konstruktion, die von Cardano bei der Entwicklung der Formel für die Suche nach Wurzeln von kubischen Gleichungen verwendet wurde.

Die Ramsey-Theorie ist ein aktives Forschungsgebiet der Kombinatorik. Sie sucht nach Mustern in Teilmengen großer Mengen. In Kapitel~\ref{c.ramsey} werden einfache Beispiele für Schur-Dreier, pythagoreische Dreiergruppen, Ramsey-Zahlen und das van der Waerden-Problem vorgestellt. Der Beweis des Satzes über pythagoräische Dreiergruppen wurde kürzlich mit Hilfe eines Computerprogramms auf der Grundlage der mathematischen Logik erbracht. Das Kapitel schließt mit einem Exkurs über das Wissen der alten Babylonier über die pythagoräischen Tripel.

C. Dudley Langford beobachtete seinen Sohn beim Spielen mit farbigen Blöcken und bemerkte, dass er sie in einer interessanten Reihenfolge angeordnet hatte. Kapitel~\ref{c.langford} stellt sein Theorem über die Bedingungen vor, unter denen eine solche Folge möglich ist.

Kapitel~\ref{c.origami-axioms} enthält die sieben Axiome des Origami, zusammen mit den detaillierten Berechnungen der analytischen Geometrie der Axiome und Charakterisierungen der Faltungen als geometrische Orte.

Kapitel~\ref{c.origami-cube} stellt die Methode von Eduard Lill und die von Margharita P. Beloch vorgeschlagene Origami-Faltung vor. Ich stelle die Lill-Methode als Zaubertrick vor, daher möchte ich sie hier nicht im Detail vorstellen.

Kapitel~\ref{c.origami-constructions} zeigt, dass mit Origami Konstruktionen ausgeführt werden können, die mit Lineal und Zirkel nicht möglich sind:  Dreiteilung eines Winkels, Verdopplung eines Würfels und Konstruktion eines Nonagons (ein regelmäßiges Vieleck mit neun Seiten).

Kapitel~\ref{c.compass} stellt das Theorem von Georg Mohr und Lorenzo Mascheroni vor, dass jede Konstruktion mit Lineal und Zirkel auch mit einem Zirkel durchgeführt werden kann.

Die entsprechende Behauptung, dass nur ein Lineal ausreicht, ist falsch, weil ein Lineal keine Längen berechnen kann, die Quadratwurzeln sind. Jean-Victor Poncelet vermutete und Jakob Steiner bewies, dass ein Lineal ausreicht, wenn es irgendwo in der Ebene einen einzigen festen Kreis gibt (Kap.~\ref{c.straightedge}).

Wenn zwei Dreiecke denselben Umfang und dieselbe Fläche haben, müssen sie dann kongruent sein? Das scheint einleuchtend zu sein, aber es stellt sich heraus, dass das nicht stimmt, auch wenn man eine ganze Menge Algebra und Geometrie braucht, um ein nicht kongruentes Paar zu finden, wie in Kap.~\ref{c.congruent} gezeigt wird.

Kapitel~\ref{c.heptadecagon} präsentiert Gauß' Meisterstück: den Beweis, dass ein Heptadeck (ein regelmäßiges Polygon mit siebzehn Seiten) mit Hilfe eines Lineals und eines Zirkels konstruiert werden kann. Durch ein geschicktes Argument über die Symmetrie der Wurzeln von Polynomen erhielt er eine Formel, die nur die vier arithmetischen Operatoren und Quadratwurzeln verwendet. Gauß hat keine explizite Konstruktion eines Heptadecks angegeben, daher wird die elegante Konstruktion von James Callagy vorgestellt. Das Kapitel schließt mit Konstruktionen eines regelmäßigen Fünfecks, die auf Gauß' Methode zur Konstruktion eines Heptadecks basieren.

Um das Buch so in sich geschlossen wie möglich zu halten, werden im Anhang Beweise für Theoreme der Geometrie und Trigonometrie gesammelt, die dem Leser vielleicht nicht geläufig sind.

\subsection*{Stil}

\begin{itemize}
\item Es wird vorausgesetzt, dass der Leser über gute Kenntnisse der Sekundarschulmathematik verfügt, einschließlich:
\begin{itemize}
\item Algebra: Polynome und Division von Polynomen, \emph{monische} Polynome---diejenigen, deren Koeffizient der höchsten Potenz $1$ ist, quadratische Gleichungen, Multiplikation von Ausdrücken mit Exponenten $a^m\cdot a^n=a^{m+n}$.
\item Euklidische Geometrie: kongruente Dreiecke $\triangle ABC \cong \triangle DEF$ und die Kriterien für Kongruenz, ähnliche Dreiecke $\triangle ABC \sim \triangle DEF$ und die Verhältnisse ihrer Seiten, Kreise und ihre Inkreis- und Mittelpunktswinkel.
\item Analytische Geometrie: die kartesische Ebene, Berechnung von Längen und Steigungen von Streckenabschnitten, die Formel für einen Kreis.
\item Trigonometrie: die Funktionen $\sin,\cos,\tan$ und die Umrechnungen zwischen ihnen, Winkel im Einheitskreis, die trigonometrischen Funktionen von um eine Achse gespiegelten Winkeln wie $\cos (180^\circ-\theta)=-\cos\theta$.
\end{itemize}
\item Zu beweisende Aussagen werden als \emph{Theoreme} bezeichnet, ohne dass versucht wird, zwischen Theoremen, Lemmata und Korollarien zu unterscheiden.
\item Wenn ein Satz auf eine Konstruktion folgt, beziehen sich die Variablen, die im Satz erscheinen, auf beschriftete Punkte, Linien und Winkel in der Abbildung, die die Konstruktion begleitet.
\item Die vollständigen Namen von Mathematikern wurden ohne biografische Informationen angegeben, die leicht in Wikipedia gefunden werden können.
\item Das Buch ist so geschrieben, dass es so weit wie möglich in sich geschlossen ist, aber gelegentlich hängt die Darstellung von fortgeschrittenen mathematischen Konzepten und Theoremen ab, die ohne Beweise gegeben werden. In solchen Fällen wird eine Zusammenfassung des Materials in Kästen präsentiert, die übersprungen werden können.
\item Es gibt keine Übungen, aber der ambitionierte Leser wird aufgefordert, jedes Theorem zu beweisen, bevor er den Beweis liest.
\item Geometrische Konstruktionen können mit Software wie Geogebra untersucht werden.
\item $\overline{AB}$ wird sowohl für den Namen eines Linienabschnitts als auch für die Länge des Abschnitts verwendet.
\item $\triangle ABC$ wird sowohl für den Namen eines Dreiecks als auch für den Flächeninhalt des Dreiecks verwendet.
\end{itemize}


\subsection*{Danksagungen}

Dieses Buch wäre nie ohne die Ermutigung von Abraham Arcavi geschrieben worden, der mich willkommen hieß, sein Terrain der mathematischen Bildung zu betreten. Er hat auch freundlicherweise das Vorwort geschrieben. Avital Elbaum Cohen und Ronit Ben-Bassat Levy waren immer bereit, mir beim (Wieder-)Erlernen der Mathematik in der Sekundarstufe zu helfen. Oriah Ben-Lulu führte mich in die Mathematik des Origami ein und arbeitete an den Beweisen mit. Ich bin Michael Woltermann dankbar für die Erlaubnis, mehrere Abschnitte seiner Neubearbeitung von Heinrich D\"{o}rrie's Buch zu verwenden. Jason Cooper, Richard Kruel, Abraham Arcavi und die anonymen Gutachter gaben hilfreiche Kommentare ab.

Ich möchte dem Team bei Springer für seine Unterstützung und Professionalität danken, insbesondere dem Herausgeber Richard Kruel.

Das Buch wird im Rahmen des Open-Access-Programms veröffentlicht, und ich möchte dem Weizmann Institute of Science für die Finanzierung der Veröffentlichung danken.

Die \LaTeX{}-Quelldateien für das Buch (einschließlich der Ti\textit{k}Z-Quelle für die Diagramme) sind verfügbar unter:
\begin{center}
\url{https://github.com/motib/surprises}
\end{center}

\medskip

\begin{flushright}
\textit{Mordechai (Moti) Ben-Ari}
\end{flushright}

\tableofcontents
