



\noindent \textsc{Mordechai Ben-Ari}\\
Département de didactique des sciences\\
Institut Weizmann \\
Rehovot, Israël\\
moti.ben.ari@gmail.com\\


\noindent  \textsc{Nicolas Baca\"er}\\
{Institut de recherche pour le d\'eveloppement}\\
nicolas.bacaer@ird.fr\\

\vspace{3cm}

\noindent Couverture: \textsc{Robert Delaunay}, Rythme, Joie de vivre (1930). Centre Pompidou, Paris. 

\vspace{3cm}

\noindent Titre original: \emph{Mathematical Surprises}\\
Publié par Springer, Cham, 2022\\
%ISBN 978-3-031-13565-1 (papier) /  978-3-031-13566-8 (électronique)\\
%\url{https://doi.org/10.1007/978-3-031-13566-8}\\
\noindent \copyright\ \url{http://creativecommons.org/licenses/by/4.0/}\\

\noindent {Pour l'\'edition française:}\\
\noindent \copyright\ {Nicolas Bacaër, Paris, 2023} \\
\noindent {ISBN: 979-10-396-4380-1} \\
{Dépôt légal: avril 2023}\\

\lhead[\fancyplain{}{\thepage}]%
{\fancyplain{}{\nouppercase{Préface }}}
\rhead[\fancyplain{}{}]%
{\fancyplain{}{\thepage}}
\cfoot{}
\pagestyle{fancyplain}



\chapter*{Préface}

\epigraph{Si tout le monde était exposé aux mathématiques dans leur état naturel, avec tous les défis, les plaisirs et les surprises que cela comporte, je pense que nous verrions un changement spectaculaire à la fois dans l'attitude des élèves envers les mathématiques et dans notre conception de ce que signifie être \og bon en maths\fg.}{Paul Lockhart}

\epigraph{J'ai vraiment faim de surprises, car chacune d'entre elles nous rend plus intelligents, même si c'est de manière très légère.}{Tadashi Tokieda}



\medskip

Les mathématiques, lorsqu'elles sont abordées de manière appropriée, peuvent nous réserver de nombreuses surprises agréables. C'est ce que confirme une recherche sur internet de l'expression \og surprises mathématiques\fg, qui donne étonnamment près d'un demi-milliard de résultats. Qu'est-ce qu'une surprise? Les origines du mot remontent au vieux français avec des racines latines : \og sur\fg{}  et \og prendre\fg. Littéralement, surprendre, c'est dépasser. En tant que substantif, la surprise est à la fois un événement ou une circonstance inattendue ou déconcertante, ainsi que l'émotion qu'elle provoque.


Prenons par exemple l'extrait d'une conférence de Maxim Bruckheimer\footnote{Maxim Bruckheimer était un mathématicien qui a été l'un des fondateurs d'\emph{Open University} au Royaume-Uni et le doyen de sa faculté de mathématiques. Il a dirigé le département de didactique des sciences à l'Institut Weizmann.} sur le cercle de Feuerbach : \og Deux points se trouvent sur une même  droite, ce n'est pas une surprise. Cependant, trois points ne sont pas nécessairement sur une même droite et si, au cours d'une exploration géométrique, trois points \og tombent\fg{} sur une  droite, c'est une surprise et nous devons souvent nous référer à ce fait comme à un théorème à démontrer. Trois points qui ne sont pas sur une  droite se trouvent sur un cercle. En revanche, si quatre points se trouvent sur le même cercle, il s'agit d'une surprise qui doit être formulée comme un théorème... Dans la mesure où le nombre de points sur une droite est supérieur à 3, le théorème est d'autant plus surprenant. De même, si le nombre de points situés sur un cercle est supérieur à 4, le théorème est d'autant plus surprenant. Ainsi, l'affirmation selon laquelle, pour tout triangle, il existe neuf points  sur le même cercle ... est très surprenante. De plus, malgré l'ampleur de la surprise, sa démonstration est élégante et facile.\fg{}

Dans ce livre, Mordechai Ben-Ari propose une riche collection de surprises mathématiques, pour la plupart moins connues que le cercle de Feuerbach et avec de bonnes raisons de les inclure. Tout d'abord, bien qu'elles soient absentes des manuels scolaires, les perles mathématiques de ce livre sont accessibles avec un simple bagage de lycée (et de la patience, ainsi que du papier et un crayon, car le plaisir n'est pas gratuit). Deuxièmement, lorsqu'un résultat mathématique remet en question ce que nous tenons pour acquis, nous sommes effectivement surpris (chap.~\ref{c.collapse} et  \ref{c.compass}). De même, nous sommes surpris par : l'intelligence d'un argument (chap.~\ref{c.trisect} et \ref{c.square}), la justification de la possibilité d'une construction géométrique par des moyens algébriques (chap.~\ref{c.heptadecagon}), une démonstration s'appuyant sur un sujet apparemment sans rapport (chap.~\ref{c.five}  et \ref{c.museum}), une étrange démonstration par récurrence  (chap.~\ref{c.induction}), de nouvelles façons de considérer un résultat bien connu (chap.~\ref{c.quadratic}), un théorème apparemment mineur devenant le fondement de tout un domaine des mathématiques (chap.~\ref{c.ramsey}), des sources d'inspiration inattendues (chap.~\ref{c.langford}), des formalisations riches qui émergent d'activités purement récréatives comme l'origami (chap.~\ref{c.origami-axioms}--\ref{c.origami-constructions}). Ce sont là autant de raisons différentes d'inclure dans ce beau livre des surprises mathématiques agréables, belles et mémorables.
   
Jusqu'à présent, j'ai abordé la manière dont le livre se rapporte à la première partie de la définition de la surprise, à savoir les raisons cognitives et rationnelles de l'inattendu. Quant au deuxième aspect, l'aspect émotionnel, ce livre est un exemple  de ce que de nombreux mathématiciens affirment concernant la raison première de faire des mathématiques : c'est fascinant! De plus, ils affirment que les mathématiques stimulent à la fois notre curiosité intellectuelle et notre sensibilité esthétique, et que la résolution d'un problème ou la compréhension d'un concept constitue une récompense spirituelle, qui nous incite à continuer à travailler sur d'autres problèmes et concepts. 

On a dit que la fonction d'un avant-propos était de dire aux lecteurs pourquoi ils devraient lire le livre. J'ai essayé de le faire, mais je crois que la réponse la plus complète viendra de vous, lecteur ou lectrice, après l'avoir lu et avoir fait l'expérience de ce que l'étymologie du mot \og surprise\fg{} suggère : être dépassé par elle!



\vspace{\baselineskip}
\begin{flushright}
{Abraham Arcavi}
\end{flushright}

\newpage

\lhead[\fancyplain{}{\thepage}]%
{\fancyplain{}{\nouppercase{Introduction }}}
\rhead[\fancyplain{}{}]%
{\fancyplain{}{\thepage}}
\cfoot{}
\pagestyle{fancyplain}

\chapter*{Introduction}

L'article de Godfried Toussaint sur le \og compas éphémère\fg{}  \cite{toussaint} m'a fait une profonde impression. Il ne me serait jamais venu à l'esprit que le compas moderne à charnière n'est pas celui utilisé à l'époque d'Euclide. Dans ce livre, je présente une sélection de résultats mathématiques qui ne sont pas seulement intéressants, mais qui m'ont surpris lorsque je les ai rencontrés pour la première fois.

Les mathématiques nécessaires à la lecture de ce livre sont celles du lycée, mais cela ne signifie pas que le contenu est simple. Certaines démonstrations sont assez longues et exigent que le lecteur soit prêt à persévérer dans son étude. La récompense est la compréhension de certains des plus beaux résultats des mathématiques. Ce livre n'est pas un manuel scolaire, car le large éventail de sujets abordés ne s'intègre pas facilement dans un programme d'études. Il convient aux activités d'enrichissement pour les  lycéens, aux séminaires de niveau universitaire et aux professeurs de mathématiques.

 Les chapitres peuvent être lus indépendamment les uns des autres. Une exception : le chapitre~\ref{c.origami-axioms} sur les axiomes de l'origami est nécessaire aux chapitres~\ref{c.origami-cube} et \ref{c.origami-constructions}, les autres chapitres sur l'origami. On présente les remarques relatives à tous les chapitres ci-dessous sous forme de liste.

\subsection*{Qu'est-ce qu'une surprise ?}

Il y avait trois critères pour inclure un sujet dans le livre :
\begin{itemize}
\item Le théorème m'a surpris. Les théorèmes sur la constructibilité avec une règle et un compas étaient particulièrement surprenants. Les mathématiques extrêmement riches de l'origami étaient presque choquantes : lorsqu'une  professeur de mathématiques m'a proposé un projet sur l'origami, j'ai d'abord refusé parce que je doutais qu'il puisse y avoir des mathématiques sérieuses associées à cette forme d'art.
D'autres sujets ont été inclus parce que, même si je connaissais les résultats, leurs démonstrations  étaient surprenantes par leur élégance et leur accessibilité, en particulier la démonstration  purement algébrique de Gauss qu'il est possible de construire un heptadécagone régulier.

\item Ce sujet n'apparaît pas dans les manuels du lycée  et du supérieur, et je n'ai trouvé ces théorèmes et démonstrations que dans des manuels avancés et dans la littérature de recherche. Il existe des articles Wikipédia sur la plupart des sujets, mais il faut savoir où chercher et les articles sont souvent sommaires.

\item Les théorèmes et les démonstrations  sont accessibles avec une bonne connaissance des mathématiques du lycée.
\end{itemize}
Chaque chapitre se termine par un paragraphe \og Quelle est la surprise?\fg{} qui explique mon choix du sujet.

\subsection*{Un aperçu du contenu}

Le chapitre~\ref{c.collapse} présente la démonstration d'Euclide que toute construction possible avec un compas à charnière est possible avec un compas éphémère. De nombreuses démonstrations ont été données, mais, comme l'a montré  Toussaint, la plupart ne sont pas correctes car elles dépendent de figures qui ne représentent pas toujours correctement la géométrie du problème. Pour souligner qu'il ne faut pas se fier aux figures, je présente la fameuse fausse démonstration  que tout triangle est isocèle. 

Au fil des siècles, les mathématiciens ont cherché sans succès à réaliser la trisection d'un angle arbitraire (c'est-à-dire à le diviser en trois parties égales) en utilisant uniquement une règle et un compas. Underwood Dudley a fait une étude complète des \og trisecteurs\fg{} qui donnent des constructions qui ne sont pas correctes; la plupart des constructions sont des approximations que l'on prétend exactes. Le chapitre~\ref{c.trisect} commence par présenter deux de ces constructions et développe les formules trigonométriques qui montrent qu'elles ne sont que des approximations. Pour montrer que la trisection à l'aide d'une simple règle et d'un compas n'a aucune importance pratique, on présente des trisections qui utilisent des outils plus complexes : la \emph{neusis} d'Archimède et la quadratrice d'Hippias. Le chapitre se termine par la démonstration  que la trisection  d'un angle arbitraire est impossible  
 avec une règle et un compas. 

La quadrature du cercle (à partir d'un cercle, construire un carré de même surface) ne peut pas être réalisée à l'aide d'une règle et d'un compas, car la valeur de $\pi$ ne peut pas être construite. Le chapitre~\ref{c.square} présente trois élégantes constructions d'approximations proches de $\pi$, une de Kocha\'{n}ski et deux de Ramanujan. Le chapitre se termine en montrant qu'on peut utiliser une quadratrice  pour résoudre la quadrature du  cercle.

Le théorème des quatre couleurs stipule qu'il est possible de colorier toute carte planaire avec quatre couleurs, de sorte que deux pays ayant une frontière commune ne soient pas coloriés avec la même couleur. La démonstration de ce théorème est extrêmement compliquée, mais la démonstration du théorème des cinq couleurs est élémentaire et élégante, comme le montre le chapitre~\ref{c.five}. Ce chapitre présente également la démonstration de Percy Heawood que la \og démonstration\fg{} du théorème des quatre couleurs d'Alfred Kempe n'est pas correcte.

Combien de gardiens 
 un musée d'art doit-il employer   pour que tous les murs soient sous l'observation constante d'au moins un gardien ? La démonstration du chapitre~\ref{c.museum} est assez astucieuse, car elle utilise la coloriage de graphes pour résoudre ce qui, à première vue, semble être un problème purement géométrique.

Le chapitre~\ref{c.induction} présente quelques résultats moins connus et leurs démonstrations  par récurrence : théorèmes sur la suite de Fibonacci et sur les nombres de Fermat, la fonction $91$ de McCarthy et le problème de Josèphe.

Le chapitre~\ref{c.quadratic} traite de la méthode de Po-Shen Loh pour résoudre les équations du second degré. Cette méthode est un élément essentiel de la démonstration algébrique de Gauss qu'un heptadécagone peut être construit à la règle et au compas (chapitre~\ref{c.heptadecagon}). Le chapitre inclut la construction géométrique d'Al-Khwârizmî pour trouver les racines des équations du second degré et une construction géométrique utilisée par Cardan dans le développement de la formule pour trouver les racines des équations du troisième degré.

La théorie de Ramsey est un sujet de la combinatoire qui constitue un domaine de recherche actif. Elle recherche des motifs parmi les sous-ensembles de grands ensembles. Le chapitre~\ref{c.ramsey} présente des exemples simples de triplets de Schur, de triplets pythagoriciens, de nombres de Ramsey et de problèmes de van der Waerden. La démonstration du théorème sur les triplets pythagoriciens a été obtenue récemment à l'aide d'un programme informatique basé sur la logique mathématique. Le chapitre se termine par une digression sur la connaissance des triplets pythagoriciens par les anciens Babyloniens.

C. Dudley Langford a observé son fils qui jouait avec des blocs de couleur et a remarqué qu'il les avait disposés dans une suite intéressante. Le chapitre~\ref{c.langford} présente son théorème sur les conditions de possibilité d'une telle suite.

Le chapitre~\ref{c.origami-axioms} contient les sept axiomes de l'origami, ainsi que les calculs détaillés de la géométrie analytique des axiomes et les caractérisations des plis en tant que lieux géométriques.

Le chapitre~\ref{c.origami-cube} présente la méthode d'Eduard Lill et le pliage d'origami proposé par Margharita P. Beloch. Je présente la méthode de Lill comme un tour de magie, je ne vais donc pas le gâcher en donnant des détails ici.

Le chapitre~\ref{c.origami-constructions} montre que l'origami permet de réaliser des constructions impossibles avec une règle et un compas : la trisection d'un angle, la duplication d'un cube et la construction d'un nonagone (polygone régulier à neuf côtés).

Le chapitre~\ref{c.compass} présente le théorème de Georg Mohr et Lorenzo Mascheroni selon lequel toute construction avec une règle et un compas peut être réalisée en utilisant uniquement un compas.

L'affirmation correspondante selon laquelle une règle suffit n'est pas  correcte, car une règle ne peut pas reproduire les longueurs qui sont des racines carrées. Jean-Victor Poncelet a conjecturé et Jakob Steiner a démontré qu'une règle est suffisante à condition qu'il existe un seul cercle fixe quelque part dans le plan (chapitre~\ref{c.straightedge}).

Si deux triangles ont le même périmètre et la même aire, sont-ils forcément isométriques? Cela semble raisonnable mais il s'avère que ce n'est pas vrai, bien qu'il faille un peu d'algèbre et de géométrie pour trouver deux pareils triangles non isométriques, comme le montre le chapitre~\ref{c.congruent}.

Le chapitre~\ref{c.heptadecagon} présente le tour de force de Gauss : la démonstration qu'un heptadécagone (un polygone régulier à dix-sept côtés) peut être construit à l'aide d'une règle et d'un compas. Par un argument astucieux sur la symétrie des racines des polynômes, il a obtenu une formule qui n'utilise que les quatre opérateurs arithmétiques et les racines carrées. Gauss n'a pas donné de construction explicite d'un heptadécagone, c'est pourquoi nous présentons l'élégante construction de James Callagy. Le chapitre se termine par la construction d'un pentagone régulier basée sur la méthode de Gauss pour la construction d'un heptadécagone.

Pour que le livre soit aussi complet que possible, l'annexe~\ref{a.trig} rassemble les démonstrations de théorèmes de géométrie et de trigonométrie qui ne sont peut-être pas familiers au lecteur.

\subsection*{Style}

\begin{itemize}
\item Nous supposons que le lecteur a une bonne connaissance des mathématiques du lycée, notamment: 
\begin{itemize}
\item en algèbre, les polynômes et la division de polynômes, les polynômes {unitaires} (ceux dont le coefficient de la plus haute puissance est $1$), les équations du second degré, la multiplication d'expressions avec des exposants $a^m\cdot a^n=a^{m+n}$.
\item en géométrie euclidienne, les  triangles isométriques  $\triangle ABC \cong \triangle DEF$ et les cas d'égalité des triangles, les triangles semblables $\triangle ABC \sim \triangle DEF$ et les rapports de leurs côtés, les cercles et leurs angles inscrits et centraux.
\item en géométrie analytique, le plan cartésien, le calcul des longueurs et des pentes des segments de droite, l'équation d'un cercle.
\item en trigonométrie, les fonctions $\sin,\cos,\tan$ et leurs relations, les angles dans le cercle unitaire, les fonctions trigonométriques des angles supplémentaires telles que $\cos (180^\circ-\theta)=-\cos\theta$.
\end{itemize}
\item Les énoncés à démontrer sont appelés \og théorèmes\fg{} sans que l'on tente de faire la distinction entre les théorèmes, les lemmes et les corollaires.
\item Lorsqu'un théorème suit une construction, les variables qui apparaissent dans le théorème font référence aux points, droites et angles marqués dans la figure qui accompagne la construction.
\item Nous avons donné les noms complets des mathématiciens sans les informations biographiques qui peuvent être trouvées facilement dans Wikipédia.
\item Le livre est écrit de manière à être aussi autonome que possible, mais la présentation dépend parfois de concepts mathématiques avancés et de théorèmes qui sont donnés sans démonstration. Dans ce cas, on trouvera un résumé technique dans des encadrés que l'on peut sauter.
\item Il n'y a pas d'exercices mais le lecteur ambitieux est invité à démontrer chaque théorème avant de lire la démonstration.
\item Les constructions géométriques peuvent être étudiées à l'aide de logiciels tels que {Geogebra}.
\item  $\overline{AB}$ est utilisé à la fois pour le nom d'un segment  et pour la longueur du segment.
\item $\triangle ABC$ est utilisé à la fois pour le nom d'un triangle et pour l'aire du triangle.
\end{itemize}


\subsection*{Remerciements}

Ce livre n'aurait jamais vu le jour sans les encouragements d'Abraham Arcavi, qui m'a permis d'empiéter sur son domaine de l'enseignement des mathématiques. Il a aussi gracieusement rédigé l'avant-propos. Avital Elbaum Cohen et Ronit Ben-Bassat Levy ont toujours été prêtes à m'aider à (ré)apprendre les mathématiques du secondaire. Oriah Ben-Lulu m'a initié aux mathématiques de l'origami et a collaboré aux démonstrations. Je suis reconnaissant à Michael Woltermann de m'avoir permis d'utiliser plusieurs sections de son adaptation du livre de Heinrich D\"{o}rrie. Jason Cooper, Richard Kruel, Abraham Arcavi et les relecteurs anonymes ont fourni des commentaires utiles.

Je tiens à remercier l'équipe de Springer pour son soutien et son professionnalisme, en particulier l'éditeur Richard Kruel.

Ce livre est publié dans le cadre du programme \og accès ouvert\fg{}. Je tiens à remercier l'Institut Weizmann des sciences pour le financement de la publication.

Les fichiers sources \LaTeX{} du livre (y compris la source Ti\textit{k}Z pour les figures) sont disponibles à l'adresse suivante :
\begin{center}
\url{https://github.com/motib/surprises}
\end{center}

\medskip

\begin{flushright}
{Mordechai (Moti) Ben-Ari}
\end{flushright}

\newpage

\lhead[\fancyplain{}{\thepage}]%
{\fancyplain{}{\nouppercase{Introduction à la traduction }}}
\rhead[\fancyplain{}{}]%
{\fancyplain{}{\thepage}}
\cfoot{}
\pagestyle{fancyplain}

\chapter*{Introduction à la traduction}

\epigraph{Sauver l'honneur, mais pour mourir: option Massada. Sauver l'avenir, mais en se reniant: option Flavius \mbox{Josèphe}.}{Régis Debray \cite{Debray}} 

Pour l'année 2020, la base de données  ZBmath, gérée depuis l'Allemagne et désormais en accès libre, recense 2\,184 livres de mathématiques\footnote{MathSciNet recense un moins grand nombre de livres. De plus, de manière assez significative, la langue de publication ne fait pas partie des critères de recherche.}. Ce nombre évolue au fur et à mesure que de nouveaux livres sont recensés dans la base, avec plus ou moins de retard. Parmi ces livres, 2\,030 sont en anglais, 106 en allemand, 33 en français, 14 en italien, 6 en espagnol, 4 en portugais, 2 en russe, 2 en polonais... Certains livres peuvent être bilingues. On remarque aussi que 1\,066 livres contenaient \og Springer \fg{} quelque part dans leur notice bibliographique, 116 \og World Scientific\fg{}, 114 \og Cambridge University Press\fg, 107 \og American Mathematical Society\fg, 77 \og Birkhäuser\fg{}, 55 \og Oxford University Press\fg, 48 \og Wiley\fg, 26 \og Elsevier\fg, 13 \og Société mathématique de France\fg{}... \'Evidemment, il y a un biais à la fois linguistique et éditorial, car Springer est une entreprise avec une base principale en Allemagne (comme ZBMath)  qui publie principalement en anglais et secondairement en allemand. De plus, des éditeurs de nombreux pays que ce soit en Chine, au Japon, au Brésil ou même en Europe ne signalent pas leurs publications à ZBMath.  Ces chiffres donnent néanmoins une certaine idée du degré de dépendance  linguistique vis-à-vis de l'anglais  et de dépendance écononomique vis-à-vis de Springer auquel est arrivée la communauté mathématique en ce début de XXI$^\text{e}$ siècle.

Deux tendances s'opposent depuis quelques années à ce double mouvement de concentration. La première vise à s'éloigner des éditeurs commerciaux  et à favoriser des structures publiques comme le centre Mersenne à Grenoble, qui gère une vingtaine de revues scientifiques, principalement en mathématiques. C'est ainsi que l'Académie des sciences à Paris a décidé de retirer la gestion de ses Comptes rendus à Elsevier pour la confier au Centre Mersenne. Cela s'inscrit dans le cadre de la promotion de la \og science ouverte\fg. Les éditeurs commerciaux se sont  adaptés à cette nouvelle donne en proposant aussi des services appelés \og accès ouvert\fg{}, où l'auteur (plus exactement son institution) paie la publication, que ce soit sous forme d'article ou  de livre. 

La deuxième tendance essaie de tirer avantage des progrès récents de la traduction automatique  pour promouvoir la diversité linguistique. Certes la traduction automatique est également utilisée pour produire toujours plus de livres en anglais: Springer a déjà traduit automatiquement puis relu et corrigé une douzaine de livres de mathématiques d'allemand en anglais, tandis que ISTE basé au Royaume-Uni a  fait de même du français vers l'anglais pour quelques livres. Mais il y a aussi des tentatives pour traduire vers des langues autres que l'anglais: voir par exemple la traduction en une vingtaine de langues de \cite{Bacaer} ou en cinq langues de \cite{Bacaer2}.

La présente traduction se situe au carrefour de ces deux tendances. Le texte source est un livre publié en accès ouvert en anglais\footnote{Il existe également une version du livre en hébreu.}  par Springer sous une licence CC-BY. L'auteur a d'ailleurs eu la bonne idée de mettre en accès libre les fichiers sources en \LaTeX{}. D'autre part, on a utilisé le traducteur automatique DeepL pour faire une première traduction rapide de l'anglais vers le français, traduction que l'on a ensuite relue et corrigée. C'est la même méthode que celle utilisée pour les traductions de \cite{Bacaer,Bacaer2}, à part que cette fois la langue cible est le français.

 Au cours des dernières années, peu de livres de mathématiques ont été traduits en français. Pour l'année 2020 par exemple, on n'a trouvé que six livres: quatre    traduits de l'anglais (deux publiés par Cassini, un par Dunod, un par les Presses universitaires de France-Comté), un traduit du néerlandais (publié par Vuibert) et un  traduit du russe (publié par les éditions du Bec de l'aigle). Nous espérons que la méthode que nous avons utilisée suscitera un plus grand nombre de traductions.

Pour compléter la bibliographie de cette traduction par des références en français, mentionnons par exemple \cite{Carrega,Lafforgue} pour les constructions à la règle et au compas, \cite{Aassila,Gowers} pour le théorème des cinq couleurs et  la théorie de Ramsey, ou \cite{Delahaye} pour les mathématiques de l'origami.

\medskip

\begin{flushright}
{Nicolas Bacaër}
\end{flushright}

\newpage

\lhead[\fancyplain{}{\thepage}]%
{\fancyplain{}{\nouppercase{Table des matières }}}
\rhead[\fancyplain{}{}]%
{\fancyplain{}{\thepage}}
\cfoot{}
\pagestyle{fancyplain}

\tableofcontents

