\chapter{Une règle et un cercle suffisent}\label{c.straightedge}

%%%%%%%%%%%%%%%%%%%%%%%%%%%%%%%%%%%%%%%%%%%%%%%%%%%%%%%%%%%%%%%

%%%%%%%%%%%%%%%%%%%%%%%%%%%%%%%%%%%%%%%%%%%%%%%%%%%%%%%%%%%%%%%

Toute construction à la règle et au  compas peut-elle être réalisée uniquement avec une règle? La réponse est non car les droites sont définies par des équations linéaires et ne peuvent pas représenter des cercles qui sont définis par des équations du second degré. En 1822, Jean-Victor Poncelet a conjecturé qu'une règle seule suffit, à condition qu'un cercle existe dans le plan. Ceci a été démontré en 1833 par Jakob Steiner.

Après avoir expliqué dans la section~\ref{s.se-what} ce que signifie effectuer une construction avec seulement une règle et un cercle, la démonstration est présentée par étapes en commençant par cinq constructions auxiliaires : la construction d'une droite parallèle à une droite donnée (sect.~\ref{s.parallel}), la construction d'une perpendiculaire à une droite donnée (sect.~\ref{s.perp}), la copie d'un segment  dans une direction donnée (sect.~\ref{s.copy}), la construction d'un segment  qui est le rapport d'autres segments (sect.~\ref{s.relative}) et la construction d'une racine carrée (sect.~\ref{s.root}). La section~\ref{s.line-circle-straight} montre comment trouver l'intersection ou les intersections d'une droite avec un cercle et la section~\ref{s.two-circles} montre comment trouver l'intersection ou les intersections de deux cercles.

\section{Qu'est-ce qu'une construction avec seulement une règle ?}\label{s.se-what}
Une construction à la règle et au  compas est une suite de trois opérations :
\begin{itemize}
\item trouver le point d'intersection de deux droites.
\item trouver  les points d'intersection d'une droite et d'un cercle.
\item trouver  les points d'intersection de deux cercles.
\end{itemize}
La première opération peut être effectuée avec une règle seule.

Un cercle est défini par un point $O$, son centre, et par un rayon $r$, un segment de longueur $r$ dont l'une des extrémités est le centre. Si nous pouvons construire les points marqués $X$ et $Y$ sur la figure~\ref{f.se-only-line-circle}, nous pouvons affirmer avoir construit avec succès les points d'intersection d'un cercle donné avec une droite donnée. De même, la construction de $X$ et $Y$ dans la figure~\ref{f.se-only-two-circles} est la construction des points d'intersection de deux cercles donnés. Les cercles dessinés en pointillés dans les figures  n'apparaissent pas réellement dans une construction ; ils sont juste utilisés pour aider à comprendre la construction.

Le cercle unique utilisé dans les constructions, appelé \emph{cercle fixe}, peut apparaître n'importe où dans le plan et peut avoir un rayon arbitraire.

\vspace{0.4cm}

\begin{minipage}{0.4\textwidth}
\centering      
\begin{tikzpicture}[scale=.7]
\coordinate (O) at (0,0);
\node[above right] at (O) {$O$};
\draw[thick,dashed,name path=circle] (0,0) circle[radius=2cm];
\draw (0,0) -- node[left] {$r$} ++(-60:2cm);
\draw[name path=line] (-3,-.5) -- ++(20:6cm);
\path [name intersections={of=circle and line,by={X,Y}}];
\node[above right,xshift=-2pt,yshift=4pt] at (X) {$X$};
\node[above left] at (Y) {$Y$};
\vertex{O};
%\vertex{X};
%\vertex{Y};
\end{tikzpicture}
%\includegraphics[width=\textwidth]{Fig14_1a}
         \captionof{figure}{$X$ et $Y$ sont les points d'intersection d'une droite et d'un cercle.}\label{f.se-only-line-circle}
         
     \end{minipage}
     \hspace{3em}
     \begin{minipage}{0.4\textwidth}
\centering         
\begin{tikzpicture}[scale=.7]
\coordinate (O1) at (0,0);
\coordinate (O2) at (3,0);
\node[above right] at (O1) {$O_1$};
\node[above right] at (O2) {$O_2$};
\draw[thick,dashed,name path=circle1] (0,0) circle[radius=2cm];
\draw[thick,dashed,name path=circle2] (3,0) circle[radius=1.7cm];
\draw (0,0) -- node[left] {$r_1$} ++(-60:2cm);
\draw (3,0) -- node[left,below] {$r_2$} ++(-20:1.7cm);
\path [name intersections={of=circle1 and circle2,by={X,Y}}];
\node[above,yshift=4pt] at (X) {$X$};
\node[below,yshift=-4pt] at (Y) {$Y$};
\vertex{O1};
\vertex{O2};
%\vertex{X};
%\vertex{Y};
\end{tikzpicture}
%\includegraphics[width=\textwidth]{Fig14_1b}
         \captionof{figure}{$X$ et $Y$ sont les points d'intersection de deux cercles.}\label{f.se-only-two-circles}
     \end{minipage}





\section{Construction d'une droite parallèle à une droite donnée}\label{s.parallel}

\begin{theorem}\label{thm.straight-parallel}
Étant donné une droite $l$ définie par deux points $A$ et $B$ et un point $P$ qui n'est pas sur la droite, il est possible de construire une droite qui passe par $P$ et qui est parallèle à $\overline{AB}$.
\end{theorem}

\begin{proof}

Il y a deux cas pour la démonstration.

\textit{Cas 1.}
 $\overline{AB}$ est un \emph{segment dirigé}, c'est-à-dire que  le milieu $M$ de $\overline{AB}$ est donné.  Construisons une demi-droite qui prolonge $\overline{AP}$ et choisissons un point quelconque $S$ sur la droite au-delà de $P$. Construisons les droites $\overline{BP}$, $\overline{SM}$ 
 et $\overline{SB}$. L'intersection de $\overline{BP}$ et de $\overline{SM}$ est notée $O$. Construisons une demi-droite qui prolonge $\overline{AO}$ et notons $Q$ l'intersection de la demi-droite $\overline{AO}$ avec $\overline{SB}$  (fig.~\ref{f.se-parallel-directed}).

Nous affirmons que  $\overline{PQ}\parallel \overline{AB}$. 

\begin{figure}[ht]
\centering
\begin{tikzpicture}
\draw[name path=pq] (-4,0) -- (4,0);
\draw (-2,-2) node[below left] {$A$} coordinate (A) -- (2,-2) node[below right] {$B$} coordinate (B);
\draw[name path=as] (A) -- ++(50:4cm) node[above] {$S$} coordinate (S);
\draw[name path=sb] (S) -- (B);
\path [name intersections={of=pq and as,by={P}}];
\path [name intersections={of=pq and sb,by={Q}}];
\node[above left] at (P) {$P$};
\node[above right] at (Q) {$Q$};
\draw[name path=pb] (P) -- (B);
\draw[name path=qa] (Q) -- (A);
\path [name intersections={of=pb and qa,by={O}}];
\node[right,xshift=2pt] at (O) {$O$};
\coordinate (M) at (0,-2);
\node[below right] at (M) {$M$};
\draw (S) -- (M);
%\vertex{O};
%\vertex{P};
%\vertex{A};
%\vertex{B};
\end{tikzpicture}
%\includegraphics[width=0.8\textwidth]{Fig14_2}

\caption{Construction d'une droite parallèle dans le cas d'un segment  dirigé.}\label{f.se-parallel-directed}
\end{figure}


La démonstration utilise le théorème de Ceva.

\textit{Théorème de Ceva   (théorème~\ref{thm.ceva}).} Si les segments  allant des sommets d'un triangle aux arêtes opposées se coupent en un point $O$ (comme sur la figure~\ref{f.se-parallel-directed}), alors les longueurs des segments satisfont 
\[
\frac{\overline{AM}}{\overline{MB}}\cdot\frac{\overline{BQ}}{\overline{QS}}\cdot\frac{\overline{SP}}{\overline{PA}} = 1\,.
\]
Dans la figure~\ref{f.se-parallel-directed}, $M$ est le milieu de $\overline{AB}$ donc 
 $\displaystyle\frac{\overline{AM}}{\overline{MB}}=1$ et l'équation devient 
\begin{align}
\frac{\overline{BQ}}{\overline{QS}}=\frac{\overline{PA}}{\overline{SP}}=\frac{\overline{AP}}{\overline{PS}}\,,\label{eq.ceva}
\end{align}
puisque l'ordre des extrémités d'un segment n'est pas important.

Nous affirmons que $\triangle ABS \sim \triangle PQS$:
\begin{align*}
\frac{\overline{BS}}{\overline{QS}}&=\frac{\overline{BQ}}{\overline{QS}}+\frac{\overline{QS}}{\overline{QS}} = \frac{\overline{BQ}}{\overline{QS}}+1\\
&\\
\frac{\overline{AS}}{\overline{PS}} &= \frac{\overline{AP}}{\overline{PS}} + \frac{\overline{PS}}{\overline{PS}} = \frac{\overline{AP}}{\overline{PS}} + 1\,.
\end{align*}
En utilisant l'équation~\ref{eq.ceva},
\[
\frac{\overline{BS}}{\overline{QS}}=\frac{\overline{BQ}}{\overline{QS}}+1=\frac{\overline{AP}}{\overline{PS}}+1=\frac{\overline{AP}}{\overline{PS}}+\frac{\overline{PS}}{\overline{PS}}=\frac{\overline{AS}}{\overline{PS}}\,,
\]
et il s'ensuit que $\triangle ABS \sim \triangle PQS$ et donc 
 $\overline{PQ}\parallel\overline{AB}$.

\textit{Cas 2:}
$\overline{AB}$ n'est pas nécessairement un segment  dirigé. Le cercle fixe $c$ a pour centre $O$ et pour rayon $r$. $P$ est le point qui n'est pas sur la droite pour laquelle il faut construire une droite parallèle à $l$ (figure~\ref{f.se-parallel-other1}).

Choisissons $M$, un point quelconque sur $l$, et construisons une demi-droite qui prolonge $\overline{MO}$ et qui coupe le cercle en $U$ et $V$.
$\overline{UV}$ est un segment  dirigé car $O$, le centre du cercle, est le milieu du diamètre $\overline{UV}$. Choisissons un point $A$ sur $l$ et utilisons la construction d'un segment  dirigé (cas 1) pour construire une droite passant par $A$ et parallèle à $\overline{UV}$ qui coupe le cercle en $X$ et $Y$  (fig.~\ref{f.se-parallel-other2}).

\vspace{0.4cm}

\begin{minipage}{0.4\textwidth}
\centering      
\begin{tikzpicture}[scale=.65]
\coordinate (O) at (0,0);
\node[below right] at (O) {$O$};
\draw[name path=circle] (O) circle[radius=2cm];
\draw[name path=l] (-4,-3) --
  node[above, near end,xshift=24pt] {$l$} +(7,0);
\path[name path=mo] (-2,-3) coordinate (M) -- 
  ($(-2,-3)!1.65!(O)$);
\node[below] at (M) {$M$};
\path [name intersections={of=circle and mo,by={V,U}}];
\node[below,xshift=2pt,yshift=-4pt] at (U) {$U$};
\node[right,xshift=4pt] at (V) {$V$};
\draw (M) -- (U) -- node[left] {$r$} (O) -- node[left] {$r$} (V);
\node at (-1.6,1.6) {$c$};
\coordinate (P) at (-4,1);
\node[above left] at (P) {$P$};
\vertex{O};
\vertex{P};
\end{tikzpicture}
%\includegraphics[width=\textwidth]{Fig14_3a}
         \captionof{figure}{Construction d'un segment  dirigé.}\label{f.se-parallel-other1}
         
     \end{minipage}
     \hspace{3em}
     \begin{minipage}{0.4\textwidth}
\centering         
\begin{tikzpicture}[scale=.65]
\coordinate (O) at (0,0);
\node[below right] at (O) {$O$};
\draw[name path=circle] (O) circle[radius=2cm];
\draw[name path=l] (-4,-3) --
  node[above,near end,xshift=24pt] {$l$} +(7,0);
\path[name path=mo] (-2,-3) coordinate (M) --
  ($(-2,-3)!1.65!(O)$);
\node[below] at (M) {$M$};
\path [name intersections={of=circle and mo,by={V,U}}];
\node[below,xshift=2pt,yshift=-4pt] at (U) {$U$};
\node[right,xshift=4pt] at (V) {$V$};
\draw (M) -- (V);
\path[name path=ax] (-3,-3) coordinate (A) --
  ($(-3,-3)!1.8!(-1,0)$);
\node[below] at (A) {$A$};
\path [name intersections={of=circle and ax,by={Y,X}}];
\node[left] at (X) {$X$};
\node[above] at (Y) {$Y$};
\node at (-1.6,1.6) {$c$};
\draw (A) -- (Y);
\coordinate (P) at (-4,1);
\node[above left] at (P) {$P$};
\vertex{O};
\vertex{P};
\end{tikzpicture}
%\includegraphics[width=\textwidth]{Fig14_3b}
         \captionof{figure}{Construction d'une droite parallèle au segment  dirigé.}\label{f.se-parallel-other2}
     \end{minipage}

\vspace{0.4cm}



\begin{figure}[htbp]
\centering
\begin{tikzpicture}[scale=.75]
\coordinate (O) at (0,0);
\node[below right] at (O) {$O$};
\draw[name path=circle] (O) circle[radius=2cm];
\draw[name path=l] (-5,-3) --
  node[above,near end,xshift=40pt] {$l$} +(9,0);
\path[name path=mo] (-2,-3) coordinate (M) -- 
  ($(-2,-3)!1.65!(O)$);
\node[below] at (M) {$M$};
\path [name intersections={of=circle and mo,by={V,U}}];
\node[below,xshift=2pt,yshift=-4pt] at (U) {$U$};
\node[above right] at (V) {$V$};
\draw (M) -- (V);
\path[name path=ax] (-3,-3) coordinate (A) -- 
  ($(-3,-3)!1.8!(-1,0)$);
\node[below] at (A) {$A$};
\path [name intersections={of=circle and ax,by={Y,X}}];
\node[left] at (X) {$X$};
\node[above] at (Y) {$Y$};
\node at (-1.6,1.6) {$c$};
\draw (A) -- (Y);
\coordinate (P) at (-4,1);
\node[above] at (P) {$P$};
\path[name path=xo] (X) -- ($(X)!2.2!(O)$);
\path[name intersections={of=circle and xo,by={Xp}}];
\node[above right] at (Xp) {$X'$};
\draw (X) -- (Xp);
\path[name path=yo] (Y) -- ($(Y)!2.2!(O)$);
\path[name intersections={of=circle and yo,by={y,Yp}}];
\node[below right] at (Yp) {$Y'$};
\draw (Y) -- (Yp);
\path[name path=xy] (Xp) -- ($(Xp)!1.6!(Yp)$);
\path[name intersections={of=l and xy,by={B}}];
\node[below] at (B) {$B$};
\draw (Xp) -- (B);
\draw[thick,dotted,name path=z] (-5,0) -- 
  (4,0) node[above,near end,xshift=40pt] {$l'$};
\draw[thick,dashed] (-5,1) -- +(9,0);
\path[name intersections={of=ax and z,by={Z}}];
\path[name intersections={of=xy and z,by={Zp}}];
\node[above left] at (Z) {$Z$};
\node[below right] at (Zp) {$Z'$};
\vertex{O};
\vertex{P};
\end{tikzpicture}
%\includegraphics[width=0.8\textwidth]{Fig14_4}
\caption{Démonstration que $l'$ est parallèle à $l$.}\label{f.se-parallel-other3}
\end{figure}

Construisons un diamètre de $X$ à $O$ qui coupe l'autre côté du cercle en $X'$, et construisons de la même façon le diamètre $\overline{YY'}$. Construisons la demi-droite de $X'$ à $Y'$ et notons $B$ son intersection avec $l$. Nous affirmons que $M$ est le milieu de $\overline{AB}$, de sorte que $\overline{AB}$ est un segment  dirigé. On peut donc construire une droite passant par $P$ et parallèle à $l$ (fig.~\ref{f.se-parallel-other3}).


$\overline{OX}$, $\overline{OX'}$, $\overline{OY}$ et $\overline{OY'}$ sont tous des rayons du cercle et $\angle XOY = \angle X'OY'$ puisqu'il s'agit d'angles opposés par le sommet, donc $\triangle XOY\cong\triangle X'OY'$ (deux côtés et un angle égaux). Définissons\footnote{Définissons, et non construisons, car nous sommes au milieu de la démonstration qu'une telle droite peut être construite.} $l'$ comme étant une droite qui passe par $O$ et parallèle à $l$ qui coupe $\overline{XY}$ en $Z$ et $\overline{X'Y'}$ en $Z'$. $\angle XOZ=\angle X'OZ'$ sont des angles opposés par le sommet, $\angle ZXO=\angle Z'X'O$ sont des angles  alternes-internes et $\overline{XO}=\overline{XO'}$ sont des rayons, donc $\triangle XOZ\cong\triangle X'OZ'$ (deux angles et un côté égaux) et $\overline{ZO}=\overline{OZ'}$. Par conséquent, $\overline{AMOZ}$ et $\overline{BMOZ'}$ sont des parallélogrammes et $\overline{AM}=\overline{ZO}=\overline{OZ'}=\overline{MB}$.
\end{proof}

\begin{theorem}
Étant donné un segment $\overline{AB}$ et un point $P$ non situé sur la droite, il est possible de construire un segment  $\overline{PQ}$ parallèle à $\overline{AB}$ et dont la longueur est égale à la longueur de $\overline{AB}$, c'est-à-dire qu'il est possible de copier $\overline{AB}$ parallèlement à lui-même avec $P$ comme l'un de ses points d'extrémité.
\end{theorem}

\begin{proof}
Nous avons démontré qu'il est possible de construire une droite $m$ qui passe par $P$ et parallèle à $\overline{AB}$ et une droite $n$ qui passe par $B$ et parallèle à $\overline{AP}$. Le quadrilatère $\overline{ABQP}$ est un parallélogramme, donc les côtés opposés sont égaux $\overline{AB}=\overline{PQ}$ (fig.~\ref{f.se-parallel-other4}).
\end{proof}

\begin{figure}[htbp]
\centering \begin{tikzpicture}[scale=.5]
\coordinate (P) at (0,0);
\coordinate (Q) at (3,0);
\coordinate (A) at (-2,2.5);
\coordinate (B) at (1,2.5);
\draw ($(P)!-.6!(Q)$) -- node[above,near end,xshift=36pt,yshift=-5pt] {$m$} ($(P)!1.8!(Q)$);
\node[below] at (P) {$P$};
\node[below left] at (Q) {$Q$};
\draw ($(A)!-.6!(B)$) -- node[above,near end,xshift=40pt,yshift=-5pt] {$l$} ($(A)!2.5!(B)$);
\node[above left] at (A) {$A$};
\node[above right] at (B) {$B$};
\draw (A) -- (P);
\draw ($(B)!-.3!(Q)$) -- node[above,near end,xshift=18pt,yshift=-18pt] {$n$} ($(B)!1.4!(Q)$);
\end{tikzpicture}
%\includegraphics[width=0.7\textwidth]{Fig14_5}
\caption{Construction d'une copie d'un segment parallèle à un segment  existant.}\label{f.se-parallel-other4}
\end{figure}

\section{Construction d'une perpendiculaire à une droite donnée.}\label{s.perp}

\begin{theorem}\label{thm.straight-perp}
Étant donné un segment  $l$ et un point $P$ non situé sur $l$, il est possible de construire une perpendiculaire à $l$ qui passe par $P$.
\end{theorem}

\begin{proof}
D'après le théorème~\ref{thm.straight-parallel}, construisons une droite $l'$ parallèle à $l$ qui coupe le cercle fixe en $U$ et $V$. Construisons le diamètre $\overline{UOU'}$ et la corde $\overline{VU'}$ (fig.~\ref{f.se-perp}). $\angle UVU'$ est un angle droit car il est sous-tendu par un diamètre. Par conséquent, $\overline{VU'}$ est perpendiculaire à $\overline{UV}$ et à $l$. De nouveau d'après le théorème~\ref{thm.straight-parallel}, construisons la parallèle à $\overline{VU'}$ qui passe par $P$.
\end{proof}

\begin{figure}[htbp]
\centering
\begin{tikzpicture}[scale=.7]
\coordinate (O) at (0,0);
\coordinate (P) at (3.5,.6);
\draw[name path=circle] (O) circle[radius=2cm];
\draw[name path=l] (-4,-3) -- node[above,near end,xshift=45pt] {$l$} ++(9,0);
\draw[name path=lp] (-3,-1) -- node[above,near end,xshift=40pt] {$l'$} ++(8,0);
\node[above left] at (O) {$O$};
\node[right] at (P) {$P$};
\path[name intersections={of=circle and lp,by={U,V}}];
\node[below left] at (U) {$U$};
\node[below right] at (V) {$V$};
\path[name path=d] (U) -- ($(U)!2.3!(O)$);
\path[name intersections={of=circle and d,by={Up}}];
\draw (U) -- (Up);
\node[above right] at (Up) {$U'$};
\draw (Up) -- (V);
\path[name path=p] (P) -- ++(0,-4);
\path[name intersections={of=p and l,by={X}}];
\draw (X) rectangle +(9pt,9pt);
\draw[rotate=90] (V) rectangle +(9pt,9pt);
\vertex{O};
\vertex{P};
\draw (P) -- ++(0,1);
\draw (P) -- (X);
\end{tikzpicture}
%\includegraphics[width=0.8\textwidth]{Fig14_6}
\caption{Construction d'une droite perpendiculaire.}\label{f.se-perp}
\end{figure}



\section{Copie d'un segment dans une direction donnée}\label{s.copy}

\begin{theorem}\label{thm.straight-direction}
Il est possible de construire une copie d'un segment  donné dans la direction d'une autre droite.
\end{theorem}

Par \og direction\fg{}, on veut dire  que la droite définie par deux points $A'$ et $H'$ a un angle $\theta$ par rapport à un certain axe. Le but est de construire $\overline{AS}=\overline{PQ}$ tel que $\overline{AS}$ ait le même angle $\theta$ par rapport à cet axe  (fig.~\ref{f.se-copy1}).

\begin{proof}
D'après le théorème~\ref{thm.straight-parallel}, il est possible de construire un segment  $\overline{AH}$ tel que $\overline{AH}\parallel\overline{A'H'}$ et de construire un segment  $\overline{AK}$ tel que $\overline{AK}\parallel\overline{PQ}$.
$\angle HAK=\theta$, il reste donc à trouver un point $S$ sur $\overline{AH}$ tel que $\overline{AS}=\overline{PQ}$.

\begin{figure}[htbp]
\centering  
\begin{tikzpicture}[scale=.7]
\coordinate (A) at (0,0);
\coordinate (P) at (3cm,2);
\coordinate (Q) at (4.5cm,2);
\draw (P) -- (Q);
\node[left] at (P) {$P$};
\node[right] at (Q) {$Q$};
\coordinate (A1) at (-3,1);
\draw (A1) -- ++(60:3cm) coordinate (H1);
\draw (A1) -- ++(0:2cm);
\node[left] at (A1) {$A'$};
\node[left] at (H1) {$H'$};
\draw (A) -- ++(60:1.5cm) coordinate (S);
\node[left] at (S) {$S$};
\draw (A) -- ++(1.5,0);
\node[left] at (A) {$A$};
\node[above right,xshift=4pt] at (A1) {$\theta$};
\node[above right,xshift=4pt] at (A) {$\theta$};
\draw (A) -- ++(60:3cm) coordinate (H);
\node[left] at (H) {$H$};
\draw (A) -- ++(1.5,0) coordinate (K);
\node[right] at (K) {$K$};
\vertex{P};
\vertex{Q};
\vertex{A};
\vertex{S};
\end{tikzpicture}
%\includegraphics[width=0.8\textwidth]{Fig14_7}
\caption{Copie d'un segment dans une direction donnée.}\label{f.se-copy1}
\end{figure}

Construisons deux rayons $\overline{OU}$ et $\overline{OV}$ du cercle fixe qui sont parallèles à $\overline{AH}$ et $\overline{AK}$ respectivement. Construisons une demi-droite passant par $K$ parallèle à $\overline{UV}$. On note $S$ son intersection avec $\overline{AH}$ (fig.~\ref{f.se-copy3}). Par construction, $\overline{AH}\parallel\overline{OU}$ et $\overline{AK}\parallel\overline{OV}$, donc $\angle SAK=\angle HAK=\angle UOV=\theta$.  $\overline{SK}\parallel\overline{UV}$ et $\triangle SAK\sim\triangle UOV$ (trois angles égaux), $\triangle UOV$ est isocèle car $\overline{OU}$ et $\overline{OV}$ sont des rayons du même cercle. Par conséquent, $\triangle SAK$ est isocèle et 
$\overline{AS}=\overline{AK}=\overline{PQ}$.
\end{proof}

\begin{figure}[htbp]
\centering
\begin{tikzpicture}[scale=.65]
\coordinate (A) at (0,0);
\coordinate (P) at (3cm,2);
\coordinate (Q) at (4.5cm,2);
\draw (P) -- (Q);
\node[left] at (P) {$P$};
\node[right] at (Q) {$Q$};
\coordinate (A1) at (-3,1);
\draw (A1) -- ++(60:3cm) coordinate (H1);
\node[left] at (A1) {$A'$};
\node[left] at (H1) {$H'$};
\node[left] at (A) {$A$};
\draw (A) -- ++(60:3cm) coordinate (H);
\node[left] at (H) {$H$};
\draw (A) -- ++(1.5,0) coordinate (K);
\node[right] at (K) {$K$};
\draw (A) -- (K);
\path (A) -- ++(60:1.5cm) coordinate (S);
\node[right] at (S) {$S$};
\draw (K) -- ($(K)!1.8!(S)$);
\node[above right,xshift=4pt] at (A) {$\theta$};
\node[above right,xshift=4pt] at (A1) {$\theta$};
\draw (A1) -- ++(1.5,0);
\vertex{P};
\vertex{Q};
\begin{scope}[xshift=3cm]
\coordinate (O) at (6,1);
\draw[name path=circle] (O) circle[radius=2.5cm];
\node[above left] at (O) {$O$};
\path[name path=u] (O) -- ++(60:2.5cm);
\path[name path=v] (O) -- ++(2.5,0);
\path[name intersections={of=circle and u,by={U}}];
\path[name intersections={of=circle and v,by={V}}];
\node[above right] at (U) {$U$};
\node[right] at (V) {$V$};
\draw (O) -- (U) -- (V) -- cycle;
\node[above right,xshift=4pt] at (O) {$\theta$};
\vertex{O};
\end{scope}
\end{tikzpicture}
%\includegraphics[width=\textwidth]{Fig14_8}
\caption{Utilisation du cercle fixe pour copier le segment.}\label{f.se-copy3}
\end{figure}

\section{Construction d'un segment  comme  rapport de segments}\label{s.relative}

\begin{theorem}\label{thm.straight-relative}
Étant donné des segments de longueurs $n$, $m$ et $s$, il est possible de construire un segment  de longueur 
\[x=\displaystyle\frac{n}{m}s\,.\]
\end{theorem}

\begin{proof}
Choisissons des points $A$, $B$ et $C$ non situés sur la même droite et construisons les demi-droites $\overline{AB}$ et $\overline{AC}$. Avec le théorème~\ref{thm.straight-direction}, il est possible de construire des points $M$, $N$ et $S$ tels que $\overline{AM}= m$, $\overline{AN} =n$ et  $\overline{AS}=s$. Avec le théorème~\ref{thm.straight-parallel}, on construit une droite qui passe par $N$, parallèle à $\overline{MS}$, qui coupe $\overline{AC}$ en $X$. On pose $x=\overline{AX}$  (fig.~\ref{f.se-three2}). $\triangle MAS\sim\triangle NAX$ (trois angles égaux) donc  $\displaystyle\frac{m}{n}=\displaystyle\frac{s}{x}$ et $x=\displaystyle\frac{n}{m}s$.
\end{proof}

\begin{figure}[htbp]
\centering
\begin{tikzpicture}[scale=.8]
\coordinate (A) at (0,0);
\draw[name path=ac] (A) node[left] {$A$} -- ++(7,0) node[right] {$C$};
\draw (A) -- ++(40:5cm) node[right] {$B$};
\path (A) -- node[above,xshift=-2pt] {$m$} ++(40:3cm) coordinate (M) node[above left] {$M$};
\path (A) -- ++(40:4cm) coordinate (N) node[above left] {$N$};
\path[name path=ms] (M) -- ++(-50:3.5cm);
\path[name path=nx] (N) -- ++(-50:4cm);
\path[name intersections={of=ac and ms,by={S}}];
\path[name intersections={of=ac and nx,by={X}}];
\node[below] at (S) {$S$};
\node[below] at (X) {$X$};
\path (A) -- node[below] {$s$} (S);
\draw (S) -- (M);
\draw (X) -- (N);
\draw[<->] ($(A)+(0,-.8)$) -- node[fill=white] {$x$} ($(X)+(0,-.8)$);
\draw[<->] ($(A)+(-.6,.8)$) -- node[fill=white] {$n$} ++(40:3.9cm);
\end{tikzpicture}
%\includegraphics[width=0.7\textwidth]{Fig14_9}

\caption{Triangles semblables pour construire le rapport des longueurs.}\label{f.se-three2}
\end{figure}

\section{Construction d'une racine carrée}\label{s.root}

\begin{theorem}\label{thm.straight-sqrt}
Étant donné des segments  de longueurs $a$ et $b$, il est possible de construire un segment  de longueur 
 $\sqrt{ab}$.
\end{theorem}

\begin{proof}
Nous voulons exprimer $x=\sqrt{ab}$ comme $x=\displaystyle\frac{n}{m}s$ afin d'utiliser le théorème~\ref{thm.straight-relative}.
\begin{itemize}
\setlength{\itemsep}{0pt}
\item Pour $n$ on utilise $d$, le diamètre du cercle fixe.
\item Pour $m$ on utilise $t=a+b$ qui peut être construit à partir de $a$ et $b$ d'après le théorème~\ref{thm.straight-direction}.
\item On pose $s=\sqrt{hk}$ où $h$ et $k$ sont définis comme des expressions qui font intervenir  les longueurs $a$, $b$, $t$ et $d$.
\end{itemize}
Posons $h=\displaystyle\frac{d}{t}a$ et $k=\displaystyle\frac{d}{t}b$. Calculons:
\begin{align*}
x&=\sqrt{ab}=\sqrt{\frac{th}{d}\frac{tk}{d}}=\sqrt{\left(\frac{t}{d}\right)^2hk}=\frac{t}{d}\sqrt{hk}=\frac{t}{d}s\\
h+k &= \frac{d}{t}a + \frac{d}{t}b = \frac{d(a+b)}{t} = \frac{dt}{t} = d\,.
\end{align*}
Avec le théorème~\ref{thm.straight-direction}, construisons $\overline{HA}= h$ sur un diamètre $\overline{HK}$ du cercle fixe. Puisque $h+k=d$, on a $\overline{AK}=k$ (fig.~\ref{f.se-sqrt}). Avec le théorème~\ref{thm.straight-perp}, on construit une perpendiculaire à $\overline{HK}$ en $A$ et on note $S$ l'intersection de cette droite avec le cercle. $\overline{OS}=\overline{OK}=d/2$ et $\overline{OA}=(d/2)-k$. 
\begin{figure}[htbp]
\centering
\begin{tikzpicture}[scale=.6]
\coordinate (O) at (0,0);
\coordinate (H) at (-3,0);
\coordinate (K) at (3,0);
\node at (-2.4,2.4) {$c$};
\draw (H) -- (K);
\draw[name path=circle] (O) circle[radius=3cm];
\node[below] at (O) {$O$};
\node[left] at (H) {$H$};
\node[right] at (K) {$K$};
\path[name path=as] (1,0) coordinate (A) -- ++(0,3.2);
\node[below] at (A) {$A$};
\path[name intersections={of=circle and as,by={S}}];
\node[above] at (S) {$S$};
\draw (A) -- node[right] {$s$} (S);
\path (H) -- node[above] {$h$} (A);
\path (A) -- node[above] {$k$} (K);
\draw (O) -- node[left,xshift=-2pt] {$\displaystyle\frac{d}{2}$} (S);
\node at (.5,-1.5) {$\displaystyle\frac{d}{2}-k$};
\draw[->] (.5, -1.2) -- ++(0,1);
\draw[rotate=90] (A) rectangle +(8pt,8pt);
\vertex{O};
\end{tikzpicture}
%\includegraphics[width=0.6\textwidth]{Fig14_10}
\caption{Construction d'une racine carrée.}\label{f.se-sqrt}
\end{figure}

D'après le théorème de Pythagore,
\begin{align*}
s^2&= \left(\frac{d}{2}\right)^2 - \left(\frac{d}{2}-k\right)^2\\
&= \left(\frac{d}{2}\right)^2 - \left(\frac{d}{2}\right)^2 + 2\frac{dk}{2} - k^2\\
&= k(d-k) = kh\\
s&=\sqrt{hk}\,.
\end{align*}
Maintenant, $x=\displaystyle\frac{t}{d}s$ peut être construit d'après le théorème~\ref{thm.straight-relative}.
\end{proof}

\section{Construction de l'intersection d'une droite et d'un cercle}\label{s.line-circle-straight}

\begin{theorem}
Étant donné une droite $l$ et un cercle $c(O,r)$, il est possible de construire leurs points d'intersection  (fig.~\ref{f.se-line-circle1}).
\end{theorem}
\begin{figure}[thbp]
\centering 
\begin{tikzpicture}[scale=.6]
\coordinate (O) at (0,0);
\node[below right] at (O) {$O$};
\vertex{O};
\draw[thick,dashed,name path=circle] (O) circle[radius=3cm];
\draw (O) -- node[above] {$r$} ++(-130:3cm) coordinate (R);
\draw[name path=l] (O) ++(170:4cm) --
  node[below, near end,xshift=30pt,yshift=10pt] {$l$} ++(20:8cm);
\path[name intersections={of=circle and l,by={Y,X}}];
\node[above left] at (X) {$X$};
\node[above right] at (Y) {$Y$};
\end{tikzpicture}
%\includegraphics[width=0.7\textwidth]{Fig14_11}

\caption{Construction des points d'intersection d'une droite et d'un cercle  (1).}\label{f.se-line-circle1}
\end{figure}

\begin{proof}
D'après le théorème~\ref{thm.straight-perp} il est possible de construire une perpendiculaire à la droite $l$ qui passe par le centre $O$ du cercle. L'intersection de $l$ avec la perpendiculaire est notée $M$.  $\overline{OM}$ coupe la corde $\overline{XY}$ en son milieu, où $X$ et $Y$ sont les intersections de la droite avec le cercle (fig.~\ref{f.se-line-circle2}). Posons $\overline{XY}=2s$ et $\overline{OM}=t$. Notons que $s$, $X$ et $Y$ ne sont que des définitions et que les entités n'ont pas été construites.


D'après le théorème de Pythagore, $s^2=r^2-t^2=(r+t)(r-t)$. D'après le théorème~\ref{thm.straight-direction}, il est possible de construire des segments de longueur $t$ à partir de $O$ dans les deux directions $\overline{OR}$ et $\overline{RO}$. On obtient ainsi deux segments de longueur $r+t$ et $r-t$.

\begin{figure}[htbp]
\centering    
\begin{tikzpicture}[scale=.6]
\coordinate (O) at (0,0);
\draw[thick,dashed,name path=circle] (O) circle[radius=3cm];
\node[below right] at (O) {$O$};
\vertex{O};
\path (O) --  ++(-130:3cm) coordinate (R);
\node[below left,yshift=2pt,xshift=2pt] at (R) {$R$};
\draw[name path=l] (O) ++(170:4cm) --
  node[below, near end,xshift=30pt,yshift=10pt] {$l$} ++(20:8cm);
\path[name intersections={of=circle and l,by={Y,X}}];
\node[above left] at (X) {$X$};
\node[above right] at (Y) {$Y$};
\draw (O) -- node[below] {$r$} (X);
\path (X) -- ($(X)!.5!(Y)$) coordinate (M);
\node[above] at (M) {$M$};
\draw (O) -- node[right] {$t$} (M);
\path (X) -- node[above] {$s$} (M);
\path (M) -- node[above] {$s$} (Y);
\draw (O) ++(170:4cm) -- ++(20:3.1cm) -- ++(-70:10pt) -- ++(20:10pt);
\draw (O) -- node[below] {$t$} +(50:2) coordinate (RTT);
\draw (O) -- node[below] {$t$} +(-130:2) coordinate (RT);
\vertex{RT};
\draw (RT) -- node[right,yshift=-2pt] {$r-t$} ($(RT)+(-130:1cm)$);
\vertex{RTT};
\draw[<->] ($(RT)+(.5cm,-1.6cm)$) -- node[fill=white] {$r+t$}+(50:5);
\end{tikzpicture}
%\includegraphics[width=0.7\textwidth]{Fig14_12}
\caption{Construction des points d'intersection d'une droite et d'un cercle 
 (2).}\label{f.se-line-circle2}
\end{figure}



D'après le théorème~\ref{thm.straight-sqrt}, on peut construire un segment  de longueur $s=\sqrt{(r+t)(r-t)}$, et d'après le théorème~\ref{thm.straight-direction} on peut construire des segments  de longueur $s$ à partir de $M$ le long de $l$ dans les deux directions. Leurs autres extrémités sont les points d'intersection de $l$ et $c$.
\end{proof}


\section{Construction de l'intersection de deux cercles}\label{s.two-circles}

\begin{theorem}
Étant donné deux cercles $c(O_1,r_1), c(O_2,r_2)$, il est possible de construire leurs points d'intersection.
\end{theorem}

\begin{proof}
Construisons $\overline{O_1O_2}$ et notons sa longueur $t$ (fig.~\ref{f.se-circle-circle1}).
On note $A$ le point d'intersection de $\overline{O_1O_2}$ et de $\overline{XY}$. On pose  $q=\overline{O_1A}$ et  $x=\overline{XA}$ (fig.~\ref{f.se-circle-circle2}). $A$ n'a pas encore été construit, mais si $q$ et $x$ sont construits, alors d'après le théorème~\ref{thm.straight-direction} le point $A$ à la distance $q$ de $O_1$ dans la direction $\overline{O_1O_2}$ peut être construit.

\begin{figure}[htbp]
\centering
\begin{tikzpicture}[scale=.9]
\coordinate (O1) at (0,0);
\coordinate (O2) at (2.5,0);
\node[below left] at (O1) {$O_1$};
\node[below right] at (O2) {$O_2$};
\vertex{O1};
\vertex{O2};
\draw[thick,dashed,name path=circle1] (O1) circle[radius=2cm];
\draw[thick,dashed,name path=circle2] (O2) circle[radius=1.6cm];
\path [name intersections={of=circle1 and circle2,by={X,Y}}];
\node[above,yshift=4pt] at (X) {$X$};
\node[below,yshift=-4pt] at (Y) {$Y$};
\draw (O1) -- node[above] {$r_1$} ++(160:2cm);
\draw (O2) -- node[above] {$r_2$} ++(30:1.6cm);
\draw (O1) -- (O2);
\node at (-1.7,1.6) {$c_1$};
\node at (3.8,1.4) {$c_2$};
\draw[<->] (0,-.6) -- node[fill=white] {$t$} +(2.5,0);
\end{tikzpicture}
%\includegraphics[width=0.7\textwidth]{Fig14_13}
\caption{Construction de l'intersection de deux cercles (1).}\label{f.se-circle-circle1}
\end{figure}

\begin{figure}[htbp]
\centering
     \begin{tikzpicture}[scale=.9]
\coordinate (O1) at (0,0);
\coordinate (O2) at (2.5,0);
\vertex{O1};
\vertex{O2};
\node[below left] at (O1) {$O_1$};
\node[below right] at (O2) {$O_2$};
\draw[thick,dashed,name path=circle1] (O1) circle[radius=2cm];
\draw[thick,dashed,name path=circle2] (O2) circle[radius=1.6cm];
\path [name intersections={of=circle1 and circle2,by={X,Y}}];
\node[above,yshift=4pt] at (X) {$X$};
\node[below,yshift=-4pt] at (Y) {$Y$};
\draw (O1) -- node[above,xshift=-4pt] {$r_1$} (X);
\draw (O2) -- node[above,xshift=4pt] {$r_2$} (X);
\draw[name path=oo] (O1) -- (O2);
\node at (-1.7,1.6) {$c_1$};
\node at (3.8,1.4) {$c_2$};
\draw[name path=xy] (X) -- (Y);
\path[name intersections={of=xy and oo,by={A}}];
\node[below left] at (A) {$A$};
\draw (A) rectangle +(6pt,6pt);
\path (O1) -- node[below,xshift=-2pt] {$q$} (A);
\path (X) -- node[left,yshift=-2pt] {$x$} (A);
\draw[<->] (0,-.6) -- node[fill=white] {$t$} +(2.5,0);
\end{tikzpicture}
%\includegraphics[width=0.7\textwidth]{Fig14_14}

\caption{Construction de l'intersection de deux cercles (2).}\label{f.se-circle-circle2}
\end{figure}

Une fois que $A$ a été construit, le théorème~\ref{thm.straight-perp} montre qu'on peut construire une perpendiculaire à $\overline{O_1O_2}$ qui passe par $A$. D'après le théorème~\ref{thm.straight-direction} il est possible de construire des segments de longueur $x$ à partir de $A$ dans les deux directions le long de la perpendiculaire. Leurs autres points d'extrémité sont les points d'intersection avec les cercles.

\noindent\textbf{Construction de la longueur $q$.} Posons $d=\sqrt{r_1^2+t^2}$. C'est  l'hypoténuse d'un triangle rectangle, qui peut être construite à partir des longueurs connues $r_1$ et $t$. Notons que $\triangle O_1XO_2$ n'est pas nécessairement un triangle rectangle ; le triangle rectangle peut être construit n'importe où dans le plan. Dans le triangle rectangle $\triangle XAO_1$, $\cos\angle XO_1A=q/r_1$. Par la loi des cosinus  pour $\triangle XO_1O_2$,
\begin{align*}
r_2^2 &= t^2 + r_1^2 - 2r_1t\cos\angle XO_1O_2\\
&= t^2 + r_1^2 - 2tq\\
2tq &= (t^2+r_1^2) - r_2^2=d^2-r_2^2\\
q&=\frac{(d+r_2)(d-r_2)}{2t}\,.
\end{align*}
D'après le théorème~\ref{thm.straight-direction}, ces longueurs peuvent être construites et, d'après le théorème~\ref{thm.straight-relative}, $q$ peut être construit à partir de $d+r_2$, $d-r_2$ et $2t$.

\medskip

\noindent\textbf{Construction de la longueur $x$.} D'après le théorème de Pythagore,
\[
x=\sqrt{r_1^2-q^2}=\sqrt{(r_1+q)(r_1-q)}\,.
\]
D'après le théorème~\ref{thm.straight-direction}, $h =r_1+ q$ et $k= r_1 - q$ peuvent être construits, de même que $x=\sqrt{hk}$ d'après le théorème~\ref{thm.straight-sqrt}.
\end{proof}



\subsection*{Quelle est la surprise ?}

Le compas est nécessaire car la règle ne peut calculer que les racines des équations linéaires et non des valeurs telles que $\sqrt{2}$, l'hypoténuse d'un triangle rectangle isocèle dont les côtés ont une longueur de $1$. Cependant, il est surprenant que l'existence d'un seul cercle, quelle que soit la position de son centre et la longueur de son rayon, suffise pour effectuer toute construction possible à la  règle et au compas.



\subsection*{Sources}

Ce chapitre se base sur le problème $34$ de \cite{dorrie1} retravaillé par Michael Woltermann \cite{dorrie2}.
