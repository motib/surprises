% !TeX root = surprises.tex


\selectlanguage{hebrew}
%\part{בניות גיאומטריות}\label{p.constructions}


\chapter{מחוגה מתמוטטת}\label{c.collapse}

%%%%%%%%%%%%%%%%%%%%%%%%%%%%%%%%%%%%%%%%%%%%%%%%%%%%%%%%%%%%%%%

\section{
\R{מחוגה קבועה ומחוגה מתמוטטת}
}

במחוגה מודרנית ניתן לקבע את המרחק בין שתי הרגליים, וכך להעתיק קטע קו או מעגל ממקום למקום. נקרא למחוגה זו: "מחוגה קבועה". בספרי לימוד גיאומטריה ניתן למצוא בנייה של אנך אמצעי לקטע קו על ידי בניית שני מעגלים שמרכזם על הקו, 
\textbf{ובלבד שהרדיוס גדול ממחצית המרחק בין המרכזים},
)תרשים שמאלי(:
\begin{center}
\selectlanguage{english}
\begin{tikzpicture}[scale=0.5]
\begin{scope}
\coordinate (A) at (0,0);
\coordinate (B) at (4,0);
\draw ($(A)!-.3!(B)$) -- ($(A)!1.3!(B)$);
\fill (A) node[below] {$A$}circle[radius=3pt];
\fill (B)  node[below] {$B$} circle[radius=3pt];
\draw[name path=larc] (A) ++(-60:3cm) arc (-60:60:3cm);
\draw[name path=rarc] (B) ++(-120:3cm) arc (-120:-240:3cm);
\path [name intersections={of=larc and rarc,by={b,t}}];
\fill (t) node[above right,xshift=-2pt,yshift=5pt] {$C$} circle[radius=3pt];
\fill (b) node[below left,xshift=2pt,yshift=-5pt] {$D$} circle[radius=3pt];
\draw ($ (b) ! 1.2 ! (t)$) -- ($ (t) ! 1.2 ! (b)$);
\end{scope}
\begin{scope}[xshift=12cm]
\coordinate (A) at (0,0);
\coordinate (B) at (4,0);
\draw ($(A)!-.3!(B)$) -- ($(A)!1.3!(B)$);
\fill (A) node[below left] {$A$}circle[radius=3pt];
\fill (B)  node[below right] {$B$} circle[radius=3pt];
\draw[name path=larc] (A) ++(-80:4cm) arc (-80:80:4cm);
\draw[name path=rarc] (B) ++(-100:4cm) arc (-100:-260:4cm);
\path [name intersections={of=larc and rarc,by={b,t}}];
\fill (t) node[above right,xshift=-2pt,yshift=3pt] {$C$} circle[radius=3pt];
\fill (b) node[below left,xshift=2pt,yshift=-3pt] {$D$}circle[radius=3pt];
\draw ($ (b) ! 1.2 ! (t)$) -- ($ (t) ! 1.2 ! (b)$);
\draw[thick,dashed] (A) -- (t) -- (B);
\end{scope}
\end{tikzpicture}
\end{center}

אוקלידס השתמש במחוגה "מתמוטטת" )%
\L{collapsing}%
(,
שרגליה מתקפלות כאשר מרימים אותן מהנייר. מחוגה המורכבת מגיר הקשור לחוט היא מחוגה מתמוטטת, כי אי-אפשר לשמור את הרדיוס כאשר מרימים אותה מהלוח. התרשים הימני למעלה מראה בנייה של אנך אמצעי באמצעות מחוגה מתמוטטת: האורך של
$\overline{AB}$
שווה כמובן לאורך של
$\overline{BA}$,
ולכן למעגלים רדיוס זהה.

בבניה עם המחוגה המתמוטטת קל להוכיח שמתקבל משולש שווה-צלעות. האורך של
$\overline{AC}$
שווה לאורכו של
$\overline{AB}$,
כי שניהם רדיוסים של אותו מעגל, ומאותה סיבה האורך של
$\overline{BC}$
שווה לאורכו של
$\overline{BA}$.
מכאן ש-%
$\overline{AC} = \overline{AB} = \overline{BA} = \overline{BC}$.
%
%\vspace{-2ex}
%
%\begin{center}
%\selectlanguage{english}
%\begin{tikzpicture}[scale=0.5]
%\begin{scope}
%\coordinate (A) at (0,0);
%\coordinate (B) at (4,0);
%\draw ($(A)!-.3!(B)$) -- ($(A)!1.3!(B)$);
%\fill (A) node[below left] {$A$}circle[radius=3pt];
%\fill (B)  node[below right] {$B$} circle[radius=3pt];
%\draw[name path=larc] (A) ++(-60:3cm) arc (-60:60:3cm);
%\draw[name path=rarc] (B) ++(-120:3cm) arc (-120:-240:3cm);
%\path [name intersections={of=larc and rarc,by={b,t}}];
%\fill (t) node[above right,xshift=-2pt,yshift=5pt] {$C$} circle[radius=3pt];
%\fill (b) node[below left,xshift=2pt,yshift=-5pt] {$D$} circle[radius=3pt];
%\draw (A) -- (t);
%\draw (B) -- (t);
%\end{scope}
%\begin{scope}[xshift=12cm]
%\coordinate (A) at (0,0);
%\coordinate (B) at (4,0);
%\draw ($(A)!-.3!(B)$) -- ($(A)!1.3!(B)$);
%\fill (A) node[below left] {$A$}circle[radius=3pt];
%\fill (B)  node[below right] {$B$} circle[radius=3pt];
%\draw[name path=larc] (A) ++(-80:4cm) arc (-80:80:4cm);
%\draw[name path=rarc] (B) ++(-100:4cm) arc (-100:-260:4cm);
%\path [name intersections={of=larc and rarc,by={b,t}}];
%\fill (t) node[above right,xshift=-2pt,yshift=3pt] {$C$} circle[radius=3pt];
%\fill (b) node[below left,xshift=2pt,yshift=-3pt] {$D$}circle[radius=3pt];
%\draw (A) -- (t);
%\draw (B) -- (t);
%\end{scope}
%\end{tikzpicture}
%\end{center}

הבניה של משולש שווה-צלעות היא המשפט הראשון בספר של אוקלידס. המשפט השני מראה שאפשר להעתיק קטע קו עם מחוגה מתמוטטת, ולכן המחוגה הקבועה לא מוסיפה יכולת חדשה. 

%%%%%%%%%%%%%%%%%%%%%%%%%%%%%%%%%%%%%%%%%%%%%%%%%%%%%%%%%%%%%%%

\section{%
העתקת קטע קו לפי אוקלידס%
}

\textbf{משפט:}
נתון קטע קו
$AB$
ונקודה
$C$
)תרשים משמאל(, ניתן לבנות )עם מחוגה מתמוטטת( בנקודה
$C$
קטע קו שאורכו שווה לאורכו של 
$AB$:


\begin{center}
\selectlanguage{english}
\begin{tikzpicture}[scale=0.6]
\begin{scope}
\coordinate (C) at (0,0);
\coordinate (A) at (2.5,0);
\coordinate (B) at (5.5,2);
\draw (A) node[below,xshift=-2pt,yshift=-2pt] {$A$} -- (B) node[right] {$B$};
\fill (A) circle[radius=3pt];
\fill (B) circle[radius=3pt];
\fill (C) node[below,xshift=2pt,yshift=-2pt] {$C$} circle[radius=3pt];
\end{scope}
\begin{scope}[xshift=12cm]
\coordinate (C) at (0,0);
\coordinate (A) at (2.5,0);
\coordinate (B) at (5.5,2);
\draw (A) node[below,xshift=-2pt,yshift=-2pt] {$A$} -- (B) node[right] {$B$};
\fill (A) circle[radius=3pt];
\fill (B) circle[radius=3pt];
\fill (C) node[below,xshift=2pt,yshift=-2pt] {$C$} circle[radius=3pt];
\draw (A) -- (C);
\path[name path=larc] (C) ++(-70:2.5cm) arc (-70:70:2.5cm);
\path[name path=rarc] (A) ++(-110:2.5cm) arc (-110:-250:2.5cm);
\path [name intersections={of=larc and rarc,by={d,D}}];
\fill (D) node[above] {$D$} circle[radius=3pt];
\draw (A) -- (D);
\draw (C) -- (D);
\end{scope}
\end{tikzpicture}
\end{center}
%\textbf{%
%בניה:%
%}
\begin{itemize}
\item
חברו בקו את הנקודות
$A$
ו-%
$C$.
\item
בנו משולש שווה צלעות שבסיסו
$\overline{AC}$
)אפשרי לפי המשפט הראשון של אוקלידס(.
סמנו את הקודקוד של המשולש ב-%
$D$
)תרשים ימני למעלה(. 
\item
בנו קרן בהמשך של
$\overline{DA}$
וקרן בהמשך של 
$DC$
)תרשים שמאלי למטה(.
\item
בנו מעגל שמרכזו 
$A$
עם רדיוס
$\overline{AB}$.
סמנו
$E$,
החיתוך של המעגל עם הקרן
$\overline{DE}$
)תרשים ימני(.

\begin{center}
\selectlanguage{english}
\begin{tikzpicture}[scale=0.6]
\begin{scope}
\coordinate (C) at (0,0);
\coordinate (A) at (2.5,0);
\coordinate (B) at (5.5,2);
\draw (A) node[below,xshift=-2pt,yshift=-2pt] {$A$} -- (B) node[right] {$B$};
\fill (A) circle[radius=3pt];
\fill (B) circle[radius=3pt];
\fill (C) node[below,xshift=2pt,yshift=-2pt] {$C$} circle[radius=3pt];
\draw (A) -- (C);
\path[name path=larc] (C) ++(-70:2.5cm) arc (-70:70:2.5cm);
\path[name path=rarc] (A) ++(-110:2.5cm) arc (-110:-250:2.5cm);
\path [name intersections={of=larc and rarc,by={d,D}}];
\fill (D) node[above] {$D$} circle[radius=3pt];
\draw (A) -- (D);
\draw (C) -- (D);
\draw[name path=ray2] (D) -- ($ (D) ! 2.7 ! (C) $);
\draw[name path=ray1] (D) -- ($ (D) ! 2.7 ! (A) $);
\end{scope}
\begin{scope}[xshift=12cm]
\coordinate (C) at (0,0);
\coordinate (A) at (2.5,0);
\coordinate (B) at (5.5,2);
\draw (A) node[below,xshift=-2pt,yshift=-2pt] {$A$} -- (B) node[right] {$B$};
\fill (A) circle[radius=3pt];
\fill (B) circle[radius=3pt];
\fill (C) node[below,xshift=2pt,yshift=-2pt] {$C$} circle[radius=3pt];
\draw (A) -- (C);
\path[name path=larc] (C) ++(-70:2.5cm) arc (-70:70:2.5cm);
\path[name path=rarc] (A) ++(-110:2.5cm) arc (-110:-250:2.5cm);
\path [name intersections={of=larc and rarc,by={d,D}}];
\fill (D) node[above] {$D$} circle[radius=3pt];
\draw (A) -- (D);
\draw (C) -- (D);
\draw[name path=ray2] (D) -- ($ (D) ! 2.7 ! (C) $);
\draw[name path=ray1] (D) -- ($ (D) ! 2.7 ! (A) $);
\node[draw,circle through=(B),name path=c1] at (A) {};
\path [name intersections={of=c1 and ray1,by={E,e}}];
\fill (E) node[right,xshift=2pt,yshift=-2pt] {$E$} circle[radius=3pt];
\end{scope}
\end{tikzpicture}
\end{center}
\item
בנו מעגל שמרכזו 
$D$
עם רדיוס 
$\overline{DE}$.
סמנו את החיתוך של
$\overline{DC}$
עם המעגל ב-%
$F$:
\end{itemize}

\begin{center}
\selectlanguage{english}
\begin{tikzpicture}[scale=0.4]
\coordinate (C) at (0,0);
\coordinate (A) at (2.5,0);
\coordinate (B) at (5.5,2);
\draw (A) node[below,xshift=-2pt,yshift=-2pt] {$A$} -- node[above] {$x$} (B) node[right] {$B$};
\fill (A) circle[radius=3pt];
\fill (B) circle[radius=3pt];
\fill (C) node[below,xshift=2pt,yshift=-2pt] {$C$} circle[radius=3pt];
\draw (A) -- (C);
\path[name path=larc] (C) ++(-70:2.5cm) arc (-70:70:2.5cm);
\path[name path=rarc] (A) ++(-110:2.5cm) arc (-110:-250:2.5cm);
\path [name intersections={of=larc and rarc,by={d,D}}];
\fill (D) node[above] {$D$} circle[radius=3pt];
\draw (A) -- node[right] {$y$} (D);
\draw (C) -- node[left] {$y$} (D);
\draw[name path=ray2] (D) -- ($ (D) ! 3 ! (C) $);
\draw[name path=ray1] (D) -- ($ (D) ! 3 ! (A) $);
\node[draw,circle through=(B),name path=c1] at (A) {};
\path [name intersections={of=c1 and ray1,by={E,e}}];
\fill (E) node[right,xshift=2pt,yshift=-2pt] {$E$} circle[radius=3pt];
\node[draw,circle through=(E),name path=c2] at (D) {};
\path [name intersections={of=c2 and ray2,by={F,f}}];
\fill (F) node[left,xshift=-2pt,yshift=-2pt] {$F$} circle[radius=3pt];
\path (A) -- node[right] {$x$} (E);
\path (C) -- node[left] {$x$} (F);
\end{tikzpicture}
\end{center}

\textbf{טענה:}
אורכו של קטע הקו
$\overline{CF}$
שווה לאורכו של קטע הקו
$\overline{AB}$.


\textbf{הוכחה:}
$\overline{DC}=\overline{DA}$
כי
$\triangle ACD$
שווה-צלעות.
$\overline{AE}=\overline{AB}$
כי שניהם רדיוסים של המעגל שמרכזו 
$A$.
$\overline{DF}=\overline{DE}$
כי שניהם רדיוסים של המעגל שמרכזו
$D$.
אורכו של
$\overline{CF}$ 
הוא:
\[
\overline{CF} = \overline{DF} - \overline{DC} = \overline{DE} - \overline{DC} = \overline{DE} - \overline{DA} = \overline{AE} = \overline{AB}\,.
\].
\qed

%%%%%%%%%%%%%%%%%%%%%%%%%%%%%%%%%%%%%%%%%%%%%%%%%%%%%%%%%%%%%%%

\section{%
העתקה שגויה של קטע קו
}\label{s.error}

\begin{itemize}
\item
בנו מעגל שמרכזו
$A$
עם רדיוס
$\overline{AB}$:
\begin{center}
\selectlanguage{english}
\begin{tikzpicture}[scale=0.6]
\begin{scope}
\coordinate (C) at (-2,0);
\coordinate (A) at (2.5,0);
\coordinate (B) at (4.5,1.5);
\draw (A) node[below,xshift=-2pt,yshift=-2pt] {$A$} -- (B) node[right] {$B$};
\fill (A) circle[radius=3pt];
\fill (B) circle[radius=3pt];
\fill (C) node[below,xshift=2pt,yshift=-2pt] {$C$} circle[radius=3pt];
\end{scope}
\begin{scope}[xshift=12cm]
\coordinate (C) at (-2,0);
\coordinate (A) at (2.5,0);
\coordinate (B) at (4.5,1.5);
\draw (A) node[below,xshift=-2pt,yshift=-2pt] {$A$} -- (B) node[right] {$B$};
\fill (A) circle[radius=3pt];
\fill (B) circle[radius=3pt];
\fill (C) node[below,xshift=2pt,yshift=-2pt] {$C$} circle[radius=3pt];
\node[draw,circle through=(B),name path=c1] at (A) {};
\end{scope}
\end{tikzpicture}
\end{center}
%\vspace*{-8ex}
\item
בנו מעגל שמרכזו
$A$
עם רדיוס
$\overline{AC}$
ומעגל שמרכזו
$C$
עם רדיוס
$\overline{AC}=\overline{CA}$.
\item
סמנו את נקודות החיתוך של המעגלים ב-%
$E,F$.
סמנו את נקודת החיתוך של המעגל שמרכזו
$C$
עם המעגל שמרכזו 
$A$
עם רדיוס
$\overline{AB}$
ב-%
$D$:
\begin{center}
\selectlanguage{english}
\begin{tikzpicture}[scale=0.6]
\coordinate (C) at (-2,0);
\coordinate (A) at (2.5,0);
\coordinate (B) at (4.5,1.5);
\draw (A) node[below right] {$A$} -- (B) node[right] {$B$};
\fill (A) circle[radius=3pt];
\fill (B) circle[radius=3pt];
\fill (C) node[left,xshift=-2pt] {$C$} circle[radius=3pt];
\node[draw,circle through=(B),name path=c1] at (A) {};
\node[draw,circle through=(C),name path=c2] at (A) {};
\node[draw,circle through=(A),name path=c3] at (C) {};
\path [name intersections={of=c1 and c3,by={D,f}}];
\path [name intersections={of=c2 and c3,by={E,F}}];
\fill (D) node[below right,xshift=4pt] {$D$} circle[radius=3pt];
\fill (E) node[above,yshift=2pt] {$E$} circle[radius=3pt];
\fill (F) node[below,yshift=-2pt] {$F$} circle[radius=3pt];
\end{tikzpicture}
\end{center}
\item
בנו מעגל שמרכזו 
$E$
עם רדיוס 
$\overline{ED}$.
סמנו ב-%
$G$
את החיתוך של המעגל עם המעגל שמרכזו
$A$
עם רדיוס 
$\overline{AC}$
)איור~%
\ref{f.collapse1}(.
\end{itemize}
\begin{figure}[tb]
\begin{center}
\selectlanguage{english}
\begin{tikzpicture}[scale=0.8]
\clip (-8,-1) rectangle (9,5);
\coordinate (C) at (-2,0);
\coordinate (A) at (2.5,0);
\coordinate (B) at (4.5,1.5);
\draw[thick] (A) node[below right] {$A$} -- (B) node[right] {$B$};
\fill (A) circle[radius=3pt];
\fill (B) circle[radius=3pt];
\fill (C) node[below left] {$C$} circle[radius=3pt];
\node[draw,circle through=(B),name path=c1] at (A) {};
\node[draw,circle through=(C),name path=c2] at (A) {};
\node[draw,circle through=(A),name path=c3] at (C) {};
\path [name intersections={of=c1 and c3,by={D,f}}];
\path [name intersections={of=c2 and c3,by={E,F}}];
\fill (D) node[below right,xshift=4pt] {$D$} circle[radius=3pt];
\fill (E) node[above,yshift=2pt] {$E$} circle[radius=3pt];
\fill (F) node[below,yshift=-2pt] {$F$} circle[radius=3pt];
\node[draw,circle through=(D),name path=c4] at (E) {};
\path [name intersections={of=c2 and c4,by={g,G}}];
\fill (G) node[below left,xshift=-4pt] {$G$} circle[radius=3pt];
\draw[thick] (C) -- (G);
\draw[thick,dashed] (G) -- (E) -- (C);
\draw[thick,dashed] (A) -- (D) -- (E) -- cycle;
\end{tikzpicture}
\end{center}
\caption{משולשים חופפים}\label{f.collapse1}
\end{figure}

\textbf{טענה:}
ארכו של 
$\overline{CG}$
שווה לאורכו של
$\overline{AB}$.

\textbf{הוכחה:}
נניח ש-%
$\triangle ADE\cong \triangle CGE$.
אם כן, 
$\overline{CG}=\overline{AD}=\overline{AB}$
כי 
$\overline{AD},\overline{AB}$
הם רדיוסים של המעגל הקטן שמרכזו 
$A$.
למעגל שמרכזו
$C$
אותו רדיוס כמו למעגל שמרכזו 
$A$
ועובר דרך 
$E$.
לכן, ניתן להתייחס אליהם כ-"אותו" מעגל.

עכשיו נוכיח את החפיפה
$\triangle ADE\cong \triangle CGE$.
$\overline{EG}=\overline{ED}$
כי הם רדיוסים של המעגל שמרכזו
$E$,
ו-%
$\overline{EC}=\overline{EA}$
כי הם רדיוסים של "אותו" מעגל. 
$\angle GCE=\angle DAE$
כי הן זוויות היקפיות על "אותו" מיתר 
$\overline{EG},\overline{ED}$.
 $\angle CGE=\angle ADE$
כי הן זוויות היקפיות על "אותו" מיתר
$\overline{EC},\overline{EA}$.
. לכן,
$\angle GEC=\angle DEA$
ו-%
$\triangle ADE\cong \triangle CGE$
לפי צ.ז.צ.
\qed

אין שום שגיאה בהוכחה! השגיאה נובעת ממקור אחר: השווין
$\overline{AB}=\overline{GC}$
מתקיים רק כאשר אורכו של 
$\overline{AB}$
קטן מאורכו של
$\overline{AC}$.
הבנייה של אוקלידס נכונה ללא קשר לאורך היחסי של הקווים ולמיקום של הנקודה
$C$
ביחס לקטע הקו
$\overline{AB}$.

%%%%%%%%%%%%%%%%%%%%%%%%%%%%%%%%%%%%%%%%%%%%%%%%%%%%%%%%%%%%%%%

\section{
בניה אחרת להעתקת קטע קו
}\label{s.copy2}
הבניה כאן להעתקת קטע קו יחסית פשוטה, אבל הוכחה מפורטת שלה מסובכת.

נתון קטע קו 
$\overline{AB}$
ונקודה
$C$,
נבנה את הנקודות
$A,B,C$.
ביחד עם הנקודה 
$D$
נקבל מקבילית עם הנקודות הללו כקודקודים שלה.
$\overline{DC}$
הוא קטע קו עם הנקודה
$C$
בקצה אחד, ו-%
$\overline{DC}=\overline{AB}$
)תרשים שמאלי(.

\vspace{-1ex}

\begin{center}
\selectlanguage{english}
\begin{tikzpicture}[scale=0.8]
\coordinate (A) at (0,0);
\coordinate (B) at (4,0);
\coordinate (C) at (5,2);
\draw (A) -- (B);
\path (A) -- node[above] {$x$} (B);
\fill (A) node[below left] {$A$} circle[radius=2pt];
\fill (B) node[below] {$B$} circle[radius=2pt];
\fill (C) node[above] {$C$} circle[radius=2pt];
\draw (B) -- node[right] {$y$} (C);
\coordinate (D) at ($(C)+(-40mm,0cm)$);
\draw (D) -- node[above] {$x$} (C);
\draw (A) -- node[left] {$y$} (D);
\fill (D) node[above] {$D$} circle[radius=2pt];
\begin{scope}[xshift=10cm]
\coordinate (A) at (0,0);
\coordinate (B) at (4,0);
\coordinate (C) at (5,2);
\draw ($ (B) ! 1.2 ! (A) $) -- ($ (A) ! 1.5 ! (B) $);
\path (A) -- (B);
\fill (A) node[below left,xshift=-4pt] {$A$} circle[radius=2pt];
\fill (B) node[below] {$B$} circle[radius=2pt];
\fill (C) node[above] {$C$} circle[radius=2pt];
\draw (B) -- (C);
\draw[name path=ray1] ($(C)+(-5cm,0cm)$) -- ($(C)+(1cm,0cm)$);
\draw[name path=ray2] ($(A)+(-.25,-.65)$) -- ($(A)+(1,2.6)$);
\path [name intersections={of=ray1 and ray2,by={D}}];
\fill (D) node[above left] {$D$} circle[radius=2pt];
\coordinate (E) at (C |- B);
\draw[thick,dashed] (C) -- (E);
\fill (E) node[below] {$E$} circle[radius=2pt];
\draw[rotate=-90] (C) rectangle +(10pt,10pt);
\draw (E) rectangle +(10pt,10pt);
\coordinate (F) at ($(A)!(B)!(D)$);
\fill (F) circle[radius=2pt];
\draw[thick,dashed] (B) -- (F);
\draw[rotate=-24] (F) rectangle +(10pt,10pt);
\draw[rotate=67] (B) rectangle +(10pt,10pt);
\end{scope}
\end{tikzpicture}
\end{center}
הבניה מוצגת בתרשים הימני.
\begin{itemize}
\item
חברו את
$B$
ו-%
$C$.
\item
בנו אנך מ-%
$C$
לקו המכיל את הקטע
$\overline{AB}$.
סמנו את נקודת החיתוך ב-%
$E$.
\item
בנו אנך לקטע
$\overline{CE}$
מהנקודה
$C$.
אנך זה מקביל ל-%
$\overline{AB}$.
\item
באותה דרך בנו קו המקביל ל-%
$\overline{BC}$
דרך 
$A$.
סמנו את נקודת החיתוך של שני הקווים ב-%
$D$.
\item
$\overline{AB}\|\overline{DC}$, $\overline{AD}\|\overline{BC}$
ולפי ההגדרה
$ABCD$
הוא מקיבילית, ולכן 
$\overline{AB}=\overline{CD}$.
\end{itemize}

\textbf{בניה עם מחובה מתמוטטת:}
נראה איך לבנות אנך דרך נקודה נתונה עם מחוגה מתמוטטת. בנו מעגל שמרכזו
$C$
עם רדיוס הגדול מהמרחק של
$C$
מהקו. סמנו את נקודות החיתוך שלו עם הקו ב-%
$D,E$.
בנו מעגלים שמרכזם
$D,E$
עם רדיוסים
$\overline{DC}=\overline{EC}$.
הקן בין נקודות החיתוך של המעגלים
$C,F$
הוא אנך לקו דרך הנקודה 
$C$.
\begin{center}
\selectlanguage{english}
\begin{tikzpicture}[scale=0.5]
\coordinate (A) at (0,0);
\coordinate (B) at (4,0);
\coordinate (C) at (5,2);
\draw[name path=ray] ($ (B) ! 1.5 ! (A) $) -- ($ (A) ! 2.5 ! (B) $);
%\fill (A) node[below] {$A$} circle[radius=3pt];
%\fill (B) node[below] {$B$} circle[radius=3pt];
\fill (C) node[right] {$C$} circle[radius=3pt];
\draw[name path=arc] (C) ++(-160:3.5cm) arc (-160:-20:3.5cm);
\path [name intersections={of=arc and ray,by={D,E}}];
\fill (D) node[below left] {$D$} circle[radius=3pt];
\fill (E) node[below right] {$E$} circle[radius=3pt];
\draw[name path=larc] (D) ++(-60:3.5cm) arc (-60:60:3.5cm);
\draw[name path=rarc] (E) ++(-120:3.5cm) arc (-120:-240:3.5cm);
\path [name intersections={of=larc and rarc,by={b,t}}];
\fill (b) node[right] {$F$} circle[radius=3pt];
\draw ($ (b) ! 1.2 ! (t)$) -- ($ (t) ! 1.2 ! (b)$);
\end{tikzpicture}
\end{center}


%%%%%%%%%%%%%%%%%%%%%%%%%%%%%%%%%%%%%%%%%%%%%%%%%%%%%%%%%%%%%%%

\section{
אין לסמוך על תרשים
}
בסעיף~%
\ref{s.error}
ראינו שאין לסמוך על ציור. הנה הוכחה "נכונה"
\textbf{\R{שכל}}
משולש שווה-שוקיים!

\begin{figure}[tb]
\begin{center}
\selectlanguage{english}
\begin{tikzpicture}[scale=1.2]
\coordinate (P) at (0,0);
\node[xshift=4mm,yshift=1mm] at (P) {$P$};
\coordinate [label=left:$B$] (B)  at (-2,-2);
\coordinate [label=right:$C$] (C)  at (4,-2);
\coordinate [label=above:$A$] (A)  at (-1,2);
\node[below,yshift=-12pt,xshift=2pt] at (A) {$\alpha$};
\node[below,yshift=-12pt,xshift=15pt] at (A) {$\alpha$};
\draw (A) -- (B);
\draw (A) -- (C);
\draw (B) -- (C);
\draw (A) -- (P);
\draw (B) -- (P);
\draw (C) -- (P);
\coordinate[label=left:$E$] (E) at ($ (A) ! .44 ! (B) $);
\draw[rotate=-100] (E) rectangle +(8pt,8pt);
\draw (P) -- (E);
\coordinate[label=right:$F$] (F) at ($ (A) ! .33 ! (C) $);
\draw[rotate=-132] (F) rectangle +(8pt,8pt);
\draw (P) -- (F);
\coordinate[label=below:$D$] (D) at ($ (B) ! .33 ! (C) $);
\draw (D) rectangle +(8pt,8pt);
\draw (P) -- (D);
\node[left] at ($ (A) ! .5 ! (E) $) {};
\node[left] at ($ (B) ! .5 ! (E) $) {};
\node[below] at ($ (B) ! .5 ! (D) $) {$a$};
\node[below] at ($ (C) ! .5 ! (D) $) {$a$};
\node[right,xshift=2pt] at ($ (A) ! .5 ! (F) $) {};
\node[right,xshift=2pt] at ($ (C) ! .5 ! (F) $) {};
%\fill (A) circle(1pt);
%\fill (B) circle(1pt);
%\fill (C) circle(1pt);
\fill (D) circle(1pt);
\fill (E) circle(1pt);
\fill (F) circle(1pt);
\fill (P) circle(1pt);
\end{tikzpicture}
\end{center}
\caption{"הוכחה" שכל משולש שווה-שוקיים}\label{f.isoceles}
\end{figure}

נתון משולש שרירותי 
$\triangle ABC$,
תהי
$P$
נקודת החיתוך בין חוצה הזווית של
$\angle BAC$
לבין האנך האמצעי של 
$\overline{BC}$
)איור
\ref{f.isoceles}(.
סמנו ב-%
$D,E,F$
את נקודות החיתוך של האנכים מ-%
$P$
לצלעות
$\overline{BC},\overline{AB},\overline{AC}$. $\triangle APE\cong \triangle APF$
כי הם משולשים ישר זווית עם זוויות שוות
$\alpha$
וצלע 
$\overline{AP}$
משותף.

$\triangle DPB\cong \triangle DPC$
לפי צ.ז.צ. כי 
$\overline{PD}$
הוא צלע משותף, 
$\angle PDB=\angle PDC=90^\circ$,
ו-%
$\overline{BD}=\overline{DC}=a$
כי 
$\overline{PD}$
הוא האנך האמצעי ל-%
$\overline{BC}$.
במשולשים ישר-זווית
$\triangle EPB\cong \triangle FPC$
כי שתי לצועות שוות:
$\overline{EP}=\overline{PF}$
לפי החפיפה הראשונה, ו-%
$\overline{PB}=\overline{PC}$
לפי החפיפה השנייה. נחבר את השוויונות ונקבל ש-%
$\triangle ABC$
שווה-שוקיים:
\[
\overline{AB}=\overline{AE}+\overline{EB}=\overline{AF}+\overline{FC}=\overline{AC}\,.
\]
\qed


הבעיה בהוכחה היא שתרשים אינו נכון כי הנקודה
$P$
נמצאת
\textbf{\R{מחוץ}}
למשולש.
% )איור
%\ref{f.isoceles-wrong}(.
%\begin{figure}[H]
\begin{center}
\selectlanguage{english}
\begin{tikzpicture}[scale=.9]
\coordinate (B)  at (0,0);
\node[left] at (B) {$B$};
\coordinate (C)  at (7,0);
\node[right] at (C) {$C$};
\path[name path=pathb] (B) -- +(80:6.5);
\path[name path=pathc] (C) -- +(140:9.5);
\path [name intersections={of=pathb and pathc,by={A}}];
\node[above] at (A) {$A$};
%\fill (A) circle(1pt) node[above] {$A$};
%\fill (B) circle(1pt) node[left] {$B$};
%\fill (C) circle(1pt) node[right] {$C$};
\draw (A) -- (B) -- (C) -- cycle;
\draw[name path=angle] (A) -- +(-70:8.5);
\draw ($(B)!.5!(C)$) |- +(0,4);
\draw[name path=perp] ($(B)!.5!(C)$) |- +(0,-3);
\draw ($(B)!.5!(C)$) rectangle +(8pt,8pt);
\path [name intersections={of=angle and perp,by={P}}];
\fill (P) circle(3pt) node[right] {$P$};
\node[below,yshift=-12pt,xshift=2pt] at (A) {$\alpha$};
\node[below,yshift=-12pt,xshift=15pt] at (A) {$\alpha$};
\end{tikzpicture}
\end{center}
%\caption{נקודת החיתוך בין חוצה הזווית והאנך האמצעי}\label{f.isoceles-wrong}
%\end{figure}

\subsection*{מקורות}

\L{Toussaint} \L{\cite{toussaint}}
הראה שפורסמו הוכחות שגויות רבות של המשפט ודווקא אוקלידס הוא זה שנתן הוכחה נכונה! הבניה בסעיף~%
\ref{s.error}
לקוחה מ-%
\L{\cite{rusty}}
ו-%
\L{\cite{toussaint}}.
הבניה בסעיף~%
\ref{s.copy2}
לקוחה מ-%
\L{\cite{roads}}.
