% !TeX root = surprises.tex

%%%%%%%%%%%%%%%%%%%%%%%%%%%%%%%%%%%%%%%%%%%%%%%%%%%%%%%%%%%%%%%%

\selectlanguage{hebrew}


\chapter[השיטה של
\L{Lill}
והקיפול של
\L{Beloch}]%
{השיטה של
\L{Lill}\\
והקיפול של
\L{Beloch}%
}\label{c.origami-cube}

\selectlanguage{hebrew}


\section{קסם}\label{s.magic}

עקבו אחר הבניה באיור שלהלן.
%\ref{f.lill1}.
בנו מסלול עם ארבעה קטעי קו באורכים הנתונים:
\[
\{a_3=1,a_2=6,a_1=11,a_06\}\,.
\]
הבניה מתחילה ממרכז מערכת הצירים בכיוון החיובי של ציר ה-%
$x$
תוך סיבוב של
$90^\circ$
בין הקטעים. בנו מסלול שני המתחיל עם קטע קו שיוצא ממרכז הצירים בזוויות
$63.4^\circ$
יחסית לציר ה-%
$x$,
וסמנו ב-%
$P$ 
את נקודת החיתוך שלו עם
$a_2$.
פנו שמאלה 
$90^\circ$
ובנו קטע קו כאשר
$Q$
היא נקודת החיתוך שלו עם
$a_1$.
פנו שמאלה 
$90^\circ$
פעם נוספת, בנו קו, ושימו לב שהוא חותך את קצה המסלול הראשון הנמצא ב-%
$(-10,0)$.

%\begin{figure}[hbt]
\begin{center}

\begin{tikzpicture}[scale=.9]
% Draw help lines and axes
\draw[step=10mm,white!50!black] (-11,-1) grid (2,7);
\draw[thick] (-11,0) -- (2,0);
\draw[thick] (0,-1) -- (0,7);
\foreach \x in {-10,...,2}
  \node at (\x-.3,-.2) {\sm{\x}};
\foreach \y in {1,...,7}
  \node at (-.2,\y-.3) {\sm{\y}};

% Draw first path
\coordinate (A) at (0,0);
\coordinate (B) at (1,0);
\coordinate (C) at (1,6);
\coordinate (D) at (-10,6);
\coordinate (E) at (-10,0);
\foreach \x in {A,B,C,D,E}
  \fill (\x) circle(2pt);
\draw[very thick] (A) --
  node[below,xshift=1pt,yshift=-10pt] {$a_3=1$} (B);
\draw[very thick,name path=bc] (B) -- 
  node[right,yshift=6pt] {$a_2=6$} (C);
\draw[very thick,name path=cd] (C) --
  node[above,xshift=4pt] {$a_1=11$}(D);
\draw[very thick,name path=de] (D) --
  node[left,xshift=3pt,yshift=6pt] {$a_0=6$}(E);

% Draw first segment of second path
\path[name path=a2] (A) -- +(63.4:4);
\path [name intersections = {of = a2 and bc, by = {A2}}];
\fill (A2) circle(2pt) node[above right] {$P$};
\draw[very thick,dashed] (A) -- (A2);
\draw ($(A) + (14pt,0)$)
  arc [start angle=0, end angle = 63.4, radius=14pt];
\node[above right,xshift=35pt,yshift=2pt] at (A) {$63.4^\circ$};
\draw[->] ($(A)+(32pt,8pt)$) -- +(-18pt,0);
\draw[rotate=153.4] (A2) rectangle +(10pt,10pt);

% Draw second segment of second path
\path[name path=b2] (A2) -- +(153.4:10);
\path [name intersections = {of = b2 and cd, by = {B2}}];
\fill (B2) circle(2pt) node[above left] {$Q$};
\draw[very thick,dashed] (A2) -- (B2);
\draw[rotate=243.4] (B2) rectangle +(10pt,10pt);

% Draw third segment of second path%
\draw[very thick,dashed] (B2) -- (E);
\end{tikzpicture}
\end{center}
%\caption{בניה של המסלול הראשון והמסלול השני}\label{f.lill1}
%\end{figure}
נחשב
$-\tan 63.4^\circ=-2$
ונציב ערך זה בפולינום שהמקדמים שלו הם אורכי הקטעים במסלול הראשון:

\begin{eqnarray*}
p(x)&=&a_3x^3+a_2x^2+a_1x+a_0\\
&=&x^3+6x^2+11x+6\\
p(-\tan 63.4^\circ)&=&(-2)^3+6(-2)^2+11(-2)+6\\
&=&0\,.
\end{eqnarray*}



בשעה טובה! מצאנו שורש של 
$x^3+6x^2+11x+6$,
פולינום ממעלה שלוש.

לפולינום 
$x^3+6x^2+11x+6$
שלושה שורשים
$-1,-2,-3$.
מחישוב השלילה של הטנגס שלהם מתקבל:
\[
\alpha=-\tan^{-1} -1 = 45^\circ,\quad \beta=-\tan^{-1}-2 = 63.4^\circ,\quad \gamma=-\tan^{-1} -3= 71.6^\circ\,.
\]
באיור~%
\ref{f.lill2}
רואים שעבור כל אחת מהזוויות, המסלול השני חותך את הקצה של המסלול הראשון.


\begin{figure}[tb]
\begin{center}

\begin{tikzpicture}[scale=.9]
% Draw help lines and axes
\draw[step=10mm,white!50!black] (-11,-1) grid (2,7);
\draw[thick] (-11,0) -- (2,0);
\draw[thick] (0,-1) -- (0,7);
\foreach \x in {-10,...,2}
  \node at (\x-.3,-.2) {\sm{\x}};
\foreach \y in {1,...,7}
  \node at (-.2,\y-.3) {\sm{\y}};
\coordinate (A) at (0,0);
\coordinate (B) at (1,0);
\coordinate (C) at (1,6);
\coordinate (D) at (-10,6);
\coordinate (E) at (-10,0);
\foreach \x in {A,B,C,D,E}
  \fill (\x) circle(2pt);
\draw[very thick] (A) --
  node[below,yshift=-5pt] {$1$} (B);
\draw[very thick,name path=bc] (B) -- 
  node[right,yshift=24pt] {$6$} (C);
\draw[very thick,name path=cd] (C) --
  node[above] {$11$}(D);
\path[name path=de] (D) -- ($(E)+(0,-.8)$);
\draw[very thick] (D) --
  node[left,yshift=6pt] {$6$} (E);


% Draw first segment of first second path
\path[name path=a1] (A) -- +(45:3);
\path [name intersections = {of = a1 and bc, by = {A1}}];
\fill (A1) circle(2pt) node[above right] {$P_1$};
\draw[very thick,dashed] (A) -- (A1);
\draw[thick] ($(A) + (16pt,0)$)
  arc [start angle=0, end angle = 45, radius=16pt];
\node[above right,xshift=50pt,yshift=-4pt] at (A) {$\alpha$};
\draw[rotate=135] (A1) rectangle +(10pt,10pt);
\draw[Stealth-] ($(A) + (16pt,6pt)$) -- +(36pt,0);

% Draw second segment of first second path
\path[name path=b1] (A1) -- +(135:8);
\path [name intersections = {of = b1 and cd, by = {B1}}];
\fill (B1) circle(2pt) node[above right] {$Q_1$};
\draw[very thick,dashed] (A1) -- (B1);
\draw[rotate=225] (B1) rectangle +(10pt,10pt);

% Draw third segment of first second path
\draw[very thick,dashed] (B1) -- (E);

% Draw first segment of second second path
\path[name path=a2] (A) -- +(63.4:4);
\path [name intersections = {of = a2 and bc, by = {A2}}];
\fill (A2) circle(2pt) node[above right] {$P_2$};
\draw[very thick,dashed] (A) -- (A2);
\draw[thick] ($(A) + (24pt,0)$)
  arc [start angle=0, end angle = 63.4, radius=24pt];
\node[above right,xshift=50pt,yshift=4pt] at (A) {$\beta$};
\draw[rotate=153.4] (A2) rectangle +(10pt,10pt);
\draw[Stealth-] ($(A) + (22pt,14pt)$) -- +(30pt,0);

% Draw second segment of second second path%
\path[name path=b2] (A2) -- +(153.4:10);
\path [name intersections = {of = b2 and cd, by = {B2}}];
\fill (B2) circle(2pt) node[above right] {$Q_2$};
\draw[very thick,dashed] (A2) -- (B2);
\draw[rotate=243.4] (B2) rectangle +(10pt,10pt);

% Draw third segment of second second path%
\draw[very thick,dashed] (B2) -- (E);

% Draw first segment of second second path%
\path[name path=a3] (A) -- +(71.6:4);
\path [name intersections = {of = a3 and bc, by = {A3}}];
\fill (A3) circle(2pt) node[above right] {$P_3$};
\draw[very thick,dashed] (A) -- (A3);
\draw[thick] ($(A) + (38pt,0)$)
  arc [start angle=0, end angle = 71.6, radius=40pt];
\node[above right,xshift=50pt,yshift=16pt] at (A) {$\gamma$};
\draw[rotate=161.6] (A3) rectangle +(10pt,10pt);
\draw[<-] ($(A) + (32pt,25pt)$) -- +(20pt,0);

% Draw second segment of second second path%
\path[name path=b3] (A3) -- +(161.6:10);
\path [name intersections = {of = b3 and cd, by = {B3}}];
\fill (B3) circle(2pt) node[above right] {$Q_3$};
\draw[very thick,dashed] (A3) -- (B3);
\draw[rotate=251.6] (B3) rectangle +(10pt,10pt);

% Draw third segment of second second path%
\draw[very thick,dashed] (B3) -- (E);
\end{tikzpicture}
\end{center}
\selectlanguage{hebrew}
\caption{שלושה מסלולים עבור שלושה שורשים}\label{f.lill2}
\end{figure}


באיור שלהלן אנו רואים שעבור זווית אחרת, נגיד
$56.3^\circ$,
שעבורה
$-\tan 56.3=-1.5$
אינו שורש, המסלול השני חותך את המשך קטע הקו עבור המקדם
$a_0$,
אבל לא ב-%
$(-10,0)$,
הקצה של המסלול הראשון.
%\begin{figure}[tb]
\begin{center}

\begin{tikzpicture}[scale=.9]
% Draw help lines and axes
\draw[step=10mm,white!50!black] (-11,-1) grid (2,7);
\draw[thick] (-11,0) -- (2,0);
\draw[thick] (0,-1) -- (0,7);
\foreach \x in {-10,...,2}
  \node at (\x-.3,-.2) {\sm{\x}};
\foreach \y in {1,...,7}
  \node at (-.2,\y-.3) {\sm{\y}};

% Draw first path
\coordinate (A) at (0,0);
\coordinate (B) at (1,0);
\coordinate (C) at (1,6);
\coordinate (D) at (-10,6);
\coordinate (E) at (-10,0);
\foreach \x in {A,B,C,D,E}
  \fill (\x) circle(2pt);
\draw[very thick] (A) --
  node[below,yshift=-5pt] {$1$} (B);
\draw[very thick,name path=bc] (B) -- 
  node[right,yshift=6pt] {$6$} (C);
\draw[very thick,name path=cd] (C) --
  node[above] {$11$}(D);
\draw[very thick] (D) --
  node[left,yshift=6pt] {$6$}(E);
\path[name path=de] (-10,-1) -- (-10,7);

% Draw first segment of second path
\path[name path=a2] (A) -- +(56.3:3);
\path [name intersections = {of = a2 and bc, by = {A2}}];
\fill (A2) circle(2pt) node[above right] {$P$};
\draw[very thick,dashed] (A) -- (A2);
\draw ($(A) + (14pt,0)$)
  arc [start angle=0, end angle = 56.3, radius=14pt];
\node[above right,xshift=10pt,yshift=6pt] at (A) {$56.3^\circ$};
\draw[rotate=146.3] (A2) rectangle +(10pt,10pt);

% Draw second segment of second path
\path[name path=b2] (A2) -- +(146.3:10);
\path [name intersections = {of = b2 and cd, by = {B2}}];
\fill (B2) circle(2pt) node[above right] {$Q$};
\draw[very thick,dashed] (A2) -- (B2);
\draw[rotate=236.3] (B2) rectangle +(10pt,10pt);

% Draw third segment of second path
\path[name path=c2] (B2) -- +(236.3:8.5);
\path [name intersections = {of = c2 and de, by = {C2}}];
\fill (C2) circle(2pt);
\draw[very thick,dashed] (B2) -- (C2);
\end{tikzpicture}
\end{center}
%\caption{מסלול שאינו מתאים לשורש}\label{f.lill3}
%\end{figure}


דוגמה זו מדגימה שיטה גרפית שבודקת אם ערך נתון הוא  שורש של פולינום. את השיטה גילה 
\L{Eduard Lill}
ב-%
$1867$.
נראה בהמשך שלשיטה של 
\L{Lill}
קשר הדוק עם אוריגמי.

\selectlanguage{hebrew}



\section{הצגת השיטה של
\L{Lill}}\label{s.method}

כדי להבין את השיטה מומלוץ לעיין בדוגמאות בסעיפים הבאים.
\begin{itemize}
\item 
נתון פולינום שרירותי
$p(x)=a_3x^3+a_2x^2+a_1x+a_0$.
\item 
בנו את המסלול הראשון: לכל מקדם
$a_3,a_2,a_1,a_0$
)בסדר זה( בנו קטע קו המתחיל במרכז הצירים
$O=(0,0)$
בכיוון החיובי של ציר ה-%
$x$.
פנו
$90^\circ$
נגד השעון בין הקטעים.
\item 
בנה את המסלול השני כך:
\begin{itemize}
\item
סמנו ב-%
$a_i$
את קטע הקו שאורכו
$a_i$.
\item
בנו קו מ-%
$O$
בזווית 
$\theta$ 
יחסית לכיוון החיובי של ציר ה-%
$x$.
סמנו ב-%
$P$
את הנקודה בה חותך הקו את 
$a_2$.
\item
פנו
$\pm 90^\circ$,
בנו קו מ-%
$P$
וסמנו ב-%
$Q$
את הנקודת החיתוך של הקו עם
$a_1$.
\item
פנו
$\pm 90^\circ$,
בנו קו מ-%
$Q$
וסמנו ב-%
$R$
את נקודת החיתוך של הקו עם
$a_0$.
\item
אם 
$R$
היא נקודת הקצה של המסלול הראשון, 
$p(-\tan\theta)=0$
ו-%
$-\tan\theta$
הוא שורש של 
$p(x)$.
\end{itemize}
\item
מקרים מיוחדים:
\begin{itemize}
\item 
בבניית המסלול הראשון, אם מקדם הוא שלילי, בנו את קטע הקו
\textbf{בכיוון ההפוך}.
\item
בבניית המסלול הראשון, אם מקדם הוא אפס, אין לבנות קטע הקו, אבל כן מבצעים את הפנייה הבאה של
$\pm90^\circ$.
\end{itemize}
\item הערות:
\begin{itemize}
\item 
"נקודת החיתוך של קו עם
$a_i$":
מותר שהחיתוך יהיה עם הקו המכיל את
$a_i$
ולא רק עם קטע הקו 
$a_i$
עצמו.
\item
כאשר בונים את המסלול השני, בחר לפנות ימינה או שמאלה ב-%
$90^\circ$
כך שלמסלול השני תהיה נקודת חיתוך עם קטעי הקו של המסלול הראשון )או של קווים המכילים את קטעי הקו(.
\end{itemize}
\end{itemize}

\selectlanguage{hebrew}


\section{מקדמים שליליים}\label{s.negative}
לפולינום 
$p(x)=x^3-3x^2-3x+1$
בסעיף%
~\ref{s.ax6}
מקדמים שליליים. נפעיל את השיטה של 
\L{Lill}
עבור פולינום זה )איור~%
\ref{f.lill4}(.
נתחיל בבניית קטע קו באורך
$1$
לכיוון החיובי של ציר ה-%
$x$. 
אחר כך נפנה שמאלה
$90^\circ$
)עם הפנים למעלה(. המקדם שלילי ולכן נבנה קטע קו באורך 
$-3$,
זאת אומרת קטע קו באורך
$3$
\textbf{למטה},
הפוך מהכיוון שאנו פונים אליו. לאחר פנייה נוספת
$90^\circ$
לשמאל, המקדם שוב שלילי כך שנבנה קו באורך
$3$
לאחור, לכיוון ימין. לבסוף, נפנה עם הפנים למטה ונבנה קטע קו באורך
$1$.

המסלול השני מתחיל עם קו בזווית
$45^\circ$
יחסית לציר ה-%
$x$.
נקודת החיתוך של הקו עם הקו המכיל את קטע ההקו
$a_2$
היא
$(1,1)$.
נפנה שוב
$-90^\circ$
)לכיוון ימין(, נבנה קו שנקודת החיתוך שלה עם הקו המכיל את קטע הקו 
$a_1$
היא
$(5,-3)$.
נפנה שוב
 $-90^\circ$,
נבנה קו שנקודת החיתוך שלו היא בנוקדת הקצה של המסלול הראשון ב-%
$(4,-4)$.

$-\tan 45^\circ=-1$,
ולכן שורש ממשי של הפולינום הוא
$-1$:
\[
p(-1)=(-1)^3-3(-1)^2-3(-1)+6=0\,.
\]
באיור שני מסלולים נוספים עבור שורשים אחרים של הפולינום 
)\ref{s.noninteger}(.
\begin{figure}[bt]
\begin{center}

\begin{tikzpicture}[scale=.9]
% Draw help lines and axes
\draw[step=10mm,white!50!black] (-1,-6) grid (6,2);
\draw (0,-6) -- (0,2);
\foreach \x in {0,...,6}
  \node at (\x-.3,-.2) {\sm{\x}};
\foreach \y in {-5,...,-1}
  \node at (-.3,\y-.3) {\sm{\y}};
\foreach \y in {1,...,2}
  \node at (-.3,\y-.3) {\sm{\y}};

% Draw first path
\coordinate (A) at (0,0);
\coordinate (B) at (1,0);
\coordinate (C) at (1,-3);
\coordinate (D) at (4,-3);
\coordinate (E) at (4,-4);
\foreach \x in {A,B,C,D,E}
  \fill (\x) circle(1pt);
\draw[very thick,{Stealth[scale=1.4,inset=2pt,reversed]}-] (A) --
  node[below,yshift=-5pt] {$1$} (B);
\draw[very thick,{Stealth[scale=1.4,inset=2pt]}-,name path=bc] (B) -- 
  node[right,xshift=3pt] {$a_2=-3$} (C);
\draw[very thick,{Stealth[scale=1.4,inset=2pt]}-,name path=cd] (C) --
  node[above,xshift=11pt] {$a_1=-3$}(D);
\draw[very thick,{Stealth[scale=1.4,inset=2pt,reversed]}-,name path=de] (D) --
  node[right] {$1$}(E);

% Draw extensions of first path
\draw[very thick,loosely dotted,name path=a] (-1,0) -- (6,0);
\draw[very thick,loosely dotted,name path=b] (1,-6) -- (1,2);
\draw[very thick,loosely dotted,name path=c] (-1,-3) -- (6,-3);

% Draw first second path
\path[name path=a1] (A) -- +(-75:5);
\path [name intersections = {of = a1 and b, by = {B1}}];
\path[name path=b1] (B1) -- +(15:5);
\path [name intersections = {of = b1 and c, by = {C1}}];
\draw[thick,loosely dashed] (A) -- (B1) -- (C1) -- (E);

% Draw second second path
\draw[very thick,dashed] (4,-4) -- (5,-3) coordinate (A2);
\fill (5,-3) circle (1.5pt) node[above right] {$P$};
\draw[very thick,dashed] (5,-3) -- (1,1) coordinate (B2);
\fill (1,1) circle (1.5pt) node[above right] {$Q$};
\draw[very thick,dashed] (1,1) -- (0,0);

% Draw third second path
\path[name path=a3] (A) -- +(-15:5);
\path [name intersections = {of = a3 and b, by = {B3}}];
\path[name path=b3] (B3) -- +(-105:5);
\path [name intersections = {of = b3 and c, by = {C3}}];
\draw[thick,loosely dashed] (A) -- (B3) -- (C3) -- (E);
\end{tikzpicture}
\end{center}
\selectlanguage{hebrew}
\caption{מסלול עבור פולינום עם מקדמים שליליים}\label{f.lill4}
\end{figure}

\selectlanguage{hebrew}



\section{מקדמים שהם אפס}\label{s.zero}

$a_2$,
המקדם של
$x^2$
ב-%
$x^3-7x-6=0$,
הוא אפס. עבור מקדם אפס, אנו "בונים" קטע קו באורך
$0$,
כלומר, אנחנו לא מציירים קו, אבל כן פונים
$\pm 90^\circ$
לפני ואחרי ש"בונים" אותו, כפי שניתן לראות באיור~%
\ref{f.lill5}.
חץ הפונה למעלה בנקודה
$(1,0)$.
קיימים שלושה מסלולים החותכים את קצה המסלול הראשון. הם מתחילים עם הזוויות:
\[
\alpha=45^\circ,\quad \beta=63.4^\circ,\quad \gamma=-71.6^\circ\,.
\]
מכאן אפשר להסיק שיש שלושה שורשים ממשיים:
\[
-\tan 45^\circ=-1,\quad -\tan 63.4^\circ =-2,\quad -\tan (-71.6^\circ)=3\,.
\]
בדיקה:
\[
(x+1)(x+2)(x-3)=(x^2+3x+2)(x-3) =x^3-7x-6\,.
\]
\begin{figure}[tb]
\begin{center}

\begin{tikzpicture}[scale=.9]
% Draw help lines and axes
\draw[step=10mm,white!50!black] (-1,-4) grid (11,7);
\foreach \x in {0,...,11}
  \node at (\x-.3,-.2) {\sm{\x}};
\foreach \y in {-3,...,-1}
  \node at (-.3,\y-.3) {\sm{\y}};
\foreach \y in {1,...,7}
  \node at (-.3,\y-.3) {\sm{\y}};

% Draw first path
\coordinate (A) at (0,0) node[above left] {$O$};
\coordinate (B) at (1,0);
\coordinate (C) at (8,0);
\coordinate (D) at (8,6);
\node[below right] at (D) {$A$};
\foreach \x in {A,B,C,D}
  \fill (\x) circle(1.5pt); 
\draw[very thick,{Stealth[scale=1.4,inset=2pt,reversed]}-] (A) --
  node[below,yshift=-5pt] {$1$} (B);
\draw[{Stealth[scale=1.4,inset=2pt,reversed]}-,very thick] (B) --
  ($(B)+(0,.1)$);
\draw[very thick,{Stealth[scale=1.4,inset=2pt]}-,name path=bc] (B) -- 
  node[below,xshift=-6pt,yshift=-5pt] {$-7$} (C);
\draw[very thick,{Stealth[scale=1.4,inset=2pt]}-,name path=cd] (C) --
  node[right,yshift=4pt] {$-6$}(D);

% Draw extensions of first path
\draw[very thick,loosely dotted] (1,-3) -- (1,7);
\draw[very thick,loosely dotted] (-1,0) -- (11,0);

% Draw first second path
\draw[very thick,dashed,->] (0,0) -- (1,-3);
\fill (1,-3) circle (1.5pt) node[below left] {$P_1$};
\draw[very thick,dashed,->] (1,-3) coordinate (A1) -- (10,0);
\fill (10,0) circle (1.5pt) node[below right] {$Q_1$};
\draw[very thick,dashed,->] (10,0) coordinate (B1) -- (D);

% Draw second second path
\draw[very thick,dashed,->] (0,0) -- (1,1) coordinate (A2);
\fill (A2) circle (1.5pt) node[above right] at (A2) {$P_2$};
\draw[very thick,dashed,->] (A2) -- (2,0) coordinate (B2);
\fill (B2) circle (1.5pt) node[below right] at (B2) {$Q_2$};
\draw[very thick,dashed,->] (B2) -- (D);

% Draw third second path
\draw[very thick,dashed,->] (0,0) -- (1,2) coordinate (A3);
\fill (A3) circle (1.5pt) node[above left] {$P_3$};
\draw[very thick,dashed,->] (A3) -- (5,0) coordinate (B3);
\fill (B3) circle (1.5pt) node[below right] at (B3) {$Q_3$};
\draw[very thick,dashed,->] (B3) -- (D);
\end{tikzpicture}
\end{center}
\selectlanguage{hebrew}
\caption{מסלול עבור פולינום עם מקדם שהוא אפס}\label{f.lill5}
\end{figure}


\selectlanguage{hebrew}


\section{שורשים שאינם מספרים שלמים}\label{s.noninteger}

נבדוק את הפולינום
$p(x)=x^3-2x+1$:
%)איור~%
%\ref{f.lill6}(.
%\begin{figure}[htb]
\begin{center}

\begin{tikzpicture}[scale=1.2]
% Draw help lines and axes
\draw[step=10mm,white!70!black,] (-1,-2) grid (4,2);
\foreach \x in {0,...,4}
  \node at (\x-.2,-.1) {\sm{\x}};
\foreach \y in {-1}
  \node at (-.1,\y-.2) {\sm{\y}};
\foreach \y in {1,2}
  \node at (-.1,\y-.2) {\sm{\y}};

% Draw first path
\coordinate (A) at (0,0);
\node[above left] at (A) {$O$};
\coordinate (B) at (1,0);
\coordinate (C) at (3,0);
\coordinate (D) at (3,-1);
\node[below right] at (D) {$A$};
\foreach \x in {A,B,C,D}
  \fill (\x) circle(1pt); 
\draw[very thick] (A) -- node[above,yshift=2pt] {$1$} (B);
\draw[{Stealth[scale=1.4,inset=2pt,reversed]}-,very thick] ($(A)+(.1,0)$) --
  ($(A)+(.15,0)$);
\draw[{Stealth[scale=1.4,inset=2pt,reversed]}-,very thick] ($(B)+(0,.05)$) --
  ($(B)+(0,.1)$);
\draw[very thick,name path=bc] (B) -- 
  node[above,xshift=-4pt,yshift=2pt] {$-2$} (C);
\draw[{Stealth[scale=1.4,inset=2pt,reversed]}-,very thick] ($(B)+(.22,0)$) --
  ($(B)+(.17,0)$);
\draw[very thick,name path=cd] (C) --
  node[right] {$1$}(D);
\draw[{Stealth[scale=1.4,inset=2pt,reversed]}-,very thick] ($(C)+(0,-.05)$) --
 ($(C)+(0,-.1)$);

% Draw extensions of first path
\draw[very thick,loosely dotted,name path=b] (1,-2) -- (1,2);
\draw[very thick,loosely dotted,name path=c] (-1,0) -- (4,0);
\draw[very thick,loosely dotted,name path=d] (3,-2) -- (3,2);

% Draw first second path
\coordinate (A1) at (1,-1);
\draw[very thick,dashed,->] (0,0) -- (A1);
\fill (A1) circle (1pt) node[below right] {$P_1$};
\coordinate (B1) at (2,0);
\draw[very thick,dashed,->] (A1) -- (B1);
\fill (B1) circle (1pt) node[above right,xshift=4pt] {$Q_1$};
\draw[very thick,dashed,->] (B1) -- (D);
\draw[rotate=45] (A1) rectangle +(6pt,6pt);
\draw[rotate=-135] (B1) rectangle +(6pt,6pt);

% Draw second second path
\path[name path=a2] (0,0) -- +(-31.7:4);
\path [name intersections = {of = a2 and b, by = {A2}}];
\draw[very thick,dashed,->] (0,0) -- (A2);
\fill (A2) circle (1pt) node[below right,yshift=2pt] {$P_2$};
\path[name path=b2] (A2) -- +(58.3:2.5);
\path [name intersections = {of = b2 and c, by = {B2}}];
\draw[very thick,dashed,->] (A2) -- (B2);
\fill (B2) circle (1pt) node[above] {$Q_2$};
\draw[very thick,dashed,->] (B2) -- (D);
\draw[rotate=58.3]   (A2) rectangle +(6pt,6pt);
\draw[rotate=-121.7] (B2) rectangle +(6pt,6pt);

% Draw third second path
\path[name path=a3] (0,0) -- +(58.3:2.5);
\path [name intersections = {of = a3 and b, by = {A3}}];
\draw[very thick,dashed,->] (0,0) -- (A3);
\fill (A3) circle (1pt) node[above left] {$P_3$};
\path[name path=b3] (A3) -- +(-31.7:4);
\path [name intersections = {of = b3 and c, by = {B3}}];
\draw[very thick,dashed,->] (A3) -- (B3);
\fill (B3) circle (1pt) node[above right] {$Q_3$};
\path[name path=c3] (B3) -- +(-121.7:4);
\draw[very thick,dashed,->] (B3) -- (D);
\draw[rotate=-121.7]   (A3) rectangle +(6pt,6pt);
\draw[rotate=-211.7]   (B3) rectangle +(6pt,6pt);
\end{tikzpicture}
\end{center}
%\caption{שורשים שאינם מספרים שלמים}\label{f.lill6}
%\end{figure}


הקטע הראשון של המסלול הראשון עובר מ-%
$(0,0)$
ל-%
$(1,0)$
ואז פונה למעלה. המקדם של
$x^2$
הוא אפס כך שלא נצייר קטע קו ונפנה שמאלה. המקדם הבא הוא
$-2$
כך שהקטע הבא נבנה לאחור מ-%
$(1,0)$
ל-%
$(3,0)$.
לבסוף, המסלול פונה למטה וקו באורך 
$1$
נבנה מ-%
$(3,0)$
ל-%
$(3,-1)$.
קל לראות ש-%
$1$
הוא שורש של
$p(x)$.
$-\tan^{-1} -45^\circ=1$,
ולכן קיים מסלול
$\overline{OP_1Q_1A}$.

אם נחלק את
$p(x)$
ב-%
$x-1$,
נקבל פולינום ריבועי
$x^2+x-1$
ששורשיו הם:
\[
\disfrac{-1\pm\sqrt{5}}{2} \approx 0.62,\; -1.62\,.
\]
לכן קיימים שני מסלולים נוספים: אחד שמתחיל בזווית
$-31.8^\circ$
כי
$-\tan^{-1} 0.62=-31.8^\circ$,
ואחד שמתחיל בזווית
$58.3^\circ$
כי
$-\tan^{-1}1.62=58.3^\circ$.

באופן דומה, לפולינום בסעיף%
~\ref{s.negative}
שני שורשים
$ 2\pm\sqrt{3}\approx 3.73, 0.27$.
הזוויות הן
$-75^\circ$
ו-%
$-15^\circ$,
כי
$-\tan (-75^\circ)\approx 3.73$ 
ו-%
$-\tan (-15^\circ)\approx 0.27$.

\selectlanguage{hebrew}


\section{השורש ממעלה שלוש של שניים}\label{s.cube-root}

כדי להכפיל קוביה עלינו למצוא
$\sqrt[3]{2}\approx 1.26$,
שורש של הפולינום ממעלה שלוש
$x^3-2$:
%)איור~%
%\ref{f.lill-cube2}(.
%\begin{figure}[htb]
\begin{center}

\begin{tikzpicture}[scale=1.2]
% Draw help lines and axes
\draw[step=10mm,white!70!black,] (-1,-2) grid (3,3);
\foreach \x in {0,...,3}
  \node at (\x-.2,-.1) {\sm{\x}};
\foreach \y in {-1}
  \node at (-.2,\y-.2) {\sm{\y}};
\foreach \y in {1,2,3}
  \node at (-.2,\y-.2) {\sm{\y}};

% Draw first path
\coordinate (A) at (0,0);
\coordinate (B) at (1,0);
\coordinate (C) at (1,2);
\foreach \x in {A,B,C}
  \fill (\x) circle(1pt); 
\draw[very thick] (A) -- node[above,yshift=2pt] {$1$} (B);

\draw[{Stealth[scale=1.4,inset=2pt,reversed]}-,very thick] ($(A)+(.05,0)$) --
  ($(A)+(.1,0)$);
\draw[{Stealth[scale=1.4,inset=2pt,reversed]}-,very thick] ($(B)+(0,.05)$) --
  ($(B)+(0,.1)$);
\draw[{Stealth[scale=1.4,inset=2pt,reversed]}-,very thick] ($(B)+(.1,.3)$) --
  ($(B)+(.08,.3)$);
\draw[{Stealth[scale=1.4,inset=2pt,reversed]}-,very thick] ($(B)+(0,.55)$) --
  ($(B)+(0,.5)$);

\draw[very thick] (B) -- 
  node[left,yshift=6pt] {$-2$} (C);

% Draw extensions of first path
\draw[very thick,loosely dotted,name path=a] (-1,0) -- (3,0);
\draw[very thick,loosely dotted,name path=b] (1,-2) -- (1,3);

% Draw first segment of second path
\path[name path=a1] (0,0) -- +(-51.6:2);
\path [name intersections = {of = a1 and b, by = {A1}}];
\draw[very thick,dashed,->] (A) -- (A1);
\fill (A1) circle (1pt) node[below left] {$P_1$};
\draw[rotate=38.4]   (A1) rectangle +(6pt,6pt);

% Draw second segment of second path
\path[name path=b1] (A1) -- +(38.4:2.5);
\path [name intersections = {of = b1 and a, by = {B1}}];
\draw[very thick,dashed,->] (A1) -- (B1);
\fill (B1) circle (1pt) node[above right] {$Q_1$};
\draw[rotate=128.4] (B1) rectangle +(7pt,7pt);

% Draw third segement of second path
\draw[very thick,dashed,->] (B1) -- (C);
\end{tikzpicture}
\end{center}
%\caption{השורש השלישי של $2$}\label{f.lill-cube2}
%\end{figure}



בבנייה של המסלול הראשון, אנו פונים פעמיים שמאלה בלי לבנות קטעי קו, כי המקדמים
$a_2$
ו-%
$a_1$
שניהם אפס. אז פונים שוב שמאלה )לכיוון למטה( ובונים קו לאחור כי 
$a_0=-2$
שלילי. הקטע הראשון של המסלול השני נבנה בזווית של
$-51.6^\circ$
ו-%
$-\tan (-51.6^\circ)\approx 1.26\approx \sqrt[3]{2}$.

\selectlanguage{hebrew}



\section{ההוכחה של השיטה של
\L{Lill}}\label{s.proof}

נגביל את הדיון לפולינומים שהמקדם הראשי שלהם הוא אחד )איור~%
\ref{f.lill-proof}(:%
\footnote{%
\R{אחרת, אפשר לחלק ב}-%
$a_3$
\R{ולפולינום המתקבל אותם שורשים}.}
\[
p(x)=x^3+a_2x^2+a_1x+a_0\,.
\]
סכום הזוויות של משולש הוא 
$180^\circ$,
ולכן אם זווית חדה אחת של משולש ישר-זווית היא
$\theta$, 
השנייה היא
$90^\circ-\theta$.
מכאן שהזווית מעל ל-%
$P$
והזווית משמאל ל-%
$Q$
שוות ל-%
$\theta$.
כעת נרשום סדרת משוואות עבור
$\tan \theta$:

\begin{eqnarray*}
\tan \theta &=& \disfrac{b_2}{1}=b_2\\
%a_2-b_2&=&a_2-\tan\theta\\
\tan \theta &=& \disfrac{b_1}{a_2-b_2}=\disfrac{b_1}{a_2-\tan\theta}\\
b_1&=&\tan\theta (a_2-\tan\theta)\\
\tan \theta &=& \disfrac{a_0}{a_1-b_1}=\disfrac{a_0}{a_1-\tan\theta(a_2-\tan\theta)}\,.
\end{eqnarray*}
נפשט את המשוואה האחרונה ונקבל:

\begin{eqnarray*}
(\tan\theta)^3-a_2(\tan\theta)^2+a_1(\tan\theta)-a_0&=&0\\
-(\tan\theta)^3+a_2(\tan\theta)^2-a_1(\tan\theta)+a_0&=&0\\
(-\tan\theta)^3+a_2(-\tan\theta)^2+a_1(-\tan\theta)+a_0&=&0\,.
\end{eqnarray*}
נסיק ש-%
$-\tan\theta$
הוא שורש ממשי של
$p(x)=x^3+a_2x^2+a_1x+a_0$.
\begin{figure}[tb]
\begin{center}

\begin{tikzpicture}[scale=.9]
% Draw grid and axes
\draw[step=10mm,white!50!black] (-11,-1) grid (2,7);
\draw[thick] (-11,0) -- (2,0);
\draw[thick] (0,-1) -- (0,7);
\foreach \x in {-10,...,2}
  \node at (\x-.3,-.2) {\sm{\x}};
\foreach \y in {1,...,7}
  \node at (-.2,\y-.3) {\sm{\y}};
  
% Draw the points of the first path
\coordinate (A) at (0,0);
\coordinate (B) at (1,0);
\coordinate (C) at (1,6);
\coordinate (D) at (-10,6);
\coordinate (E) at (-10,0);
\foreach \x in {A,B,C,D,E}
  \fill (\x) circle(2pt);
\draw[rotate=90] (B) rectangle +(10pt,10pt);
  
% Draw A -- B and arrow
\draw[very thick] (A) --(B);
\draw[thick,<->] ($(A)+(0,-16pt)$) --
  node[fill=white] {$1$} ($(B)+(0,-16pt)$);

% Draw B -- C and arrow
\draw[very thick,name path=bc] (B) -- (C);
\draw[thick,<->] ($(B)+(36pt,0)$) --
  node[fill=white] {$a_2$} ($(C)+(36pt,0)$);

% Draw C -- D and arrow
\draw[very thick,name path=cd] (C) --(D);
\draw[thick,<->] ($(C)+(0,24pt)$) -- 
  node[fill=white] {$a_1$} ($(D)+(0,24pt)$);

% Draw D -- E and arrow
\draw[very thick,name path=de] (D) -- (E);
\draw[thick,<->] ($(D)+(-16pt,0)$) --
  node[fill=white] {$a_0$} ($(E)+(-16pt,0)$);

% Draw first angled segment of the second path and intersection A2 with BC
\path[name path=a2] (A) -- +(63.4:4);
\path [name intersections = {of = a2 and bc, by = {A2}}];
\fill (A2) circle(2pt) node[above right] {$P$};
\draw[very thick,dashed] (A) -- (A2);
\path (B) -- node[right] {$b_2$} (A2);
\path (A2) -- node[right,yshift=8pt] {$a_2-b_2$} (C);
\draw[rotate=153.4] (A2) rectangle +(10pt,10pt);

% Draw second segment of the second path and intersection B2 with CD
\path[name path=b2] (A2) -- +(153.4:10);
\path [name intersections = {of = b2 and cd, by = {B2}}];
\fill (B2) circle(2pt) node[above right] {$Q$};
\draw[very thick,dashed] (A2) -- (B2);
\draw[rotate=243.4] (B2) rectangle +(10pt,10pt);
\path (D) -- node[above] {$a_1-b_1$} (B2); 
\path (B2) -- node[above] {$b_1$} (C);

% Draw third segment of the second path to E
\draw[very thick,dashed] (B2)-- (E);

% Label A, A2, B2 with theta
\draw ($(A) + (14pt,0)$)
  arc [start angle=0, end angle = 63.4, radius=14pt];
\node[above right,xshift=10pt,yshift=8pt] at (A) {$\theta$};
\draw ($(A2) + (0,14pt)$)
  arc [start angle=90, end angle = 153.4, radius=14pt];
\node[above left,xshift=-4pt,yshift=14pt] at (A2) {$\theta$};
\draw ($(B2) + (-14pt,0)$)
  arc [start angle=180, end angle = 243.4, radius=14pt];
\node[below left,xshift=-14pt,yshift=-4pt] at (B2) {$\theta$};
\end{tikzpicture}
\end{center}
\selectlanguage{hebrew}
\caption{הוכחה של השיטה של Lill}\label{f.lill-proof}
\end{figure}


%%%%%%%%%%%%%%%%%%%%%%%%%%%%%%%%%%%%%%%%%%%%%%%%%%%%%%%%%%%%%%%%

\selectlanguage{hebrew}



\section{%
הקיפול של
\L{Beloch}}\label{s.beloch-fold}

\L{Margharita P. Beloch}
גילתה קשר מרתק בין אוריגמי והשיטה של
\L{Lill}
למציאת שורשים של פולינומים ממעלה שלוש. היא מצאה שהפעלה אחת בלבד של אקסיומה%
~$6$
מאפשרת מציאת שורש ממשי של כל פולינום ממעלה שלוש. לכבודה, לעתים מכנים את הפעולה של האקסיומה "הקיפול של
\L{Beloch}".

נדגים את השיטה על הפולינום
$p(x)=x^3+6x^2+11x+6$
מסעיף%
~\ref{s.magic}.
הקיפול
$\overline{RS}$
יהיה אנך אמצעי גם ל-%
$\overline{QQ'}$
וגם ל-%
$\overline{PP'}$
)איור~%
\ref{f.beloch-fold1}(,
כאשר 
$P',Q'$
הם השיקופים של
$P,Q$.
לפי השיטה של
\L{Lill},
הנקודה
$R$
תהיה על הקו
$a_2$
והנקודה
$S$
תהיה על הקו
$a_1$.
נבנה קו
$a_2'$
מקביל ל-%
$a_2$
ובאותו מרחק מ-%
$a_2$
כמו המרחק של
$a_2$
מ-%
$P$,
ונבנה את הקו
$a_1'$
מקביל ל-%
$a_1$
ובאותו מרחק מ-%
$a_1$
כמו המרחק של
$a_1$
מ-%
$Q$.
נפעיל את אקסיומה%
~$6$
כדי להניח בו-זמנית את 
$P$
ב-%
$P'$
על 
$a_2'$
ואת 
$Q$
ב-%
$Q'$
על
$a_1'$.
הקיפול 
$\overline{RS}$
הוא האנך האמצעי של הקווים
$\overline{PP'}$
ו-%
$\overline{QQ'}$,
ולכן הזוויות ב-%
$R$
ו-%
$S$
הן ישרות כפי שמתחייב.



\begin{figure}
\begin{center}

\begin{tikzpicture}[scale=.8]
% Draw help lines and axes
\draw[step=10mm,white!60!black] (-11,-1) grid (3,13);
\draw[thick] (-11,0) -- (3,0);
\draw[thick] (0,-1) -- (0,13);
\foreach \x in {-10,...,3}
  \node at (\x-.3,-.2) {\sm{\x}};
\foreach \y in {1,...,13}
  \node at (-.2,\y-.3) {\sm{\y}};
  
% Draw first path with five points
\coordinate (A) at (0,0);
\coordinate (B) at (1,0);
\coordinate (C) at (1,6);
\coordinate (D) at (-10,6);
\coordinate (E) at (-10,0);
\foreach \x in {A,B,C,D,E}
  \fill (\x) circle(2pt);
\node[below right,yshift=-6pt] at (A) {$P$};
\node[below left,yshift=-6pt] at (E) {$Q$};

\draw[thick] (A) -- (B);
\draw[thick,name path=bc] (B) -- node[right,near end] {$a_2$} (C);
\draw[thick,name path=cd] (C) -- node[above] {$a_1$} (D);
\draw[thick,name path=de] (D) -- (E);

% Draw parallel lines
\draw[ultra thick,dotted,name path=bpcp] ($(B)+(1,-1)$) --
  node[above right] {$a_2'$}
  ($(C)+(1,7)$);
\draw[ultra thick,dotted,name path=cpdp] ($(C)+(2,6)$) -- 
  node[above left,xshift=-24pt] {$a_1'$} 
  ($(D)+(-1,6)$);

% Draw first segment of second path
\path[name path=a2] (A) -- +(63.4:4);
\path [name intersections = {of = a2 and bc, by = {A2}}];
\draw[ultra thick,dotted] (A) -- (A2);
\fill (A2) circle(2pt) node[above right,xshift=4pt] {$R$};
\draw[rotate=153.4] (A2) rectangle +(10pt,10pt);

% Draw second segment of second path
\path[name path=b2] (A2) -- +(153.4:10);
\path [name intersections = {of = b2 and cd, by = {B2}}];
\fill (B2) circle(2pt) node[above left]  {$S$};
\draw[very thick,dashed] (A2) -- (B2);
\draw[rotate=243.4] (B2) rectangle +(10pt,10pt);

% Draw third segment of second path
\draw[ultra thick,dotted] (B2) -- (E);

% Locate reflections on parallel lines and draw lines
\coordinate (PP) at ($(A2)+(1,2)$);
\fill (PP) circle(2pt) node[above right] {$P'$};
\draw[ultra thick,dotted] (A2) -- (PP);

\coordinate (QP) at ($(B2)+(3,6)$);
\fill (QP) circle(2pt) node[above right] {$Q'$};
\draw[ultra thick,dotted] (B2) -- (QP);
\end{tikzpicture}
\end{center}
\selectlanguage{hebrew}
\caption{הקיפול של Beloch עבור $x^3+6x^2+11x+6$}\label{f.beloch-fold1}
\end{figure}



ננסה את הקיפול של 
\L{Beloch}
על הפולינום
$x^3-3x^2-3x+1$
מסעיף%
~\ref{s.negative}.
%)איור~%
%\ref{f.beloch-fold2}(.
%\begin{figure}[htb]
\begin{center}

\begin{tikzpicture}[scale=.9]
% Draw help lines and axes
\draw[step=10mm,white!50!black] (-1,-5) grid (6,2);
\foreach \x in {0,...,6}
  \node at (\x-.3,-.2) {\sm{\x}};
\foreach \y in {-4,...,-1}
  \node at (-.3,\y-.3) {\sm{\y}};
\foreach \y in {1,...,2}
  \node at (-.3,\y-.3) {\sm{\y}};

% Draw first path
\coordinate (A) at (0,0);
\coordinate (B) at (1,0);
\coordinate (C) at (1,-3);
\coordinate (D) at (4,-3);
\coordinate (E) at (4,-4);
\node[above left] at (A) {$P$};
\node[below right] at (E) {$Q$};
\foreach \x in {A,B,C,D,E}
  \fill (\x) circle(2pt);

\draw[very thick,{Stealth[scale=1.4,inset=2pt,reversed]}-] (A) --
  (B);
\draw[very thick,{Stealth[scale=1.4,inset=2pt]}-,name path=bc] (B) -- 
  node[left] {$a_2$} (C);
\draw[very thick,{Stealth[scale=1.4,inset=2pt]}-,name path=cd] (C) --
  node[above] {$a_1$}(D);
\draw[very thick,{Stealth[scale=1.4,inset=2pt,reversed]}-,name path=de] (D) --
 (E);

% Draw extensions of first path
\draw[very thick,loosely dotted,name path=b] (1,-4) -- (1,2);
\draw[very thick,loosely dotted,name path=c] (-1,-3) -- (6,-3);

% Draw reflected points
\coordinate (PP) at (2,2);
\coordinate (QP) at (6,-2);
\fill (PP) circle(2pt) node[above left] {$P'$};
\fill (QP) circle(2pt) node[below right] {$Q'$};

% Midpoints of bisected lines
\coordinate (R) at (1,1);
\coordinate (S) at (5,-3);
\fill (R) circle(2pt) node[above left] {$R$};
\fill (S) circle(2pt) node[below right] {$S$};

% Draw reflected lines
\draw[ultra thick,dotted] ($(B)+(1,2)$) --
  node[right,very near end,yshift=-8pt] {$a_2'$} ($(C)+(1,-2)$);
\draw[ultra thick,dotted] ($(C)+(-2,1)$) --
  node[above,very near start,xshift=-8pt,yshift=-1pt] {$a_1'$} ($(D)+(2,1)$);
\draw[ultra thick,dotted] (A) -- (PP);
\draw[ultra thick,dotted] (E) -- (QP);

% Draw fold
\draw[very thick,dashed] (R) -- (S);
\draw[rotate=-45] (R) rectangle +(8pt,8pt);
\draw[rotate=45] (S) rectangle +(8pt,8pt);
\end{tikzpicture}
\end{center}
%\caption{הקיפול של \L{Beloch} עבור $x^3-3x^2+-3x+1$}\label{f.beloch-fold2}
%\end{figure}



$a_2$
הוא קטע מהקו האנכי 
$x=1$,
והקו המקביל לו הוא
$a_2'$
הקו
$x=2$.
$a_1$
הוא קטע מהקו האופקי 
$y=-3$,
והקו המקביל לו הוא
$a_1'$
הקו
$y=-2$.
$RS$
הוא האנך האמצעי גם של
$\overline{PP'}$
וגם של
$\overline{QQ'}$.
המסלול
$\overline{PRSQ}$
זהה למסלול בסעיף%
~\ref{s.negative}.
%\begin{figure}[htb]
%\begin{center}
%
%\begin{tikzpicture}[scale=.9]
% Draw help lines and axes
%\draw[step=10mm,white!50!black] (-1,-5) grid (6,2);
%\foreach \x in {0,...,6}
%  \node at (\x-.3,-.2) {\sm{\x}};
%\foreach \y in {-4,...,-1}
%  \node at (-.3,\y-.3) {\sm{\y}};
%\foreach \y in {1,...,2}
%  \node at (-.3,\y-.3) {\sm{\y}};
%
% Draw first path
%\coordinate (A) at (0,0);
%\coordinate (B) at (1,0);
%\coordinate (C) at (1,-3);
%\coordinate (D) at (4,-3);
%\coordinate (E) at (4,-4);
%\node[above left] at (A) {$P$};
%\node[below right] at (E) {$Q$};
%\foreach \x in {A,B,C,D,E}
%  \fill (\x) circle(2pt);
%
%\draw[very thick,{Stealth[scale=1.4,inset=2pt,reversed]}-] (A) --
%  (B);
%\draw[very thick,{Stealth[scale=1.4,inset=2pt]}-,name path=bc] (B) -- 
%  node[left] {$a_2$} (C);
%\draw[very thick,{Stealth[scale=1.4,inset=2pt]}-,name path=cd] (C) --
%  node[above] {$a_1$}(D);
%\draw[very thick,{Stealth[scale=1.4,inset=2pt,reversed]}-,name path=de] (D) --
% (E);
%
% Draw extensions of first path
%\draw[very thick,loosely dotted,name path=b] (1,-4) -- (1,2);
%\draw[very thick,loosely dotted,name path=c] (-1,-3) -- (6,-3);
%
% Draw reflected points
%\coordinate (PP) at (2,2);
%\coordinate (QP) at (6,-2);
%\fill (PP) circle(2pt) node[above left] {$P'$};
%\fill (QP) circle(2pt) node[below right] {$Q'$};
%
% Midpoints of bisected lines
%\coordinate (R) at (1,1);
%\coordinate (S) at (5,-3);
%\fill (R) circle(2pt) node[above left] {$R$};
%\fill (S) circle(2pt) node[below right] {$S$};
%
% Draw reflected lines
%\draw[ultra thick,dotted] ($(B)+(1,2)$) --
%  node[right,very near end,yshift=-8pt] {$a_2'$} ($(C)+(1,-2)$);
%\draw[ultra thick,dotted] ($(C)+(-2,1)$) --
%  node[above,very near start,xshift=-8pt,yshift=-1pt] {$a_1'$} ($(D)+(2,1)$);
%\draw[ultra thick,dotted] (A) -- (PP);
%\draw[ultra thick,dotted] (E) -- (QP);
%
% Draw fold
%\draw[very thick,dashed] (R) -- (S);
%\draw[rotate=-45] (R) rectangle +(8pt,8pt);
%\draw[rotate=45] (S) rectangle +(8pt,8pt);
%\end{tikzpicture}
%\end{center}
%\caption{הקיפול של \L{Beloch} עבור $x^3-3x^2+-3x+1$}\label{f.beloch-fold2}
%\end{figure}

\selectlanguage{hebrew}



\section{הריבוע של
\L{Beloch}}\label{s.beloch-square}

ניתן להציג את הבנייה בסעיף הקודם לפי הריבוע של
\L{Beloch}:
%\begin{figure}[thb]
\begin{center}

\begin{tikzpicture}[scale=.7]
% Draw help lines and axes
\draw[step=10mm,white!60!black] (-12,-7) grid (2,7);
%\draw[thick] (-12,0) -- (2,0);
%\draw[thick] (0,-7) -- (0,7);
\foreach \x in {-11,...,3}
  \node at (\x-.3,-.2) {\sm{\x}};
\foreach \y in {1,...,7}
  \node at (-.4,\y-.3) {\sm{\y}};
\foreach \y in {-6,...,-1}
  \node at (-.4,\y-.3) {\sm{\y}};

% Draw first path
\coordinate (A) at (0,0);
\coordinate (B) at (1,0);
\coordinate (C) at (1,6);
\coordinate (D) at (-10,6);
\coordinate (E) at (-10,0);
\foreach \x in {A,B,C,D,E}
  \fill (\x) circle(2pt);
\draw[thick] (A) -- (B);
\draw[thick,name path=bc] (B) -- node[right,near end] {$a_2$} (C);
\draw[thick,name path=cd] (C) -- node[above] {$a_1$} (D);
\draw[thick,name path=de] (D) -- (E);

% Find first segment of second path
\path[name path=a2] (A) -- +(63.4:4);
\path [name intersections = {of = a2 and bc, by = {A2}}];
\draw[rotate=153.4] (A2) rectangle +(10pt,10pt);

% Draw second segment of second path
\path[name path=b2] (A2) -- +(153.4:10);
\path [name intersections = {of = b2 and cd, by = {B2}}];
\draw[very thick,dashed] (A2) -- (B2);
\draw[rotate=243.4] (B2) rectangle +(10pt,10pt);

% Draw square
\draw[very thick,dashed] (B2) -- +(243.4:8.94) coordinate (AB);
\draw[very thick,dashed] (A2) -- +(243.4:8.94) coordinate (BB);
\fill (AB) circle (2pt) node[below left] {$B$};
\fill (BB) circle (2pt) node[below right] {$A$};
\draw[very thick,dashed] (AB) -- (BB);

\path[fill=white!85!black,rotate=-26.6] (AB) rectangle +(8.94,8.94);

% Draw labels of points
\fill (A2) circle(2pt) node[above right,xshift=4pt] {$R$};
\fill (B2) circle(2pt) node[above left]  {$S$};
\fill (A)  circle(2pt) node[below right,yshift=-6pt]  {$P$};
\fill (E)  circle(2pt) node[above left]  {$Q$};
\end{tikzpicture}
\end{center}
%\caption{הריבוע של \L{Beloch}}\label{f.beloch-square}
%\end{figure}

נתונות שתי נקודות
$P,Q$
ושני קווים
$a_2,a_1$,
בנו ריבוע
$\overline{ARSB}$
כך ש:
\begin{itemize}
\item 
צלע אחד הוא
$\overline{RS}$
כאשר
$R$
נמצאת על
$a_2$
ו-%
$S$
נמצאת על
$a_1$;
\item $P$
נמצאת על
$\overline{RA}$
ו-%
$Q$
נמצאת על
$\overline{SB}$.
\end{itemize}
האיור
%\ref{f.beloch-square}
מדגים את הריבוע של
\L{Beloch}
עבור 
$x^3+6x^2+11x+6$.
האורך של
$RS$
הוא
$\sqrt{80}=4\sqrt{5}\approx 8.94$. 


\selectlanguage{hebrew}


\subsection*{מקורות}

פרק זה מבוסס על
\L{\cite{bradford, hull-beloch, riaz}}.
