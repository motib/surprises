% !TeX root = surprises.tex

\selectlanguage{hebrew}


\chapter[\R{האם משולשים עם אותו שטח ואותו היקף חופפים?}]{האם משולשים עם אותו שטח\\
ואותו היקף חופפים?}
\label{c.congruent}
%%%%%%%%%%%%%%%%%%%%%%%%%%%%%%%%%%%%%%%%%%%%%%%%%%%%%%%%%%%%%%%


האם משולשים עם אותו שטח ואותו היקף חופפים? לא בהכרח: לשני המשולשים הלא-חופפים עם הצלעות
$(17,25,28)$
ו-%
$(20,21,27)$
היקף
$70$
ושטח 
$210$.
פרק זה מראה שנתון משולש עם אורכי צלעות רציונליים, ניתן לבנות משולש לא-חופף עם אורכי צלעות רציונליים, ועם אותו היקף ושטח.
בסוף הפרק הבאתי הוכחה אלגנטית לנוסחה של הרון לשטח של משולש.


\section{ממשולש לעקומה אליפטית}

נתון משולש עם קודקודים
$A,B,C$
וצלעות
$a,b,c$,
חוצי הזוויות נפגשים בקודה אחת
$O$
שהוא המרכז של מעגל החסום על ידי המשולש )איור~%
\ref{f.inscribed}(.%
\footnote{במעגל חסום וצלעות משיקים למעגל והעובדה שהם חוצי הזוויות נובעת מהתכונות של משיקים.}
$OA',OB',OC'$
הם הגבהים של המשולש. באיור סומנו גם זוגות של זוויות מרכזיות
$\alpha/2,\beta/2,\gamma/2$.
השוויון של הזוויות נובע ממשולשים ישר-זווית חופפים:
\[
\triangle AOB'\cong \triangle AOC',\quad \triangle BOA'\cong \triangle BOC', \quad \triangle COA'\cong \triangle COB'\,.
\]
החפיפת המשולשים מתקבל גם שוויון בין קטעי הקו 
$u,v,w$
המחברים את הצלעות עם נקודת ההשקה שלהן עם המעגל.



\begin{figure}[htb]
\begin{center}
\begin{tikzpicture}[baseline=-6mm,scale=2.25]
% Draw base and path two lines at known angles
\draw (0,0) coordinate (a) node[xshift=-6pt] {$A$} -- (0:6) coordinate (b) node[xshift=6pt] {$B$};
\path[name path=ac] (a) -- +(50:4);
\path[name path=bc] (b) -- +(150:5);
% Get their intersection and draw lines between vertices
\path[name intersections={of=ac and bc,by=c}];
\node[above] at (c) {$C$};
\draw (a) -- (c) -- (b) -- (a);
% Label angles with tick marks
\draw (a) ++(0:4mm) arc (0:50:4mm);
\draw (a) ++(10:3.5mm) -- +(10:1mm);
\draw (a) ++(15:3.5mm) -- +(15:1mm);
\draw (a) ++(35:3.5mm) -- +(35:1mm);
\draw (a) ++(40:3.5mm) -- +(45:1mm);
\draw (b) ++(150:5mm) arc (150:180:5mm);
\draw (b) ++(157.5:4.5mm) -- +(157.5:1mm);
\draw (b) ++(172.5:4.5mm) -- +(172.5:1mm);
\draw (c) ++(230:3mm) arc (230:330:3mm);
\draw (c) ++(250:2.5mm) -- +(250:1mm);
\draw (c) ++(255:2.5mm) -- +(255:1mm);
\draw (c) ++(260:2.5mm) -- +(260:1mm);
\draw (c) ++(300:2.5mm) -- +(300:1mm);
\draw (c) ++(305:2.5mm) -- +(305:1mm);
\draw (c) ++(310:2.5mm) -- +(310:1mm);
% Path bisectors of two lines
\path[name path=bia] (a) -- +(25:3.5);
\path[name path=bib] (b) -- +(165:5);
% Intersection of angle bisectors
\path [name intersections={of=bia and bib,by=center}];
% Draw angle bisectors to center
\draw (a) -- (center);
\draw (c) -- (center);
\draw (b) -- (center);
% Draw radii
\draw (center) -- node[left] {$r$} ($(a)!(center)!(b)$) node[below,yshift=-2pt] {$C'$} coordinate (ap);
\draw (center) -- node[left,yshift=-4pt] {$r$} ($(a)!(center)!(c)$) node[above left] {$B'$} coordinate (bp);
\draw (center) -- node[right] {$r$} ($(b)!(center)!(c)$) node[above right] {$A'$} coordinate (cp);
% Draw dots
\fill (center) circle (.5pt) node[above,xshift=3pt,yshift=6pt] {$O$};
\fill (a) circle (.5pt);
\fill (b) circle (.5pt);
\fill (c) circle (.5pt);
\fill (ap) circle (.5pt);
\fill (bp) circle (.5pt);
\fill (cp) circle (.5pt);
% Draw right angle squares
\draw (ap) -- ++(90:4pt) -- ++(0:4pt) -- ++(-90:4pt);
\draw (bp) -- ++(-40:4pt) -- ++(-130:4pt) -- ++(-220:4pt);
\draw (cp) -- ++(-30:4pt) -- ++(-120:4pt) -- ++(-210:4pt);
% Labels of angles
\node[above,xshift=5pt,yshift=21pt] at (center) {$\gamma/2$};
\node[above left,xshift=-4pt,yshift=21pt] at (center) {$\gamma/2$};
\node[above right,xshift=4pt,yshift=-5pt] at (center) {$\beta/2$};
\node[below right,yshift=-6pt] at (center) {$\beta/2$};
\node[left,xshift=-8pt,yshift=3pt] at (center) {$\alpha/2$};
\node[below left,xshift=2pt,yshift=-6pt] at (center) {$\alpha/2$};
% Labels of line segments (names of points are weird...)
\path (a) -- node[below,yshift=-2pt] {$u$} (ap);
\path (a) -- node[left, xshift=-2pt] {$u$} (bp);
\path (b) -- node[above,yshift=2pt]  {$v$} (cp);
\path (b) -- node[below,xshift=-2pt] {$v$} (ap);
\path (c) -- node[above,xshift=-2pt] {$w$} (bp);
\path (c) -- node[above,xshift=2pt]  {$w$} (cp);
% Labels of sides
\draw[<->] ($(a)+(0,-15pt)$) -- node[fill=white] {$c$} 
           ($(b)+(0,-15pt)$);
\draw[<->] ($(a)+(-12pt,10pt)$) -- node[fill=white] {$b$}
           ($(c)+(-12pt,10pt)$);
\draw[<->] ($(b)+(6pt,13pt)$) -- node[fill=white] {$c$}
           ($(c)+(6pt,13pt)$);
% Inscribed circle
\node[thick,dotted,draw,circle through=(ap)] at (center) {};
\end{tikzpicture}
\end{center}
\selectlanguage{hebrew}
\caption{מעגל חסום המוגדר על ידי חיתוך חוצי הזווית משולש}\label{f.inscribed}
\end{figure}



השטח 
$S_{\triangle ABC}$
הוא סכום השטחים של 
$\triangle AOC, \triangle BOC, \triangle AOB$:

\begin{eqnlabels}
S_{\triangle ABC} &=& \frac{1}{2}(w+v)r + \frac{1}{2}(v+u)r + \frac{1}{2}(u+w)r\label{eq.area1}\\
S_{\triangle ABC}&=& \frac{1}{2}\cdot 2(u+v+w)r = rs\,, \label{eq.area2}
\end{eqnlabels}
כאשר 
$s$
הוא מחצית ההיקף:
\[
s=\disfrac{1}{2}(a+b+c)=\disfrac{1}{2}(2u+2v+2w)=u+v+w\,.
\]
נבטא את 
$u,v,w$
באמצעות פונקציות טריגונומטריות של זוויות המעגל והרדיוס
$r$:




\begin{eqnlabels}
\tan \frac{\alpha}{2} &=& \frac{u}{r}\label{eq.alpha}\\
\tan \frac{\beta}{2} &=& \frac{v}{r}\label{eq.beta}\\
\tan \frac{\gamma}{2} &=& \frac{w}{r}\label{eq.gamma}\,.
\end{eqnlabels}
נסכם ונקבל ביטוי של 
$s$
כתלות של הזוויות והרדיוס:
\[
s = u+v+w = r\tan \frac{\alpha}{2}+r\tan \frac{\beta}{2}+r\tan \frac{\gamma}{2} = r\left(\tan \frac{\alpha}{2}+\tan \frac{\beta}{2}+\tan \frac{\gamma}{2}\right)\,,
\]
ולפי המשוואה~%
\ref{eq.area2}
$S_{\triangle ABC}=rs$:

\begin{eqnlabels}
S_{\triangle ABC} &=&r\cdot r\left(\tan \frac{\alpha}{2}+\tan \frac{\beta}{2}+\tan \frac{\gamma}{2}\right)\label{eq.area3}\\
\frac{S_{\triangle ABC}}{r^2} &=& \tan \frac{\alpha}{2}+\tan \frac{\beta}{2}+\tan \frac{\gamma}{2} \label{eq.area4}\\
\frac{s^2}{S_{\triangle ABC}} &=& \tan \frac{\alpha}{2}+\tan \frac{\beta}{2}+\tan \frac{\gamma}{2} \label{eq.area5}\,.
\end{eqnlabels}
סכום הזוויות
$\alpha,\beta,\gamma$
הוא
$2\pi$,
ולכן:

\begin{eqnlabels}
%\gamma &=& 2\pi - (\alpha + \beta)\\
%\gamma/2 &=& \pi - (\alpha/2 + \beta/2)\\
\tan\gamma/2 &=& \tan(\pi - (\alpha/2 + \beta/2))\\
\tan\gamma/2&=& -\tan (\alpha/2 + \beta/2)\\
\tan\gamma/2&=& \frac{\tan\alpha/2 + \tan\beta/2}{\tan\alpha/2 \, \tan\beta/2-1}\,.\label{eq.tangent1}
\end{eqnlabels}
השתמשנו בזהות טריגונומטרית שהוכחה בסעיף~
\ref{s.tangent}.

נפשט את הסימון על ידי הגדרת נעלמים עבור הטנגנסים:
\[
x=\tan \frac{\alpha}{2},\quad
y=\tan \frac{\beta}{2},\quad
z=\tan \frac{\gamma}{2}\,.
\]
עם סימון זה משוואה~%
\ref{eq.tangent1}
היא:

\begin{equation}
z = \frac{x+y}{xy-1}\,.\label{eq.xy1}
\end{equation}
ומשוואה~%
\ref{eq.area5}
היא:

\begin{equation}
x+y+\frac{x+y}{xy-1}=\frac{s^2}{S_{\triangle ABC}}\,.\label{eq.xy2}
\end{equation}



האם קיימים פתרונות שונים למשוואה%
~\ref{eq.xy2}?
עבור משולש ישר-הזווית
$(3,4,5)$:

\begin{eqnlabels}
x+y+\frac{x+y}{xy-1}=\frac{6^2}{6}=6\\
x^2y + xy^2 -6xy + 6 = 0\,.\label{eq.elliptic}
\end{eqnlabels}



בסעיף הבא נחפש פתרונות נוספים למשוואה זו.

\section{פתרון המשוואה לעקומה אליפטית}

העקומה באיור~%
\ref{f.two-secants}
היא של ציור חלקי של המשוואה
\ref{eq.elliptic}.
כל נקודה בעקומה ברביע הראשון היא פתרון, כי אורכי הצלעות חייבים להיות חיוביים. 
$A,B,D$
מתאימות למשולש
$(3,4,5)$
כפי שנראה בהמשך. כדי למצוא פתרונות רציונליים נוספים, נשתמש ב-%
\textbf{שיטת שני סקנסים}
\L{(\textbf{method of two secants})}.
\begin{figure}[tb]
\begin{center}

\begin{tikzpicture}[scale=1.5]
\draw[thin,step=10mm] (-4,-4) grid (4,4);
\draw[thick] (-4,0) -- (4,0);
\draw[thick] (0,-4) -- (0,4);
\foreach \x in {-3,...,4}
  \node at (\x-.2,-.2) {\sm{\x}};
\foreach \y in {-3,...,-1}
  \node at (+.2,\y-.3) {\sm{\y}};
\foreach \y in {1,...,4}
  \node at (+.2,\y-.3) {\sm{\y}};

\draw[very thick,domain=.936:3.306,samples=200] plot (\x,{
(
  (6*\x-\x*\x)+
  sqrt(
   (\x*\x-6*\x)^2 -
   4*\x*6
  )
)/
(2*\x)
});

\draw[very thick,domain=.936:3.306,samples=100] plot (\x,{
(
  (6*\x-\x*\x)-
  sqrt(
   (\x*\x-6*\x)^2 -
   4*\x*6
  )
)/
(2*\x)
});

\draw[very thick,domain=-2.5:-.25,samples=100] plot (\x,{
(
  (6*\x-\x*\x)+
  sqrt(
   (\x*\x-6*\x)^2 -
   4*\x*6
  )
)/
(2*\x)
});

\coordinate (A) at (2,3);
\coordinate (B) at (1,2);
\coordinate (C) at (-1.5,-0.5);
\coordinate (D) at (3,2);
\coordinate (E) at (1.5,1.2);

\draw[very thick,dashed,red]  ($(C)!-.4!(A)$) -- ($(C)!1.2!(A)$);
\draw[very thick,dashed,blue] ($(C)!-.4!(D)$) -- ($(C)!1.2!(D)$);

\fill (A) circle(1pt) node[right,xshift=	12pt,yshift=-8pt] {A=(2,3)};
\fill (B) circle(1pt) node[above left,xshift=2pt] {B=(1,2)};
\fill (C) circle(1pt) node[right,xshift=20pt,yshift=-4] {C=(-1.5,-0.5)};
\fill (D) circle(1pt) node[right,xshift=6pt,yshift=-6pt] {D=(3,2)};
\fill (E) circle(1pt) node[below,xshift=8pt,yshift=-12pt] {E=(1.5,1.2)};

\end{tikzpicture}
\end{center}
\selectlanguage{hebrew}
\caption{שיטת שני הסקנטים}\label{f.two-secants}
\end{figure}

ציירו סקנס דרך הנקודות
$A=(2,3)$
ו-%
$B=(1,2)$.
הוא חותך את העקומה ב-%
$C=(-1.5,-0.5)$.
נקודה זו אינה פתרון כי הקואורדינטות שליליים. אם נצייר סקנס שני מ-%
$C$
ל-%
$D=(3,2)$,
החיתוך שלו עם העקומה ב-%
$E$
כן מהווה פתרון נוסף.%
\footnote{$(1.5,1.2)$
\R{הוא קירוב. בהמשך נחשב את הקואורדינטות המדוייקות}.}

המשוואה של הקו )האדום( דרך 
$A,B$
היא
$y=x+1$. 
נציב עבור 
$y$
במשוואה%
~\ref{eq.elliptic}:
\[
x^2(x+1) + x(x+1)^2 -6x(x+1) +6 =0,\,
\]
ונפשט:
\[
2x^3 -3x^2 -5x +6 =0\,.
\]
מהנקודות
$A,B$
אנו יודעים שני שורשים
$x=2,x=1$,
כך שאפשר לפרק את הפולינום מדרגה שלוש כך:
\[
(x-2)(x-1)(ax+b)=0\,,
\]
כאשר רק השורש השלישי לא ידוע. נכפיל את הגורמים ונראה מיד שהמקדם של הגורם מדרגה שלוש
$x^3$,
חייב להיות
$2$,
ו-%
$2b$,
הקבוע, חייב להיות
$6$.
לכן, הגורם השלישי הוא
$2x+3$
ומכאן שהשורש השלישי הוא
$x=-\disfrac{3}{2}$.
נחשב
$y=x+1=-\disfrac{1}{2}$.
הקואורדינטות של הנקודה
$C$
הן
$\left(-\disfrac{3}{2},-\disfrac{1}{2}\right)$.

המשוואה של הסקנס שני דרך
$D,C$
)כחול מקווקוו( היא:

\begin{equation}
y = \frac{5}{9}x + \frac{1}{3}\,.\label{eq.second-secant}
\end{equation}
נציב עבור 
$y$
במשוואה 
~\ref{eq.elliptic}:
\[
x^2\left(\frac{5}{9}x + \frac{1}{3}\right) + x\left(\frac{5}{9}x + \frac{1}{3}\right)^2 -6x\left(\frac{5}{9}x + \frac{1}{3}\right) +6 =0\,,
\]
ונפשט:
\[
\frac{70}{81}x^3 - \frac{71}{27}x^2 - \frac{17}{9}x +6 =0\,.
\]
שוב יש לנו שני שורשים
$x=3,x=-\disfrac{3}{2}$,
וניתן לפרק את הפולינום מדרגה שלוש:
\[
(x-3)\left(x+\frac{3}{2}\right)(ax+b)=0\,.
\]
נשווה את המקדם של 
$x^3$
ונשווה את הקובע ונקבל:
\[
\frac{70}{81}x - \frac{4}{3}=0\,,
\]
ולכן:
\[
x=\frac{81}{70}\cdot \frac{4}{3}= \frac{27\cdot 4}{70} = \frac{54}{35}\approx 1.543\,.
\]
נחשב את
$y$
ממשוואה~%
\ref{eq.second-secant}
ונקבל:

\[
y=\frac{25}{21}\approx 1.190\,.
\]
הערכים הללו קרובים למה שקראנו מאיור~%
\ref{f.two-secants}:
$(1.5,1.2)$.

לבסוף, נחשב את
$z$
ממשוואה
~\ref{eq.xy1}:
\[
z=\frac{x+y}{xy-1}=%
\left(\disfrac{54}{35} + \disfrac{25}{21}\right)%
 \, / \,%
\left(\disfrac{54}{35}\cdot\disfrac{25}{21}-1\right)=%
\frac{2009}{615} = \frac{49}{15}\,.
\]

\section{פיתוח משולש מהעקומה האליפטית}
מ-%
$x,y,z,a,b,c$, 
ניתן לחשב את אורכי הצלעות של המשולש
$\triangle ABC$:

\begin{eqn}
a&=&w+v = r(z+y)=(z+y)\\
b&=&u+w= r(x+z)=(x+z)\\
c&=&u+v=r(x+y)=(x+y)\,,
\end{eqn}
כי
$r=\disfrac{A}{s}=\disfrac{6}{6}=1$.

עבור הפתרון 
$A=(2,3)$
של העקומה ערכו של
$z$
הוא:
\[
z=\frac{x+y}{xy-1}=\frac{2+3}{2\cdot 3-1}=1\,,
\]
והצלעות של המשולש הן:

\begin{eqn}
a &=& z+y = 1+3 = 4\\
b &=& x+z = 2+1=3\\
c &=& x+y = 2+3=5\,.
\end{eqn}



המשולש ישר-זווית עם
$s=A=6$.
חישוב הצלעות המתאימים ל-%
$B$
ו-%
$D$
נותן את אותו משולש.
עבור
$E$:

\begin{eqn}
a &=& z+y = \frac{49}{15} + \frac{25}{21} = \frac{243+125}{105}= \frac{156}{35}\\
b &=& x+z = \frac{54}{35} + \frac{49}{15} = \frac{810+1715}{525}=\frac{101}{21}\\
c &=& x+y = \frac{54}{35} + \frac{25}{21}  = \frac{1134+875}{735}=\frac{41}{15}\,,
\end{eqn}

נבדוק את התוצאה. מחצית ההיקף היא:
\[
s=\frac{1}{2}\left(\frac{156}{35} + \frac{101}{21}+\frac{41}{15}\right) = \frac{1}{2}\left(\frac{468+505+287}{105}\right) = \frac{1}{2}\left(\frac{1260}{105}\right)= 6\,.
\]
נחשב את השטח באמצעות הנוסחה של הרון
$S_{\triangle ABC}= \sqrt{s(s-a)(s-b)(s-c)}$:

\begin{eqn}
%A &=& \sqrt{s(s-a)(s-b)(s-c)}\\
S_{\triangle ABC}&=& \sqrt{6 \left(6-\frac{156}{35}\right) \left(6-\frac{101}{21}\right) \left(6-\frac{41}{15}\right)}\\
&=& \sqrt{6 \cdot \frac{54}{35}\cdot \frac{25}{21} \cdot \frac{49}{15}}\\
&=& \sqrt{\frac{396900}{11025}}= \sqrt{36} = 6\,.
\end{eqn}
קשה להאמין שהיינו "מנחשים" שלמשולש:
\[
\left(\frac{156}{35},\, \frac{101}{21},\,\frac{41}{15}\right)
\]
אותו היקף ושטח כמו
$(3,4,5)$!

\subsection*{מקורות}

פרק זה מבוסס על 
\L{\cite{mccallum}}.
ברבש
\L{\cite{marita}}
מראה שאם נתון משולש שווה-צלעות, קיימים משולשים לא חופפים עם אותו היקף ואותו שטח, אולם ההוכחה שלה לא כוללת בנייה. 
המשוואה~%
\ref{eq.elliptic}
היא
\textbf{עקומה אליפטית}
נושא שנחקר לעומק על ידי מתמטיקאים.
