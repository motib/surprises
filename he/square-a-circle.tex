% !TeX root = surprises.tex

\selectlanguage{hebrew}
\chapter{איך לרבע את המעגל}
\label{c.square}


\section{
קירובים ל-%
$\pi$}\label{s.square-intro}

כדי לרבע מעגל יש לבנות את האורך 
$\sqrt{\pi}$,
אבל
$\pi$
הוא טרנסנדנטי, כלומר, הוא אינו פתרון של אף משוואה אלגברית. 
פרק זה מביא שלוש בניות של קירובים ל-%
$\pi$.
הטבלה שלהלן מביא את הנוסחאות של האורכים שנבנה, ערכם המספרי, ההפרש בין ערכים הללו והערך של 
$\pi$,
והשגיאה )במטרים( אם משתמשים בקירוב כדי לחשב את היקף כדור הארץ, כאשר נתון שהרדיוס הוא
$6378$
ק"מ.
\[
\selectlanguage{english}
%\renewcommand{\arraystretch}{1.2}
\begin{array}{|l|c|c|c|c|}
\hline
\textrm{\R{הבנייה}} & \textrm{\R{הנוסחה}} &\textrm{\R{הערך}} & \textrm{\R{ההפרש}} & \textrm{\R{השגיאה )מ(}}\\\hline
\pi & \textrm{\rule[-15pt]{0pt}{25pt}}& 3.14159265359 & - & -\\\hline
\textrm{Kochansky} & \textrm{\rule[-15pt]{0pt}{35pt}}\sqrt{\disfrac{40}{3}-2\sqrt{3}}&
  3.14153338705 & 5.932 \times 10^{-5} & 756\\\hline
\textrm{Ramanujan}\; 1 & \textrm{\rule[-15pt]{0pt}{35pt}}\disfrac{355}{113} &
  3.14159292035 &2.667  \times 10^{-7}&3.4\\\hline
\textrm{Ramanujan}\; 2 &\textrm{\rule[-15pt]{0pt}{35pt}}\left(9^2+\disfrac{19^2}{22}\right)^{1/4}&
  3.14159265258 & 1.007 \times 10^{-9}& 0.013\\\hline
\end{array}
\]
בבניות בפרק זה נצטרך לחלק
\textbf{קטע קו}
לשלושה חלקים. ניתן לבנות קטע קו בכל אורך רציונלי. נתון קטע קו באורך 
$1$
וקטעי קו באורכים 
$a,b$,
לפי משולשים דומים
$1/b=\overline{OD}/a$
ולכן
$\overline{OD}=a/b$.
%\begin{figure}[H]
\begin{center}
\selectlanguage{english}
\begin{tikzpicture}
%\clip (-.6,-1) rectangle (4,3);
\draw[name path=horz] (0,0) coordinate (o) -- (7,0);
\fill (o) circle(1.5pt) node[left] {$O$};
\fill (6,0) circle(1.5pt) coordinate (a) node[below] {$A$};
\draw (0,0) -- (30:5.5);
\fill (30:3) circle(1.5pt) coordinate (c) node[above] {$C$};
\fill (30:5) circle(1.5pt) coordinate (b) node[above] {$B$};
\draw (a) -- (b);
\path[name path=par] (c) -- +($(a)-(b)$);
\path[name intersections={of=par and horz,by=d}];
\fill (d) circle(1.5pt) node[below] {$D$};
\draw (c) -- (d);
\draw[<->] (-.4,.5) -- node[fill=white] {$b$} +(30:5.2);
\path (o) -- node[above] {$1$} (c);
\draw[<->] (0,-.8) -- node[fill=white] {$a$} +(6,0);
\path (o) -- node[below] {$a/b$} (d);
\end{tikzpicture}
\end{center}
%\caption{חילוק במשולשים דומים}\label{f.similar}
%\end{figure}

\newpage


%%%%%%%%%%%%%%%%%%%%%%%%%%%%%%%%%%%%%%%%%%%%%%%%%%%%%%%%%%%%%%%

\section{הבניה של
\L{Kochansky}}

\begin{itemize}
\item
בנו מעגל יחידה שמרכזו 
$O$,
עם קוטר
$\overline{AB}$
ובנו משיק למעגל ב-%
$A$.
\item
בנו מעגל יחידה שמרכזו
$A$.
סמנו את החיתוך עם המעגל הראשון ב-%
$C$.\footnote{%
עבור המעגל השני והשלישי, האיור מראה רק את הקשת החותך את המעגל הקודם.%
}
\item
בנו מעגל יחידה שמרכזו 
$C$.
סמנו את החיתוך שלו עם המעגל השני ב-%
$D$. 
\item
בנו
$\overline{OD}$
וסמנו את החיתוך שלו עם המשיק ב-%
$E$.
\item
מ-%
$E$
בנו
$F,G,H$,
כל אחת במרחק 
$1$
מהנקודה הקודמת, כך ש-%
$\overline{AH}=3-\overline{EA}$.
\item
בנו
$\overline{BH}$.
\end{itemize}
%\begin{figure}[bt]
\begin{center}
\selectlanguage{english}
\begin{tikzpicture}[scale=.5]
% Scale at 4

% Coordinates of circle
\coordinate (O) at (0,0);
\coordinate (A) at (0,-4);
\coordinate (B) at (0,4);
\fill (O) circle(2pt) node[above right] {$O$};
\fill (A) circle(2pt) node[below right] {$A$};
\fill (B) circle(2pt) node[above right] {$B$};
\draw (A) rectangle +(12pt,12pt);

% Draw circle and diameter
\node [thick,draw,circle through=(A),name path=circle] at (O) {};
\draw [thick] (A) -- (B);

% Draw tangent at A
\draw[thick,name path=tangent] ($(A)+(-4.5,0)$) -- ($(A)+(10.5,0)$);

% Draw arc centered at A which intersects circle at C
\draw[thick,name path=Aarc] (O)
   arc [start angle=90,end angle=220,radius=4];
\path[name intersections={of=circle and Aarc,by=C}];
\fill (C) circle(2pt) node[left,xshift=-4pt] {$C$};

% Draw arc centered at C which intersects the first arc at D
\draw[thick,name path=Carc] ($(C)+(260:4)$)
   arc [start angle=260,end angle=280,radius=4];
\path[name intersections={of=Carc and Aarc,by=D}];
\fill (D) circle(2pt) node[below left] {$D$};

% Draw O--D which intersects the tangent at E
\draw[thick,name path=OD] (O) -- (D);
\path[name intersections={of=tangent and OD,by=E}];
\fill (E) circle(2pt) node[above left] {$E$};

% Find point H at length 3 from E
\coordinate (F) at ($(E)+(4,0)$);
\fill (F) circle(2pt) node[above right,xshift=4pt] {$F$};
\coordinate (G) at ($(F)+(4,0)$);
\fill (G) circle(2pt) node[above] {$G$};
\coordinate (H) at ($(G)+(4,0)$);
\fill (H) circle(2pt) node[above] {$H$};

% Draw BH of length approximately pi
\draw[thick] (B) -- (H);
\end{tikzpicture}
\end{center}
%\caption{הבניה של \L{Kochansky}}\label{f.kochansky}
%\end{figure}

\textbf{טענה:}
$\overline{BH}=\sqrt{\disfrac{40}{3}-2\sqrt{3}}\approx \pi$.

\textbf{הוכחה:}
איור~%
\ref{f.kochansky}
מתמקד בחלק מהאיור למעלה. קטעי הקו המקווקווים נוספו. מפני שכל המעגלים הם מעגלי היחידה, קל לראות שאורך כל אחד מהקטעים המקווקווים הוא 
$1$.
מכאן ש-%
$AOCD$
הוא מעויין, ולכן האלכסונים שלו ניצבים זה לזה וחוצים זה את זה ב-%
$K$,
כך ש-%
$\overline{AK}=\frac{1}{2}$.
\begin{figure}
\begin{center}
\selectlanguage{english}
\begin{tikzpicture}[scale=1.2]
% Scale at 4

\clip (-4.5,-6.5) rectangle +(5,7);
% Coordinates of circle
\coordinate (O) at (0,0);
\coordinate (A) at (0,-4);
\coordinate (B) at (0,4);

% Draw circle
\node [circle through=(A),name path=circle] at (O) {};
\draw ($(O)+(200:4)$) arc [start angle=200,end angle=280,radius=4];

% Draw tangent at A
\draw[thick,name path=tangent] ($(A)+(-4.5,0)$) -- ($(A)+(10.5,0)$);
\draw (A) rectangle +(6pt,6pt);

% Draw arc centered at O which intersects circle at C
\draw[thin,name path=Aarc] (O)
   arc [start angle=90,end angle=220,radius=4];
\path[name intersections={of=circle and Aarc,by=C}];

% Draw arc centered at C which intersects the first arc at D
\draw[thin,name path=Carc] ($(C)+(260:4)$)
   arc [start angle=260,end angle=280,radius=4];
\path[name intersections={of=Carc and Aarc,by=D}];

% Draw O--D which intersects the tangent at E
\draw[thick,name path=OD] (O) -- (D);
\path[name intersections={of=tangent and OD,by=E}];

% Find point H at length 3 from E
\coordinate (F) at ($(E)+(4,0)$);
\coordinate (G) at ($(F)+(4,0)$);
\coordinate (H) at ($(G)+(4,0)$);

% Draw BH of length approximately pi
\draw[thick] (B) -- (H);

\draw[ultra thick,dashed] (A) -- (O) -- (C) -- (D) -- cycle;
\draw[ultra thick,dashed,name path=AC] (A) -- (C);

\node[above left,yshift=10pt,xshift=2pt] at (A) {$60^\circ$};
\node[left,yshift=10pt,xshift=-20pt] at (A) {$30^\circ$};

\path[name intersections={of=AC and OD,by=K}];
\draw[thick,rotate=-30] (K) rectangle +(6pt,6pt);

\path (A) -- node[above,xshift=2pt,yshift=2pt] {$1/2$} (K);
\path (A) -- node[below,xshift=-4pt] {$1/\sqrt{3}$} (E);

\fill (O) circle(1.25pt) node[above right] {$O$};
\fill (A) circle(1.25pt) node[below right] {$A$};
\fill (B) circle(1.25pt) node[above right] {$B$};
\fill (C) circle(1.25pt) node[left,xshift=-4pt] {$C$};
\fill (D) circle(1.25pt) node[below left] {$D$};
\fill (E) circle(1.25pt) node[above left] {$E$};
\fill (F) circle(1.25pt) node[above right,xshift=4pt] {$F$};
\fill (G) circle(1.25pt) node[above] {$G$};
\fill (H) circle(1.25pt) node[above] {$H$};
\fill (K) circle(1.25pt) node[above,yshift=4pt] {$K$};
\end{tikzpicture}
\end{center}
\caption{הוכחת הבניה של \L{Kochansky}}\label{f.kochansky}
\end{figure}

האלכסון
$\overline{AC}$
מייצר משולשים שווי-צלעות
$\triangle OAC, \triangle DAC$
כך ש-%
$\angle OAC=60^\circ$.
הזווית בין המשיק לרדיוס
$\overline{OA}$
היא זווית ישרה ולכן
$\angle KAE=30^\circ$.
נחשב:
\erh{12pt}
\begin{equationarray*}{rcl}
\disfrac{1/2}{\overline{EA}}&=&
\cos 30^\circ=\disfrac{\sqrt{3}}{2}\\
\overline{EA}&=&\disfrac{1}{\sqrt{3}}\\
\overline{AH}&=&3-\overline{EA}\\
&=&\left(3-\disfrac{1}{\sqrt{3}}\right)
=\disfrac{3\sqrt{3}-1}{\sqrt{3}}
\end{equationarray*}
נחזור לאיור הראשון.
$\triangle ABH$
הוא משולש ישר-זווית, ולפי משפט פיתרוגס:
\erh{12pt}
\begin{equationarray*}{rcl}
\overline{BH}^2&=&\overline{AB}^2+\overline{AH}^2\\
&=&2^2+\left(\disfrac{3\sqrt{3}-1}{\sqrt{3}}\right)^2\\
&=&4+\disfrac{9\cdot 3 -6\sqrt{3}+1}{3}\\
&=&\disfrac{40}{3}-2\sqrt{3}\\
\overline{BH}&=&\sqrt{\disfrac{40}{3}-2\sqrt{3}}\approx 3.141533387\approx \pi\,.
\end{equationarray*}

\newpage
%%%%%%%%%%%%%%%%%%%%%%%%%%%%%%%%%%%%%%%%%%%%%%%%%%%%%%%%%%%%%%%%

\section{הבניה הראשונה של
\L{Ramanujan}}
\begin{itemize}
\item
בנו מעגל יחידה שמרכזו 
$O$
עם קוטר
$\overline{PR}$.
\item
בנו נקודה
$H$
שחוצה את
$\overline{PO}$
ונקודה
$T$
כך ש-%
$\overline{TR}=\disfrac{1}{3}\overline{OR}=\disfrac{1}{3}$.
\item
בנו ניצב ב-%
$T$
שחותך את המעגל ב-%
$Q$.
\item
בנו את המיתר
$\overline{RS}=\overline{QT}$.
\item
בנו את המיתר
$\overline{PS}$.
\item
בנו קו מקביל ל-%
$\overline{RS}$
העובר דרך
$T$,
וסמנו ב-%
$N$
את החיתוך שלו עם 
$\overline{PS}$.
\item
בנו קו מקביל ל-%
$\overline{RS}$
העובר דרך
$O$,
וסמנו ב-%
$M$
את החיתוך שלו עם 
$\overline{PS}$.
\item
בנו את המיתר
$\overline{PK}=\overline{PM}$.
\item
בנו משיק ב-%
$P$
שאורכו
$\overline{PL}=\overline{MN}$.
\item
חברו את הנקודות
$K,L,R$.
\item
מצאו נקודה 
$C$
כך ש-%
$\overline{RC}=\overline{RH}$.
\item
בנו קו
$\overline{CD}$
המקביל ל-%
$\overline{KL}$
שחותך את
$\overline{LR}$ 
ב-%
$D$. 
\end{itemize}

%\begin{figure}
\begin{center}
\selectlanguage{english}
\begin{tikzpicture}[scale=1,align=left]
\clip (-6,-5.1) rectangle +(11.5,10.2);
% Draw circle and horizontal diameter
\draw[name path=circle] (0,0)  coordinate (o) node[below] {$O$} circle[radius=5cm];
\draw (-5,0) coordinate (p) node[left] {$P$} -- (5,0) coordinate (r) node[right] {$R$};
\fill (o) circle (1pt);
\fill (p) circle (1pt);
\fill (r) circle (1pt);
\fill (-2.5,0) coordinate (h) node[below] {$H$} circle (1pt);
\fill (10/3,0) coordinate (t) node[below left] {$T$} circle (1pt);
\path (p) -- node[above,xshift=10pt] {$1/2$} (h) -- node[above] {$1/2$} (o) -- node[above] {$2/3$} (t) -- node[above] {$1/3$} (r);

% Draw perpendicular TQ
\path[name path=tq] (t) -- +(0,5);
\path[name intersections={of=tq and circle,by=q}];
\draw (t) -- (q) node[above] {$Q$};
\fill (q) circle (1pt);

% Draw chord RS and line PS
\path[name path=tq] (t) -- +(0,5);
\path[name intersections={of=tq and circle,by=q}];
\path[name path=rcirc] (r) let \p1 = ($ (t) - (q) $) in circle ({veclen(\x1,\y1)});
\path[name intersections={of=rcirc and circle,by=s}];
\draw (r) -- (s);
\fill (s) node[above right] {$S$} circle (1pt);
\draw[name path=ps] (p) -- (s);

% Draw TN
\path[name path=tn] (t) -- +($(s)-(r)$);
\path[name intersections={of=ps and tn,by=n}];
\draw (t) -- (n);
\fill (n) node[above] {$N$} circle (1pt);

% Draw OM
\path[name path=om] (o) -- +($(s)-(r)$);
\path[name intersections={of=ps and om,by=m}];
\draw (o) -- (m);
\fill (m) node[above left] {$M$} circle (1pt);
\path (p) -- (m);
\path (m) -- (n);

% Draw chord PK
\draw (p) -- +(-62.3:4.64) coordinate (k) node[below left] {$K$};

% Draw tangent PL
\draw let \p1 = ($ (m) - (n) $), \n1 = {veclen(\x1,\y1)} in (p) -- (-5,-\n1) coordinate (l) node[left] {$L$};

% Connect L and K to R
\draw (r) -- (l) -- (k) -- cycle;

% Find point C on RK
\coordinate (c) at ($(r)!7.5cm!(k)$);
\path (r) -- (c);
\fill (c) node[below] {$C$} circle (1pt);

% Draw CD
\path[name path=cd] (c) -- +($(l)-(k)$);
\path[name path=lr] (l) -- (r);
\path[name intersections={of=cd and lr,by=d}];
\draw (c) -- (d);
\fill (d) node[above,xshift=2pt] {$D$} circle (1pt);
\path (r) -- (d);

\draw[rotate=90] (t) rectangle +(8pt,8pt);
\draw[rotate=-160] (s) rectangle +(8pt,8pt);
\draw[rotate=-90] (p) rectangle +(8pt,8pt);
\draw[rotate=30] (k) rectangle +(8pt,8pt);
\draw[dashed] (o) -- (q);
\end{tikzpicture}
\end{center}
%\caption{הבניה של \L{Ramanujan} ל-%
%$\disfrac{355}{113}$}\label{f.ramanujan1}
%\end{figure}

\newpage

\textbf{טענה:}
$\overline{RD}^2=\disfrac{355}{113}\approx \pi$.

\textbf{הוכחה:}
לפי משפט פיתגורס ב-%
$\triangle QOT$:
\[
\overline{QT} = \sqrt{1^2-\left(\frac{2}{3}\right)^2}=\frac{\sqrt{5}}{3}\,.
\]
לפי הבניה
$\overline{QT}=\overline{RS}$
ולפי משפט פיתגורס ב-%
$\triangle PSR$:
\[
\overline{PS} = \sqrt{2^2-\left(\frac{\sqrt{5}}{3}\right)^2}=\sqrt{4-\frac{5}{9}}=\frac{\sqrt{31}}{3}\,.
\]
לפי הבניה 
$\overline{MO} \| \overline{RS}$
כך ש-%
$\triangle MPO\sim \triangle SPR$
ולכן:
\erh{12pt}
\begin{equationarray*}{rcl}
\disfrac{\overline{PM}}{\overline{PO}}&=&\disfrac{\overline{PS}}{\overline{PR}}\\
\disfrac{\overline{PM}}{1}&=&\disfrac{\sqrt{31}/3}{2}\\
\overline{PM}&=&\disfrac{\sqrt{31}}{6}\,.
\end{equationarray*}
לפי הבניה
$\overline{NT}\|\overline{RS}$
כך ש-%
$\triangle NPT\sim \triangle SPR$
ולכן:
\erh{16pt}
\begin{equationarray*}{rcl}
\disfrac{\overline{PN}}{\overline{PT}}&=&\disfrac{\overline{PS}}{\overline{PR}}\\
\disfrac{\overline{PN}}{5/3}&=&\disfrac{\sqrt{31}/3}{2}\\
\overline{PN}&=&\disfrac{5\sqrt{31}}{18}\\
\overline{MN}&=&\overline{PN}-\overline{PM}\\
&=&\sqrt{31}\left(\disfrac{5}{18}-\disfrac{1}{6}\right) = \disfrac{\sqrt{31}}{9}\,.
\end{equationarray*}
$\triangle PKR$
הוא משולש ישר-זווית כי הוא נשען על קוטר. לפי הבניה 
$\overline{PK}=\overline{PM}$
ולכן לפי משפט פיתגורס:
\[
\overline{RK}=\sqrt{2^2-\left(\frac{\sqrt{31}}{6}\right)^2} = \frac{\sqrt{113}}{6}\,.
\]
$\triangle PLR$
הוא משולש ישר-זווית כי 
$\overline{PL}$
הוא משיק. לפי הבניה
$\overline{PL}=\overline{MN}$
ולכן לפי משפט פיתגורס:
\[
\overline{RL}=\sqrt{2^2+\left(\frac{\sqrt{31}}{9}\right)^2} = \frac{\sqrt{355}}{9}\,.
\]

%\newpage

לפי הבניה 
$\overline{RC}=\overline{RH}=\frac{3}{2}$.
$\overline{CD}$
מקביל ל-%
$\overline{LK}$,
ולכן
$\triangle{LRK}\sim\triangle{DRC}$
ו:
\erh{16pt}
\begin{equationarray*}{rcl}
\disfrac{\overline{RD}}{\overline{RC}}&=&\disfrac{\overline{RL}}{\overline{RK}}\\
\disfrac{\overline{RD}}{3/2}&=&\disfrac{\sqrt{355}/9}{\sqrt{113}/6}\\
\overline{RD}&=&\sqrt{\disfrac{355}{113}}\,.
\end{equationarray*}
באיור~%
%\ref{f.ramanujan1a}
שלהלן אורכי קטעי הקו מסומנים:
%\begin{figure}
\begin{center}
\selectlanguage{english}
\begin{tikzpicture}[scale=1,align=left]
\clip (-6,-5.1) rectangle +(11.5,10.2);
% Draw circle and horizontal diameter
\draw[name path=circle] (0,0)  coordinate (o) node[below] {$O$} circle[radius=5cm];
\draw (-5,0) coordinate (p) node[left] {$P$} -- (5,0) coordinate (r) node[right] {$R$};
\fill (o) circle (1pt);
\fill (p) circle (1pt);
\fill (r) circle (1pt);
\fill (-2.5,0) coordinate (h) node[below] {$H$} circle (1pt);
\fill (10/3,0) coordinate (t) node[below] {$T$} circle (1pt);
\path (p) -- node[above,xshift=10pt] {$1/2$} (h) -- node[above] {$1/2$} (o) -- node[above] {$2/3$} (t) -- node[above] {$1/3$} (r);
% Draw chord RS and line PS
\path[name path=tq] (t) -- +(0,5);
\path[name intersections={of=tq and circle,by=q}];
\path[name path=rcirc] (r) let \p1 = ($ (t) - (q) $) in circle ({veclen(\x1,\y1)});
\path[name intersections={of=rcirc and circle,by=s}];
\draw (r) -- node[right] {$\sqrt{5}/3$} (s);
\fill (s) node[above right] {$S$} circle (1pt);
\draw[name path=ps] (p) -- node[above right,yshift=16pt] {$\sqrt{31}/3$} (s);
% Draw TN
\path[name path=tn] (t) -- +($(s)-(r)$);
\path[name intersections={of=ps and tn,by=n}];
\draw (t) -- (n);
\fill (n) node[above] {$N$} circle (1pt);
% Draw OM
\path[name path=om] (o) -- +($(s)-(r)$);
\path[name intersections={of=ps and om,by=m}];
\draw (o) -- (m);
\fill (m) node[above left] {$M$} circle (1pt);
\path (p) -- node[below,xshift=32pt,yshift=12pt] {$\sqrt{31}/6$} (m);
\path (m) -- node[below,xshift=5pt,yshift=-6pt] {$\sqrt{31}/9$} (n);
% Draw chord PK
\draw (p) -- node[right,xshift=-2pt,yshift=10pt] {$\sqrt{31}/6$} +(-62.3:4.64) coordinate (k) node[below left] {$K$};
% Draw tangent PL
\draw let \p1 = ($ (m) - (n) $), \n1 = {veclen(\x1,\y1)} in (p) -- node[left] {$\disfrac{\sqrt{31}}{9}$} (-5,-\n1) coordinate (l) node[left] {$L$};
% Connect L and K to R
\draw (r) -- (l) -- (k) -- cycle;
% Find point C on RK
\coordinate (c) at ($(r)!7.5cm!(k)$);
%\path (r) -- node[below,yshift=-16pt] {$RC=3/2$\\$RK=\sqrt{113}/6$} (c);
\path (r) -- node[below,xshift=16pt,yshift=-2pt] {$\overline{RC}=3/2$} (c);
\path (r) -- node[below,xshift=-4pt,yshift=-20pt] {$\overline{RK}=\sqrt{113}/6$} (k);
\fill (c) node[below] {$C$} circle (1pt);
% Draw CD
\path[name path=cd] (c) -- +($(l)-(k)$);
\path[name path=lr] (l) -- (r);
\path[name intersections={of=cd and lr,by=d}];
\draw (c) -- (d);
\fill (d) node[above,xshift=2pt] {$D$} circle (1pt);
%\path (r) -- node[above,xshift=-40pt,yshift=-8pt] {$RD=\sqrt{355/113}$\\$RL=\sqrt{355}/9$} (d);
\path (r) -- node[above,xshift=-40pt,yshift=-4pt] {$\overline{RD}=\sqrt{355/113}$} (d);
\path (r) -- node[below,xshift=-44pt,yshift=-20pt] {$\overline{RL}=\sqrt{355}/9$} (l);
\draw[rotate=-90] (p) rectangle +(8pt,8pt);
\draw[rotate=30] (k) rectangle +(8pt,8pt);
\draw (t) -- node[below right,xshift=-2pt,yshift=-4pt] {$\sqrt{5}/3$} (q) node[above] {$Q$};
\draw[dashed] (o) -- (q);
\fill (q) circle (1pt);
\end{tikzpicture}
\end{center}
%\caption{הבניה עם האורכים של קטעי הקו}\label{f.ramanujan1a}
%\end{figure}

\newpage
%%%%%%%%%%%%%%%%%%%%%%%%%%%%%%%%%%%%%%%%%%%%%%%%%%%%%%%%%%%

\section{הבניה השנייה של
\L{Ramanujan}}


\begin{itemize}
\item
בנו מעגל יחידה שמרכזו
$O$
עם קוטר
$\overline{AB}$,
וסמנו ב-%
$C$
את החיתוך של הניצב ב-%
$O$
עם המעגל.
\item
בנו את הנקודה
$T$
כך ש-%
$\overline{AT}=1/3$ 
ו-%
$\overline{TO}=2/3$.
\item
בנו
$\overline{BC}$
ומצאו נקודות
$M,N$
כך ש-%
$\overline{CM}=\overline{MN}=\overline{AT}=1/3$.
\item
בנו 
$\overline{AM}$
ו-%
$\overline{AN}$
וסמנו ב-%
$P$
את הנקודה על
$\overline{AN}$
כך ש-%
$\overline{AP}=\overline{AM}$.
\item
בנו קו המקביל ל-%
$\overline{MN}$
שעובר דרך
$P$,
וסמנו ב-%
$Q$
את נקודת החיתוך שלו עם
$\overline{AM}$.
\item
בנו
$\overline{OQ}$
ובנו קו המקביל ל-%
$\overline{OQ}$
שעובר דרך 
$T$,
וסמנו ב-%
$R$
את נקודת החיתוך שלו עם
$\overline{AM}$.
\item
בנו משיק
$\overline{AS}$
כך ש-%
$\overline{AS}=\overline{AR}$.
\item
בנו
$\overline{SO}$.
\end{itemize}



%\begin{figure}[bt]
\begin{center}
\selectlanguage{english}
\begin{tikzpicture}[scale=1.3]
\clip (-4.4,-4.2) rectangle +(8.8,8.6);
% Scale at 4

% Coordinates of circle
\coordinate (O) at (0,0);
\coordinate (A) at (-4,0);
\coordinate (B) at (4,0);
\coordinate (C) at (0,4);

% Draw circle and diameter
\node [thick,draw,circle through=(A),name path=circle] at (O) {};
\draw [thick] (A) -- (B);
\draw [thick,dashed] (C) -- (O);
\draw (O) rectangle +(8pt,8pt);
\draw[rotate=-90] (A) rectangle +(8pt,8pt);

\coordinate (T) at (-2.667,0);
\path (A) -- node[below] {$1/3$} (T);
\path (T) -- node[below] {$2/3$} (O);
\path (O) -- node[below] {$1$} (B);

\draw (C) -- node[right] {$1/3$} +(-45:1.333) coordinate (M);
\draw (M) -- node[right] {$1/3$} +(-45:1.333) coordinate (N);
\draw (N) -- node[left] {$\sqrt{2}\!-\!2/3$}(B);

\draw[name path=AM] (A) -- (M);
\draw[name path=AN] (A) -- (N);

\node [circle through=(M),name path=AMcircle] at (A) {};

\path[name intersections={of=AMcircle and AN,by=P}];

\path[name path=PQ] (P) -- +(135:2);
\path[name intersections={of=PQ and AM,by=Q}];
\draw (P) -- (Q) -- (O);

\path[name path=QT] (T) -- ($(Q)+(-2.667,0)$) -- (Q);
\path[name intersections={of=QT and AM,by=R}];
\draw (T) -- (R);

\node [circle through=(R),name path=ARcircle] at (A) {};
\path[name path=AS] (A) -- ($(A)+(0,-2.5)$);
\path[name intersections={of=ARcircle and AS,by=S}];
\draw (A) -- (S);

\draw (S) -- (O);

\fill (O) circle(1.2pt) node[below right] {$O$};
\fill (A) circle(1.2pt) node[left] {$A$};
\fill (B) circle(1.2pt) node[right] {$B$} node[above left,xshift=-8pt] {$45^\circ$};
\fill (C) circle(1.2pt) node[above] {$C$};
\fill (T) circle(1.2pt) node[below] {$T$};
\fill (M) circle(1.2pt) node[right] {$M$};
\fill (N) circle(1.2pt) node[right] {$N$};
\fill (P) circle(1.2pt) node[below] {$P$};
\fill (Q) circle(1.2pt) node[above left] {$Q$};
\fill (R) circle(1.2pt) node[above left] {$R$};
\fill (S) circle(1.2pt) node[above left] {$S$};

\end{tikzpicture}
\end{center}
%\caption{הבניה של \L{Ramanujan} ל-%
%$\left(9^2+\disfrac{19^2}{22}\right)^{1/4}$
%}\label{f.ramanujan2}
%\end{figure}

\newpage

\textbf{טענה:}
$3\sqrt{\overline{SO}}=\left(9^2+\disfrac{19^2}{22}\right)^{1/4}\approx \pi$.

\textbf{הוכחה:}
$\triangle COB$
הוא משולש ישר-זווית ו-%
$\overline{OB}=\overline{OC}=1$.
לפי משפט פיתגורס
$\overline{CB}=\sqrt{2}$
ו-%
$\overline{NB}=\sqrt{2}-2/3$.
$\triangle COB$
הוא המשולש שווה-שוקיים ולכן
$\angle NBA =\angle MBA=45^\circ$.
נשתמש במשפט הקוסינוסים על 
$\triangle NBA$
כדי לחשב את
$\overline{AN}$:
\erh{16pt}
\begin{equationarray*}{rcl}
\overline{AN}^2&=&\overline{AB}^2 + \overline{BN}^2-2\cdot\overline{AB}\cdot\overline{BN}\cdot\cos \angle NBA\\
&=&2^2+\left(\sqrt{2}-\disfrac{2}{3}\right)^2-2\cdot 2 \cdot \left(\sqrt{2}-\disfrac{2}{3}\right)\cdot \disfrac{\sqrt{2}}{2}\\
&=&4+2-\disfrac{4\sqrt{2}}{3}+\disfrac{4}{9} - 4 + \disfrac{4\sqrt{2}}{3}=\disfrac{22}{9}\\
\overline{AN}&=&\sqrt{\disfrac{22}{9}}\,.
\end{equationarray*}
באופן דומה, נשתמש במשפט הקוסינוסים על
$\triangle MBA$
כדי לחשב את
$\overline{AM}$:
\begin{equationarray*}{rcl}
\overline{AM}^2&=&\overline{AB}^2 + \overline{BM}^2-2\cdot\overline{AB}\cdot\overline{BM}\cdot\cos \angle MBA\\
&=&2^2+\left(\sqrt{2}-\disfrac{1}{3}\right)^2-2\cdot 2 \cdot \left(\sqrt{2}-\disfrac{1}{3}\right)\cdot \disfrac{\sqrt{2}}{2}\\
&=&4+2-\disfrac{2\sqrt{2}}{3}+\disfrac{1}{9} - 4 + \disfrac{2\sqrt{2}}{3}=\disfrac{19}{9}\\
\overline{AM}&=&\sqrt{\disfrac{19}{9}}\,.
\end{equationarray*}
לפי הבנייה
$\overline{QP}\parallel \overline{MN}$
כך ש-%
$\triangle MAN\sim \triangle QAP$,
ולפי הבניה
$\overline{AP}=\overline{AM}$,
ולכן:
\erh{12pt}
\begin{equationarray*}{rcl}
\disfrac{\overline{AQ}}{\overline{AM}}&=&\disfrac{\overline{AP}}{\overline{AN}}=\disfrac{\overline{AM}}{\overline{AN}}\\
\overline{AQ}&=&\disfrac{\overline{AM}^2}{\overline{AN}}=\disfrac{19/9}{\sqrt{22/9}}=\disfrac{19}{3\sqrt{22}}\,.
\end{equationarray*}
לפי הבנייה
$\overline{TR}\parallel \overline{OQ}$
ולכן
$\triangle RAT\sim \triangle QAO$
כך ש:
\erh{12pt}
\begin{equationarray*}{rcl}
\disfrac{\overline{AR}}{\overline{AQ}}&=&\disfrac{\overline{AT}}{\overline{AO}}\\
\overline{AR}&=&\overline{AQ}\cdot\disfrac{\overline{AT}}{\overline{AO}}\\
&=&\disfrac{19}{3\sqrt{22}}\cdot\disfrac{1/3}{1}=\disfrac{19}{9\sqrt{22}}\,.
\end{equationarray*}

לפי הבנייה
$\overline{AS}=\overline{AR}$
ו-%
$\triangle OAS$ 
הוא משולש ישר-זווית כי 
$\overline{AS}$
הוא משיק. לפי משפט פיתגורס:
\erh{12pt}
\begin{equationarray*}{rcl}
\overline{SO}&=&\sqrt{1^2+\left(\disfrac{19}{9\sqrt{22}}\right)^2}\\
3\sqrt{\overline{SO}}&=&3\left(1+\disfrac{19^2}{9^2\cdot 22}\right)^\frac{1}{4}\\
&=&\left(3^4+\disfrac{3^4\cdot 19^2}{9^2\cdot 22}\right)^\frac{1}{4}\\
&=&\left(9^2+\disfrac{19^2}{22}\right)^\frac{1}{4}\\
&\approx& 3.14159265262\approx \pi\,.
\end{equationarray*}
באיור שלהן אורכי קטעי הקו מסומנים:
%\begin{figure}
\begin{center}
\selectlanguage{english}
\begin{tikzpicture}[scale=1.3]
\clip (-5.2,-4.2) rectangle +(10,8.7);
% Scale at 4

% Coordinates of circle
\coordinate (O) at (0,0);
\coordinate (A) at (-4,0);
\coordinate (B) at (4,0);
\coordinate (C) at (0,4);

% Draw circle and diameter
\node [thick,draw,circle through=(A),name path=circle] at (O) {};
\draw [thick] (A) -- (B);
\draw [thick,dashed] (C) -- (O);
\draw (O) rectangle +(8pt,8pt);
\draw[rotate=-90] (A) rectangle +(8pt,8pt);

\coordinate (T) at (-2.667,0);
\path (A) -- node[below] {$1/3$} (T);
\path (T) -- node[below] {$2/3$} (O);
\path (O) -- node[below] {$1$} (B);

\draw (C) -- node[right] {$1/3$} +(-45:1.333) coordinate (M);
\draw (M) -- node[right] {$1/3$} +(-45:1.333) coordinate (N);
\draw (N) -- node[left] {$\sqrt{2}\!-\!2/3$}(B);

\draw[name path=AM] (A) -- node[above,xshift=15pt,yshift=24pt] {$\overline{AM}=\sqrt{19/9}$} (M);
\draw[name path=AN] (A) -- node[below,yshift=-10pt] {$\overline{AN}=\sqrt{22/9}$} (N);

\node [circle through=(M),name path=AMcircle] at (A) {};

\path[name intersections={of=AMcircle and AN,by=P}];

\path[name path=PQ] (P) -- +(135:2);
\path[name intersections={of=PQ and AM,by=Q}];
\draw (P) -- (Q) -- (O);
\path (A) -- node[above,xshift=-14pt,yshift=10pt] {$\overline{AQ}=19/3\sqrt{22}$} (Q);

\path[name path=QT] (T) -- ($(Q)+(-2.667,0)$) -- (Q);
\path[name intersections={of=QT and AM,by=R}];
\draw (T) -- (R);
\path (A) -- node[above,xshift=-31pt,yshift=6pt] {$\overline{AR}=19/9\sqrt{22}$} (R);

\node [circle through=(R),name path=ARcircle] at (A) {};
\path[name path=AS] (A) -- ($(A)+(0,-2.5)$);
\path[name intersections={of=ARcircle and AS,by=S}];
\draw (A) -- node[xshift=10pt,yshift=6pt] {$\overline{AS}=\,19/9\sqrt{22}$} (S);

\draw (S) -- node[right,xshift=-2pt,yshift=-10pt] {$\overline{SO}=\sqrt{1+
  \left(\disfrac{19^2}{9^2\cdot 22}\right)}$} (O);

\fill (O) circle(1.2pt) node[below right] {$O$};
\fill (A) circle(1.2pt) node[left] {$A$};
\fill (B) circle(1.2pt) node[right] {$B$} node[above left,xshift=-8pt] {$45^\circ$};
\fill (C) circle(1.2pt) node[above] {$C$};
\fill (T) circle(1.2pt) node[below] {$T$};
\fill (M) circle(1.2pt) node[right] {$M$};
\fill (N) circle(1.2pt) node[right] {$N$};
\fill (P) circle(1.2pt) node[below] {$P$};
\fill (Q) circle(1.2pt) node[above left] {$Q$};
\fill (R) circle(1.2pt) node[above left] {$R$};
\fill (S) circle(1.2pt) node[above left] {$S$};

\end{tikzpicture}
\end{center}
%\caption{אורכי הקווים בבניה  \L{Ramanujan}}\label{f.ram}
%\end{figure}

\subsection*{מקורות}

הבניה של
\L{Kochansky}
לקוחה מ-%
\L{\cite{bold}},
והבניות של 
\L{Ramanujan} 
לקוחות מ-%
\L{\cite{ramanujan1,ramanujan2}}.
