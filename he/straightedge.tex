% !TeX root = surprises.tex

\chapter{אפשר להסתפק בסרגל ביחד עם מעגל אחד}\label{c.straightedge}

%%%%%%%%%%%%%%%%%%%%%%%%%%%%%%%%%%%%%%%%%%%%%%%%%%%%%%%%%%%%%%%

האם כל בנייה עם סרגל ומחוגה ניתנת לבנייה עם סרגל בלבד? התשובה היא שלילית כי קווים הוגדרים על ידי משוואות ליניאריות ולא יכולים להגדיר מעגלים שמשוואותיהם ריבועיות. ב-%
$1822$
\L{Jean-Victor Poncelet}
שיער שכן ניתן להסתפק בסרגל בלבד בתנאי שקיים במישור מעגל אחד בלבד. המשפט הוכח ב-%
$1833$
על ידי
\L{Jakob Steiner}.

לאחר שנסביר בסעיף%
~\ref{s.se-what}
מה המשמעות של בנייה רק עם סרגל ומעגל אחד, ההוכחה מוצגת בשלבים, תחילה עם חמש בניות עזר: בניית קו המקביל לקו נתון (סעיף%
~\ref{s.parallel}),
בניית ניצב לקו נתון (סעיף%
~\ref{s.perpendicular}),
העתקת קטע קו בכיוון נתון (סעיף%
~\ref{s.direction}), 
בניית קטע קו כיחס בין קטעים אחרים (סעיף%
~\ref{s.relative-straight})
ובניית שורש ריבועי (סעיף%
~\ref{s.root}).
סעיף%
~\ref{s.line-circle-straight}
מראה איך למצוא את החיתוכים של קו ומעגל וסעיף%
~\ref{s.circle-circle}
מראה איך למצוא את החיתוכים של שני מעגלים.

%%%%%%%%%%%%%%%%%%%%%%%%%%%%%%%%%%%%%%%%%%%%%%%%%%%%%%%%%%%%%%%

\section{מהי בנייה עם סרגל בלבד?}\label{s.se-what}

כל צעד בבנייה עם סרגל ומחוגה הוא אחת משלושת הפעולות הללו:
\begin{itemize}
\setlength{\itemsep}{0pt}
\item
מציאת נקודת החיתוך של שני קווים.
\item
מציאת נקודות החיתוך של קו עם מעגל.
\item
מציאת נקודות החיתוך של שני מעגלים.
\end{itemize}
ניתן לבצע את הפעולה הראשונה עם סרגל בלבד.

מעגל מוגדר על ידי נקודה
$O$,
מרכזו, ועל ידי קטע קו באורך
$r$,
הרדיוס, שאחת מהנקודות הקצה שלו היא
$O$.
אם נצליח לבנות את הנקודות
$X,Y$
המסומנות ב-%
\ref{f.se-only-line-circle},
נוכל לטעון שהצלחנו לבנות את נקודות החיתוך של מעגל נתון עם קו נתון. באופן דומה, הבנייה של
$X,Y$
ב-%
\ref{f.se-only-two-circles},
היא בניית נקודות החיתוך של שני מעגלים נתונים. המעגלים המצויירים בקווים מקווקווים לא מופיעים בבנייה והם רק עוזרים להבנתה.

המעגל היחיד בבנייה ייקרא המעגל הקבוע ויכול להופיע בכל מקום במישור עם רדיוס שרירותי.

\begin{figure}[tb]
\begin{center}
\begin{subfigure}{.4\textwidth}
\begin{tikzpicture}[scale=.9]
\fill (0,0) node[above right] {$O$} circle[radius=1.5pt];
\draw[thick,dashed,name path=circle] (0,0) circle[radius=2cm];
\draw (0,0) -- node[left] {$r$} ++(-60:2cm);
\fill (0,0) ++(-60:2cm) circle[radius=1.5pt];
\draw[name path=line] (-3,-.5) -- ++(20:6cm);
\path [name intersections={of=circle and line,by={X,Y}}];
\fill (X) node[above right,xshift=-2pt,yshift=4pt] {$X$} circle[radius=1.5pt];
\fill (Y) node[above left] {$Y$} circle[radius=1.5pt];
\end{tikzpicture}
\selectlanguage{hebrew}
\caption{$X$, $Y$
הם נקודות החיתוך של קו ומעגל}
\label{f.se-only-line-circle}
\end{subfigure}
\hspace{3em}
\begin{subfigure}{.4\textwidth}
\begin{tikzpicture}[scale=.9]
\fill (0,0) node[above right] {$O_1$} circle[radius=1.5pt];
\fill (3,0) node[above right] {$O_2$} circle[radius=1.5pt];
\draw[thick,dashed,name path=circle1] (0,0) circle[radius=2cm];
\draw[thick,dashed,name path=circle2] (3,0) circle[radius=2cm];
\draw (0,0) -- node[left] {$r_1$} ++(-70:2cm);
\draw (3,0) -- node[left,below] {$r_2$} ++(-20:2cm);
\fill (0,0) ++(-70:2cm) circle[radius=1.5pt];
\fill (3,0) ++(-20:2cm) circle[radius=1.5pt];
\path [name intersections={of=circle1 and circle2,by={X,Y}}];
\fill (X) node[above,yshift=4pt] {$X$} circle[radius=1.5pt];
\fill (Y) node[below,yshift=-4pt] {$Y$} circle[radius=1.5pt];
\end{tikzpicture}
\selectlanguage{hebrew}
\caption{$X$, $Y$
הם נקודות החיתוך של שני מעגלים}
\label{f.se-only-two-circles}
\end{subfigure}
\end{center}
\end{figure}

%%%%%%%%%%%%%%%%%%%%%%%%%%%%%%%%%%%%%%%%%%%%%%%%%%%%%%%%%%%%%%%

\section{בניית קו המקביל לקו נתון}\label{s.parallel}

\begin{theorem}\label{thm.parallel-line}
נתון קו
$l$
העובר דרך שתי נקודות
$A,B$,
ונתונה נקודה 
$P$
שאיננה על הקו, ניתן לבנות קו דרך
$P$
המקביל ל-%
$\overline{AB}$.
\end{theorem}
\begin{proof}
ההוכחה היא עבור שני מקרים בנפרד.

\textbf{מקרה ראשון:}
$\overline{AB}$
נקרא קו מכוון אם נתונה
$M$,
נקודת האמצע של הקו. נבנה קרן הממשיכה את
$\overline{AP}$,
ונבחר
$S$,
נקודה כלשהי על הקרן מעבר ל-%
$P$.
נבנה את הקווים
$\overline{SB}$, $\overline{SM}$, $\overline{BP}$.
נסמן ב-%
$O$
את נקודת החיתוך של 
$\overline{BP}$
עם
$\overline{SM}$.
נבנה קרן הממשיכה את
$\overline{AO}$
ונסמן ב-%
$Q$
את החיתוך של הקרן עם
$\overline{SB}$
(איור%
~\ref{f.se-parallel-directed}).
טענה:
$\overline{PQ}\|\overline{AB}$.

\begin{figure}[tb]
\begin{center}
\begin{tikzpicture}
\draw[name path=pq] (-4,0) -- (4,0);
\draw (-2,-2) node[below left] {$A$} coordinate (A) -- (2,-2) node[below right] {$B$} coordinate (B);
\fill (A) circle[radius=1.5pt];
\fill (B) circle[radius=1.5pt];
\draw[name path=as] (A) -- ++(50:4cm) node[above] {$S$} coordinate (S);
\fill (S) circle[radius=1.5pt];
\draw[name path=sb] (S) -- (B);
\path [name intersections={of=pq and as,by={P}}];
\path [name intersections={of=pq and sb,by={Q}}];
\fill (P) node[above left] {$P$} circle[radius=1.5pt];
\fill (Q) node[above right] {$Q$} circle[radius=1.5pt];
\draw[name path=pb] (P) -- (B);
\draw[name path=qa] (Q) -- (A);
\path [name intersections={of=pb and qa,by={O}}];
\fill (O) node[right,xshift=2pt] {$O$} circle[radius=1.5pt];
\fill (0,-2) coordinate (M) node[below right] {$M$} circle[radius=1.5pt];
\draw (S) -- (M);
\end{tikzpicture}
\selectlanguage{hebrew}
\caption{בניית קו מקביל לקו מכוון}\label{f.se-parallel-directed}
\end{center}
\end{figure}
הוכחת הטענה משתמשת במשפט 
\L{Ceva}
(משפט%
~\ref{thm.ceva}): 
אם קטעי הקו מקודקודי משולש לצלעות הנגדיות שנפגשים בנקודה
$M$
(כמו באיור%
~\ref{f.se-parallel-directed}),
האורכים של הקטעי הצלעות מקיימים את היחס:
\[
\frac{\overline{AM}}{\overline{MB}}\cdot\frac{\overline{BQ}}{\overline{QS}}\cdot\frac{\overline{SP}}{\overline{PA}} = 1\,.
\]
באיור%
~\ref{f.se-parallel-directed}
$M$
היא נקודת האמצע של
$\overline{AB}$
ולכן
$\disfrac{\overline{AM}}{\overline{MB}}=1$,
ומכאן ש:
\begin{equation}
\frac{\overline{BQ}}{\overline{QS}}=\frac{\overline{PA}}{\overline{SP}}=\frac{\overline{AP}}{\overline{PS}}\,,\label{eq.ceva}
\end{equation}
כי סדר נקודות הקצה של קטעי הקו אינו חשוב.

נוכיח ש-%
$\triangle ABS\sim\triangle PQS$:
\begin{eqn}
\disfrac{\overline{BS}}{\overline{QS}}&=&\disfrac{\overline{BQ}}{\overline{QS}}+\disfrac{\overline{QS}}{\overline{QS}} = \disfrac{\overline{BQ}}{\overline{QS}}+1\\
\disfrac{\overline{AS}}{\overline{PS}} &=& \disfrac{\overline{AP}}{\overline{PS}} + \disfrac{\overline{PS}}{\overline{PS}} = \disfrac{\overline{AP}}{\overline{PS}} + 1\,.
\end{eqn}
לפי משוואה%
~\ref{eq.ceva}:
\[
\disfrac{\overline{BS}}{\overline{QS}}=\disfrac{\overline{BQ}}{\overline{QS}}+1=\disfrac{\overline{AP}}{\overline{PS}} + 1=\disfrac{\overline{AS}}{\overline{PS}}\,,
\]
ולכן 
$\triangle ABS\sim\triangle PQS$
ו-%
$\overline{PQ}\|\overline{AB}$.

%%%%%%%%%%%%%%%%%%%%%%%%%%%%%%%%%%%%%%%%%%%%%%%%%%%%%%%%%%%%%%%

\textbf{מקרה שני:}
$\overline{AB}$
אינו בהכרח קו מכוון. למגעל הקבוע 
$c$
מרכז 
$O$
ורדיוס
$r$.
$P$
היא נקודה שאיננה נמצאת כל הקן דרכה יש לבנות קו המקביל ל-%
$l$
(\ref{f.se-parallel-other1}).

נבחר
$M$,
נקודה שרירותית על
$l$
ונבנה קרן 
$\overline{MO}$
שחותך את המעגל הקבוע ב-%
$U,V$.
קו זה הוא קו מכוון כי 
$O$,
מרכז המעגל, חוצה את הקוטר
$\overline{UV}$.
נבחר נקודה
$A$
על 
$l$
ולפי הבנייה עבור קו מכוון, ניתן לבנות קו המקביל ל-%
$\overline{UV}$
דרך 
$A$.
שחותך את המעגל ב-%
$X,Y$
(\ref{f.se-parallel-other2}).
\begin{figure}[tb]
\begin{center}
\begin{subfigure}{.45\textwidth}
\begin{tikzpicture}[scale=.8]
\coordinate (O) at (0,0);
\fill (O) node[below right] {$O$} circle[radius=1.5pt];
\draw[name path=circle] (O) circle[radius=2cm];
\draw[name path=l] (-4,-3) -- node[above, near end] {$l$} +(9,0);
\path[name path=mo] (-2,-3) coordinate (M) -- ($(-2,-3)!1.65!(O)$);
\fill (M) node[below] {$M$} circle[radius=1.5pt];
\path [name intersections={of=circle and mo,by={V,U}}];
\fill (U) node[below,xshift=2pt,yshift=-4pt] {$U$} circle[radius=1.5pt];
\fill (V) node[right,xshift=4pt] {$V$} circle[radius=1.5pt];
\draw (M) --   (U) -- node[above] {$r$} (O) -- node[above] {$r$} (V);
%\node at (-1.6,1.6) {$c$};
\fill (-4,1) node[above left] {$P$} circle[radius=1.5pt];
\end{tikzpicture}
\selectlanguage{hebrew}
\caption{בניית קו מכוון}\label{f.se-parallel-other1}
\end{subfigure}
\begin{subfigure}{.45\textwidth}
\begin{tikzpicture}[scale=.9]
\coordinate (O) at (0,0);
\fill (O) node[below right] {$O$} circle[radius=1.5pt];
\draw[name path=circle] (O) circle[radius=2cm];
\draw[name path=l] (-4,-3) -- node[above,near end,xshift=24pt] {$l$} +(9,0);
\path[name path=mo] (-2,-3) coordinate (M) -- ($(-2,-3)!1.65!(O)$);
\fill (M) node[below] {$M$} circle[radius=1.5pt];
\path [name intersections={of=circle and mo,by={V,U}}];
\fill (U) node[below,xshift=2pt,yshift=-4pt] {$U$} circle[radius=1.5pt];
\fill (V) node[right,xshift=4pt] {$V$} circle[radius=1.5pt];
\draw (M) -- (V);
\path[name path=ax] (-3,-3) coordinate (A) -- ($(-3,-3)!1.8!(-1,0)$);
\fill (A) node[below] {$A$} circle[radius=1.5pt];
\path [name intersections={of=circle and ax,by={Y,X}}];
\fill (X) node[left] {$X$} circle[radius=1.5pt];
\fill (Y) node[above] {$Y$} circle[radius=1.5pt];
%\node at (-1.6,1.6) {$c$};
\draw (A) -- (Y);
\fill (-4,1) node[above left] {$P$} circle[radius=1.5pt];
\end{tikzpicture}
\selectlanguage{hebrew}
\caption{בניית קו מקביל לקו מכוון}\label{f.se-parallel-other2}
\end{subfigure}
\end{center}
\end{figure}
נבנה קוטר
$\overline{XX'}$
דרך 
$X$
במעגל 
$O$
שחותך את הצד השני של המעגל ב-%
$X'$,
ובאופן דומה נבנה קוטר
$\overline{YY'}$.
נבנה קרן מ-%
$X'$
שעבור דרך
$Y'$
ונסמן ב-%
$B$
את נקודת החיתוך שלה עם 
$l$
(איור%
~\ref{f.se-parallel-other3}).

\begin{figure}[tb]
\begin{center}
\begin{tikzpicture}[scale=.9]
\coordinate (O) at (0,0);
\fill (O) node[below right] {$O$} circle[radius=1.5pt];
\draw[name path=circle] (O) circle[radius=2cm];
\draw[name path=l] (-4,-3) -- node[above,near end,xshift=24pt] {$l$} +(9,0);
\path[name path=mo] (-2,-3) coordinate (M) -- ($(-2,-3)!1.65!(O)$);
\fill (M) node[below] {$M$} circle[radius=1.5pt];
\path [name intersections={of=circle and mo,by={V,U}}];
\fill (U) node[below,xshift=2pt,yshift=-4pt] {$U$} circle[radius=1.5pt];
\fill (V) node[right,xshift=4pt] {$V$} circle[radius=1.5pt];
\draw (M) -- (V);
\path[name path=ax] (-3,-3) coordinate (A) -- ($(-3,-3)!1.8!(-1,0)$);
\fill (A) node[below] {$A$} circle[radius=1.5pt];
\path [name intersections={of=circle and ax,by={Y,X}}];
\fill (X) node[left] {$X$} circle[radius=1.5pt];
\fill (Y) node[above] {$Y$} circle[radius=1.5pt];
%\node at (-1.6,1.6) {$c$};
\draw (A) -- (Y);
\fill (-4,1) node[above left] {$P$} circle[radius=1.5pt];
\path[name path=xo] (X) -- ($(X)!2.2!(O)$);
\path[name intersections={of=circle and xo,by={Xp}}];
\fill (Xp) node[right,xshift=2pt,yshift=-2pt] {$X'$} circle[radius=1.5pt];
\draw (X) -- (Xp);
\path[name path=yo] (Y) -- ($(Y)!2.4!(O)$);
\path[name intersections={of=circle and yo,by={Yp,y}}];
\fill (Yp) node[below right] {$Y'$} circle[radius=1.5pt];
\draw (Y) -- (Yp);
\path[name path=xy] (Xp) -- ($(Xp)!1.6!(Yp)$);
\path[name intersections={of=l and xy,by={B}}];
\fill (B) node[below] {$B$} circle[radius=1.5pt];
\draw (Xp) -- (B);
\draw[thick,dashed,name path=z] (-4,0) -- (4,0) node[above,near end,xshift=40pt] {$l'$};
\path[name intersections={of=ax and z,by={Z}}];
\path[name intersections={of=xy and z,by={Zp}}];
\fill (Z) node[above left] {$Z$} circle[radius=1.5pt];
\fill (Zp) node[below right] {$Z'$} circle[radius=1.5pt];
\end{tikzpicture}
\selectlanguage{hebrew}
\caption{הוכחה ש-%
$l'$
מקביל ל-%
$l$}\label{f.se-parallel-other3}
\end{center}
\end{figure}
$\overline{OX},\overline{OX'},\overline{OY},\overline{OY'}$
הם כולם רדיוסים של המעגל ו-%
$\angle XOY = \angle X'OY'$
כי הן זוויות קודקודיות. לכן
$\triangle XOY\cong \triangle X'OY'$
לפי צלע-זווית-צלע. נגדיר%
\footnote{%
נגדיר, לא נבנה, כי אנו באמצע הוכחה שאותו קו הוא בן-בנייה.}
קו
$l'$
כקן המקביל ל-%
$l$
דרך
$O$
שחותך את 
$\overline{XY}$
ב-%
$Z$
ושחותך את 
$X',Y'$
ב-%
$Z'$.
$\angle XOZ=\angle X'OZ'$
כי הן זוויות קודקודיות, ולכן
$\triangle XOZ\cong \triangle X'OZ'$
לפי זווית-צלע-זווית. מכאן ש-%
$\overline{ZO}=\overline{OZ'}$.
המרובעים
$AMOZ$
ו-%
$BMOZ'$
הם מקביליות ולכן
$\overline{AM}=\overline{ZO}=\overline{OZ'}=\overline{MB}$.
\end{proof}

\begin{theorem}
נתון קטע קו
$\overline{AB}$
ונקודה 
$P$
שאיננה נמצאת על הקו, ניתן לבנות קטע קו
$\overline{PQ}$
מקביל ל-%
$\overline{AB}$
שאורכו שווה לאורכו של
$\overline{AB}$.
במילים אחרות, ניתן להעתיק 
$\overline{AB}$
מקביל לעצמו כאשר 
$P$
היא אחת מנקודות הקצה שלה.
\end{theorem}

\begin{proof}
הוכחנו שניתן לבנות קו
$m$
מקביל ל-%
$\overline{AB}$
דרך 
$P$,
וגם קו
$n$
מקביל ל-%
$\overline{AP}$
דרך
$B$.
המרובע
$\overline{ABQP}$
הוא מקבילית שצלעות הנגדיות שלה שוות 
$\overline{AB}=\overline{PQ}$
(איור%
~\ref{f.se-parallel-other4}).
\end{proof}

\begin{figure}[tb]
\begin{center}
\begin{tikzpicture}[scale=.7]
\coordinate (A) at (0,0);
\coordinate (B) at (3,0);
\coordinate (P) at (-2,2.5);
\coordinate (Q) at (1,2.5);
\draw ($(P)!-.6!(Q)$) -- node[above,near end,xshift=36pt] {$m$} ($(P)!2.2!(Q)$);
\fill (P) node[above] {$P$} circle[radius=1.5pt];
\fill (Q) node[above right] {$Q$} circle[radius=1.5pt];
\draw ($(A)!-.6!(B)$) -- node[above,near end,xshift=40pt] {$l$} ($(A)!2.5!(B)$);
\fill (A) node[below] {$A$} circle[radius=1.5pt];
\fill (B) node[below left] {$B$} circle[radius=1.5pt];
\draw (A) -- (P);
\draw ($(B)!-.3!(Q)$) -- node[above,near end,xshift=24pt,yshift=-24pt] {$n$} ($(B)!1.4!(Q)$);
\end{tikzpicture}
\selectlanguage{hebrew}
\caption{בניית העתק של קו מקביל לקו קיים}\label{f.se-parallel-other4}
\end{center}
\end{figure}

%%%%%%%%%%%%%%%%%%%%%%%%%%%%%%%%%%%%%%%%%%%%%%%%%%%%%%%%%%%%%%%

\section{בניית אנך לקו נתון}\label{s.perpendicular}

\begin{theorem}\label{thm.perpendicular}
נתון קו
$l$
ונקודה
$P$
שאיננה על הקו ניתן לבנות אנך ל-%
$l$
דרך
$P$.%
\end{theorem}

\begin{proof}
לפי משפט
\ref{thm.parallel-line}
נבנה קו
$l$
מקביל ל-%
$l'$
שחותך את המעגל הקבוע ב-%
$U,V$
(איור%
~\ref{f.se-perp}).
נבנה את הקוטר
$\overline{UOU'}$
והמיתר
$\overline{U'V}$.
$\angle UVU'$
היא זווית ישרה כי היא נשענת על קוטר. מכאן ש-%
$\overline{VU'}$
ניצב ל-%
$\overline{UV}$
ול-%
$l$.
שוב לפי משפט%
~\ref{thm.parallel-line}
נבנה קו מקביל ל-%
$\overline{VU'}$
דרף
$P$.
\end{proof}

\begin{figure}[tb]
\begin{center}
\begin{tikzpicture}[scale=.7]
\coordinate (O) at (0,0);
\coordinate (P) at (3.5,.6);
\draw[name path=circle] (O) circle[radius=2cm];
\draw[name path=l] (-4,-3) -- node[above,near end,xshift=45pt] {$l$} ++(9,0);
\draw[name path=lp] (-3,-1) -- node[above,near end,xshift=40pt] {$l'$} ++(8,0);
\node[above left] at (O) {$O$};
\node[right] at (P) {$P$};
\path[name intersections={of=circle and lp,by={U,V}}];
\node[below left] at (U) {$U$};
\node[below right] at (V) {$V$};
\path[name path=d] (U) -- ($(U)!2.3!(O)$);
\path[name intersections={of=circle and d,by={Up}}];
\draw (U) -- (Up);
\node[above right] at (Up) {$U'$};
\draw (Up) -- (V);
\path[name path=p] (P) -- ++(0,-4);
\path[name intersections={of=p and l,by={X}}];
\draw (X) rectangle +(9pt,9pt);
\draw[rotate=90] (V) rectangle +(9pt,9pt);
\vertex{O};
\vertex{P};
\draw (P) -- ++(0,1);
\draw (P) -- (X);
\end{tikzpicture}
\end{center}
\caption{בניית ניצב}\label{f.se-perp}
\end{figure}

%%%%%%%%%%%%%%%%%%%%%%%%%%%%%%%%%%%%%%%%%%%%%%%%%%%%%%%%%%%%%%%

\section{העתקת קטע קו נתון בכיוון נתון}\label{s.direction}

\begin{theorem}\label{thm.angle}
נתון קטע קו ניתן לבנות עותק שלו בכיוון של קו אחר.
\end{theorem}
המשמעות של "כיוון" היא שקו שמוגדר על ידי שתי נקודות
$A',H'$
נמצא בזווית
$\theta$
יחסית לציר כלשהו והמטרה היא לבנות
$\overline{AS}=\overline{PQ}$
כך של-%
$\overline{AS}$
יהיה אותו זווית
$\theta$
יחסיות לציר (איור%
~\ref{f.se-copy1}).

%
%בסוף סעיף
%\ref{s.parallel}
%הראינו שאפשר להעתיק קטע קו מקביל לעצמו. כאן המטרה היא להעתיק את קטע הקו
%$\overline{PQ}$
%ל-%
%$\overline{AS}$,
%כך ש-%
%$\overline{AS}$
%יהיה באותה זווית
%$\theta$
%יחסית לאותו ציר. באיור
%$\overline{PQ}$
%נמצא על ציר ה-%
%$x$
%אבל אין לזה חשיבות
%(איור%
%~\ref{f.se-copy1}).


\begin{proof}
לפי משפט%
~\ref{thm.parallel-line}
ניתן לבנות קטע קו
$\overline{AH}\|\overline{A'H'}$,
וקטע קו
$\overline{AK}$
כך ש-%
$\overline{AK}\|\overline{PQ}$.
$\angle HAK=\theta$
ולכן מה שנשאר הוא למצוא נקודה
$S$
על
$\overline{AH}$
כך ש-%
$\overline{AS}=\overline{PQ}$.
\begin{figure}[tb]
\begin{center}
\begin{tikzpicture}[scale=.7]
\coordinate (A) at (0,0);
\coordinate (P) at (3cm,2);
\coordinate (Q) at (4.5cm,2);
\draw (P) -- (Q);
\node[left] at (P) {$P$};
\node[right] at (Q) {$Q$};
\coordinate (A1) at (-3,1);
\draw (A1) -- ++(60:3cm) coordinate (H1);
\draw (A1) -- ++(0:2cm);
\node[left] at (A1) {$A'$};
\node[left] at (H1) {$H'$};
\draw (A) -- ++(60:1.5cm) coordinate (S);
\node[left] at (S) {$S$};
\draw (A) -- ++(1.5,0);
\node[left] at (A) {$A$};
\node[above right,xshift=4pt] at (A1) {$\theta$};
\node[above right,xshift=4pt] at (A) {$\theta$};
\draw (A) -- ++(60:3cm) coordinate (H);
\node[left] at (H) {$H$};
\draw (A) -- ++(1.5,0) coordinate (K);
\node[right] at (K) {$K$};
\vertex{P};
\vertex{Q};
\vertex{A};
\vertex{S};
\end{tikzpicture}
\end{center}
\selectlanguage{hebrew}
\caption{העתקת קו בכיוון נתון}\label{f.se-copy1}
\end{figure}
נבנה שני רדיוסים במעגל הקבוע 
$\overline{OU},\overline{OV}$
שמקביליים ל-%
$\overline{AH},\overline{AK}$,
בהתאמה, ונבנה קרן דרך
$K$
המקבילה ל-%
$\overline{UV}$.
נסמן את נקודת החיתוך של הקו עם
$\overline{AH}$
ב-%
$S$
(איור%
~\ref{f.se-copy3}).
לפי הבנייה
$\overline{AH}\|\overline{OU}$
ו-%
$\overline{AK}\|\overline{OV}$,
ולכן
$\angle SAK=\angle UOV=\theta$.
$\overline{SK}\|\overline{UV}$
ו-%
$\triangle SAK\sim \triangle UOV$
לפי זווית-זווית-זווית,
$\triangle UOV$
הוא שווה-שוקיים כי
$\overline{OU}$, $\overline{OV}$
הם רדיוסים של אותו מעגל. מכאן ש-%
$\triangle SAK$
הוא שווה-שוקיים ו-%
$\overline{AS}=\overline{AK}=\overline{PQ}$.
\end{proof}

\begin{figure}[tb]
\begin{center}
\begin{tikzpicture}[scale=.7]
\coordinate (A) at (0,0);
\coordinate (P) at (3cm,2);
\coordinate (Q) at (4.5cm,2);
\draw (P) -- (Q);
\node[left] at (P) {$P$};
\node[right] at (Q) {$Q$};
\coordinate (A1) at (-3,1);
\draw (A1) -- ++(60:3cm) coordinate (H1);
\node[left] at (A1) {$A'$};
\node[left] at (H1) {$H'$};
\node[left] at (A) {$A$};
\draw (A) -- ++(60:3cm) coordinate (H);
\node[left] at (H) {$H$};
\draw (A) -- ++(1.5,0) coordinate (K);
\node[right] at (K) {$K$};
\draw (A) -- (K);
\path (A) -- ++(60:1.5cm) coordinate (S);
\node[right] at (S) {$S$};
\draw (K) -- ($(K)!1.8!(S)$);
\node[above right,xshift=4pt] at (A) {$\theta$};
\node[above right,xshift=4pt] at (A1) {$\theta$};
\draw (A1) -- ++(1.5,0);
\vertex{P};
\vertex{Q};
\begin{scope}[xshift=3cm]
\coordinate (O) at (6,1);
\draw[name path=circle] (O) circle[radius=2.5cm];
\node[above left] at (O) {$O$};
\path[name path=u] (O) -- ++(60:2.5cm);
\path[name path=v] (O) -- ++(2.5,0);
\path[name intersections={of=circle and u,by={U}}];
\path[name intersections={of=circle and v,by={V}}];
\node[above right] at (U) {$U$};
\node[right] at (V) {$V$};
\draw (O) -- (U) -- (V) -- cycle;
\node[above right,xshift=4pt] at (O) {$\theta$};
\vertex{O};
\end{scope}
\end{tikzpicture}
\end{center}
\selectlanguage{hebrew}
\caption{שימוש במעגל הקבוע כדי להעתיק קטע קו}\label{f.se-copy3}
\end{figure}

%%%%%%%%%%%%%%%%%%%%%%%%%%%%%%%%%%%%%%%%%%%%%%%%%%%%%%%%%%%%%%%

\section{בניית קטע קו יחסית לקטעי קו אחרים}\label{s.relative-straight}

\begin{theorem}\label{thm.three-lines}
נתונים שלושה קטעי קו באורכים
$n, m, s$,
ניתן לבנות קטע קו באורך:
\[
x=\disfrac{n}{m}s\,.
\]
\end{theorem}

\begin{proof}
נבחר נקודות 
$A,B,C$
שאינן על אותו קו ונבנה קרונות
$\overline{AB},\overline{AC}$.
לפי משפט%
~\ref{thm.angle}
ניתן לבנות
$M,N,S$
כך ש-%
$\overline{AM}= m,\overline{AN} =n, \overline{AS}=s$.
לפי משפט%
~\ref{thm.parallel-line}
נבנה קו המקביל ל-%
$\overline{MS}$
דרך 
$N$
שחותך את
$\overline{AC}$
ב-%
$X$,
ונסמן את אורכו של ב-%
$x$
(איור%
~\ref{f.se-three2}).
$\triangle MAS\sim \triangle NAX$
לפי זווית-זווית-זווית ולכן:
\[
\frac{m}{n}=\frac{s}{x}, \quad\quad x=\disfrac{n}{m}s\,.
\]
\end{proof}

\begin{figure}[tb]
\begin{center}
\begin{tikzpicture}[scale=.8]
\coordinate (A) at (0,0);
\draw[name path=ac] (A) node[left] {$A$} -- ++(7,0) node[right] {$C$};
\draw (A) -- ++(40:5cm) node[right] {$B$};
\path (A) -- node[above,xshift=-2pt] {$m$} ++(40:3cm) coordinate (M) node[above left] {$M$};
\path (A) -- ++(40:4cm) coordinate (N) node[above left] {$N$};
\path[name path=ms] (M) -- ++(-50:3.5cm);
\path[name path=nx] (N) -- ++(-50:4cm);
\path[name intersections={of=ac and ms,by={S}}];
\path[name intersections={of=ac and nx,by={X}}];
\node[below] at (S) {$S$};
\node[below] at (X) {$X$};
\path (A) -- node[below] {$s$} (S);
\draw (S) -- (M);
\draw (X) -- (N);
\draw[<->] ($(A)+(0,-.8)$) -- node[fill=white] {$x$} ($(X)+(0,-.8)$);
\draw[<->] ($(A)+(-.6,.8)$) -- node[fill=white] {$n$} ++(40:3.9cm);
\end{tikzpicture}
\end{center}
\caption{בניית יחס אורכים באמצעות משולשים דומים}\label{f.se-three2}
\end{figure}

%%%%%%%%%%%%%%%%%%%%%%%%%%%%%%%%%%%%%%%%%%%%%%%%%%%%%%%%%%%%%%%

\section{בניית שורש ריבועי}\label{s.root}

\begin{theorem}\label{thm.root}
נתון קטעי קו באורכים
$a,b$,
ניתן לבנות קטע קו שאורכו
$\sqrt{ab}$.
\end{theorem}

\begin{proof}
אם נבטא את
$x=\sqrt{ab}$
בצורה
$x=\disfrac{n}{m}s$
נוכל להשתמש במשפט~%
\ref{thm.three-lines}.
\begin{itemize}
\item
עבור
$n$
נשתמש ב-%
$d$,
הקוטר של המעגל הקבוע.

עבור
$m$
נשתמש ב-%
$t=a+b$
שניתן לבנות מ-%
$a,b$
לפי משפט
\ref{thm.angle}.
\item
נגדיר את
$s=\sqrt{hk}$
כאשר 
$h,k$
$a,b,t,d$.
\end{itemize}
נגדיר
$h=\disfrac{d}{t}a$, $k=\disfrac{d}{t}b$,
ונחשב:
\begin{eqn}
x&=&\sqrt{ab}=\sqrt{\frac{th}{d}\frac{tk}{d}}=\sqrt{\left(\frac{t}{d}\right)^2hk}=\frac{t}{d}\sqrt{hk}=\frac{t}{d}s\\
h+k &=& \frac{d}{t}a + \frac{d}{t}b = \frac{d(a+b)}{t} = \frac{dt}{t} = d\,.
\end{eqn}
לפי משפט%
~\ref{thm.angle}
נבנה
$\overline{HA}= h$
על קוטר
$\overline{HK}$
של המעגל הקבוע. מ-%
$h+k=d$
אפשר להסיק ש-%
$\overline{AK}=k$
(איור%
~\ref{f.se-sqrt}).
לפי משפט~%
\ref{thm.perpendicular}
ניתן לבנות דרך
$A$
אנך ל-%
$\overline{HK}$.
נסמן ב-%
$S$
את החיתוך שלו עם המעגל הקבוע.
$\overline{OS}=\overline{OK}=d/2$
הם רדיוסים של המעגל, ו-%
$\overline{OA}=(d/2)-k$.

\begin{figure}[tb]
\begin{center}
\begin{tikzpicture}[scale=.7]
\coordinate (O) at (0,0);
\coordinate (H) at (-3,0);
\coordinate (K) at (3,0);
\node at (-2.4,2.4) {$c$};
\draw (H) -- (K);
\draw[name path=circle] (O) circle[radius=3cm];
\node[below] at (O) {$O$};
\node[left] at (H) {$H$};
\node[right] at (K) {$K$};
\path[name path=as] (1,0) coordinate (A) -- ++(0,3.2);
\node[below] at (A) {$A$};
\path[name intersections={of=circle and as,by={S}}];
\node[above] at (S) {$S$};
\draw (A) -- node[right] {$s$} (S);
\path (H) -- node[above] {$h$} (A);
\path (A) -- node[above] {$k$} (K);
\draw (O) -- node[left,xshift=-2pt] {$\displaystyle\frac{d}{2}$} (S);
\node at (.5,-1.5) {$\displaystyle\frac{d}{2}-k$};
\draw[->] (.5, -1.2) -- ++(0,1);
\draw[rotate=90] (A) rectangle +(8pt,8pt);
\vertex{O};
\end{tikzpicture}
\end{center}
\caption{בניית שורש ריבועי}\label{f.se-sqrt}
\end{figure}
לפי משפט פיתגורס:
\begin{eqn}
s^2=\overline{SA}^2 &=& \left(\disfrac{d}{2}\right)^2 - \left(\disfrac{d}{2}-k\right)^2\\
&=& \left(\disfrac{d}{2}\right)^2 - \left(\disfrac{d}{2}\right)^2 + 2\disfrac{dk}{2} - k^2\\
&=& k(d-k)=kh\\
s&=&\sqrt{hk}\,.
\end{eqn}
כעת ניתן לבנות
$x=\disfrac{t}{d}s$
לפי משפט~%
\ref{thm.three-lines}.
\end{proof}

%%%%%%%%%%%%%%%%%%%%%%%%%%%%%%%%%%%%%%%%%%%%%%%%%%%%%%%%%%%%%%%

\section{בניית נקודות חיתוך של קו עם מעגל}\label{s.line-circle-straight}

\begin{theorem}\label{thm.line-circle}
נתון קו
$l$
ומעגל
$c$
שמרכזו
$O$
ורדיוס
$r$.
ניתן לבנות את נקודות החיתוך של
$l$
עם
$c$.
(איור%
~\ref{f.se-line-circle1}).
\end{theorem}
\begin{figure}[tb]
\begin{center}
\begin{tikzpicture}[scale=.7]
\coordinate (O) at (0,0);
\node[below right] at (O) {$O$};
\vertex{O};
\draw[thick,dashed,name path=circle] (O) circle[radius=3cm];
\draw (O) -- node[above] {$r$} ++(-130:3cm) coordinate (R);
\draw[name path=l] (O) ++(170:4cm) --
  node[below, near end,xshift=30pt,yshift=10pt] {$l$} ++(20:8cm);
\path[name intersections={of=circle and l,by={Y,X}}];
\node[above left] at (X) {$X$};
\node[above right] at (Y) {$Y$};
\end{tikzpicture}
\end{center}
\caption{בניית נקודות החיתוך של קו ומעגל (1)}\label{f.se-line-circle1}
\end{figure}
\begin{proof}
לפי משפט~%
\ref{thm.perpendicular}
ניתן לבנות אנך ממרכז המעגל
$O$
לקו
$l$.
נסמן ב-%
$M$
את נקודת החיתוך של
$l$
עם האנך. 
$OM$
חוצה של המיתר 
$\overline{XY}$,
כאשר 
$X,Y$
הן נקודות החיתוך של הקו עם המעגל
(איור%
~\ref{f.se-line-circle2}).
נגדיר 
$\overline{XY}=2s$
ו-%
$\overline{OM}=t$.
שימו לב שבאיור
$s,X,Y$
הם רק הגדרות וטרם בנינו את נקודות החיתוך.

\begin{figure}[tb]
\begin{center}
\begin{tikzpicture}[scale=.7]
\coordinate (O) at (0,0);
\draw[thick,dashed,name path=circle] (O) circle[radius=3cm];
\node[below right] at (O) {$O$};
\vertex{O};
\path (O) --  ++(-130:3cm) coordinate (R);
\node[below left,yshift=2pt,xshift=2pt] at (R) {$R$};
\draw[name path=l] (O) ++(170:4cm) --
  node[below, near end,xshift=30pt,yshift=10pt] {$l$} ++(20:8cm);
\path[name intersections={of=circle and l,by={Y,X}}];
\node[above left] at (X) {$X$};
\node[above right] at (Y) {$Y$};
\draw (O) -- node[below] {$r$} (X);
\path (X) -- ($(X)!.5!(Y)$) coordinate (M);
\node[above] at (M) {$M$};
\draw (O) -- node[right] {$t$} (M);
\path (X) -- node[above] {$s$} (M);
\path (M) -- node[above] {$s$} (Y);
\draw (O) ++(170:4cm) -- ++(20:3.1cm) -- ++(-70:10pt) -- ++(20:10pt);
\draw (O) -- node[below] {$t$} +(50:2) coordinate (RTT);
\draw (O) -- node[below] {$t$} +(-130:2) coordinate (RT);
\vertex{RT};
\draw (RT) -- node[right,yshift=-2pt] {$r-t$} ($(RT)+(-130:1cm)$);
\vertex{RTT};
\draw[<->] ($(RT)+(.5cm,-1.6cm)$) -- node[fill=white] {$r+t$}+(50:5);
\end{tikzpicture}
\end{center}
\caption{בניית נקודות החיתוך של קו ומעגל (2)}\label{f.se-line-circle2}
\end{figure}
לפי משפט פיתגורס
$s^2=r^2-t^2=(r+t)(r-t)$.
לפי משפט~%
\ref{thm.angle}
ניתן לבנות קטעי קו באורך
$r$
מהנקודה 
$O$
בשני הכיוונים
$\overline{RO}$
ו-%
$\overline{OR}$.
התוצאה היא שני קטעי קו שאורכם
$r+t,r-t$.

לפי משפט~%
\ref{thm.root}
ניתן לבנות קטע קו באורך
$s=\sqrt{(r+t)(r-t)}$.
שוב לפי משפט~%
\ref{thm.angle},
ניתן לבנות קטעי קו באורך 
$s$
על הקו הנתון
$l$
מהנקודה
$M$
בשני הכיוונים. הקצה השני של כל אחד מקטעי הקו האלה הוא נקודת חיתוך של 
$l$
עם המעגל.
\end{proof}

%%%%%%%%%%%%%%%%%%%%%%%%%%%%%%%%%%%%%%%%%%%%%%%%%%%%%%%%%%%%%%%

\section{בניית נקודות החיתוך של שני מעגלים}\label{s.circle-circle}

\begin{theorem}\label{thm.two-circles}
נתונים שני מעגלים עם מרכזים
$O_1,O_2$
ורדיוסים
$r_1,r_2$,
ניתן לבנות את נקודות החיתוך שלהם.
\end{theorem}
\begin{proof}
נבנה את
$\overline{O_1O_2}$
ונסמן את אורכו ב-%
$t$
(איור%
~\ref{f.se-circle-circle1}).
נסמן ב-%
$A$
את נקודת החיתוך של
$\overline{O_1O_2}$
עם
$\overline{XY}$,
ונסמן
$q=\overline{O_1A},x=\overline{XA}$
(איור%
~\ref{f.se-circle-circle2}).
טרם בנינו את הנקודה
$A$,
אבל אם נצליח לבנות את האורכים
$q,x$,
לפי משפט~% 
\ref{thm.angle}
נוכל לבנות את 
$A$
באורך
$q$
מהנקודה
$O_1$
לכיוון
$\overline{O_1O_2}$.
\begin{figure}[tb]
\begin{center}
\begin{tikzpicture}[scale=.9]
\coordinate (O1) at (0,0);
\coordinate (O2) at (2.5,0);
\node[below left] at (O1) {$O_1$};
\node[below right] at (O2) {$O_2$};
\vertex{O1};
\vertex{O2};
\draw[thick,dashed,name path=circle1] (O1) circle[radius=2cm];
\draw[thick,dashed,name path=circle2] (O2) circle[radius=1.6cm];
\path [name intersections={of=circle1 and circle2,by={X,Y}}];
\node[above,yshift=4pt] at (X) {$X$};
\node[below,yshift=-4pt] at (Y) {$Y$};
\draw (O1) -- node[above] {$r_1$} ++(160:2cm);
\draw (O2) -- node[above] {$r_2$} ++(30:1.6cm);
\draw (O1) -- (O2);
\node at (-1.7,1.6) {$c_1$};
\node at (3.8,1.4) {$c_2$};
\draw[<->] (0,-.6) -- node[fill=white] {$t$} +(2.5,0);
\end{tikzpicture}
\end{center}
\caption{בניית החיתוך של שני מעגלים (1)}\label{f.se-circle-circle1}
\end{figure}
לאחר שבנינו את
$A$,
לפי משפט~%
\ref{thm.perpendicular}
ניתן לבנות את האנך ל-%
$\overline{O_1O_2}$
בנקודה
$A$,
ושוב לפי משפט~%
\ref{thm.angle}
ניתן לבנות קטעי קו באורך
$x$
מהנקודה
$A$
בשני הכיוונים לאורך האנך.
$X,Y$,
הקצות של קטעי הקו, הם נקודות החיתוך של שני המעגלים.

\begin{figure}[tb]
\begin{center}
\begin{tikzpicture}[scale=.9]
\coordinate (O1) at (0,0);
\coordinate (O2) at (2.5,0);
\vertex{O1};
\vertex{O2};
\node[below left] at (O1) {$O_1$};
\node[below right] at (O2) {$O_2$};
\draw[thick,dashed,name path=circle1] (O1) circle[radius=2cm];
\draw[thick,dashed,name path=circle2] (O2) circle[radius=1.6cm];
\path [name intersections={of=circle1 and circle2,by={X,Y}}];
\node[above,yshift=4pt] at (X) {$X$};
\node[below,yshift=-4pt] at (Y) {$Y$};
\draw (O1) -- node[above,xshift=-4pt] {$r_1$} (X);
\draw (O2) -- node[above,xshift=4pt] {$r_2$} (X);
\draw[name path=oo] (O1) -- (O2);
\node at (-1.7,1.6) {$c_1$};
\node at (3.8,1.4) {$c_2$};
\draw[name path=xy] (X) -- (Y);
\path[name intersections={of=xy and oo,by={A}}];
\node[below left] at (A) {$A$};
\draw (A) rectangle +(6pt,6pt);
\path (O1) -- node[below,xshift=-2pt] {$q$} (A);
\path (X) -- node[left,yshift=-2pt] {$x$} (A);
\draw[<->] (0,-.6) -- node[fill=white] {$t$} +(2.5,0);
\end{tikzpicture}
\end{center}
\caption{בניית החיתוך של שני מעגלים (2)}\label{f.se-circle-circle2}
\end{figure}



\textbf{בניית האורך
$q$:}
נגדיר
$d=\sqrt{r_1^2+t^2}$, 
היתר של משולש ישר-זווית שניתן לבנות מהאורכים הידועים
$r_1,t$.
שימו לב ש-%
$\triangle O_1XO_2$
הוא לא בהכרח משולש ישר-זווית, ואת המשולש אפשר לבנות בכל מקום במישור. במשולש ישר-זווית 
$\triangle XAO_1$, $\cos\angle XO_1A=q/r_1$.
לפי משפט הקוסינוסים ב-%
$\triangle O_1O_2X$:
\begin{eqn}
r_2^2 &=& t^2 + r_1^2 - 2r_1t\cos\angle XO_1O_2\\
&=& t^2 + r_1^2 - 2tq\\
2tq &=& (t^2+r_1^2) - r_2^2=d^2-r_2^2\\
q&=&\frac{(d+r_2)(d-r_2)}{2t}\,.
\end{eqn}
לפי משפט~%
\ref{thm.angle}
ניתן לבנות את האורכים האלה, ולפי משפט~%
\ref{thm.three-lines}
ניתן לבנות את
$q$
מהביטויים
$d+r_2,d-r_2,2t$.

\textbf{בניית האורך
$x$:}
לפי משפט פיתגורס:
\[
x^2=r_1^2-q^2 =\sqrt{(r_1+q)(r_1-q)}\,.
\]
לפי משפט~%
\ref{thm.angle}
ניתן לבנות
$h =r_1+ q$
ו-%
$k= r_1 - q$,
ולפי משפט~%
\ref{thm.root}
ניתן לבנות
$x= \sqrt{hk}$. 
\end{proof}

%%%%%%%%%%%%%%%%%%%%%%%%%%%%%%%%%%%%%%%%%%%%%%%%%%

\subsection*{מה ההפתעה?}

חובה להשתמש במחוגה כי עם סרגל אפשר רק לחשב שורשים  משוואות ליניאריות ולא ערכים כגון
$\sqrt{2}$,
היתר של משולש ישר-זווית ושווה-שוקיים עם צלעות באורך
$1$.
לכן, מפתיע שקיום של מעגל אחד בלבד, ללא תלות במקומו של המרכז או של הרדיוס שלו, מספיק כדי לבצע כל בנייה שאפשרית עם סרגל ומחוגה.

%%%%%%%%%%%%%%%%%%%%%%%%%%%%%%%%%%%%%%%%%%%%%%%%%%

\subsection*{מקורות}
הפרק מבוסס על בעיה
$34$
ב-%
\L{\cite{dorrie1}}
שעובדה על ידי
\L{Michael Woltermann} \L{\cite{dorrie2}}. 
