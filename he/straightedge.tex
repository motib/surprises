% !TeX root = surprises.tex

\chapter{אפשר להסתפק בסרגל ביחד עם מעגל אחד}\label{c.straightedge}

%%%%%%%%%%%%%%%%%%%%%%%%%%%%%%%%%%%%%%%%%%%%%%%%%%%%%%%%%%%%%%%

האם כל בניה עם סרגל ומחוגה ניתנת לבניה עם סרגל בלבד? התשובה היא שלילית כי קווים הוגדרים על ידי משוואות ליניאריות ולא יכולים להציג מעגלים המוגדרים על ידי משוואות ריבועיות. ב-%
$1822$
\L{Jean-Victor Poncelet}
שיער שכן ניתן להסתפק בסרגל בלבד בתנאי שקיים במישור 
\textbf{מעגל אחד בלבד}.
המשפט הוכח ב-%
$1833$
על ידי
\L{Jakob Steiner}.

לאחר שנסביר בסעיף%
~\ref{s.se-what}
מה המשמעות של בנייה רק עם סרגל ומעגל אחד, ההוכחה מוצגת בשלבים, תחילה עם חמש בניות עזר: בניית קו המקביל לקו נתון (סעיף%
~\ref{s.parallel}),
בניית ניצב לקו נתון (סעיף%
~\ref{s.perpendicular}),
העתקת קטע קו בכיוון נתון (סעיף%
~\ref{s.direction}), 
בניית קטע קו כיחס בין קטעים אחרים (סעיף%
~\ref{s.relative-straight})
ובניית שורש ריבועי (סעיף%
~\ref{s.root}).
סעיף%
~\ref{s.line-circle-straight}
מראה איך למצוא את החיתוכים של קו ומעגל וסעיף%
~\ref{s.circle-circle}
מראה איך למצוא את החיתוכים של שני מעגלים.

%%%%%%%%%%%%%%%%%%%%%%%%%%%%%%%%%%%%%%%%%%%%%%%%%%%%%%%%%%%%%%%

\section{מהי בנייה עם סרגל בלבד?}\label{s.se-what}

כל צעד בבניה עם סרגל ומחוגה הוא אחת משלושת הפעולות הללו:
\begin{itemize}
\setlength{\itemsep}{0pt}
\item
מציאת נקודת החיתוך של שני קווים.
\item
מציאת נקודות החיתוך של קו עם מעגל.
\item
מציאת נקודות החיתוך של שני מעגלים.
\end{itemize}
ברור שניתן לבצע את הפעולה הראשונה עם סרגל בלבד. עלינו להראות שעבור שתי הפעולות האחרות ניתן למוצא בניה שקולה שמשתמשת רק בסרגל עם מעגל אחד.

%%%%%%%%%%%%%%%%%%%%%%%%%%%%%%%%%%%%%%%%%%%%%%%%%%%%%%%%%%%%%%%

מה המשמעות של בניה עם סרגל בלבד? מעגל מוגדר על ידי נקודה
$O$,
שהיא מרכז המעגל, וקטע קו באורך
$r$,
הרדיוס, שאחת מהנקודות הקצה שלו היא
$O$.
אם נצליח לבנות את הנקודות
$X,Y$
המסומנות באיור שלהלן, נוכל לטעון שהצלחנו לבנות את נקודות החיתוך של מעגל נתון עם קו נתון ושל שני מעגלים נתונים
(איור%
~\ref{f.se-only-line-circle}).
בהמשך, המעגל הנתון יצוייר בקו רגיל והמעגלים שמשמשים רק להדגמת הבניה יצויירו מקווקווים 
(איור%
~\ref{f.se-only-two-circles}).
\begin{figure}[htb]
\begin{center}
\begin{subfigure}{.4\textwidth}
\begin{tikzpicture}[scale=.9]
\fill (0,0) node[above right] {$O$} circle[radius=1.5pt];
\draw[thick,dashed,name path=circle] (0,0) circle[radius=2cm];
\draw (0,0) -- node[left] {$r$} ++(-60:2cm);
\fill (0,0) ++(-60:2cm) circle[radius=1.5pt];
\draw[name path=line] (-3,-.5) -- ++(20:6cm);
\path [name intersections={of=circle and line,by={X,Y}}];
\fill (X) node[above right,xshift=-2pt,yshift=4pt] {$X$} circle[radius=1.5pt];
\fill (Y) node[above left] {$Y$} circle[radius=1.5pt];
\end{tikzpicture}
\selectlanguage{hebrew}
\caption{$X$, $Y$
הם נקודות החיתוך של קו ומעגל}
\label{f.se-only-line-circle}
\end{subfigure}
\hspace{3em}
\begin{subfigure}{.4\textwidth}
\begin{tikzpicture}[scale=.9]
\fill (0,0) node[above right] {$O_1$} circle[radius=1.5pt];
\fill (3,0) node[above right] {$O_2$} circle[radius=1.5pt];
\draw[thick,dashed,name path=circle1] (0,0) circle[radius=2cm];
\draw[thick,dashed,name path=circle2] (3,0) circle[radius=2cm];
\draw (0,0) -- node[left] {$r_1$} ++(-70:2cm);
\draw (3,0) -- node[left,below] {$r_2$} ++(-20:2cm);
\fill (0,0) ++(-70:2cm) circle[radius=1.5pt];
\fill (3,0) ++(-20:2cm) circle[radius=1.5pt];
\path [name intersections={of=circle1 and circle2,by={X,Y}}];
\fill (X) node[above,yshift=4pt] {$X$} circle[radius=1.5pt];
\fill (Y) node[below,yshift=-4pt] {$Y$} circle[radius=1.5pt];
\end{tikzpicture}
\selectlanguage{hebrew}
\caption{$X$, $Y$
הם נקודות החיתוך של שני מעגלים}
\label{f.se-only-two-circles}
\end{subfigure}
\end{center}
\end{figure}

%%%%%%%%%%%%%%%%%%%%%%%%%%%%%%%%%%%%%%%%%%%%%%%%%%%%%%%%%%%%%%%
תחילה נביא חמש בניות עזר נחוצות )סעיפים
\ref{s.parallel}--\ref{s.root}%
(,
ואחר כך נראה איך למצוא את נקודות החיתוך של קו עם מעגל )סעיף
\ref{s.line-circle-straight}(
ושל שני מעגלים )סעיף
\ref{s.circle-circle}(.


%%%%%%%%%%%%%%%%%%%%%%%%%%%%%%%%%%%%%%%%%%%%%%%%%%%%%%%%%%%%%%%

\section{בניית קו המקביל לקו נתון}\label{s.parallel}

\begin{theorem}\label{thm.parallel-line}
נתון קו
$l$
העובר דרך שתי נקודות
$A,B$,
ונתונה נקודה 
$P$
)שאיננה על הקו(, ניתן לבנות קו דרך
$P$
המקביל ל-%
$\overline{AB}$.
\end{theorem}
\begin{proof}
נפריד את הבניה לשני מקרים:
\begin{enumerate}
\item
"קו מכוון": נתונה גם הנקודה
$M$
שהיא נקודת האמצע של קטע הקו
$\overline{AB}$.
\item
כל קו אחר.
\end{enumerate}



\textbf{מקרה ראשון:}
בנו קרן הממשיכה את
$\overline{AP}$,
ובחרו
$S$,
נקודה כלשהי על הקרן מעבר ל-%
$P$.
בנו את הקווים
$\overline{SB}$, $\overline{SM}$, $\overline{BP}$.
סמנו ב-%
$O$
את נקודת החיתוך של 
$\overline{BP}$
עם
$\overline{SM}$.
בנו קרן הממשיכה את
$\overline{AO}$
וסמנו ב-%
$Q$
את החיתוך של הקרן עם
$\overline{SB}$
(איור%
~\ref{f.se-parallel-directed}).

\begin{figure}[htb]
\begin{center}
\begin{tikzpicture}
\draw[name path=pq] (-4,0) -- (4,0);
\draw (-2,-2) node[below left] {$A$} coordinate (A) -- (2,-2) node[below right] {$B$} coordinate (B);
\fill (A) circle[radius=1.5pt];
\fill (B) circle[radius=1.5pt];
\draw[name path=as] (A) -- ++(50:4cm) node[above] {$S$} coordinate (S);
\fill (S) circle[radius=1.5pt];
\draw[name path=sb] (S) -- (B);
\path [name intersections={of=pq and as,by={P}}];
\path [name intersections={of=pq and sb,by={Q}}];
\fill (P) node[above left] {$P$} circle[radius=1.5pt];
\fill (Q) node[above right] {$Q$} circle[radius=1.5pt];
\draw[name path=pb] (P) -- (B);
\draw[name path=qa] (Q) -- (A);
\path [name intersections={of=pb and qa,by={O}}];
\fill (O) node[right,xshift=2pt] {$O$} circle[radius=1.5pt];
\fill (0,-2) coordinate (M) node[below right] {$M$} circle[radius=1.5pt];
\draw (S) -- (M);
\end{tikzpicture}
\selectlanguage{hebrew}
\caption{בניית קו מקביל לקו מכוון}\label{f.se-parallel-directed}
\end{center}
\end{figure}

\textbf{%
טענה:
$\overline{PQ}\|\overline{AB}$.}

להוכחת הטענה נשתמש במשפט
\L{Ceva}
שנוכיח בהמשך.


נתונים שלושה קטעי קו מקודקודי משולש לצלעות הנגדיות שנפגשים בנקודה
$M$
)כמו באיור לעיל, אבל 
$M$
לא בהכרח חוצה את הצלע(. קטעי הצלעות מקיימים את היחס:
\[
\frac{\overline{AM}}{\overline{MB}}\cdot\frac{\overline{BQ}}{\overline{QS}}\cdot\frac{\overline{SP}}{\overline{PA}} = 1\,.
\]

\textbf{הוכחה של הטענה:}
בבניה למעלה 
$M$
חוצה את
$\overline{AB}$
ולכן
$\disfrac{\overline{AM}}{\overline{MB}}=1$.
מכאן ש:
\begin{equation}
\frac{\overline{BQ}}{\overline{QS}}=\frac{\overline{PA}}{\overline{SP}}=\frac{\overline{AP}}{\overline{PS}}\,.\label{eq.ceva}
\end{equation}



נוכיח ש-%
$\triangle ABS\sim\triangle PQS$,
ולכן
$\overline{PQ}\|\overline{AB}$
כי
$\angle ABS = \angle PQS$.


\textbf{הוכחה שהמשולשים דומים:}

\begin{eqn}
\overline{BS}&=&\overline{BQ}+\overline{QS}\\
\disfrac{\overline{BS}}{\overline{QS}}&=&\disfrac{\overline{BQ}}{\overline{QS}}+\disfrac{\overline{QS}}{\overline{QS}} = \disfrac{\overline{BQ}}{\overline{QS}}+1\\
\overline{AS}&=&\overline{AP}+\overline{PS}\\
\disfrac{\overline{AS}}{\overline{PS}} &=& \disfrac{\overline{AP}}{\overline{PS}} + \disfrac{\overline{PS}}{\overline{PS}} = \disfrac{\overline{AP}}{\overline{PS}} + 1\\
\disfrac{\overline{BS}}{\overline{QS}}=\disfrac{\overline{BQ}}{\overline{QS}}+1&=&\disfrac{\overline{AP}}{\overline{PS}} + 1=\disfrac{\overline{AS}}{\overline{PS}}\,,
\end{eqn}
כאשר המשוואה האחרונה מתקבלת ממשפט
\ref{thm.ceva}.

%\textbf{הוכחה של משפט \L{Ceva}:}
%\begin{center}
%
%\vspace*{-4pt}
%\begin{tikzpicture}
%\path[name path=pq] (-4,0) -- (4,0);
%\draw (-2,-2) node[below left] {$A$} coordinate (A) -- (2,-2) node[below right] {$B$} coordinate (B);
%\coordinate (M) at (0,-2);
%\draw[name path=as] (A) -- ++(50:4cm) node[above] {$S$} coordinate (S);
%\draw[name path=sb] (S) -- (B);
%\path [name intersections={of=pq and as,by={P}}];
%\path [name intersections={of=pq and sb,by={Q}}];
%\path[name path=pb] (P) -- (B);
%\path[name path=qa] (Q) -- (A);
%\path [name intersections={of=pb and qa,by={O}}];
%\draw[fill=gray!40] (B) -- (O) -- (Q);
%\draw[fill=gray!70] (S) -- (O) -- (Q);
%\draw (B) -- (O) -- (A);
%\draw (S) -- (O) -- (A);
%\draw (A) -- (B) -- (S) -- cycle;
%\draw (S) -- (O);
%\draw (B) -- (O);
%\fill (A) circle[radius=1.5pt];
%\fill (B) circle[radius=1.5pt];
%\fill (S) circle[radius=1.5pt];
%\fill (Q) node[above right] {$Q$} circle[radius=1.5pt];
%\fill (O) node[above left] {$O$} circle[radius=1.5pt];
%\path[name path=al1] (O) -- ($(Q)!(O)!(B)$);
%\draw[rotate=-155] ($(Q)!(O)!(B)$) rectangle +(7pt,7pt);
%\path [name intersections={of=al1 and sb,by={A1}}];
%\draw[thick,dashed] (O) -- (A1);
%\begin{scope}[xshift=6cm]
%\path[name path=pq] (-4,0) -- (4,0);
%\draw (-2,-2) node[below left] {$A$} coordinate (A) -- (2,-2) node[below right] {$B$} coordinate (B);
%\coordinate (M) at (0,-2);
%\draw[name path=as] (A) -- ++(50:4cm) node[above] {$S$} coordinate (S);
%\draw[name path=sb] (S) -- (B);
%\path [name intersections={of=pq and as,by={P}}];
%\path [name intersections={of=pq and sb,by={Q}}];
%\draw[name path=pb] (P) -- (B);
%\draw[name path=qa] (Q) -- (A);
%\path [name intersections={of=pb and qa,by={O}}];
%\draw (B) -- (O) -- (Q);
%\draw (A) -- (Q) -- (B);
%\draw[fill=gray!40] (B) -- (Q) -- (A);
%\draw[fill=gray!70] (S) -- (Q) -- (A);
%\draw (A) -- (B) -- (S) -- cycle;
%\draw (S) -- (O);
%\draw (B) -- (O);
%\fill (A) circle[radius=1.5pt];
%\fill (B) circle[radius=1.5pt];
%\fill (S) circle[radius=1.5pt];
%\fill (Q) node[above right] {$Q$} circle[radius=1.5pt];
%\fill (O) node[above left] {$O$} circle[radius=1.5pt];
%\path[name path=al2] (A) -- ($(Q)!(A)!(B)$);
%\draw[rotate=-155] ($(Q)!(A)!(B)$) rectangle +(7pt,7pt);
%\path [name intersections={of=al2 and sb,by={A2}}];
%\draw[thick,dashed] (A) -- (A2);
%\end{scope}
%\end{tikzpicture}
%\vspace*{-6pt}
%\end{center}
%אם הגבהים של שני משולשים שווים, יחס השטחים שווה ליחס הבסיסים.
%%\[
%%A_1 = \frac{1}{2}hb_1,\quad A_2 = \frac{1}{2}hb_2, \quad \frac{A_1}{A_2}=\frac{b_1}{b_2}\,.
%%\]
%הגבהים של המשולשים 
%$\triangle BQO, \triangle SQO$
%שווים, כמו גם
%$\triangle BQA, \triangle SQA$.
%לכן:%
%\footnote{%
%\R{נשתמש בשם המשולש כקיצור לשטחו}.}
%\begin{eqnlabels}
%\frac{\triangle BQO}{\triangle SQO} &=& \frac{\overline{BQ}}{\overline{QS}}\label{eq.ratio-of-areas1}\\
%\frac{\triangle BQA}{\triangle SQA} &=& \frac{\overline{BQ}}{\overline{QS}}\,.\label{eq.ratio-of-areas2}
%\end{eqnlabels}
%על ידי חיסור נקבל יחס בין המשולשים המסומנים באפור:
%\begin{center}
%
%\vspace*{-4pt}
%\begin{tikzpicture}
%\path[name path=pq] (-4,0) -- (4,0);
%\draw (-2,-2) node[below left] {$A$} coordinate (A) -- (2,-2) node[below right] {$B$} coordinate (B);
%\coordinate (M) at (0,-2);
%\draw[name path=as] (A) -- ++(50:4cm) node[above] {$S$} coordinate (S);
%\draw[name path=sb] (S) -- (B);
%\path [name intersections={of=pq and as,by={P}}];
%\path [name intersections={of=pq and sb,by={Q}}];
%\path[name path=pb] (P) -- (B);
%\draw[thick,name path=qa] (Q) -- (A);
%\path [name intersections={of=pb and qa,by={O}}];
%\draw[fill=gray!50] (B) -- (O) -- (A);
%\draw[fill=gray!70] (S) -- (O) -- (A);
%\draw (B) -- (O) -- (A);
%\draw (S) -- (O) -- (A);
%\draw (A) -- (B) -- (S) -- cycle;
%\draw (S) -- (O);
%\draw (B) -- (O);
%\fill (A) circle[radius=1.5pt];
%\fill (B) circle[radius=1.5pt];
%\fill (S) circle[radius=1.5pt];
%\fill (Q) node[above right] {$Q$} circle[radius=1.5pt];
%\fill (O) node[right,xshift=2pt] {$O$} circle[radius=1.5pt];
%\end{tikzpicture}
%\end{center}
%\[
%\frac{\triangle BQA - \triangle BQO}{\triangle SQA-\triangle SQO} = \frac{\triangle BOA}{\triangle SOA} = \frac{\overline{BQ}}{\overline{QS}}\,.\label{eq.diff-of-areas}
%\]
%נסביר את החישוב תוך שימוש בסימונים פשוטים יותר:
%\[
%a=\overline{BQ},\, b=\overline{QS},\,
%c=\triangle{BQA},\, d=\triangle{SQA},\,
%e=\triangle{BQO},\,f=\triangle{SQO}\,.
%\]
%מהמשוואות
%\ref{eq.ratio-of-areas1}, \ref{eq.ratio-of-areas2}:
%
%\begin{eqnlabels}
% \disfrac{c}{d} &=&\disfrac{a}{b}\\
% \disfrac{e}{f} &=&\disfrac{a}{b}\,.
%\end{eqnlabels}
%מחישוב פשוט:
%
%\begin{eqn}
%c-e &=& \disfrac{ad}{b} - \disfrac{af}{b}= \disfrac{a}{b}(d-f)\\
%\disfrac{c-e}{d-f} &=& \disfrac{a}{b}
%\end{eqn}
%מתקבלת המשוואה
%\ref{eq.diff-of-areas}.
%
%באופן דומה ניתן להוכיח:
%\[
%\frac{AM}{MB} = \frac{\triangle AOS}{\triangle BOS}\;,\quad\quad \frac{SP}{PA} =\frac{\triangle SOB}{\triangle AOB}\;,
%\]
%
%ומכאן:
%\[
%\frac{AM}{MB}\cdot\frac{BQ}{QS}\cdot\frac{SP}{PA} = \frac{\triangle AOS}{\triangle BOS}\frac{\triangle BOA}{\triangle SOA}\frac{\triangle SOB}{\triangle AOB}=1\,,
%\]
%
%כי סדר הקודקודים  לא משנה
%$\triangle AOS\!=\!\triangle SOA, \triangle BOA\!=\!\triangle AOB, \triangle SOB\!=\!\triangle BOS$.
%\qed
%%%%%%%%%%%%%%%%%%%%%%%%%%%%%%%%%%%%%%%%%%%%%%%%%%%%%%%%%%%%%%%

\medskip

\textbf{מקרה שני:}
נתון הקו
$l$
ונתונה הנקודה
$P$
שאיננה נמצאת על הקו. סמנו את המרכז של המעגל הקבוע ב-%
$O$
והרדיוס שלו ב-%
$r$
(איור%
~\ref{f.se-parallel-other1}).
בחרו נקודה
$M$
על הקו 
$l$
ובנו קרן דרך 
$\overline{MO}$
שחותך את המעגל הקבוע ב-%
$U,V$.

\begin{figure}[htb]
\begin{center}
\begin{subfigure}{.45\textwidth}
\begin{tikzpicture}[scale=.8]
\coordinate (O) at (0,0);
\fill (O) node[below right] {$O$} circle[radius=1.5pt];
\draw[name path=circle] (O) circle[radius=2cm];
\draw[name path=l] (-4,-3) -- node[above, near end] {$l$} +(9,0);
\path[name path=mo] (-2,-3) coordinate (M) -- ($(-2,-3)!1.65!(O)$);
\fill (M) node[below] {$M$} circle[radius=1.5pt];
\path [name intersections={of=circle and mo,by={V,U}}];
\fill (U) node[below,xshift=2pt,yshift=-4pt] {$U$} circle[radius=1.5pt];
\fill (V) node[right,xshift=4pt] {$V$} circle[radius=1.5pt];
\draw (M) --   (U) -- node[above] {$r$} (O) -- node[above] {$r$} (V);
%\node at (-1.6,1.6) {$c$};
\fill (-4,1) node[above left] {$P$} circle[radius=1.5pt];
\end{tikzpicture}
\selectlanguage{hebrew}
\caption{בניית קו מכוון}\label{f.se-parallel-other1}
\end{subfigure}
\begin{subfigure}{.45\textwidth}
\begin{tikzpicture}[scale=.9]
\coordinate (O) at (0,0);
\fill (O) node[below right] {$O$} circle[radius=1.5pt];
\draw[name path=circle] (O) circle[radius=2cm];
\draw[name path=l] (-4,-3) -- node[above,near end,xshift=24pt] {$l$} +(9,0);
\path[name path=mo] (-2,-3) coordinate (M) -- ($(-2,-3)!1.65!(O)$);
\fill (M) node[below] {$M$} circle[radius=1.5pt];
\path [name intersections={of=circle and mo,by={V,U}}];
\fill (U) node[below,xshift=2pt,yshift=-4pt] {$U$} circle[radius=1.5pt];
\fill (V) node[right,xshift=4pt] {$V$} circle[radius=1.5pt];
\draw (M) -- (V);
\path[name path=ax] (-3,-3) coordinate (A) -- ($(-3,-3)!1.8!(-1,0)$);
\fill (A) node[below] {$A$} circle[radius=1.5pt];
\path [name intersections={of=circle and ax,by={Y,X}}];
\fill (X) node[left] {$X$} circle[radius=1.5pt];
\fill (Y) node[above] {$Y$} circle[radius=1.5pt];
%\node at (-1.6,1.6) {$c$};
\draw (A) -- (Y);
\fill (-4,1) node[above left] {$P$} circle[radius=1.5pt];
\end{tikzpicture}
\selectlanguage{hebrew}
\caption{בניית קו מקביל לקו מכוון}\label{f.se-parallel-other2}
\end{subfigure}
\end{center}
\end{figure}
קו זה הוא
\textbf{%
קו מכוון%
}
כי 
$O$,
מרכז המעגל, חוצה את הקוטר
$\overline{UV}$.
בחרו נקודה שנייה 
$A$
על 
$l$.
לפי הבניה של המקרה הראשון, ניתן לבנות קו המקביל ל-%
$\overline{UV}$
דרך 
$A$.
יש לבחור את
$A$
כך שהקו המקביל חותך את המעגל בשתי נקודות שנסמן
$X,Y$
(איור%
~\ref{f.se-parallel-other2}).

בנו קוטר
$\overline{XX'}$
וקוטר
$\overline{YY'}$.
בנו קרן מ-%
$X'$
עבור דרך
$Y'$
וסמנו ב-%
$B$
את נקודת החיתוך שלה עם 
$l$
(איור%
~\ref{f.se-parallel-other3}).

\begin{figure}[htb]
\begin{center}
\begin{tikzpicture}[scale=.9]
\coordinate (O) at (0,0);
\fill (O) node[below right] {$O$} circle[radius=1.5pt];
\draw[name path=circle] (O) circle[radius=2cm];
\draw[name path=l] (-4,-3) -- node[above,near end,xshift=24pt] {$l$} +(9,0);
\path[name path=mo] (-2,-3) coordinate (M) -- ($(-2,-3)!1.65!(O)$);
\fill (M) node[below] {$M$} circle[radius=1.5pt];
\path [name intersections={of=circle and mo,by={V,U}}];
\fill (U) node[below,xshift=2pt,yshift=-4pt] {$U$} circle[radius=1.5pt];
\fill (V) node[right,xshift=4pt] {$V$} circle[radius=1.5pt];
\draw (M) -- (V);
\path[name path=ax] (-3,-3) coordinate (A) -- ($(-3,-3)!1.8!(-1,0)$);
\fill (A) node[below] {$A$} circle[radius=1.5pt];
\path [name intersections={of=circle and ax,by={Y,X}}];
\fill (X) node[left] {$X$} circle[radius=1.5pt];
\fill (Y) node[above] {$Y$} circle[radius=1.5pt];
%\node at (-1.6,1.6) {$c$};
\draw (A) -- (Y);
\fill (-4,1) node[above left] {$P$} circle[radius=1.5pt];
\path[name path=xo] (X) -- ($(X)!2.2!(O)$);
\path[name intersections={of=circle and xo,by={Xp}}];
\fill (Xp) node[right,xshift=2pt,yshift=-2pt] {$X'$} circle[radius=1.5pt];
\draw (X) -- (Xp);
\path[name path=yo] (Y) -- ($(Y)!2.4!(O)$);
\path[name intersections={of=circle and yo,by={Yp,y}}];
\fill (Yp) node[below right] {$Y'$} circle[radius=1.5pt];
\draw (Y) -- (Yp);
\path[name path=xy] (Xp) -- ($(Xp)!1.6!(Yp)$);
\path[name intersections={of=l and xy,by={B}}];
\fill (B) node[below] {$B$} circle[radius=1.5pt];
\draw (Xp) -- (B);
\draw[thick,dashed,name path=z] (-4,0) -- (4,0) node[above,near end,xshift=40pt] {$l'$};
\path[name intersections={of=ax and z,by={Z}}];
\path[name intersections={of=xy and z,by={Zp}}];
\fill (Z) node[above left] {$Z$} circle[radius=1.5pt];
\fill (Zp) node[below right] {$Z'$} circle[radius=1.5pt];
\end{tikzpicture}
\selectlanguage{hebrew}
\caption{הוכחה ש-%
$l'$
מקביל ל-%
$l$}\label{f.se-parallel-other3}
\end{center}
\end{figure}

\textbf{%
טענה:%
}
$l$
הוא
\textbf{קו מכוון}.

$\overline{OX},\overline{OX'},\overline{OY},\overline{OY'}$
הם כולם רדיוסים של המעגל, ו-%
$\angle XOY = \angle X'OY'$
כי הן זוויות קודקודיות. לכן,
$\triangle XOY\cong \triangle X'OY'$
לפי צלע-זווית-צלע. נגדיר )לא נבנה!( קו
$l'\|l$
שחותך את 
$\overline{XY}$
ב-%
$Z$
ושחותך את 
$X',Y'$
ב-%
$Z'$.
$\angle XOZ=\angle X'OZ'$
כי הן זוויות קודקודיות, ולכן
$\triangle XOZ\cong \triangle X'OZ'$
לפי זווית-צלע-זווית. מכאן ש-%
$\overline{ZO}=\overline{OZ'}$. 
$AMOZ$
ו-%
$BMOZ'$
מקביליות ולכן
$\overline{AM}=\overline{ZO}=\overline{OZ'}=\overline{MB}$
ו-%
$M$
חוצה את
$\overline{AB}$.
\end{proof}

\begin{proof}
לפי הבניה של המקרה הראשון ניתן לבנות את הקו
$l'$
דרך 
$P$
שמקביל ל-%
$l$.
\end{proof}

משפט~%
\ref{thm.parallel-line}
אומר שניתן לבנות 
\textbf{קו}
דרך 
$P$
המקביל ל-%
$\overline{AB}$.
למעשה גם אפשר לבנות
\textbf{קטע קו}
המקביל ל-%
$\overline{AB}$
שאורכו שווה לאורך של
$\overline{AB}$.

\begin{theorem}
ניתן להעתיק קטע קו מקביל לעצמו כך שקצה אחד יהיה נקודה כלשהי.
\end{theorem}

\begin{proof}
בנו קו דרך
$P$
המקביל ל-%
$\overline{AB}$,
חברו את
$P$
ו-%
$A$,
ובנו קו
$n$
דרך 
$B$
המקביל ל-%
$\overline{AP}$.
$ABQP$
הוא מקבילית ו-%
$\overline{PQ}=\overline{AB}$.
\end{proof}

\begin{figure}[htb]
\begin{center}
\begin{tikzpicture}[scale=.7]
\coordinate (A) at (0,0);
\coordinate (B) at (3,0);
\coordinate (P) at (-2,2.5);
\coordinate (Q) at (1,2.5);
\draw ($(P)!-.6!(Q)$) -- node[above,near end,xshift=36pt] {$m$} ($(P)!2.2!(Q)$);
\fill (P) node[above] {$P$} circle[radius=1.5pt];
\fill (Q) node[above right] {$Q$} circle[radius=1.5pt];
\draw ($(A)!-.6!(B)$) -- node[above,near end,xshift=40pt] {$l$} ($(A)!2.5!(B)$);
\fill (A) node[below] {$A$} circle[radius=1.5pt];
\fill (B) node[below left] {$B$} circle[radius=1.5pt];
\draw (A) -- (P);
\draw ($(B)!-.3!(Q)$) -- node[above,near end,xshift=24pt,yshift=-24pt] {$n$} ($(B)!1.4!(Q)$);
\end{tikzpicture}
\selectlanguage{hebrew}
\caption{בניית העתק של קו מקביל לקו קיים}\label{f.se-parallel-other4}
\end{center}
\end{figure}

%%%%%%%%%%%%%%%%%%%%%%%%%%%%%%%%%%%%%%%%%%%%%%%%%%%%%%%%%%%%%%%

\section{בניית אנך לקו נתון}\label{s.perpendicular}

\begin{theorem}\label{thm.perpendicular}
נתון קו
$l$
ונקודה
$P$
)שאיננה על הקו( ניתן לבנות אנך ל-%
$l$
דרך
$P$.%
\end{theorem}

\begin{proof}
בנו לפי משפט
\ref{thm.parallel-line}
קו
$l'\|l$
החותך את המעגל הקבוע ב-%
$U,V$
(איור%
~\ref{f.se-perp}).
בנו את הקוטר
$\overline{UOU'}$
והמיתר
$\overline{U'V}$.
$\angle UVU'$
היא זווית ישרה כי היא נשענת על קוטר. מכאן ש-%
$\overline{U'V}\perp l'$.
בנו קו מקביל ל-%
$\overline{U'V}$
דרף
$P$
לפי משפט~%
\ref{thm.parallel-line}.
\end{proof}

\begin{figure}[htb]
\begin{center}
\begin{tikzpicture}[scale=.7]
\coordinate (O) at (0,0);
\coordinate (P) at (3.5,.6);
\draw[name path=circle] (O) circle[radius=2cm];
\draw[name path=l] (-4,-3) -- node[above,near end,xshift=45pt] {$l$} ++(9,0);
\draw[name path=lp] (-3,-1) -- node[above,near end,xshift=40pt] {$l'$} ++(8,0);
\node[above left] at (O) {$O$};
\node[right] at (P) {$P$};
\path[name intersections={of=circle and lp,by={U,V}}];
\node[below left] at (U) {$U$};
\node[below right] at (V) {$V$};
\path[name path=d] (U) -- ($(U)!2.3!(O)$);
\path[name intersections={of=circle and d,by={Up}}];
\draw (U) -- (Up);
\node[above right] at (Up) {$U'$};
\draw (Up) -- (V);
\path[name path=p] (P) -- ++(0,-4);
\path[name intersections={of=p and l,by={X}}];
\draw (X) rectangle +(9pt,9pt);
\draw[rotate=90] (V) rectangle +(9pt,9pt);
\vertex{O};
\vertex{P};
\draw (P) -- ++(0,1);
\draw (P) -- (X);
\end{tikzpicture}
\end{center}
\caption{בניית ניצב}\label{f.se-perp}
\end{figure}

%\begin{center}
%
%\begin{tikzpicture}[scale=.8]
%\coordinate (O) at (0,0);
%\coordinate (P) at (3.5,.6);
%%\node at (-1.6,1.6) {$c$};
%\draw[name path=circle] (O) circle[radius=2cm];
%\draw[name path=l] (-4,-3) -- node[above,near end,xshift=45pt] {$l$} ++(9.5,0);
%\draw[name path=lp] (-3,-1) -- node[above,near end,xshift=45pt] {$l'$} ++(7.5,0);
%\fill (O) node[above left] {$O$} circle[radius=1.5pt];
%\fill (P) node[right] {$P$} circle[radius=1.5pt];
%\path[name intersections={of=circle and lp,by={U,V}}];
%\fill (U) node[below left] {$U$} circle[radius=1.5pt];
%\fill (V) node[below right] {$V$} circle[radius=1.5pt];
%\path[name path=d] (U) -- ($(U)!2.3!(O)$);
%\path[name intersections={of=circle and d,by={Up}}];
%\draw (U) -- (Up);
%\fill (Up) node[above right] {$U'$} circle[radius=1.5pt];
%\draw (Up) -- (V);
%\path[name path=p] (P) -- ++(0,-4);
%\draw[name intersections={of=p and l,by={X}}];
%\fill (X) circle[radius=1.5pt];
%\draw[thick,dashed] (P) -- (X);
%\draw ($(U)!.9!(V)$) -- ++(0,.3) -| (V);
%\end{tikzpicture}
%\vspace*{-8pt}
%\end{center}
%
%%%%%%%%%%%%%%%%%%%%%%%%%%%%%%%%%%%%%%%%%%%%%%%%%%%%%%%%%%%%%%%

\section{העתקת קטע קו נתון בכיוון נתון}\label{s.direction}

\begin{theorem}\label{thm.angle}
נתון נקודה
$A$,
קטע קו
$\overline{PQ}$
וזווית
$\theta$,
ניתן לבנות קטע קו
$\overline{AS}=\overline{PQ}$
בכיוון 
$\theta$.
\end{theorem}

%\begin{center}
%
%\vspace*{-8pt}
%\begin{tikzpicture}[scale=.8]
%\coordinate (A) at (0,0);
%\coordinate (P) at (1,-1.5);
%\coordinate (Q) at (2.5,-1.5);
%\draw (P) -- node[below] {$a$} (Q);
%\fill (P) node[left] {$P$} circle[radius=1.5pt];
%\fill (Q) node[right] {$Q$} circle[radius=1.5pt];
%\coordinate (A1) at (-4,1);
%\draw[thick,dashed] (A1) -- ++(60:3cm) coordinate (H1);
%\draw[thick,dashed] (A1) -- ++(0:3cm);
%\fill (A1) node[left] {$A'$} circle[radius=1.5pt];
%\fill (H1) node[left] {$H'$} circle[radius=1.5pt];
%\draw (A) -- node[left] {$a$} ++(60:1.5cm) coordinate (S);
%\fill (S) node[above right] {$S$} circle[radius=1.5pt];
%\draw[thick,dashed] (A) -- ++(3,0);
%\fill (A) node[left] {$A$} circle[radius=1.5pt];
%\node[above right,xshift=4pt] at (A1) {$\theta$};
%\node[above right,xshift=4pt] at (A) {$\theta$};
%\end{tikzpicture}
%\vspace*{-12pt}
%\end{center}

המשמעות של "בכיוון 
$\theta$"
היא שהזווית בין 
$\overline{AS}$
לקו המקביל ל-%
$\overline{PQ}$
דרך
$A$
היא
$\theta$.

בסוף סעיף
\ref{s.parallel}
הראינו שאפשר להעתיק קטע קו מקביל לעצמו. כאן המטרה היא להעתיק את קטע הקו
$\overline{PQ}$
ל-%
$\overline{AS}$,
כך ש-%
$\overline{AS}$
יהיה באותה זווית
$\theta$
יחסית לאותו ציר. באיור
$\overline{PQ}$
נמצא על ציר ה-%
$x$
אבל אין לזה חשיבות
(איור%
~\ref{f.se-copy1}).


\begin{proof}
נניח שהזווית הנתונה
$\theta$
היא הזווית בין קטע הקו
$\overline{A'H'}$
לקו מקביל ל-%
$\overline{PQ}$
המכיל את
$A'$.
לפי משפט~%
\ref{thm.parallel-line}
ניתן לבנות קטע קו
$\overline{AH}$
כך ש-%
$\overline{AH}\|\overline{A'H'}$,
וקטע קו
$\overline{AK}$
כך ש-%
$\overline{AK}\|\overline{PQ}$
ו-%
$\overline{AK}=\overline{PQ}=a$.

\begin{figure}[htb]
\begin{center}
\begin{tikzpicture}[scale=.7]
\coordinate (A) at (0,0);
\coordinate (P) at (3cm,2);
\coordinate (Q) at (4.5cm,2);
\draw (P) -- (Q);
\node[left] at (P) {$P$};
\node[right] at (Q) {$Q$};
\coordinate (A1) at (-3,1);
\draw (A1) -- ++(60:3cm) coordinate (H1);
\draw (A1) -- ++(0:2cm);
\node[left] at (A1) {$A'$};
\node[left] at (H1) {$H'$};
\draw (A) -- ++(60:1.5cm) coordinate (S);
\node[left] at (S) {$S$};
\draw (A) -- ++(1.5,0);
\node[left] at (A) {$A$};
\node[above right,xshift=4pt] at (A1) {$\theta$};
\node[above right,xshift=4pt] at (A) {$\theta$};
\draw (A) -- ++(60:3cm) coordinate (H);
\node[left] at (H) {$H$};
\draw (A) -- ++(1.5,0) coordinate (K);
\node[right] at (K) {$K$};
\vertex{P};
\vertex{Q};
\vertex{A};
\vertex{S};
\end{tikzpicture}
\end{center}
\caption{העתקת קו בכיוון נתון}\label{f.se-copy1}
\end{figure}

%\begin{center}
%
%\vspace*{-4pt}
%\begin{tikzpicture}[scale=.8]
%\coordinate (A) at (0,0);
%\coordinate (P) at (1,-1.5);
%\coordinate (Q) at (2.5,-1.5);
%\draw (P) -- node[below] {$a$} (Q);
%\fill (P) node[left] {$P$} circle[radius=1.5pt];
%\fill (Q) node[right] {$Q$} circle[radius=1.5pt];
%\coordinate (A1) at (-3,1);
%\draw (A1) -- ++(60:3cm) coordinate (H1);
%\draw[thick,dashed] (A1) -- ++(0:1.5cm);
%\fill (A1) node[left] {$A'$} circle[radius=1.5pt];
%\fill (H1) node[left] {$H'$} circle[radius=1.5pt];
%\draw (A) -- ++(60:3cm) coordinate (H);
%\fill (H) node[left] {$H$} circle[radius=1.5pt];
%\draw (A) -- ++(1.5,0) coordinate (K);
%\fill (K) node[below right] {$K$} circle[radius=1.5pt];
%\draw[dashed] (A) -- ($(A)!2!(K)$);
%\draw (A) -- node[below] {$a$} (K);
%\fill (A) node[left] {$A$} circle[radius=1.5pt];
%\node[above right,xshift=4pt] at (A1) {$\theta$};
%\node[above right,xshift=4pt] at (A) {$\theta$};
%\end{tikzpicture}
%\vspace*{-8pt}
%\end{center}
כעת יש למצוא נקודה
$S$
על
$\overline{AH}$
כך ש-%
$\overline{AS}=\overline{AK}$.
במעגל הקבוע בנו שני רדיוסים
$\overline{OU}$
ו-%
$\overline{OV}$
מקביליים ל-%
$\overline{AH}$
ו-%
$\overline{AK}$,
בהתאמה, ובנו קרן דרך
$K$
המקבילה ל-%
$\overline{UV}$.
סמנו את נקודת החיתוך של הקו עם
$\overline{AH}$
ב-%
$S$.
(איור%
~\ref{f.se-copy3}).

לפי הבניה
$\overline{AH}\|\overline{OU}$
ו-%
$\overline{AK}\|\overline{OV}$,
ולכן
$\angle SAK=\angle UOV=\theta$.
$\overline{SK}\|\overline{UV}$,
ו-%
$\triangle SAK\sim \triangle UOV$
לפי זווית-זווית-זווית.
$\triangle UOV$
הוא שווה-שוקיים כי
$\overline{OU}$, $\overline{OV}$
הם רדיוסים של אותו מעגל. מכאן ש-%
$\triangle SAK$
הוא שווה-שוקיים ו-%
$\overline{AS}=\overline{AK}=\overline{PQ}$.
\end{proof}

\begin{figure}[htb]
\begin{center}
\begin{tikzpicture}[scale=.7]
\coordinate (A) at (0,0);
\coordinate (P) at (3cm,2);
\coordinate (Q) at (4.5cm,2);
\draw (P) -- (Q);
\node[left] at (P) {$P$};
\node[right] at (Q) {$Q$};
\coordinate (A1) at (-3,1);
\draw (A1) -- ++(60:3cm) coordinate (H1);
\node[left] at (A1) {$A'$};
\node[left] at (H1) {$H'$};
\node[left] at (A) {$A$};
\draw (A) -- ++(60:3cm) coordinate (H);
\node[left] at (H) {$H$};
\draw (A) -- ++(1.5,0) coordinate (K);
\node[right] at (K) {$K$};
\draw (A) -- (K);
\path (A) -- ++(60:1.5cm) coordinate (S);
\node[right] at (S) {$S$};
\draw (K) -- ($(K)!1.8!(S)$);
\node[above right,xshift=4pt] at (A) {$\theta$};
\node[above right,xshift=4pt] at (A1) {$\theta$};
\draw (A1) -- ++(1.5,0);
\vertex{P};
\vertex{Q};
\begin{scope}[xshift=3cm]
\coordinate (O) at (6,1);
\draw[name path=circle] (O) circle[radius=2.5cm];
\node[above left] at (O) {$O$};
\path[name path=u] (O) -- ++(60:2.5cm);
\path[name path=v] (O) -- ++(2.5,0);
\path[name intersections={of=circle and u,by={U}}];
\path[name intersections={of=circle and v,by={V}}];
\node[above right] at (U) {$U$};
\node[right] at (V) {$V$};
\draw (O) -- (U) -- (V) -- cycle;
\node[above right,xshift=4pt] at (O) {$\theta$};
\vertex{O};
\end{scope}
\end{tikzpicture}
\end{center}
\caption{שימוש במעגל הקבוע כדי להעתיק קטע קו}\label{f.se-copy3}
\end{figure}


%\begin{center}
%\begin{tikzpicture}[scale=.8]
%\coordinate (A) at (0,0);
%\coordinate (P) at (1,-1.5);
%\coordinate (Q) at (2.5,-1.5);
%\draw (P) -- (Q);
%\fill (P) node[left] {$P$} circle[radius=1.5pt];
%\fill (Q) node[right] {$Q$} circle[radius=1.5pt];
%\coordinate (A1) at (-3,1);
%\draw (A1) -- ++(60:3cm) coordinate (H1);
%\fill (A1) node[left] {$A'$} circle[radius=1.5pt];
%\fill (H1) node[left] {$H'$} circle[radius=1.5pt];
%\draw (A) -- ++(60:3cm) coordinate (H);
%\fill (A) node[left] {$A$} circle[radius=1.5pt];
%\fill (H) node[left] {$H$} circle[radius=1.5pt];
%\coordinate (O) at (6,1);
%%\node at (4.8,3.4) {$c$};
%\draw[name path=circle] (O) circle[radius=2.5cm];
%\fill (O) node[above left] {$O$} circle[radius=1.5pt];
%\draw (A) -- ++(1.5,0) coordinate (K);
%\fill (K) node[below right] {$K$} circle[radius=1.5pt];
%\draw (A) -- (K);
%\path[name path=u] (O) -- ++(60:2.5cm);
%\path[name path=v] (O) -- ++(2.5,0);
%\path[name intersections={of=circle and u,by={U}}];
%\path[name intersections={of=circle and v,by={V}}];
%\fill (U) node[above right] {$U$} circle[radius=1.5pt];
%\fill (V) node[right] {$V$} circle[radius=1.5pt];
%\draw (O) -- (U) -- (V) -- cycle;
%\path (A) -- ++(60:1.5cm) coordinate (S);
%\fill (S) node[right] {$S$} circle[radius=1.5pt];
%\draw (K) -- ($(K)!1.8!(S)$);
%\draw (A) -- (S);
%\node[above right,xshift=4pt] at (A) {$\theta$};
%\node[above right,xshift=4pt] at (O) {$\theta$};
%\node[above right,xshift=4pt] at (A1) {$\theta$};
%\draw[thick,dashed] (A1) -- ++(1.5,0);
%\end{tikzpicture}
%%\vspace*{-4pt}
%\end{center}

%%%%%%%%%%%%%%%%%%%%%%%%%%%%%%%%%%%%%%%%%%%%%%%%%%%%%%%%%%%%%%%

\section{בניית קטע קו יחסית לקטעי קו אחרים}\label{s.relative-straight}

\begin{theorem}\label{thm.three-lines}
נתונים שלושה קטעי קו באורכים
$n, m, s$,
ניתן לבנות קטע קו באורך
$x=\disfrac{n}{m}s$.
\end{theorem}

\begin{proof}
קטעי הקו הנתונים נמצאים במקומות שרירותיים במישור:
%\begin{center}
%
%\begin{tikzpicture}[scale=.9]
%\draw (0,0) -- node[above] {$s$} ++(30:1.5cm);
%\draw (2,1.2) -- node[above] {$m$} ++(-10:2.5cm);
%\draw (-2,1.5) -- node[above] {$n$} ++(5:2cm);
%\fill (0,0) circle[radius=1.5pt];
%\fill (2,1.2) circle[radius=1.5pt];
%\fill (-2,1.5) circle[radius=1.5pt];
%\fill (0,0) ++(30:1.5cm) circle[radius=1.5pt];
%\fill (2,1.2) ++(-10:2.5cm) circle[radius=1.5pt];
%\fill (-2,1.5) ++(5:2cm) circle[radius=1.5pt];
%\end{tikzpicture}
%\vspace*{-8pt}
%\end{center}
בחרו נקודה 
$A$
במישור ובנו שני קטעי קו 
$\overline{AB},\overline{AC}$.
לפי משפט
\ref{thm.angle}
ניתן לבנות
$M,N,S$
כך ש-%
$\overline{AM}= m,\overline{AN} =n, \overline{AS}=s$.
בנו דרך
$N$
קו המקביל ל-%
$\overline{MS}$
שחותך את
$\overline{AC}$
ב-%
$X$,
וסמנו את אורכו של ב-%
$x$
(איור%
~\ref{f.se-three2}).

$\triangle MAS\sim \triangle NAX$
לפי זווית-זווית-זווית, ולכן:
\[
\frac{m}{n}=\frac{s}{x}, \quad\quad x=\disfrac{n}{m}s\,.
\]
\end{proof}

\begin{figure}[htb]
\begin{center}
\begin{tikzpicture}[scale=.8]
\coordinate (A) at (0,0);
\draw[name path=ac] (A) node[left] {$A$} -- ++(7,0) node[right] {$C$};
\draw (A) -- ++(40:5cm) node[right] {$B$};
\path (A) -- node[above,xshift=-2pt] {$m$} ++(40:3cm) coordinate (M) node[above left] {$M$};
\path (A) -- ++(40:4cm) coordinate (N) node[above left] {$N$};
\path[name path=ms] (M) -- ++(-50:3.5cm);
\path[name path=nx] (N) -- ++(-50:4cm);
\path[name intersections={of=ac and ms,by={S}}];
\path[name intersections={of=ac and nx,by={X}}];
\node[below] at (S) {$S$};
\node[below] at (X) {$X$};
\path (A) -- node[below] {$s$} (S);
\draw (S) -- (M);
\draw (X) -- (N);
\draw[<->] ($(A)+(0,-.8)$) -- node[fill=white] {$x$} ($(X)+(0,-.8)$);
\draw[<->] ($(A)+(-.6,.8)$) -- node[fill=white] {$n$} ++(40:3.9cm);
\end{tikzpicture}
\end{center}
\caption{בניית יחס אורכים באמצעות משולשים דומים}\label{f.se-three2}
\end{figure}


%\begin{center}
%
%\vspace*{-10pt}
%\begin{tikzpicture}
%\coordinate (A) at (0,0);
%\draw[name path=ac] (A) node[left] {$A$} -- ++(7,0) node[right] {$C$};
%\draw (A) -- ++(40:5cm) node[right] {$B$};
%\fill (A) circle[radius=1.5pt];
%\fill (A) ++(40:5cm) circle[radius=1.5pt];
%\fill (A) ++(7,0) circle[radius=1.5pt];
%\path (A) -- node[above,xshift=-2pt] {$m$} ++(40:3cm) coordinate (M) node[above left] {$M$};
%\path (A) -- ++(40:4cm) coordinate (N) node[above left] {$N$};
%\fill (M) circle[radius=1.5pt];
%\fill (N) circle[radius=1.5pt];
%\path[name path=ms] (M) -- ++(-50:3.5cm);
%\path[name path=nx] (N) -- ++(-50:4cm);
%\path[name intersections={of=ac and ms,by={S}}];
%\path[name intersections={of=ac and nx,by={X}}];
%\fill (S) circle[radius=1.5pt] node[below] {$S$};
%\fill (X) circle[radius=1.5pt] node[below] {$X$};
%\path (A) -- node[below] {$s$} (S);
%\draw (S) -- (M);
%\draw (X) -- (N);
%\draw[<->] ($(A)+(0,-16pt)$) -- node[fill=white] {$x$} ($(X)+(0,-16pt)$);
%\draw[<->] ($(A)+(-14pt,-2pt)$) -- node[fill=white] {$n$} +(40:4cm);
%\draw[<->] ($(A)+(0,32pt)$) -- node[fill=white] {$n$} +(40:4cm);
%\draw[thick,dotted] (A) -- ($(A)+(0,32pt)$);
%\draw[thick,dotted] (N) -- ($(N)+(0,32pt)$);
%\node at (7,2.5) {$AN=n$};
%\node at (7,2) {$AX=x$};
%\end{tikzpicture}
%\end{center}

%%%%%%%%%%%%%%%%%%%%%%%%%%%%%%%%%%%%%%%%%%%%%%%%%%%%%%%%%%%%%%%

\section{בניית שורש ריבועי}\label{s.root}

\begin{theorem}\label{thm.root}
נתון קטעי קו באורכים
$a,b$,
ניתן לבנות קטע קו שאורכו
$\sqrt{ab}$.
\end{theorem}

\begin{proof}
אם נבטא את
$x=\sqrt{ab}$
בצורה
$x=\disfrac{n}{m}s$
נוכל להשתמש במשפט~%
\ref{thm.three-lines}.

עבור
$n$
נשתמש ב-%
$d$,
הקוטר של המעגל הקבוע.

עבור
$m$
נשתמש ב-%
$t=a+b$
שניתן לבנות מ-%
$a,b$
לפי משפט
\ref{thm.angle}.

נבנה את
$s=\sqrt{hk}$
כאשר
$h=\disfrac{d}{t}a$, $k=\disfrac{d}{t}b$,
ונחשב:
%

\[
x=\sqrt{ab}=\sqrt{\frac{th}{d}\frac{tk}{d}}=\sqrt{\left(\frac{t}{d}\right)^2hk}=\frac{t}{d}\sqrt{hk}=\frac{t}{d}s\,.
\]
%

נחשב גם: 
\[
h+k = \frac{d}{t}a + \frac{d}{t}b = \frac{d(a+b)}{t} = \frac{dt}{t} = d\,.
\]


לפי משפט~%
\ref{thm.angle}
בנו
$\overline{HA}= h$
על הקוטר
$\overline{HK}$
של המעגל הקבוע. מ-%
$h+k=d$
אפשר להסיק ש-%
$\overline{AK}=k$
(איור%
~\ref{f.se-sqrt}).


\begin{figure}[htb]
\begin{center}
\begin{tikzpicture}[scale=.7]
\coordinate (O) at (0,0);
\coordinate (H) at (-3,0);
\coordinate (K) at (3,0);
\node at (-2.4,2.4) {$c$};
\draw (H) -- (K);
\draw[name path=circle] (O) circle[radius=3cm];
\node[below] at (O) {$O$};
\node[left] at (H) {$H$};
\node[right] at (K) {$K$};
\path[name path=as] (1,0) coordinate (A) -- ++(0,3.2);
\node[below] at (A) {$A$};
\path[name intersections={of=circle and as,by={S}}];
\node[above] at (S) {$S$};
\draw (A) -- node[right] {$s$} (S);
\path (H) -- node[above] {$h$} (A);
\path (A) -- node[above] {$k$} (K);
\draw (O) -- node[left,xshift=-2pt] {$\displaystyle\frac{d}{2}$} (S);
\node at (.5,-1.5) {$\displaystyle\frac{d}{2}-k$};
\draw[->] (.5, -1.2) -- ++(0,1);
\draw[rotate=90] (A) rectangle +(8pt,8pt);
\vertex{O};
\end{tikzpicture}
\end{center}
\caption{בניית שורש ריבועי}\label{f.se-sqrt}
\end{figure}

%\begin{center}
%
%\vspace*{-4pt}
%\begin{tikzpicture}[scale=.8]
%\coordinate (O) at (0,0);
%\coordinate (H) at (-3,0);
%\coordinate (K) at (3,0);
%\node at (-2.4,2.4) {$c$};
%\draw (H) -- (K);
%\draw[name path=circle] (O) circle[radius=3cm];
%\fill (O) node[below] {$O$} circle[radius=1.5pt];
%\fill (H) node[left] {$H$} circle[radius=1.5pt];
%\fill (K) node[right] {$K$} circle[radius=1.5pt];
%\path[name path=as] (1,0) coordinate (A) -- ++(0,3.2);
%\fill (A) node[below] {$A$} circle[radius=1.5pt];
%\path[name intersections={of=circle and as,by={S}}];
%\fill (S) node[above] {$S$} circle[radius=1.5pt];
%\draw (A) -- node[right,yshift=-6pt] {$\sqrt{hk}$} node[right,near end,yshift=-6pt] {$s=$} (S);
%\path (H) -- node[below] {$h$} (A);
%\path (A) -- node[below] {$k$} (K);
%\draw[thick,dashed] (O) -- node[left,xshift=-4pt] {$\disfrac{d}{2}$} (S);
%\node at (.5,-1.5) {$\disfrac{d}{2}-k$};
%\draw[thick,->] (.5, -1.2) -- ++(0,1);
%\draw (.8,0) -- ++(0,.2) -- ++(.2,0);
%\end{tikzpicture}
%\end{center}
לפי משפט~%
\ref{thm.perpendicular}
ניתן לבנות דרך
$A$
אנך ל-%
$\overline{HK}$.
סמנו ב-%
$S$
את החיתוך שלו עם המעגל הקבוע.
$\overline{OS}\!=\!\overline{OK}\!=\!\disfrac{d}{2}$
הם רדיוסים של המעגל, ו-%
$\overline{OA}\!=\!\disfrac{d}{2}\!-\!k$.
לפי משפט פיתגורס:

\begin{eqn}
s^2=\overline{SA}^2 &=& \left(\disfrac{d}{2}\right)^2 - \left(\disfrac{d}{2}-k\right)^2\\
&=& \left(\disfrac{d}{2}\right)^2 - \left(\disfrac{d}{2}\right)^2 + 2\disfrac{dk}{2} - k^2\\
&=& k(d-k)=kh\\
s&=&\sqrt{hk}\,.
\end{eqn}
כעת ניתן לבנות
$x=\disfrac{t}{d}s$
לפי משפט~%
\ref{thm.three-lines}.
\end{proof}

%%%%%%%%%%%%%%%%%%%%%%%%%%%%%%%%%%%%%%%%%%%%%%%%%%%%%%%%%%%%%%%

\section{בניית נקודות חיתוך של קו עם מעגל}\label{s.line-circle-straight}

\begin{theorem}\label{thm.line-circle}
נתון קו
$l$
ומעגל
$c$
שמרכזו
$O$
ורדיוס
$r$.
ניתן לבנות את נקודות החיתוך של
$l$
עם
$c$.
(איור%
~\ref{f.se-line-circle1}).
\end{theorem}

\begin{figure}[htb]
\begin{center}
\begin{tikzpicture}[scale=.7]
\coordinate (O) at (0,0);
\node[below right] at (O) {$O$};
\vertex{O};
\draw[thick,dashed,name path=circle] (O) circle[radius=3cm];
\draw (O) -- node[above] {$r$} ++(-130:3cm) coordinate (R);
\draw[name path=l] (O) ++(170:4cm) --
  node[below, near end,xshift=30pt,yshift=10pt] {$l$} ++(20:8cm);
\path[name intersections={of=circle and l,by={Y,X}}];
\node[above left] at (X) {$X$};
\node[above right] at (Y) {$Y$};
\end{tikzpicture}
\end{center}
\caption{בניית נקודות החיתוך של קו ומעגל (1)}\label{f.se-line-circle1}
\end{figure}


%\begin{center}
%
%\begin{tikzpicture}[scale=.7]
%\coordinate (O) at (0,0);
%\node at (2.6,-2) {$c$};
%\draw[thick,dashed] (O) circle[radius=3cm];
%\fill (O) node[below right] {$O$} circle[radius=1.5pt];
%\draw (O) -- node[right] {$r$} ++(-110:3cm) coordinate (R);
%\fill (R) circle[radius=1.5pt] node[above right,xshift=2pt] {$R$};
%\draw (O) +(170:4cm) -- node[below, near end,xshift=40pt,yshift=8pt] {$l$} ++(30:4.5cm);
%\end{tikzpicture}
%\end{center}

\begin{proof}
לפי משפט~%
\ref{thm.perpendicular}
ניתן לבנות אנך ממרכז המעגל
$O$
לקו
$l$.
סמנו ב-%
$M$
את נקודת החיתוך של
$l$
עם האנך. 
$M$
חוצה של המיתר 
$\overline{XY}$,
כאשר 
$X,Y$
הן נקודות החיתוך של הקו עם המעגל
(איור%
~\ref{f.se-line-circle2}).

סמנו ב-%
$2s$
את אורך המיתר. שימו לב שבאיור
$s,X,Y$
הם רק הגדרות וטרם בנינו את נקודות החיתוך.

\begin{figure}[htb]
\begin{center}
\begin{tikzpicture}[scale=.7]
\coordinate (O) at (0,0);
\draw[thick,dashed,name path=circle] (O) circle[radius=3cm];
\node[below right] at (O) {$O$};
\vertex{O};
\path (O) --  ++(-130:3cm) coordinate (R);
\node[below left,yshift=2pt,xshift=2pt] at (R) {$R$};
\draw[name path=l] (O) ++(170:4cm) --
  node[below, near end,xshift=30pt,yshift=10pt] {$l$} ++(20:8cm);
\path[name intersections={of=circle and l,by={Y,X}}];
\node[above left] at (X) {$X$};
\node[above right] at (Y) {$Y$};
\draw (O) -- node[below] {$r$} (X);
\path (X) -- ($(X)!.5!(Y)$) coordinate (M);
\node[above] at (M) {$M$};
\draw (O) -- node[right] {$t$} (M);
\path (X) -- node[above] {$s$} (M);
\path (M) -- node[above] {$s$} (Y);
\draw (O) ++(170:4cm) -- ++(20:3.1cm) -- ++(-70:10pt) -- ++(20:10pt);
\draw (O) -- node[below] {$t$} +(50:2) coordinate (RTT);
\draw (O) -- node[below] {$t$} +(-130:2) coordinate (RT);
\vertex{RT};
\draw (RT) -- node[right,yshift=-2pt] {$r-t$} ($(RT)+(-130:1cm)$);
\vertex{RTT};
\draw[<->] ($(RT)+(.5cm,-1.6cm)$) -- node[fill=white] {$r+t$}+(50:5);
\end{tikzpicture}
\end{center}
\caption{בניית נקודות החיתוך של קו ומעגל (2)}\label{f.se-line-circle2}
\end{figure}


%\begin{center}
%
%\begin{tikzpicture}[scale=.7]
%\coordinate (O) at (0,0);
%\node at (2.6,-2) {$c$};
%\draw[thick,dashed,name path=circle] (O) circle[radius=3cm];
%\fill (O) node[below right] {$O$} circle[radius=1.5pt];
%\draw (O) -- node[right] {$r$} ++(-110:3cm) coordinate (R);
%\fill (R) node[above right,xshift=2pt] {$R$} circle[radius=1.5pt];
%\draw[name path=l] (O) ++(170:4cm) -- node[below, near end,xshift=40pt,yshift=12pt] {$l$} ++(20:8cm);
%\path[name intersections={of=circle and l,by={Y,X}}];
%\fill (X) node[above left] {$X$} circle[radius=1.5pt];
%\fill (Y) node[above right,yshift=2pt] {$Y$} circle[radius=1.5pt];
%\draw (O) -- node[below] {$r$} (X);
%\path (X) -- ($(X)!.5!(Y)$) coordinate (M);
%\fill (M) node[above] {$M$} circle[radius=1.5pt];
%\draw (O) -- node[right] {$t$} (M);
%\path (X) -- node[above] {$s$} (M);
%\path (M) -- node[above] {$s$} (Y);
%\draw (O) ++(170:4cm) -- ++(20:3.1cm) -- ++(-70:10pt) -- ++(20:10pt);
%\end{tikzpicture}
%\end{center}
$\triangle OMX$
הוא מעגל ישר-זווית ולכן
$s^2=r^2-t^2=(r+t)(r-t)$.
$r$
נתון כרדיוס המעגל ו-%
$t$
הוגדר כאורך של
$\overline{OM}$.
לפי משפט~%
\ref{thm.angle}
ניתן לבנות קטעי קו באורך
$r$
מהנקודה 
$O$
בשני הכיוונים
$\overline{MO}$
ו-%
$\overline{OM}$.
התוצאה היא שני קטעי קו שאורכם
$r+t,r-t$.

לפי משפט~%
\ref{thm.root}
ניתן לבנות קטע קו באורך
$s=\sqrt{(r+t)(r-t)}$.
שוב לפי משפט~%
\ref{thm.angle},
ניתן לבנות קטעי קו באורך 
$s$
על הקו הנתון
$l$
מהנקודה
$M$
בשני הכיוונים. הקצה השני של כל אחד מקטעי הקו האלה הוא נקודת חיתוך של 
$l$
עם המעגל.
\end{proof}

%%%%%%%%%%%%%%%%%%%%%%%%%%%%%%%%%%%%%%%%%%%%%%%%%%%%%%%%%%%%%%%

\section{בניית נקודות החיתוך של שני מעגלים}\label{s.circle-circle}

\begin{theorem}\label{thm.two-circles}
נתונים שני מעגלים עם מרכזים
$O_1,O_2$
ורדיוסים
$r_1,r_2$.
ניתן לבנות את נקודות החיתוך שלהם
$X,Y$.
\end{theorem}

\begin{proof}
בנו את קטע הקו
$\overline{O_1O_2}$
המחבר את שני המרכזים. סמנו את אורכו ב-%
$t$
(איור%
~\ref{f.se-circle-circle1}).


\begin{figure}[htb]
\begin{center}
\begin{tikzpicture}[scale=.9]
\coordinate (O1) at (0,0);
\coordinate (O2) at (2.5,0);
\node[below left] at (O1) {$O_1$};
\node[below right] at (O2) {$O_2$};
\vertex{O1};
\vertex{O2};
\draw[thick,dashed,name path=circle1] (O1) circle[radius=2cm];
\draw[thick,dashed,name path=circle2] (O2) circle[radius=1.6cm];
\path [name intersections={of=circle1 and circle2,by={X,Y}}];
\node[above,yshift=4pt] at (X) {$X$};
\node[below,yshift=-4pt] at (Y) {$Y$};
\draw (O1) -- node[above] {$r_1$} ++(160:2cm);
\draw (O2) -- node[above] {$r_2$} ++(30:1.6cm);
\draw (O1) -- (O2);
\node at (-1.7,1.6) {$c_1$};
\node at (3.8,1.4) {$c_2$};
\draw[<->] (0,-.6) -- node[fill=white] {$t$} +(2.5,0);
\end{tikzpicture}
\end{center}
\caption{בניית החיתוך של שני מעגלים (1)}\label{f.se-circle-circle1}
\end{figure}

%\begin{center}
%
%\begin{tikzpicture}[scale=1.2]
%\coordinate (O1) at (0,0);
%\coordinate (O2) at (2.5,0);
%\fill (O1) node[below left] {$O_1$} circle[radius=1.5pt];
%\fill (O2) node[below right] {$O_2$} circle[radius=1.5pt];
%\draw[thick,dashed,name path=circle1] (O1) circle[radius=2cm];
%\draw[thick,dashed,name path=circle2] (O2) circle[radius=1.6cm];
%\path [name intersections={of=circle1 and circle2,by={X,Y}}];
%\draw (O1) -- node[above] {$r_1$} ++(160:2cm);
%\draw (O2) -- node[above] {$r_2$} ++(30:1.6cm);
%\fill (O1) ++(160:2cm) circle[radius=1.5pt];
%\fill (O2) ++(30:1.6cm) circle[radius=1.5pt];
%\draw (O1) -- (O2);
%\node at (-1.7,1.6) {$c_1$};
%\node at (3.8,1.4) {$c_2$};
%\draw[<->] (0,-1) -- node[fill=white] {$t$} (2.5,-1);
%%\node at (6,0) {$t=O_1O_2$};
%\end{tikzpicture}
%\end{center}
סמנו ב-%
$A$
את נקודת החיתוך של
$\overline{O_1O_2}$
עם
$\overline{XY}$,
וסמנו את
$q=\overline{O_1A},x=\overline{XA}$
(איור%
~\ref{f.se-circle-circle2}).

טרם בנינו את הנקודה
$A$,
אבל אם נצליח לבנות את האורכים
$q,x$,
לפי משפט~% 
\ref{thm.angle}
נוכל לבנות את 
$A$
באורך
$q$
מהנקודה
$O_1$
לכיוון
$\overline{O_1O_2}$.
לפי משפט~%
\ref{thm.perpendicular}
ניתן לבנות את האנך ל-%
$\overline{O_1O_2}$
בנקודה
$A$,
ושוב לפי משפט~%
\ref{thm.angle}
ניתן לבנות קטעי קו באורך
$x$
מהנקודה
$A$
בשני הכיוונים לאורך האנך.
$X,Y$,
הקצות של קטעי הקו, הם נקודות החיתוך של שני המעגלים.

\begin{figure}[htb]
\begin{center}
\begin{tikzpicture}[scale=.9]
\coordinate (O1) at (0,0);
\coordinate (O2) at (2.5,0);
\vertex{O1};
\vertex{O2};
\node[below left] at (O1) {$O_1$};
\node[below right] at (O2) {$O_2$};
\draw[thick,dashed,name path=circle1] (O1) circle[radius=2cm];
\draw[thick,dashed,name path=circle2] (O2) circle[radius=1.6cm];
\path [name intersections={of=circle1 and circle2,by={X,Y}}];
\node[above,yshift=4pt] at (X) {$X$};
\node[below,yshift=-4pt] at (Y) {$Y$};
\draw (O1) -- node[above,xshift=-4pt] {$r_1$} (X);
\draw (O2) -- node[above,xshift=4pt] {$r_2$} (X);
\draw[name path=oo] (O1) -- (O2);
\node at (-1.7,1.6) {$c_1$};
\node at (3.8,1.4) {$c_2$};
\draw[name path=xy] (X) -- (Y);
\path[name intersections={of=xy and oo,by={A}}];
\node[below left] at (A) {$A$};
\draw (A) rectangle +(6pt,6pt);
\path (O1) -- node[below,xshift=-2pt] {$q$} (A);
\path (X) -- node[left,yshift=-2pt] {$x$} (A);
\draw[<->] (0,-.6) -- node[fill=white] {$t$} +(2.5,0);
\end{tikzpicture}
\end{center}
\caption{בניית החיתוך של שני מעגלים (2)}\label{f.se-circle-circle2}
\end{figure}

%\begin{center}
%
%\begin{tikzpicture}[scale=1.2]
%\coordinate (O1) at (0,0);
%\coordinate (O2) at (2.5,0);
%\fill (O1) node[below left] {$O_1$} circle[radius=1.5pt];
%\fill (O2) node[below right] {$O_2$} circle[radius=1.5pt];
%\draw[thick,dashed,name path=circle1] (O1) circle[radius=2cm];
%\draw[thick,dashed,name path=circle2] (O2) circle[radius=1.6cm];
%\path [name intersections={of=circle1 and circle2,by={X,Y}}];
%\fill (X) node[above,yshift=4pt] {$X$} circle[radius=1.5pt];
%\fill (Y) node[below,yshift=-4pt] {$Y$} circle[radius=1.5pt];
%\draw[thick,dashed] (O1) -- node[above,xshift=-4pt] {$r_1$} (X);
%\draw[thick,dashed] (O2) -- node[above,xshift=4pt] {$r_2$} (X);
%\draw[name path=oo] (O1) -- (O2);
%\node at (-1.7,1.6) {$c_1$};
%\node at (3.8,1.4) {$c_2$};
%\draw[name path=xy] (X) -- (Y);
%\path[name intersections={of=xy and oo,by={A}}];
%\fill (A) node[below left] {$A$} circle[radius=1.5pt];
%\path (O1) -- node[below,xshift=-2pt] {$q$} (A);
%\path (X) -- node[left,yshift=-2pt] {$x$} (A);
%\draw[<->] (0,-1) -- node[fill=white] {$t$} (2.5,-1);
%\node at (6,.5) {$t=\overline{O_1O_2}$};
%\node at (6,0) {$q=\overline{O_1A}$};
%\node at (6,-.5) {$x=\overline{XA}$};
%\end{tikzpicture}
%\end{center}
\textbf{בניית האורך
$q$:}
בנו
$d=\sqrt{r_1^2+t^2}$, 
אורך היתר של משולש ישר-זווית מהאורכים הידועים
$r_1,t$:
על קו כלשהי במישור בנו קטע קו
$\overline{RS}$
באורך
$r_1$,
אחר כך בנו אנך ל-%
$\overline{RS}$
דרך
$R$,
ולבסוף בנו קטע קו
$\overline{RT}$
באורך 
$t$
מ-%
$R$
על האנך.

לפי משפט הקוסינוסים ב-%
$\triangle O_1O_2X$:

\begin{eqn}
r_2^2 &=& t^2 + r_1^2 - 2r_1t\cos\angle XO_1O_2\\
\disfrac{q}{r_1}&=&\cos\angle O_1O_2X\\
r_2^2&=& t^2 + r_1^2 - 2tq\\
2tq &=& (r_1^2+t^2) - r_2^2\\
q&=&\disfrac{(d+r_2)(d-r_2)}{2t}\,.
\end{eqn}


לפי משפט~%
\ref{thm.angle}
ניתן לבנות את האורכים האלה, ולפי משפט~%
\ref{thm.three-lines}
ניתן לבנות את
$q$
מהביטויים
$d+r_2,d-r_2,2t$.

\textbf{בניית האורך
$x$:}
$\triangle AO_1X$
הוא משולש ישר-זווית ולכן:
\[
x^2=r_1^2-q^2 =\sqrt{(r_1+q)(r_1-q)}\,.
\]
לפי משפט~%
\ref{thm.angle}
ניתן לבנות
$h =r_1+ q$
ו-%
$k= r_1 - q$,
ולפי משפט~%
\ref{thm.root}
ניתן לבנות
$x= \sqrt{hk}$. 
\end{proof}

%%%%%%%%%%%%%%%%%%%%%%%%%%%%%%%%%%%%%%%%%%%%%%%%%%

\subsection*{What Is the Surprise?}

חובה להשתמש במחוגה כי עם סרגל אפשר רק לחשב שורשים  משוואות ליניאריות ולא ערכים כגון
$\sqrt{2}$,
היתר של משולש ישר-זווית ושווה-שוקיים עם צלעות באורך
$1$.
לכן, מפתיע שקיום של מעגל אחד בלבד, ללא תלות במקומו של המרכז או של הרדיוס שלו מספיק כדי לבצע כל בניית שאפשרית עם סרגל ומחוגה.

%%%%%%%%%%%%%%%%%%%%%%%%%%%%%%%%%%%%%%%%%%%%%%%%%%

\subsection*{מקורות}
הפרק מבוסס על בעיה
$34$
ב-%
\L{\cite{dorrie1}}
שעובדה על ידי
\L{Michael Woltermann} \L{\cite{dorrie2}}. 
