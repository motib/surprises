% !TeX root = surprises.tex

\selectlanguage{hebrew}


\chapter{איך לרבע את המעגל}
\label{c.square}

\selectlanguage{hebrew}

לרבע את המעגל היא אחת מבעיות הבנייה שהיוונים ניסו לפתור ולא הצליחו. בניגוד לחלוקה זווית לשלושה חלקים ולהפכלת הקוביוה שאינן ניתנת לבנייה בגלל התכונות של השורשים של פולינומים, לא ניתן לרבע את המעגל בגלל ש-%
$\pi$
הוא טרנסנדנטי: הוא אינו פתרון של אף פולינום עם מקדמים רציונליים. ההוכחה מסובכת וניתנה רק בי-%
$1882$
כל ידי 
\L{Carl von Lindemann}.

קירובים ל-%
$\pi\approx 3.14159265359$
היו ידועים גם בעולם העתיק. קירובים די מדוייקים הם:
\[
\displaystyle\frac{22}{7}\approx 3.142857,\quad \displaystyle\frac{333}{106}\approx 3.141509,\quad \displaystyle\frac{355}{113}\approx 3.141593\,.
\]
נביא שלוש בניות על ידי סרגל ומעגל של קירובים ל-%
$\pi$.
בנייה אחת של
\L{Adam Kocha\'{n}ski}
(סעיף%
~\ref{s.square-kochanski})
ושתי בניות של
\L{Ramanujan}
(סעיפים%
~\ref{s.square-ramanujan-first}, \ref{s.square-ramanujan-second}).
סעיף%
~\ref{s.square-quad}
מסביר איך לרבע את המעגל באמצעות קוודרטריקס.


הטבלה שלהלן מביא את הנוסחאות של האורכים שננבנה, ערכם המקורב, ההפרש בין ערכים הללו והערך של 
$\pi$,
והשגיאה (במטרים) אם משתמשים בקירוב כדי לחשב את היקף כדור הארץ, כאשר נתון שהרדיוס הוא
$6378$
ק"מ.
\[
\renewcommand{\arraystretch}{1.1}
\begin{array}{lcccc}
\hline
\textrm{\R{בנייה}} & \textrm{\R{נוסחה}} &\textrm{\R{ערך}} & \textrm{\R{הפרש}} & \textrm{\R{שגיאה (מ)}}\\\hline\hline
\pi & \textrm{\rule[-15pt]{0pt}{25pt}}& 3.14159265359 & - & -\\
\textrm{Kochansky} & \textrm{\rule[-15pt]{0pt}{35pt}}\sqrt{\disfrac{40}{3}-2\sqrt{3}}&
  3.1415333871 & 5.932 \times 10^{-5} & 756\\
\textrm{Ramanujan}\; 1 & \textrm{\rule[-15pt]{0pt}{35pt}}\disfrac{355}{113} &
  3.14159292035 &2.667  \times 10^{-7}&3.4\\
\textrm{Ramanujan}\; 2 &\textrm{\rule[-15pt]{0pt}{35pt}}\left(9^2+\disfrac{19^2}{22}\right)^{1/4}&
  3.14159265258 & 1.007 \times 10^{-9}& 0.013\\\hline
\end{array}
\]


%%%%%%%%%%%%%%%%%%%%%%%%%%%%%%%%%%%%%%%%%%%%%%%%%%%%%%%%%%%%%%%

\section{הבנייה של
\L{\large Kochansky}}\label{s.square-kochanski}

\textbf{%
בנייה (איור%
~\ref{f.square-kochansky}):}
\begin{itemize}
\item
נבנה מעגל יחידה שמרכזו 
$O$
עם קוטר
$\overline{AB}$
ונבנה משיק למעגל ב-%
$A$.
\item
נבנה מעגל יחידה שמרכזו
$A$
ונסמן את החיתוך עם המעגל הראשון ב-%
$C$.
נבנה מעגל יחידה שמרכזו 
$C$
ונסמן את החיתוך שלו עם המעגל השני ב-%
$D$. 
\item
נבנה
$\overline{OD}$
ונסמן את החיתוך שלו עם המשיק ב-%
$E$.
\item
מ-%
$E$
נבנה
$F,G,H$,
כל אחת במרחק 
$1$
מהנקודה הקודמת.
\item
נבנה
$\overline{BH}$.
\end{itemize}

\begin{figure}[tb]
\begin{center}
\begin{tikzpicture}[scale=.55]
% Scale at 4

% Coordinates of circle
\coordinate (O) at (0,0);
\coordinate (A) at (0,-4);
\coordinate (B) at (0,4);
\node[above right] at (O) {$O$};
\vertex{O};
\node[below right] at (A) {$A$};
\node[above right] at (B) {$B$};
\draw (A) rectangle +(14pt,14pt);

% Draw circle and diameter
\node [draw,circle through=(A),name path=circle] at (O) {};
\draw (A) --  node[right] {$1$} (O) -- node[right] {$1$} (B);

% Draw tangent at A
\draw[name path=tangent] ($(A)+(-4.5,0)$) -- ($(A)+(10.5,0)$);

% Draw arc centered at A which intersects circle at C
\draw[name path=Aarc] (O)
   arc [start angle=90,end angle=220,radius=4];
\path[name intersections={of=circle and Aarc,by=C}];
\node[left,xshift=-4pt] at (C) {$C$};

% Draw arc centered at C which intersects the first arc at D
\draw[name path=Carc] ($(C)+(200:4)$)
   arc [start angle=200,end angle=310,radius=4];
\path[name intersections={of=Carc and Aarc,by=D}];
\node[below left] at (D) {$D$};

% Draw O--D which intersects the tangent at E
\draw[name path=OD] (O) -- (D);
\path[name intersections={of=tangent and OD,by=E}];
\node[above left] at (E) {$E$};

% Find point H at length 3 from E
\coordinate (F) at ($(E)+(4,0)$);
\vertex{F};
\node[above right,xshift=6pt] at (F) {$F$};
\coordinate (G) at ($(F)+(4,0)$);
\vertex{G};
\node[above] at (G) {$G$};
\coordinate (H) at ($(G)+(4,0)$);
\node[above,xshift=2pt] at (H) {$H$};

% Draw BH of length approximately pi
\draw (B) -- (H);

\draw[<->] ($(E)+(0,-.8)$) -- node[fill=white] {$1$} ($(F)+(0,-.8)$);
\draw[<->] ($(F)+(0,-.8)$) -- node[fill=white] {$1$} ($(G)+(0,-.8)$);
\draw[<->] ($(G)+(0,-.8)$) -- node[fill=white] {$1$} ($(H)+(0,-.8)$);

\end{tikzpicture}
\end{center}
\selectlanguage{hebrew}
\caption{הקירוב של
\L{Kocha\'{n}ski}
ל-%
$\pi$}
\label{f.square-kochansky}
\end{figure}


%\begin{figure}[bt]
%\begin{center}
%
%\begin{tikzpicture}[scale=.5]
%% Scale at 4
%
%% Coordinates of circle
%\coordinate (O) at (0,0);
%\coordinate (A) at (0,-4);
%\coordinate (B) at (0,4);
%\fill (O) circle(2pt) node[above right] {$O$};
%\fill (A) circle(2pt) node[below right] {$A$};
%\fill (B) circle(2pt) node[above right] {$B$};
%\draw (A) rectangle +(12pt,12pt);
%
%% Draw circle and diameter
%\node [thick,draw,circle through=(A),name path=circle] at (O) {};
%\draw [thick] (A) -- (B);
%
%% Draw tangent at A
%\draw[thick,name path=tangent] ($(A)+(-4.5,0)$) -- ($(A)+(10.5,0)$);
%
%% Draw arc centered at A which intersects circle at C
%\draw[thick,name path=Aarc] (O)
%   arc [start angle=90,end angle=220,radius=4];
%\path[name intersections={of=circle and Aarc,by=C}];
%\fill (C) circle(2pt) node[left,xshift=-4pt] {$C$};
%
%% Draw arc centered at C which intersects the first arc at D
%\draw[thick,name path=Carc] ($(C)+(260:4)$)
%   arc [start angle=260,end angle=280,radius=4];
%\path[name intersections={of=Carc and Aarc,by=D}];
%\fill (D) circle(2pt) node[below left] {$D$};
%
%% Draw O--D which intersects the tangent at E
%\draw[thick,name path=OD] (O) -- (D);
%\path[name intersections={of=tangent and OD,by=E}];
%\fill (E) circle(2pt) node[above left] {$E$};
%
%% Find point H at length 3 from E
%\coordinate (F) at ($(E)+(4,0)$);
%\fill (F) circle(2pt) node[above right,xshift=4pt] {$F$};
%\coordinate (G) at ($(F)+(4,0)$);
%\fill (G) circle(2pt) node[above] {$G$};
%\coordinate (H) at ($(G)+(4,0)$);
%\fill (H) circle(2pt) node[above] {$H$};
%
%% Draw BH of length approximately pi
%\draw[thick] (B) -- (H);
%\end{tikzpicture}
%\end{center}
%\caption{הבנייה של \L{Kochansky}}\label{f.kochansky}
%\end{figure}

\begin{theorem}
$\overline{BH}=\sqrt{\disfrac{40}{3}-2\sqrt{3}}\approx \pi$.
\end{theorem}

\begin{proof}
איור~%
\ref{f.kochansky-proof}
מתמקד בחלק מאיור%
~\ref{f.square-kochansky}
כאשר נוספו קטעי הקו המקווקווים. בגלל שכל המעגלים הם מעגלי היחידה, אורכו של כל אחד מהם הוא 
$1$.
מכאן ש-%
$AOCD$
הוא מעויין ולכן האלכסונים שלו ניצבים זה לזה וחוצים זה את זה בנקודה שסומנה
$K$,
$\overline{AK}=1/2$.
\begin{figure}[tb]
\begin{center}

\begin{tikzpicture}[scale=1.2]
% Scale at 4

\clip (-4.5,-6.5) rectangle +(5,7);
% Coordinates of circle
\coordinate (O) at (0,0);
\coordinate (A) at (0,-4);
\coordinate (B) at (0,4);

% Draw circle
\node [circle through=(A),name path=circle] at (O) {};
\draw ($(O)+(200:4)$) arc [start angle=200,end angle=280,radius=4];

% Draw tangent at A
\draw[thick,name path=tangent] ($(A)+(-4.5,0)$) -- ($(A)+(10.5,0)$);
\draw (A) rectangle +(6pt,6pt);

% Draw arc centered at O which intersects circle at C
\draw[thin,name path=Aarc] (O)
   arc [start angle=90,end angle=220,radius=4];
\path[name intersections={of=circle and Aarc,by=C}];

% Draw arc centered at C which intersects the first arc at D
\draw[thin,name path=Carc] ($(C)+(260:4)$)
   arc [start angle=260,end angle=280,radius=4];
\path[name intersections={of=Carc and Aarc,by=D}];

% Draw O--D which intersects the tangent at E
\draw[thick,name path=OD] (O) -- (D);
\path[name intersections={of=tangent and OD,by=E}];

% Find point H at length 3 from E
\coordinate (F) at ($(E)+(4,0)$);
\coordinate (G) at ($(F)+(4,0)$);
\coordinate (H) at ($(G)+(4,0)$);

% Draw BH of length approximately pi
\draw[thick] (B) -- (H);

\draw[ultra thick,dashed] (A) -- (O) -- (C) -- (D) -- cycle;
\draw[ultra thick,dashed,name path=AC] (A) -- (C);

\node[above left,yshift=10pt,xshift=2pt] at (A) {$60^\circ$};
\node[left,yshift=10pt,xshift=-20pt] at (A) {$30^\circ$};

\path[name intersections={of=AC and OD,by=K}];
\draw[thick,rotate=-30] (K) rectangle +(6pt,6pt);

\path (A) -- node[above,xshift=2pt,yshift=2pt] {$1/2$} (K);
\path (A) -- node[below,xshift=-4pt] {$1/\sqrt{3}$} (E);

\fill (O) circle(1.25pt) node[above right] {$O$};
\fill (A) circle(1.25pt) node[below right] {$A$};
\fill (B) circle(1.25pt) node[above right] {$B$};
\fill (C) circle(1.25pt) node[left,xshift=-4pt] {$C$};
\fill (D) circle(1.25pt) node[below left] {$D$};
\fill (E) circle(1.25pt) node[above left] {$E$};
\fill (F) circle(1.25pt) node[above right,xshift=4pt] {$F$};
\fill (G) circle(1.25pt) node[above] {$G$};
\fill (H) circle(1.25pt) node[above] {$H$};
\fill (K) circle(1.25pt) node[above,yshift=4pt] {$K$};
\end{tikzpicture}
\selectlanguage{hebrew}
\caption{הוכחת הבנייה של Kochansky}\label{f.kochansky-proof}
\end{center}
\end{figure}

האלכסון
$\overline{AC}$
מייצר שני משולשים שווי-צלעות
$\triangle OAC, \triangle DAC$
כך ש-%
$\angle OAC=60^\circ$.
הזווית בין המשיק לרדיוס
$\overline{OA}$
היא זווית ישרה ולכן
$\angle KAE=30^\circ$.
נחשב:
\begin{eqn}
\disfrac{1/2}{\overline{EA}}&=&
\cos 30^\circ=\disfrac{\sqrt{3}}{2}\\
\overline{EA}&=&\disfrac{1}{\sqrt{3}}\\
\overline{AH}&=&3-\overline{EA}
=\left(3-\disfrac{1}{\sqrt{3}}\right)
=\disfrac{3\sqrt{3}-1}{\sqrt{3}}
\end{eqn}
$\triangle ABH$
הוא משולש ישר-זווית ו-%
$\overline{AH}=3-\overline{EA}$,
ןלכן לפי משפט פיתרגורס:
\begin{eqn}
\overline{BH}^2&=&\overline{AB}^2+\overline{AH}^2\\
%&=&2^2+\left(\disfrac{3\sqrt{3}-1}{\sqrt{3}}\right)^2\\
&=&4+\disfrac{27 -6\sqrt{3}+1}{3}=\disfrac{40}{3}-2\sqrt{3}\\
\overline{BH}&=&\sqrt{\disfrac{40}{3}-2\sqrt{3}}\approx 3.141533387\approx \pi\,.
\end{eqn}
\end{proof}

%%%%%%%%%%%%%%%%%%%%%%%%%%%%%%%%%%%%%%%%%%%%%%%%%%%%%%%%%%%%%%%%

\section{הבנייה הראשונה של
\L{\large Ramanujan}}\label{s.square-ramanujan-first}

\textbf{בנייה (איור%
~\ref{f.ramanujan1}):}
\begin{itemize}
\item
נבנה מעגל יחידה שמרכזו 
$O$
עם קוטר
$\overline{PR}$.
\item
נבנה נקודה
$H$
שחוצה את
$\overline{PO}$
ונקודה
$T$
כך ש-%
$\overline{TR}$
מחלק את 
$\overline{OR}$
לשלושה קטעים שווים.
\item
נבנה ניצב ב-%
$T$
שחותך את המעגל ב-%
$Q$.
\item
נבנה את המיתרים
$\overline{RS}=\overline{QT}$
ו-%
$\overline{PS}$.
\item
נבנה קו מקביל ל-%
$\overline{RS}$
שמתחיך ב-%
$T$
שחותך את
$\overline{PS}$
ב-%
$N$.
\item
נבנה קו מקביל ל-%
$\overline{RS}$
שמתחיך ב-%
$O$
שחותך את
$\overline{PS}$
ב-%
$M$.
\item
נבנה מיתר
$\overline{PK}=\overline{PM}$.
\item
נבנה משיק ב-%
$P$
שאורכו
$\overline{PL}=\overline{MN}$.
\item
נחבר את הנקודות
$K,L,R$.
\item
נמצא נקודה 
$C$
כך ש-%
$\overline{RC}=\overline{RH}$.
\item
נבנה קו
$\overline{CD}$
המקביל ל-%
$\overline{KL}$
שחותך את
$\overline{LR}$ 
ב-%
$D$. 
\end{itemize}

\begin{figure}[tb]
\begin{center}

\begin{tikzpicture}[scale=1,align=left]
\clip (-6,-5.1) rectangle +(11.5,10.2);
% Draw circle and horizontal diameter
\draw[name path=circle] (0,0)  coordinate (o) node[below] {$O$} circle[radius=5cm];
\draw (-5,0) coordinate (p) node[left] {$P$} -- (5,0) coordinate (r) node[right] {$R$};
\fill (o) circle (1pt);
\fill (p) circle (1pt);
\fill (r) circle (1pt);
\fill (-2.5,0) coordinate (h) node[below] {$H$} circle (1pt);
\fill (10/3,0) coordinate (t) node[below left] {$T$} circle (1pt);
\path (p) -- node[above,xshift=10pt] {$1/2$} (h) -- node[above] {$1/2$} (o) -- node[above] {$2/3$} (t) -- node[above] {$1/3$} (r);

% Draw perpendicular TQ
\path[name path=tq] (t) -- +(0,5);
\path[name intersections={of=tq and circle,by=q}];
\draw (t) -- (q) node[above] {$Q$};
\fill (q) circle (1pt);

% Draw chord RS and line PS
\path[name path=tq] (t) -- +(0,5);
\path[name intersections={of=tq and circle,by=q}];
\path[name path=rcirc] (r) let \p1 = ($ (t) - (q) $) in circle ({veclen(\x1,\y1)});
\path[name intersections={of=rcirc and circle,by=s}];
\draw (r) -- (s);
\fill (s) node[above right] {$S$} circle (1pt);
\draw[name path=ps] (p) -- (s);

% Draw TN
\path[name path=tn] (t) -- +($(s)-(r)$);
\path[name intersections={of=ps and tn,by=n}];
\draw (t) -- (n);
\fill (n) node[above] {$N$} circle (1pt);

% Draw OM
\path[name path=om] (o) -- +($(s)-(r)$);
\path[name intersections={of=ps and om,by=m}];
\draw (o) -- (m);
\fill (m) node[above left] {$M$} circle (1pt);
\path (p) -- (m);
\path (m) -- (n);

% Draw chord PK
\draw (p) -- +(-62.3:4.64) coordinate (k) node[below left] {$K$};

% Draw tangent PL
\draw let \p1 = ($ (m) - (n) $), \n1 = {veclen(\x1,\y1)} in (p) -- (-5,-\n1) coordinate (l) node[left] {$L$};

% Connect L and K to R
\draw (r) -- (l) -- (k) -- cycle;

% Find point C on RK
\coordinate (c) at ($(r)!7.5cm!(k)$);
\path (r) -- (c);
\fill (c) node[below] {$C$} circle (1pt);

% Draw CD
\path[name path=cd] (c) -- +($(l)-(k)$);
\path[name path=lr] (l) -- (r);
\path[name intersections={of=cd and lr,by=d}];
\draw (c) -- (d);
\fill (d) node[above,xshift=2pt] {$D$} circle (1pt);
\path (r) -- (d);

\draw[rotate=90] (t) rectangle +(8pt,8pt);
\draw[rotate=-160] (s) rectangle +(8pt,8pt);
\draw[rotate=-90] (p) rectangle +(8pt,8pt);
\draw[rotate=30] (k) rectangle +(8pt,8pt);
\draw[dashed] (o) -- (q);
\end{tikzpicture}
\end{center}
\selectlanguage{hebrew}
\caption{הבנייה של \L{Ramanujan}}\label{f.ramanujan1}
\end{figure}

\begin{theorem}
$\overline{RD}^2=\disfrac{355}{113}\approx \pi$.
\end{theorem}
\begin{proof}
לפי הבנייה
$\overline{RS}=\overline{QT}$
ולפי משפט פיתגורס ב-%
$\triangle QOT$:
\[
\overline{RS}=\overline{QT} = \sqrt{1^2-\left(\frac{2}{3}\right)^2}=\frac{\sqrt{5}}{3}\,.
\]
הזווית
$\angle PSR$
נשען על קוטר כך ש-%
$\triangle PSR$
הוא משולש ישר-זווית ולפי משפט פיתגורס:
\[
\overline{PS} = \sqrt{2^2-\left(\frac{\sqrt{5}}{3}\right)^2}=\sqrt{4-\frac{5}{9}}=\frac{\sqrt{31}}{3}\,.
\]
לפי הבנייה 
$\overline{MO} \| \overline{RS}$
כך ש-%
$\triangle MPO\sim \triangle SPR$
ול:
\begin{eqn}
\disfrac{\overline{PM}}{\overline{PO}}&=&\disfrac{\overline{PS}}{\overline{PR}}\\
\disfrac{\overline{PM}}{1}&=&\disfrac{\sqrt{31}/3}{2}\\
\overline{PM}&=&\disfrac{\sqrt{31}}{6}\,.
\end{eqn}
לפי הבנייה
$\overline{NT}\|\overline{RS}$
כך ש-%
$\triangle NPT\sim \triangle SPR$
ו:
\begin{eqn}
\disfrac{\overline{PN}}{\overline{PT}}&=&\disfrac{\overline{PS}}{\overline{PR}}\\
\disfrac{\overline{PN}}{5/3}&=&\disfrac{\sqrt{31}/3}{2}\\
\overline{PN}&=&\disfrac{5\sqrt{31}}{18}\\
\overline{MN}&=&\overline{PN}-\overline{PM}=\sqrt{31}\left(\disfrac{5}{18}-\disfrac{1}{6}\right) = \disfrac{\sqrt{31}}{9}\,.
\end{eqn}
$\triangle PKR$
הוא משולש ישר-זווית כי
$\angle PKR$
נשען על קוטר. לפי הבנייה 
$\overline{PK}=\overline{PM}$
ולכן לפי משפט פיתגורס:
\[
\overline{RK}=\sqrt{2^2-\left(\frac{\sqrt{31}}{6}\right)^2} = \frac{\sqrt{113}}{6}\,.
\]
$\triangle PLR$
הוא משולש ישר-זווית כי 
$\overline{PL}$
הוא משיק ולכן
$\angle PLR$
הוא משולש-ישר זווית. לפי הבנייה
$\overline{PL}=\overline{MN}$
ולפי משפט פיתגורס:
\[
\overline{RL}=\sqrt{2^2+\left(\frac{\sqrt{31}}{9}\right)^2} = \frac{\sqrt{355}}{9}\,.
\]
לפי הבנייה 
$\overline{RC}=\overline{RH}=3/2$
ו-%
$\overline{CD}\parallel \overline{LK}$.
לפי משולשים דומים:
\begin{eqn}
\disfrac{\overline{RD}}{\overline{RC}}&=&\disfrac{\overline{RL}}{\overline{RK}}\\
\disfrac{\overline{RD}}{3/2}&=&\disfrac{\sqrt{355}/9}{\sqrt{113}/6}\\
\overline{RD}&=&\sqrt{\disfrac{355}{113}}\\
\overline{RD}^2&=&\disfrac{355}{113}\approx 3.14159292035\approx \pi\,.
\end{eqn}
באיור~%
\ref{f.ramanujan1a}
אורכי קטעי הקו מסומנים.
\end{proof}

\begin{figure}[tb]
\begin{center}

\begin{tikzpicture}[scale=1,align=left]
\clip (-6,-5.1) rectangle +(11.5,10.2);
% Draw circle and horizontal diameter
\draw[name path=circle] (0,0)  coordinate (o) node[below] {$O$} circle[radius=5cm];
\draw (-5,0) coordinate (p) node[left] {$P$} -- (5,0) coordinate (r) node[right] {$R$};
\fill (o) circle (1pt);
\fill (p) circle (1pt);
\fill (r) circle (1pt);
\fill (-2.5,0) coordinate (h) node[below] {$H$} circle (1pt);
\fill (10/3,0) coordinate (t) node[below] {$T$} circle (1pt);
\path (p) -- node[above,xshift=10pt] {$1/2$} (h) -- node[above] {$1/2$} (o) -- node[above] {$2/3$} (t) -- node[above] {$1/3$} (r);
% Draw chord RS and line PS
\path[name path=tq] (t) -- +(0,5);
\path[name intersections={of=tq and circle,by=q}];
\path[name path=rcirc] (r) let \p1 = ($ (t) - (q) $) in circle ({veclen(\x1,\y1)});
\path[name intersections={of=rcirc and circle,by=s}];
\draw (r) -- node[right] {$\sqrt{5}/3$} (s);
\fill (s) node[above right] {$S$} circle (1pt);
\draw[name path=ps] (p) -- node[above right,yshift=16pt] {$\sqrt{31}/3$} (s);
% Draw TN
\path[name path=tn] (t) -- +($(s)-(r)$);
\path[name intersections={of=ps and tn,by=n}];
\draw (t) -- (n);
\fill (n) node[above] {$N$} circle (1pt);
% Draw OM
\path[name path=om] (o) -- +($(s)-(r)$);
\path[name intersections={of=ps and om,by=m}];
\draw (o) -- (m);
\fill (m) node[above left] {$M$} circle (1pt);
\path (p) -- node[below,xshift=32pt,yshift=12pt] {$\sqrt{31}/6$} (m);
\path (m) -- node[below,xshift=5pt,yshift=-6pt] {$\sqrt{31}/9$} (n);
% Draw chord PK
\draw (p) -- node[right,xshift=-2pt,yshift=10pt] {$\sqrt{31}/6$} +(-62.3:4.64) coordinate (k) node[below left] {$K$};
% Draw tangent PL
\draw let \p1 = ($ (m) - (n) $), \n1 = {veclen(\x1,\y1)} in (p) -- node[left] {$\disfrac{\sqrt{31}}{9}$} (-5,-\n1) coordinate (l) node[left] {$L$};
% Connect L and K to R
\draw (r) -- (l) -- (k) -- cycle;
% Find point C on RK
\coordinate (c) at ($(r)!7.5cm!(k)$);
%\path (r) -- node[below,yshift=-16pt] {$RC=3/2$\\$RK=\sqrt{113}/6$} (c);
\path (r) -- node[below,xshift=16pt,yshift=-2pt] {$\overline{RC}=3/2$} (c);
\path (r) -- node[below,xshift=-4pt,yshift=-20pt] {$\overline{RK}=\sqrt{113}/6$} (k);
\fill (c) node[below] {$C$} circle (1pt);
% Draw CD
\path[name path=cd] (c) -- +($(l)-(k)$);
\path[name path=lr] (l) -- (r);
\path[name intersections={of=cd and lr,by=d}];
\draw (c) -- (d);
\fill (d) node[above,xshift=2pt] {$D$} circle (1pt);
%\path (r) -- node[above,xshift=-40pt,yshift=-8pt] {$RD=\sqrt{355/113}$\\$RL=\sqrt{355}/9$} (d);
\path (r) -- node[above,xshift=-40pt,yshift=-4pt] {$\overline{RD}=\sqrt{355/113}$} (d);
\path (r) -- node[below,xshift=-44pt,yshift=-20pt] {$\overline{RL}=\sqrt{355}/9$} (l);
\draw[rotate=-90] (p) rectangle +(8pt,8pt);
\draw[rotate=30] (k) rectangle +(8pt,8pt);
\draw (t) -- node[below right,xshift=-2pt,yshift=-4pt] {$\sqrt{5}/3$} (q) node[above] {$Q$};
\draw[dashed] (o) -- (q);
\fill (q) circle (1pt);
\end{tikzpicture}
\end{center}
\selectlanguage{hebrew}
\caption{הבנייה עם האורכים של סימון של ארכי קטעי הקו}\label{f.ramanujan1a}
\end{figure}


%%%%%%%%%%%%%%%%%%%%%%%%%%%%%%%%%%%%%%%%%%%%%%%%%%%%%%%%%%%

\section{הבנייה השנייה של
\L{\large Ramanujan}}\label{s.square-ramanujan-second}

\textbf{%
בנייה (איור%
~\ref{f.ramanujan2})}

\begin{itemize}
\item
נבנה מעגל יחידה שמרכזו
$O$
עם קוטר
$\overline{AB}$
ונסמן ב-%
$C$
את החיתוך של הניצב ל-%
$\overline{AB}$
ב-%
$O$
עם המעגל.
\item
נחלק את הקטע
$\overline{AO}$
כך ש-%
$\overline{AT}=1/3$ 
ו-%
$\overline{TO}=2/3$.
\item
נבנה
$\overline{BC}$
ונמצא נקודות
$M,N$
כך ש-%
$\overline{CM}=\overline{MN}=\overline{AT}=1/3$.
\item
נבנה 
$\overline{AM}$
ו-%
$\overline{AN}$
ונסמן ב-%
$P$
את הנקודה על
$\overline{AN}$
כך ש-%
$\overline{AP}=\overline{AM}$.
\item
נבנה קו המקביל ל-%
$\overline{MN}$
שעובר דרך
$P$
ונסמן ב-%
$Q$
את נקודת החיתוך שלו עם
$\overline{AM}$.
\item
נבנה
$\overline{OQ}$
ונבנה קו המקביל ל-%
$\overline{OQ}$
שעובר דרך 
$T$
ונסמן ב-%
$R$
את נקודת החיתוך שלו עם
$\overline{AM}$.
\item
נבנה משיק
$\overline{AS}$
כך ש-%
$\overline{AS}=\overline{AR}$.
\item
נבנה
$\overline{SO}$.
\end{itemize}



\begin{figure}[tb]
\begin{center}

\begin{tikzpicture}[scale=1.3]
\clip (-4.4,-4.2) rectangle +(8.8,8.6);
% Scale at 4

% Coordinates of circle
\coordinate (O) at (0,0);
\coordinate (A) at (-4,0);
\coordinate (B) at (4,0);
\coordinate (C) at (0,4);

% Draw circle and diameter
\node [thick,draw,circle through=(A),name path=circle] at (O) {};
\draw [thick] (A) -- (B);
\draw [thick,dashed] (C) -- (O);
\draw (O) rectangle +(8pt,8pt);
\draw[rotate=-90] (A) rectangle +(8pt,8pt);

\coordinate (T) at (-2.667,0);
\path (A) -- node[below] {$1/3$} (T);
\path (T) -- node[below] {$2/3$} (O);
\path (O) -- node[below] {$1$} (B);

\draw (C) -- node[right] {$1/3$} +(-45:1.333) coordinate (M);
\draw (M) -- node[right] {$1/3$} +(-45:1.333) coordinate (N);
\draw (N) -- node[left] {$\sqrt{2}\!-\!2/3$}(B);

\draw[name path=AM] (A) -- (M);
\draw[name path=AN] (A) -- (N);

\node [circle through=(M),name path=AMcircle] at (A) {};

\path[name intersections={of=AMcircle and AN,by=P}];

\path[name path=PQ] (P) -- +(135:2);
\path[name intersections={of=PQ and AM,by=Q}];
\draw (P) -- (Q) -- (O);

\path[name path=QT] (T) -- ($(Q)+(-2.667,0)$) -- (Q);
\path[name intersections={of=QT and AM,by=R}];
\draw (T) -- (R);

\node [circle through=(R),name path=ARcircle] at (A) {};
\path[name path=AS] (A) -- ($(A)+(0,-2.5)$);
\path[name intersections={of=ARcircle and AS,by=S}];
\draw (A) -- (S);

\draw (S) -- (O);

\fill (O) circle(1.2pt) node[below right] {$O$};
\fill (A) circle(1.2pt) node[left] {$A$};
\fill (B) circle(1.2pt) node[right] {$B$} node[above left,xshift=-8pt] {$45^\circ$};
\fill (C) circle(1.2pt) node[above] {$C$};
\fill (T) circle(1.2pt) node[below] {$T$};
\fill (M) circle(1.2pt) node[right] {$M$};
\fill (N) circle(1.2pt) node[right] {$N$};
\fill (P) circle(1.2pt) node[below] {$P$};
\fill (Q) circle(1.2pt) node[above left] {$Q$};
\fill (R) circle(1.2pt) node[above left] {$R$};
\fill (S) circle(1.2pt) node[above left] {$S$};

\end{tikzpicture}
\end{center}
\caption{הבנייה השנייה של \L{Ramanujan}}\label{f.ramanujan2}
\end{figure}

\begin{theorem}
$3\sqrt{\overline{SO}}=\left(9^2+\disfrac{19^2}{22}\right)^{1/4}\approx \pi$.
\end{theorem}
\begin{proof}
$\triangle COB$
הוא משולש ישר-זווית ולפי משפט פיתגורס
$\overline{CB}=\sqrt{2}$
ו-%
\[
\overline{NB}=\sqrt{2}-2/3\,.
\]
$\triangle COB$
הוא המשולש שווה-שוקיים ולכן
$\angle NBA =\angle MBA=45^\circ$.
לפי משפט הקוסינוסים:
\begin{eqn}
\overline{AN}^2&=&\overline{AB}^2 + \overline{BN}^2-2\cdot\overline{AB}\cdot\overline{BN}\cdot\cos \angle NBA\\
&=&2^2+\left(\sqrt{2}-\disfrac{2}{3}\right)^2-2\cdot 2 \cdot \left(\sqrt{2}-\disfrac{2}{3}\right)\cdot \disfrac{\sqrt{2}}{2}=\disfrac{22}{9}\\
%&=&4+2-\disfrac{4\sqrt{2}}{3}+\disfrac{4}{9} - 4 + \disfrac{4\sqrt{2}}{3}=\disfrac{22}{9}\\
\overline{AN}&=&\sqrt{\disfrac{22}{9}}\,.
\end{eqn}
שוב לפי משפט הקוסינוסים:
\begin{eqn}
\overline{AM}^2&=&\overline{AB}^2 + \overline{BM}^2-2\cdot\overline{AB}\cdot\overline{BM}\cdot\cos \angle MBA\\
&=&2^2+\left(\sqrt{2}-\disfrac{1}{3}\right)^2-2\cdot 2 \cdot \left(\sqrt{2}-\disfrac{1}{3}\right)\cdot \disfrac{\sqrt{2}}{2}=\sqrt{\disfrac{19}{9}}\\
%&=&4+2-\disfrac{2\sqrt{2}}{3}+\disfrac{1}{9} - 4 + \disfrac{2\sqrt{2}}{3}=\disfrac{19}{9}\\
\overline{AM}&=&\sqrt{\disfrac{19}{9}}\,.
\end{eqn}
לפי הבנייה
$\overline{QP}\parallel \overline{MN}$
כך ש-%
$\triangle MAN\sim \triangle QAP$,
ולפי הבנייה
$\overline{AP}=\overline{AM}$,
ולכן:

\begin{eqn}
\disfrac{\overline{AQ}}{\overline{AM}}&=&\disfrac{\overline{AP}}{\overline{AN}}=\disfrac{\overline{AM}}{\overline{AN}}\\
\overline{AQ}&=&\disfrac{\overline{AM}^2}{\overline{AN}}=\disfrac{19/9}{\sqrt{22/9}}=\disfrac{19}{3\sqrt{22}}\,.
\end{eqn}
לפי הבנייה
$\overline{TR}\parallel \overline{OQ}$
ולכן
$\triangle RAT\sim \triangle QAO$
ו:
\begin{eqn}
\disfrac{\overline{AR}}{\overline{AQ}}&=&\disfrac{\overline{AT}}{\overline{AO}}\\
\overline{AR}&=&\overline{AQ}\cdot\disfrac{\overline{AT}}{\overline{AO}}
=\disfrac{19}{3\sqrt{22}}\cdot\disfrac{1/3}{1}=\disfrac{19}{9\sqrt{22}}\,.
\end{eqn}
לפי הבנייה
$\overline{AS}=\overline{AR}$
ו-%
$\triangle OAS$ 
הוא משולש ישר-זווית כי 
$\overline{AS}$
הוא משיק. לפי משפט פיתגורס:
\begin{eqn}
\overline{SO}&=&\sqrt{1^2+\left(\disfrac{19}{9\sqrt{22}}\right)^2}\\
3\sqrt{\overline{SO}}&=&3\left(1+\disfrac{19^2}{9^2\cdot 22}\right)^{1/4}=\left(9^2+\disfrac{19^2}{22}\right)^{1/4}\approx 3.14159265258\approx \pi\,.
%&=&\left(3^4+\disfrac{3^4\cdot 19^2}{9^2\cdot 22}\right)^\frac{1}{4}\\
%&=&\left(9^2+\disfrac{19^2}{22}\right)^\frac{1}{4}\\
%&\approx& 3.14159265262\approx \pi\,.
\end{eqn}
באיור~%
\ref{f.ramanujan2a}
אורכי קטעי הקו מסומנים.
\end{proof}
\begin{figure}[tb]
\begin{center}

\begin{tikzpicture}[scale=1.3]
\clip (-5.2,-4.2) rectangle +(10,8.7);
% Scale at 4

% Coordinates of circle
\coordinate (O) at (0,0);
\coordinate (A) at (-4,0);
\coordinate (B) at (4,0);
\coordinate (C) at (0,4);

% Draw circle and diameter
\node [thick,draw,circle through=(A),name path=circle] at (O) {};
\draw [thick] (A) -- (B);
\draw [thick,dashed] (C) -- (O);
\draw (O) rectangle +(8pt,8pt);
\draw[rotate=-90] (A) rectangle +(8pt,8pt);

\coordinate (T) at (-2.667,0);
\path (A) -- node[below] {$1/3$} (T);
\path (T) -- node[below] {$2/3$} (O);
\path (O) -- node[below] {$1$} (B);

\draw (C) -- node[right] {$1/3$} +(-45:1.333) coordinate (M);
\draw (M) -- node[right] {$1/3$} +(-45:1.333) coordinate (N);
\draw (N) -- node[left] {$\sqrt{2}\!-\!2/3$}(B);

\draw[name path=AM] (A) -- node[above,xshift=15pt,yshift=24pt] {$\overline{AM}=\sqrt{19/9}$} (M);
\draw[name path=AN] (A) -- node[below,yshift=-10pt] {$\overline{AN}=\sqrt{22/9}$} (N);

\node [circle through=(M),name path=AMcircle] at (A) {};

\path[name intersections={of=AMcircle and AN,by=P}];

\path[name path=PQ] (P) -- +(135:2);
\path[name intersections={of=PQ and AM,by=Q}];
\draw (P) -- (Q) -- (O);
\path (A) -- node[above,xshift=-14pt,yshift=10pt] {$\overline{AQ}=19/3\sqrt{22}$} (Q);

\path[name path=QT] (T) -- ($(Q)+(-2.667,0)$) -- (Q);
\path[name intersections={of=QT and AM,by=R}];
\draw (T) -- (R);
\path (A) -- node[above,xshift=-31pt,yshift=6pt] {$\overline{AR}=19/9\sqrt{22}$} (R);

\node [circle through=(R),name path=ARcircle] at (A) {};
\path[name path=AS] (A) -- ($(A)+(0,-2.5)$);
\path[name intersections={of=ARcircle and AS,by=S}];
\draw (A) -- node[xshift=10pt,yshift=6pt] {$\overline{AS}=\,19/9\sqrt{22}$} (S);

\draw (S) -- node[right,xshift=-2pt,yshift=-10pt] {$\overline{SO}=\sqrt{1+
  \left(\disfrac{19^2}{9^2\cdot 22}\right)}$} (O);

\fill (O) circle(1.2pt) node[below right] {$O$};
\fill (A) circle(1.2pt) node[left] {$A$};
\fill (B) circle(1.2pt) node[right] {$B$} node[above left,xshift=-8pt] {$45^\circ$};
\fill (C) circle(1.2pt) node[above] {$C$};
\fill (T) circle(1.2pt) node[below] {$T$};
\fill (M) circle(1.2pt) node[right] {$M$};
\fill (N) circle(1.2pt) node[right] {$N$};
\fill (P) circle(1.2pt) node[below] {$P$};
\fill (Q) circle(1.2pt) node[above left] {$Q$};
\fill (R) circle(1.2pt) node[above left] {$R$};
\fill (S) circle(1.2pt) node[above left] {$S$};

\end{tikzpicture}
\end{center}
\caption{הבנייה עם האורכים של סימון של ארכי קטעי הקו}\label{f.ramanujan2a}
\end{figure}

%%%%%%%%%%%%%%%%%%%%%%%%%%%%%%%%%%%%%%%%%%%%%%%%%%%%%%%

\section{לרבע את המעגל באמצעות קוודרטיקס}\label{s.square-quad}

הקוודרטיקס מתואר בסעיף%
~\ref{s.q}.

יהי
$t=\overline{DE}$
המרחק שהסרגל האופקי זז כאשר הוא יורד בציר ה-%
$y$
ויהי 
$\theta$
הזווית שנוצרה בין הסרגל המסתובב לבין ציר ה-%
$x$.
יהי 
$P$
המיקום של הציר המחבר את שני הסרגל. המקום הגיאומטרי של
$P$
הוא עקומת הקוודרטיקס.

יהי 
$F$
ההיטל של
$P$
על ציר ה-%
$x$
ויהי 
$G$
המקום של הציר כאשר שני המקלות מגיעים לציר ה-%
$x$,
כלומר, 
$G$
היא נקודה החיתוך בין עקומת הקוודרטיקס לבין ציר ה-%
$x$
(איור%
~\ref{f.square-quad}).
\begin{figure}[tb]
\begin{center}
\begin{tikzpicture}[scale=.7,domain=.01:1.57,samples=100]
\draw (0,0) node[below left] {$A$} node [above right,xshift=6pt] {$\theta$} -- (8,0) node[below right] {$B$} -- (8,8) node[above right] {$C$} -- (0,8) node[above left] {$D$} -- cycle;
\draw[name path=horiz] (0,5) -- (8,5);
\draw[name path=slant] (0,0) -- (61:8);
\path[name intersections={of=horiz and slant,by=joint}];
\draw (joint) -- node[right] {$y$} (joint |- 0,0) coordinate (f);
\path (0,0) -- node[below] {$x$} (0,0 -| joint) node[below] {$F$};
\coordinate (e) at (0,5);
\path (e) node[left] {$E$} -- node[left] {$t$} (0,8);
\path (0,0) -- node[left] {$1-t$} (0,5);
\node[above right,xshift=4pt] at (joint) {$P$};
\node[below] at (4.28,0) {$G$};
\draw[name path=curve,thick] plot (4.3*.637*\x,{12.5*.637*\x*cot(\x r)});
\end{tikzpicture}
\end{center}
\selectlanguage{hebrew}
\caption{לרבע את המעגל באמצעות קוודרטיקס}\label{f.square-quad}
\end{figure}

\begin{theorem}
$\overline{AG}=2/\pi$.
\end{theorem}
\begin{proof}
יהי
$y=\overline{PF}=\overline{EA}=1-t$.
על קוודרטיקס
$\theta$
יורד בקצב זהה של העלייה ב-%
$t$:
\begin{eqn}
\frac{1-t}{1} &=& \frac{\theta}{\pi/2}\\
&&\\
\theta &=&\frac{\pi}{2}(1-t)\,.
\end{eqn}
יהי
$x=\overline{AF}=\overline{EP}$. 
אזי
$\tan \theta = y/x$
ולכן:
\begin{equation}\label{eq.quad-square}
x = \frac{y}{\tan\theta}=y\cot\theta=y\cot \frac{\pi}{2}(1-t)=y\cot \frac{\pi}{2}y\,.
\end{equation}
נהוג לבטא פונקציה כ-%
$y=f(x)$
אבל ניתן לבטא אותה גם כ-%
$x=f(y)$. 

כדי לקבל 
$x=\overline{AG}$
לא ניתן פשוט להציב
$y=0$
במשוואה%
~\ref{eq.quad-square}
כי
$\cot 0$
לא מוגדר. נחשב את הגבול של
$x$
כאשר
$y$
שואף ל-%
$0$. 
תחילה נציב
$z=(\pi/2)y$
ונקבל:
\[
x = y\cot \frac{\pi}{2}y = \frac{2}{\pi} \left(\frac{\pi}{2}y\cot \frac{\pi}{2}y\right)=\frac{2}{\pi}(z\cot z)\,,
\]
ונחשב את הגבול:
\[
\lim_{z\rightarrow 0} x=\frac{2}{\pi}\lim_{z\rightarrow 0} (z\cot z) = \frac{2}{\pi}\lim_{z\rightarrow 0} \left(\frac{z\cos z}{\sin z}\right) = \frac{2}{\pi}\lim_{z\rightarrow 0} \left(\frac{\cos z}{(\sin z)/z}\right) = \frac{2}{\pi}\frac{\cos 0}{1} = \frac{2}{\pi}\,,
\]
כאשר השתמשנו ב-%
$\lim_{z\rightarrow 0} (\sin z/z)=1$ (משפט~\ref{thm.limit-sine-over}).
\end{proof}

\subsection*{מה ההפתעה?}

מפתיע שניתן לנבנהת קירובים כל כל מדוייקים ל-%
$\pi$.
כמובן, אנו נדהמים מהבניות של
\L{Ramanujan}.

\subsection*{מקורות}
הבנייה של
\L{Kocha\'{n}ski}
מופיעה ב-%
\L{\cite{bold}}.
הבניות של
\L{Ramanujan}
נמצאות ב-%
\L{\cite{ramanujan1,ramanujan2}}.
ריבוע המגעל באמצעות הקוודרטיקס מבוסס על
\L{\cite[pp.~48--49]{martin}}
ו-%
\L{\cite{wiki:quad}}.
