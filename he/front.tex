% !TeX root = surprises.tex

\pagestyle{empty}

\begin{center}
\textbf{\Huge הפתעות מתמטיות}
 
\bigskip
\bigskip
\bigskip
\bigskip

\textbf{\Large מוטי בן-ארי}

\L{\url{http://www.weizmann.ac.il/sci-tea/benari/}}
\bigskip
\bigskip

\textbf{\Large עריכה: רחל זקס}

\bigskip
\bigskip

\today{}
\end{center}

%%%%%%%%%%%%%%%%%%%%%%%%%%%%%%%%%%%%%%%%%%%%%%%%%%%%%%

\vfill

\begin{center}
\copyright{} מוטי בן-ארי
2022-24
 \end{center}

\begin{english}
\begin{small}
This work is licensed under Attribution 4.0 International. To view a copy of this license, visit \url{http://creativecommons.org/licenses/by/4.0/}.

\end{small}
\end{english}


%%%%%%%%%%%%%%%%%%%%%%%%%%%%%%%%%%%%%%%%%%%%%%%%%%%%%%

\thispagestyle{empty}
\pagestyle{plain}
\pagenumbering{roman}

\chapter*{פתח דבר}
\thispagestyle{empty}

\begin{flushleft}
\parbox{7cm}{
\begin{small}
\begin{flushleft}
לו כל אחד היה נחשף למתמטיקה במצבה הטבעי, עם כל ההנאה המאתגרת וההפתעות שבה, לדעתי היינו רואים שינוי מרשים הן בדעות התלמידים כלפי מתמטיקה והן בתפיסה שלנו של מה זה להיות "טוב במתמטיקה".\\
\L{Paul Lockhard}

\bigskip

אני ממש רעב להפתעות, כי כל אחת מצעידה אותנו צעד קטן אך משמעותי להיות חכמים יותר.\\
\L{Tadashi Tokieda}
\end{flushleft}
\end{small}
}
\end{flushleft}

כאשר ניגשים למתמטיקה בדרך נאותה, היא עשויה לספק לנו הפתעות רבות ומהנות. אישור לכך ניתן לקבל בחיפוש של 
\L{mathematical surprises}
בגוגל, שמחזיר (וזה מפתיע) כחצי מיליארד תוצאות. מהי הפתעה?
\L{(surprise)}?
מקור המילה בצרפתית עתיקה עם שורשים בלטינית: 
\L{sur},
(מעל) ו-
\L{prendre}
(לקחת, לאחוז, לתפוס). באופן מילולי, להפתיע הוא להשיג. כשם עצם, הפתעה היא גם אירוע או מצב בלתי צפוי או מבלבל וגם הרגש שהוא גורם.


קחו לדוגמה קטע מהרצאה של מקסים ברוקהיימר.
\L{Maxim Bruckheimer}%
\footnote{מקסים ברוקהיימר היה מתמטיקאי, ממייסדי האוניברסיטה הפתוחה בבריטניה ודיקן הפקולטה למתמטיקה שלה. 
וראש המחלקה להוראת מדעים במכון ויצמן למדע.}
על מעגל פיירבאך
\L{Feuerbach}:
"שתי נקודות נמצאות על קו ישר אחד בלבד, אין זו הפתעה. אולם בהינתן שלוש נקודות שאינן בהכרח על קו ישר אחד, אם במהלך החקר הגיאומטרי שלוש הנקודות 'נופלות' על קו ישר, זו הפתעה, ולעיתים קרובות עלינו להתייחס לעובדה זו כאל משפט שדורש הוכחה. כל שלוש נקודות שאינן על קו ישר נמצאות על מעגל יחיד. אם ארבע נקודות נמצאות על אותו מעגל, זו הפתעה שיש לנסחה כמשפט.
$\ldots$
ככל שמספר הנקודות על קו ישר גדול משלוש, כך המשפט מפתיע יותר. באופן דומה, ככל שמספר הנקודות על מעגל גדול מארבע,
$4$,
כך המשפט מפתיע עוד יותר. לכן, הטענה שעבור כל משולש קיימות תשע נקודות קשורות זו לזו הנמצאות על אותו מעגל 
$\ldots$
היא מפתיעה ביותר. בנוסף, למרות עוצמת ההפתעה, ההוכחה פשוטה ואלגנטית".

בספר מציע מרדכי בן-ארי אוסף עשיר של הפתעות מתמטיות, רובן מוכרות פחות ממעגל פיירבאך 
,
ובעלות סיבות מוצקות להכללתן. ראשית, למרות שהן נעדרות מספרי לימוד, אבני החן בספר נגישות עם רקע במתמטיקה של בית ספר תיכון בלבד (וסבלות ונייר ועפרון, כי הנאה לא מגיעה בחינם). שנית, כאשר תוצאה מתמטית מאתגרת את מה שהנחנו, אנו באמת מופתעים (פרקים~%
\ref{c.collapse}, \ref{c.compass}).
באופן דומה אנו מופתעים מ: הוכחות נבונות (פרקים~%
~\ref{c.trisect}, \ref{c.square}),
הוכחה אלגברית של האפשרות לבנייה גאומטרית (פרק~%
~\ref{c.heptadecagon}),
הוכחות המתבססות על נושאים בלתי קשורים לכאורה (פרקים~%
~\ref{c.five}, \ref{c.museum}),
הוכחה מוזרה באינדוקציה (פרק~%
~\ref{c.induction}),
דרכים חדשות להסתכל על תוצאה ידועה היטב (פרק~%
~\ref{c.quadratic}),
משפט שנראה שולי והופך להיות בסיס לתחום רחב במתמטיקה (פרק~%
~\ref{c.ramsey}),
מקורות בלתי צפויים להשראה (פרק~%
\ref{c.langford}),
מערכת אקסיומטית הנובעת מפעילות פנאי כגון אוריגמי (פרקים~%
~\ref{c.origami-axioms}-\ref{c.origami-constructions}).
אלו הסיבות השונות להכללת הפתעות מתמטית מהנות, יפות ובלתי נשכחות בספר נפלא זה.

עד כאן התייחסתי לצורה שבה הספר מטפל בחלק הראשון של הגדרת ההפתעה, הסיבות הקוגניטיביות והרצניוליות לבלתי צפוי. בקשר להיבט השני, ההיבט הרגשי, הספר הוא מקרה מאיר של הטענה של מתמטיקאים לסיבה המרכזית לעסוק במתמטיקה: היא מרתקת! בנוסף, הם טוענים שמתמטיקה מעוררת גם את הסקרנות האינטלקטואלית שלנו וגם רגישות אסטטית, ושפתרון בעיות או הבנת מושג מספקים תגמול רוחני המפתה אותנו להמשיך לעבוד על בעיות ועל ומושגים נוספים.

אומרים שתפקידו של פתח דבר הוא לספר לקוראים למה כדאי להם לקרוא את הספר. ניסיתי למלא תפקיד זה, אבל אני מאמין שתשובה מלאה יותר תגיע מכם הקוראים, לאחר שתקראו ותחוו את מה שמשתמע ממקור המילה הפתעה: שיתפוס אתכם!

\bigskip

\begin{flushleft}
אברהם הרכבי
\end{flushleft}

%%%%%%%%%%%%%%%%%%%%%%%%%%%%%%%%%%%%%%%%%%%%%%%%%%%%%%

\chapter*{הקדמה}
\thispagestyle{empty}

המאמר של 
\cite{toussaint} \L{Godfried Toussaint}
על "מחוגה מתמוטטת" עשה עליי רושם חזק. מעולם לא עלה על דעתי שהמחוגה המודרנית עם ציר חיכוך איננה אותה מחוגה שהייתה קיימת בימיו של אוקלידס. 
בספר זה אני מציג מבחר נושאים מתמטיים שהם לא רק מעניינים, אלא שהפתיעו אותי כאשר נתקלתי בהם בפעם הראשונה. 

המתמטיקה הדרושה לקריאת הספר היא ברמת בית-ספר תיכון, אבל אין זה אומר שהחומר פשוט. חלק מההוכחות הן ארוכות למדי ונדרשת מהקורא נכונות להשקיע ולהתמיד. הפרס הוא הבנה של כמה מהנושואים היפים יותר במתמטיקה. הספר אינו ספר לימוד כי מגוון הנושאים העשיר אינו מתאים לסילבוס. הוא כן מתאים לפעילויות העשרה של תלמידי תיכון, לסמינרים אוניברסיטאיים ולמורים למתמטיקה.


הפרקים אינם תלויים זה בזה (פרט לפרק~%
\ref{c.origami-axioms}
על אקסיומות האוריגמי שיש לקרוא אותו לפני פרקים~%
\ref{c.origami-cube}, \ref{c.origami-constructions},
הפרקים האחרים על אוריגמי).

\subsection*{מהי הפתעה?}

שלושה קריטריונים הנחו אותי בבחירת נושאים לספר:
\begin{itemize}
\item 
המשפט הפתיע אותי. מפתיעים במיוחד היו המשפטים על בנייה בסרגל ובמחוגה. העושר המתמטי של אוריגמי היה כמעט הלם. כאשר מורה למתמטיקה הציעה פרויקט בנושא, סירבתי כי פקפקתי באפשרות שקיימת מתמטיקה רצינית בתחום זה של אומנות. נושאים אחרים נכללו מכיוון, שלמרות שהכרתי את התוצאות, הופתעתי מהאלגנטיות של ההוכחות ומהנגישות שלהן. בלטה במיוחד ההוכחה 
\textbf{האלגברית}
של גאוס (Gauss) שניתן לבנות הפטדקגון heptadecagon (מצולע משוכלל בעל $17$ צלעות).

\item
הנושא אינו מופיע בספרי לימוד לבתי-ספר תיכון או לאוניברסיטה. את המשפטים וההכחות מצאתי רק בספרים מתקדמים או בספרות מחקר. קיימים מאמרי ויקיפדיה לרוב הנושאים, אבל עליך לדעת איפה לחפש אותם ולעיתים קרובות הם אינם יורדים לפרטים.

\item
המשפטים וההוכחות נגישים עם ידע טוב במתמטיקה של בית ספר תיכון.
\end{itemize}
כל פרק מסתיים בסעיף 
\textbf{מה ההפתעה?}
המסביר את בחירתי בנושא.

\subsection*{סקירת התוכן}


פרק~%
\ref{c.collapse}
מביא את ההוכחה של אוקלידס שעבור כל בנייה במחוגה קבועה, קיימת בנייה שקולה ב"מחוגה מתמוטטת". לאורך השנים ניתנו הוכחות שגויות רבות המבוססות על תרשימים שאינם נכונים בכל מצב. כדי להדגיש שאין לסמוך על תרשימים, הבאתי את ה"הוכחה" המפורסמת לכך שכל משולש הוא שווה-שוקיים.

לאורך שנים רבות, מתמטיקאים חיפשו לשווא בנייה שתחלק זווית שרירותית לשלושה חלקים שווים
\L{trisection}.
\L{Underwood Dudley}
חקר לעומק אנשים שהקדישו את חייהם לחיפוש אחר בנייה. לרוב הבניות הן קירובים שממציאיהם טוענים לנכונותם. פרק~%
\ref{c.trisect}
מתחיל בהצגת שתי בניות ובפיתוח הנוסחאות הטריגונומטריות המראות שמדובר בקירובים בלבד. כדי להראות שאין משמעות להגבלה לסרגל ומחוגה בלבד, מוצגת חלוקת זווית לשלושה חלקים שווים בעזרת כלים משוכללים יותר: ה-%
\L{neusis}
של
\L{Archimedes}
וה-%
\L{quadratrix}
של
\L{Hippias}.
בסוף הפרק מובאת הוכחה שלא ניתן לחלק זווית שרירותית לשלושה חלקים שווים בעזרת סרגל ומחוגה.

לא ניתן לרבע מעגל (לבנות ריבוע ששטחו זהה לשטח מעגל נתון) בעזרת סרגל ומחוגה. הבנייה בלתי אפשרית כי הערך של 
$\pi$.
אינו ניתן לבנייה. 
פרק~%
\ref{c.square}
מביא שלוש בניות אלגנטיות של קירובים טובים ל-%
$\pi$,
אחת של
\L{Kochansky}
ושתיים של
\L{Ramanujan}.
בסוף הפרק נסביר איך לרבע מעגל באמצעות
\L{quadratrix}.

%%%%%%%%%%%%%%%%%%%%%%%%%%%%%%%%%%%%%%%%%%%%%%%%%%%%%%%%%%%%%%%

לפי משפט ארבעת-הצבעים ניתן לצבוע כל מפה במישור בארבעה צבעים, כך ששתי ארצות שיש להן גבול משותף צבועות בצבעים שונים. ההוכחה של משפט זה מסובכת ביותר, אבל ההוכחה של משפט חמשת הצבעים פשוטה ואלגנטית (פרק%
~\ref{c.five}).
הפרק מביא גם את ה"הוכחה" של 
\L{Alfred Kempe}
לבעיית ארבעת הצבעים ואת ההדגמה של 
\L{Percy Heawood}
לכך שההוכחה שגויה.

%%%%%%%%%%%%%%%%%%%%%%%%%%%%%%%%%%%%%%%%%%%%%%%%%%%%%%%%%%%%%%%

כמה שומרים דרושים לשמירה על מוזיאון לאומנות, כך שכל הקירות נמצאים תחת השגחה רציפה? ההוכחה בפרק%
~\ref{c.museum}
מתוחכמת, כי היא משתמשת בצביעת גרפים כדי לפתור בעיה שנראית במבט ראשון כבעיה גאומטרית.

%%%%%%%%%%%%%%%%%%%%%%%%%%%%%%%%%%%%%%%%%%%%%%%%%%%%%%%%%%%%%%%

פרק~%
\ref{c.induction}
מביא משפטים פחות מוכרים שהוכחותיהם באינדוקציה. המשפטים הם בנושאים: מספרי 
\L{Fibonacci}, 
מספרי
\L{Fermat},
פונקציה 
$91$
של 
\L{McCarthy}
ובעיית
\L{Josephus}
(יוסף בן-מתתיהו).

%%%%%%%%%%%%%%%%%%%%%%%%%%%%%%%%%%%%%%%%%%%%%%%%%%%%%%%%%%%%%%%

פרק~%
\ref{c.quadratic}
עוסק בשיטה של
\L{Po-Shen Loh}
למציאת שורשים של משוואות ריבועיות. לשיטה חשיבות רבה בהוכחה האלגברית של
\L{Gauss}
לבניית 
\L{heptadecagon}.
בפרק כלולות שתי בניות גאומטריות לפתרון בעיות אלגבריות. הפתרון של
\L{Khwarizmi}
למציאת שורשים של משוואות ריבועיות ובנייה ש-%
\L{Cardano}
השמתמש בה בפיתוח הנוסחה לשורשי משוואות ממעלה שלישית.

%%%%%%%%%%%%%%%%%%%%%%%%%%%%%%%%%%%%%%%%%%%%%%%%%%%%%%%%%%%%%%%

תיאוריית 
\L{Ramsey}
היא נושא בקומבינטוריקה שמהווה תחום מחקר פעיל. בתיאוריה מחפשים תבניות בקבוצות גדולות. פרק~%
\ref{c.ramsey}
מציג דוגמאות פשוטות של שלשות
\L{Schur},
שלשות פיתגוריות,
מספרי 
\L{Ramsey},
ובעייתו של
\L{van der Waerden}.
הוכחת המשפט על שלשות פיתגוריות היא תוצאה חדשה שהשתמשה בתוכנת מחשב המבוססת על לוגיקה מתמטית. בסוף הפרק אנו סוטים מעט מהדרך הישרה כדי להציג את הידע של הבבלים על שלשות פיתגוריות.

\L{C. Dudley Langford}
צפה יום אחד בבנו שסידר קוביות צבעוניות בסדר מעניין. 
פרק~%
\ref{c.langford}
מביא משפט שלו הקובע את התנאים שבהם סידור זה אפשרי.

%%%%%%%%%%%%%%%%%%%%%%%%%%%%%%%%%%%%%%%%%%%%%%%%%%%%%%%%%%%%%%%

\newpage

בפרק%
~\ref{c.origami-axioms}
מוצגות שבע אקסיומות האוריגמי עם חישובים מפורטים בגאומטרייה אנליטית של משוואות האקסיומות ואפיון הקפלים כמוקדים גאומטריים.

פרק~%
\ref{c.origami-cube}
מביא את השיטה של
\L{Eduard Lill}
ואת הקיפול של
\L{Margharita P. Beloch}.
אני מציג את השיטה של
\L{Lill}
כקסם, ולכן לא אפרט יותר כאן.

פרק~%
\ref{c.origami-constructions}
מראה שבאמצעות אוריגמי ניתן לבצע בניות שאינן אפשריות בבסרגל ומחוגה: חלוקת זווית לשלושה חלקים שווים, הכפלת קובייה ובניית 
\L{nonagon},
מצולע משוכלל בעל תשע צלעות.

%%%%%%%%%%%%%%%%%%%%%%%%%%%%%%%%%%%%%%%%%%%%%%%%%%%%%%%%%%%%%%%

פרק~%
\ref{c.compass}
מביא את המשפט של 
\L{Lorenzo Mascheroni}
ו-%
\L{Georg Mohr} 
שכל בנייה בסרגל ומחוגה ניתן לבצע במחוגה בלבד.

הטענה המקבילה, שניתן להסתפק בסרגל, אינה נכונה, כי בסרגל לא ניתן לחשב ערכים שהם שורש ריבועי.
\L{Jean-Victor Poncelet}
שיער ו-%
\L{Jakob Steiner}
הוכיח שאפשר להסתפק בסרגל בתנאי שקיים מעגל אחד אי-שם במישור (פרק~%
\ref{c.straightedge}).

%%%%%%%%%%%%%%%%%%%%%%%%%%%%%%%%%%%%%%%%%%%%%%%%%%%%%%%%%%%%%%%

האם שני משולשים בעלי אותו שטח ואותו היקף הם בהכרח חופפים? הטענה מתקבלת על הדעת אבל איננ’ה נכונה, אף שמציאת זוגות לא-חופפים מחייבת מסע דרך הרבה אלגברה וגאומטרייה כפי שמתואר בפרק~%
\ref{c.congruent}.

%%%%%%%%%%%%%%%%%%%%%%%%%%%%%%%%%%%%%%%%%%%%%%%%%%%%%%%%%%%%%%%

פרק~%
\ref{c.heptadecagon}
מביא את ההישג המדהים של 
\L{Gauss}:
הוכחה שניתן להשתמש בסרגל ומחוגה כדי לבנות 
\L{heptadecagon}
(מצולע משוכלל עם $17$ צלעות). באמצעות טיעון מבריק על הסמטרייה של שורשים של פולינומים, הוא מצא נוסחה המכילה רק את ארבע פעולות החשבון ושורש ריבועי. 
\L{Gauss}
לא סיפק בנייה גאומטרית, ולכן הפרק מביא בנייה אלגנטית של 
\L{James Callagy}.
בסיום הפרק מוצגות בניות של מחומש משוכלל המבוססות על השיטה של
\L{Gauss}.


%%%%%%%%%%%%%%%%%%%%%%%%%%%%%%%%%%%%%%%%%%%%%%%%%%%%%%%%%%%%%%%

 על מנת שהספר יהיה בלתי תלוי ככל האפשר בהוכחות של משפטים ושל נוסחאות אחרים, נספח%
~\ref{a.trig}
אוסף הוכחות של משפטים בגאומטרייה ובטריגונומטריה שייתכן שאינם מוכרים לקורא.

%%%%%%%%%%%%%%%%%%%%%%%%%%%%%%%%%%%%%%%%%%%%%%%%%%%%%%%%%%%%%%%

\subsection*{סגנון}

\begin{itemize}
\item
הרקע הנדרש מהקורא הוא מתמטיקה ברמת בית-ספר תיכון, הכוללת:

\begin{itemize}
\item
אלגברה: פולינומים, חילוק של פולינומים, פולינומים 
\L{monic}
(פולינומים שבהם מקדם החזקה הגבוהה ביותר הוא $1$), משוואות ריבועיות, מכפלה של חזקות 
$a^m\cdot a^n=a^{m+n}$.

\item
גאומטרייה אוקלידית: משולשים חופפים 
$\triangle ABC \cong \triangle DEF$
והקריטריונים לחפיפה, משולשים דומים
$\triangle ABC \sim \triangle DEF$
והיחסים בין הצלעות שלהם, מעגלים והזוויות ההיקפיות והמרכזיות שלהם.

\item
גיאומטרייה אנלטית: המישור הקרטזי, חישוב אורכים ושיפועים של קטעי קו, נוסחת המעגל.

\item
טריגונומטריה: הפונקציות 
$\sin,\cos,\tan$
וההמרות ביניהן, זוויות במעגל היחידה, פונקציות טריגונומטריות של זוויות לאחר שיקוף סביב ציר כגון
$\cos (180^\circ-\theta)=-\cos\theta$.
\end{itemize}

\item
כל טענה להוכחה נקראת "משפט" ואין ניסיון לסווג טענה כמשפט, כלמה או כמסקנה.

\item
כאשר משפט מופיע לאחר בנייה, המשתנים המופיעים במשפט מתייחסים לנקודות, לקווים ולזוויות המסומנים באיור הנלווה לבנייה.

\item
שמות המתמטיקאים ניתנים במלואם ללא מידע ביוגרפי שניתן למצוא בקלות בוויקיפדיה.

\item
הספר נכתב כדי שיהיה בלתי תלוי ככל האפשר במקורות אחרים. פה ושם נחוץ שימוש במושגים ובמשפטים הניתנים ללא הוכחה. הסברים קצרים ניתנים בתוך מסגרות וניתן לדלג עליהם.

\item
אין תרגילים, אבל הקורא השאפתן מוזמן לנסות להוכיח כל משפט לפני קריאת ההוכחה.

\item
ניתן להתעמק בבניות גאומטריות באמצעות תוכנה כגון גיאוגברה.

\item
$\overline{AB}$
מסמן גם שם של קטע קו וגם את אורכו.

\item
$\triangle ABC$
מסמן גם שם של משולש וגם את שטחו.
\end{itemize}

%%%%%%%%%%%%%%%%%%%%%%%%%%%%%%%%%%%%%%%%%%%%%%%%%%%%%%%%%%%%%%%

\subsection*{הבעת תודה}

הספר נכתב בעידודו של אברהם הרכבי שקיבל בברכה את הסגת הגבול שלו בחינוך מתמטי. הוא גם התנדב לכתוב את פתח הדבר. אביטל אלבאום-כהן ורונית בן-בסט לוי היו נכונות תמיד לעזור לי ללמוד (מחדש) מתמטיקה של בית-ספר תיכון. אוריה בן-לולו הכירה לי את המתמטיקה של אוריגמי ועזרה לי בכתיבת ההוכחות. אני מודה ל-%
\L{Michael Woltermann}
שהרשה לי להשתמש בעיבוד שלו לספרו של
\L{Heinrich D\"{o}rrie}.
ג'ייסון קופר, אברהם הרכבי,
\L{Richard Kruel}
והשופטים האנונימיים העירו הערות מועילות.

ברצוני להודות לצוות ב-%
\L{Springer}
עבור התמיכה והמקצועונות בתהליך ההוצאה לאור, במיוחד לעורך
\L{Richard Kruel}.

הספר פורסם באנגלית כ-%
\L{\textit{Mathematical Surprises}, Springer, 2022}
וניתן להורידו בחינם מ:\\
\L{\url{https://link.springer.com/book/10.1007/978-3-031-13566-8}}.

אני מודה למכון ויצמן למדע על מימון ההוצאה לאור.

\subsection*{קבצי המקור}
קובצי המקור של הספר ב-%
\L{\LaTeX{}}
(כולל קבצי המקור לאיורים ב-%
\L{Ti\textit{k}Z}%
)
זמינים ב:
\begin{center}
\L{\url{https://github.com/motib/surprises}}
\end{center}

\medskip

\begin{flushleft}
מוטי בן-ארי
\\
רחובות
2022
\end{flushleft}

\tableofcontents
