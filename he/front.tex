% !TeX root = surprises.tex

\selectlanguage{hebrew}
\thispagestyle{empty}

\begin{center}
\textbf{\Huge הפתעות מתמטיות}
 
\bigskip
\bigskip
\bigskip
\bigskip

\textbf{\Large מוטי בן-ארי}

\bigskip
\bigskip

\selectlanguage{english}
\url{http://www.weizmann.ac.il/sci-tea/benari/}
\end{center}

% !TeX root = surprises.tex

\begin{center}
\begin{bfseries}
\bigskip
\bigskip
גרסה
$1.0$

\bigskip

\today

\end{bfseries}
\end{center}


%%%%%%%%%%%%%%%%%%%%%%%%%%%%%%%%%%%%%%%%%%%%%%%%%%%%%%

\newpage
\thispagestyle{empty}
\selectlanguage{english}

This book was prepared from \LaTeX{} source files.  The Cumulus Hebrew fonts were used. Mathematics (\verb+$$+, \verb+\[\]+, etc.) were written on separate lines to avoid difficulties with the cursor movement in Hebrew.

Source files: \url{https://github.com/motib/mathematics/}.

Fonts:
\url{http://www.ma.huji.ac.il/~sameti/tex/culmusmiktex.html}

\vfill

\begin{center}
\copyright{} Moti Ben-Ari $2021$ 
\end{center}

\begin{footnotesize}
This work is licensed under the Creative Commons Attribution-ShareAlike 3.0 Unported License. To view a copy of this license, visit \url{http://creativecommons.org/licenses/by-sa/3.0/} or send a letter to Creative Commons, 444 Castro Street, Suite 900, Mountain View, California, 94041, USA.
\end{footnotesize}

%\bigskip
%
%\begin{center}
%\includegraphics[width=.15\textwidth]{by-sa.png}
%\end{center}

\newpage

%%%%%%%%%%%%%%%%%%%%%%%%%%%%%%%%%%%%%%%%%%%%%%%%%%%%%%

\selectlanguage{hebrew}
\tableofcontents
\thispagestyle{empty}



\chapter{הקדמה}

\setcounter{page}{1}



המאמר של 
\L{Godfried Toussaint}
\L{\cite{toussaint}}
על "מחוגה מתמוטטת" עשה עלי רושם חזק. לעולם לא עלה על דעתי שהמחוגה המודרנית איננה אותה מחוגה שהיתה קיימת בימיו של אוקלידס. 
בספר זה אני מציג את המחוגה המתמוטטת ומגוון רחב של נושאים מתמטיים אחרים שהפתיעו אותי. המתמטיקה היא ברמה של חמש יחידות בבית ספר תיכון, אבל החישובים וההוכחות אינם בהכרח פשוטים ודרושה מהקורא נכונות להשקיעה ולהתמיד.


ארבעת הפרקים הראשונים עוסקים בבניות גיאומטריות. פרק~%
\ref{c.collapse}
מביא את ההוכחה של אוקלידס שעבור כל בניה עם מחוגה קבועה, קיימת בניה שקולה עם מחוגה מתמוטטת. לאורך השנים ניתנו הוכחות שגויות רבות  שמבוססות על תרשימים שאינם נכונים בכל מצב. כדי להדגיש שאין לסמוך על תרשימים, הבאתי את "ההוכחה" המפורסמת שכל משולש שווה-שוקיים.

היוונים חיפשו בניה שתחלק זווית שרירותית לשלושה חלקים שווים. רק במאה ה-%
$19$
הוכח שהבניה אינה אפשרית. למעשה, אין לבעיה שום משמעות מעשית, כי ניתן לחלק זווית לשלושה חלקים שווים עם כלים מעט יותר משוכללים מסרגל ומחוגה, כפי שמוסבר בפרק~%
\ref{c.trisect}.


בעיה שנייה שאיתגר את היוונים היתה ריבוע המעגל: נתון מעגל, בנה ריבוע עם שטח זהה. הבניה שקולה לבניית קטע קו באורך 
$\pi$.
גם בניה זו הוכחה כבלתי אפשרית. פרק~%
\ref{c.square}
מביא שלוש בניות של קירובים מדוייקים להפליא ל-%
$\pi$,
אחת של
\L{Kochansky}
מ-%
$1685$,
ושתיים של
\L{Ramanujan}
מ-%
$1913$.

%%%%%%%%%%%%%%%%%%%%%%%%%%%%%%%%%%%%%%%%%%%%%%%%%%%%%%%%%%%%%%%

פרקים
\ref{c.five}--\ref{c.langford}
עוסקים בבעיות הקשורות בצביעת ישויות כגון גרפים.
ב-%
$1976$
פורסמה הוכחה מסובכת ביותר שניתן לצבוע כל מפה )גרף מישורי( עם ארבעה צבעים. אולם, כבר במאה ה-%
$19$,
הופיעה הוכחה פשוטה יחסית שניתן לצבוע גרף מישורי בששה ואף בחמישה צבעים. פרק~%
\ref{c.five}
מביא את ההוכחה ביחד עם הוכחת הנוסחה של 
\L{Euler}
הדרושה להוכחת הצביעה של גרפים.


כמה שומרים נחוצים כדי לשמור על מוזיאון? כלומר, נתון שטח במישור שתחום בקירות שרירותיים, מה מספר הנקודות הקטן ביותר שמהן ניתן לראות את כל הקירות?
פרק~%
\ref{c.museum}
מציג את הפתרון והוכחה אלגנטית ביותר לפתרון שמבוססת על צביעת גרפים.


המתמטיקאי
\L{C. Dudley Langford}
צפה יום אחד בבנו שסידר קוביות צבעוניות בסדר מעניין. 
פרק~%
\ref{c.langford}
מביא משפט שלו הקובע מתי סידור זה אפשרי.

%%%%%%%%%%%%%%%%%%%%%%%%%%%%%%%%%%%%%%%%%%%%%%%%%%%%%%%%%%%%%%%

פרק~%
\ref{c.quadratic}
עוסק בפתרון משוואות ריבועיות, מאבני היסוד המוכרות ביותר בקורסי מבוא במתמטיקה, ומסביר הדרך המעט שונה של
\L{Po-Shen Loh}
לפתרון המשוואות.

\newpage

פרק~%
\ref{c.induction}
מביא הוכחות למשפטים פחות מוכרים שמשתמשות באינדוקציה. המשפטים הם בנושאים: מספרי 
\L{Fibonacci}, 
מספרי
\L{Fermat}
ופונקציה 
$91$
של 
\L{McCarthy}.


%%%%%%%%%%%%%%%%%%%%%%%%%%%%%%%%%%%%%%%%%%%%%%%%%%%%%%%%%%%%%%%


אוריגמי הוא אומנות בה האומן מייצר חפצים יפים על ידי קיפולי נייר. לקראת סוף המאה ה-%
$20$,
מתמטיקאים גילו שאפשר לאפיין את הכל הקיפולים האפשריים באמצעות שבע אקסיומות. פרק%
~\ref{c.origami-axioms}
מפתח את המשוואות של האקסיומות ביחד עם דוגמאות נומריות.



פעולות הקיפול יכולות לבנות כל בניה שניתן לבנות עם סרגל ומחוגה. בנוסף, ניתן לבנות שורשים ממעלה שלוש. פרק~%
\ref{c.origami-cube}
מביא את השיטה הגיאומטרית של
\L{Eduard Lill}
לבדיקת שורשים ממשיים של פולינומים, וכן את הקיפול של
\L{Margharita P. Beloch}
למציאת שורשים ממשיים של פולינום ממעלה שלוש.

העובדה שניתן למצוא שורשים ממעלה שלוש מאפשרת לבנות בקיפולי אוריגמי בניות שלא ניתן לבנות עם סרגל ומחוגה. בפרק~%
\ref{c.origami-constructions}
נביא את הבניות: חלוקת זווית לשלושה לחלקים שווים, הכפלת קוביה, ובניית מתושע )מצולע משוכלל עם תשע צלעות(.

%%%%%%%%%%%%%%%%%%%%%%%%%%%%%%%%%%%%%%%%%%%%%%%%%%%%%%%%%%%%%%%

אנו מסיימים עם שלושה פרקים על בניות גיאומטריות מתקדמות. המתמטיקה בחלק זה היא עדיין ברמה של בית ספר תיכון, אבל ההוכחות מאוד ארוכות. פרק~%
\ref{c.compass}
מביא את המשפט המפתיע ביותר של 
\L{Lorenzo Mascheroni}
מ-%
$1797$
ו-%
\L{Georg Mohr} 
מ-%
$1672$
שאין צורך בסרגל, וניתן להסתפק במחוגה בלבד.

בעקבות משפט זה ניתן לשאול: האם צריך מחוגה? התשובה היא לא
כי עם סרגל בלבד אפשר לבנות ערכים המתקבלים רק מחישובים לינאריים, לעומת בניות עם מחוגה שמאפשרת חישובים עם שורש ריבועי. ב-%
$1833$
\L{Jakob Steiner}
הוכיח שאפשר להסתפק בסרגל בלבד, בתנאי שקיים אי-שם במישור מעגל אחד. ההוכחה נמצאת בפרק~%
\ref{c.straightedge}.

שאלה מעניינת בגיאומטריה היא: האם שני משולשים עם אותו שטח ועם אותו היקף חייבים להיות חופפים? התשובה היא לא, אבל מציאת זוגות לא חופפים מחייבת מסע דרך הרבה טריגונומטריה, כפי שמתואר בפרק~%
\ref{c.congruent}.
לפרק הוספתי הוכחה אלגנטית לנוסחה של 
\L{Heron}
לשטח של משולש.

