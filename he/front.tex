% !TeX root = surprises.tex

\pagestyle{empty}

\begin{center}
\textbf{\Huge הפתעות מתמטיות}
 
\bigskip
\bigskip
\bigskip
\bigskip

\textbf{\Large מוטי בן-ארי}

\bigskip
\bigskip


\L{\url{http://www.weizmann.ac.il/sci-tea/benari/}}

\bigskip
\bigskip

\today{}
\end{center}

%%%%%%%%%%%%%%%%%%%%%%%%%%%%%%%%%%%%%%%%%%%%%%%%%%%%%%

\vfill

\begin{center}
\copyright{} מוטי בן-ארי
2022
 \end{center}

\begin{english}
\begin{small}
This work is licensed under Attribution 4.0 International. To view a copy of this license, visit \url{http://creativecommons.org/licenses/by/4.0/}.

\end{small}
\end{english}


%%%%%%%%%%%%%%%%%%%%%%%%%%%%%%%%%%%%%%%%%%%%%%%%%%%%%%

\thispagestyle{empty}
\pagestyle{plain}
\pagenumbering{roman}

\chapter*{פתח דבר}
\thispagestyle{empty}

\begin{flushleft}
\parbox{7cm}{
\begin{small}
\begin{flushleft}
לו כל אחד היה נחשף למתמטיקה במצבו הטבעי, עם כל ההנאה המאתגרת וההפתעות בה, לדעתי היינו רואים שינוי מרשים גם בדיעות של תלמידים כלפי מתמטיקה, וכם בתפיסה שלנו של מה זה נקרא "טוב במתמטיקה".\\
\L{Paul Lockhard}

\bigskip

אני ממש רעב להפתעות כי כל אחת מצעיד אותנו צעד קטן אך משמעותי להיות חכמים יותר.\\
\L{Tadashi Tokieda}
\end{flushleft}
\end{small}
}
\end{flushleft}

כאשר ניגשים למתמטיקה בצורה נאותה היא יכולה לספק לנו הפתעות רבות ומהנות. אישור לכך ניתן לקבל בחיפוש בגוגל של 
\L{mathematical surprises}
שמחזיר (וזה מפתיע) כחצי מיליארד תוצאות. מהי הפתעה
\L{(surprises)}?
מקור המילה בצרפתית עתיקה עם שורשים בלטינית: 
\L{sur},
(מעל) ו-%
\L{prendre}
(לקחת, לאחוז, לתפוס). באופן מילולי, להפתיע הוא להשיג. כשם עצם, הפתעה היא גם אירוע או מצב לא צפוי או מבלבל, וגם הרגש שהוא גורם.


קחו לדוגמה, קטע מהרצאה של
\L{Maxim Bruckheimer}%
\footnote{מקסים ברוקהיימר היה מתמטיקאי ממייסדי האוניברסיטה הפתוחה בבריטניה ודיקן הפקולטה למתמטיקה שלה. 
הוא היה ראש המחלקה להוראת מדעים במכון ויצמן למדע.}
על המעגל של
\L{Feuerbach}:
"שתי נקודות נמצאות על קו אחד בלבד, אין זו הפתעה. אולם, נתון שלוש נקודות, שאינן בהכרח על קו ישר, אם במהלך החקר הגיאומטרי, שלוש הנקודות 'נופלות' על קו ישר, זו הפתעה, ולעתים קרובות עלינו להתייחס לעובדה כמשפט שיש להוכיח. כל שלוש נקודות שאינן על קו ישר נמצאות על מעגל יחיד. אם ארבע נקודות נמצאות על אותו מעגל, זו הפתעה שיש לנסחה כמשפט.
$\ldots$
ככל שמספר הנקודות בקו ישר גדול מ-$3$, כך המשפט מפתיע יותר. באופן דומה, ככל שמספר הנקודות על מעגל גדול מ-%
$4$,
כך המשפט מפתיע עוד יותר. לכן, הטענה שעבור כל משולש קיימות תשע נקודות קשורות אחת לשניה שנמצאות על אותו מעגל  
$\ldots$
היא מפתיעה ביותר. בנוסף, למרות עוצמת ההפתעה, ההוכחה פשוטה ואלגנטית".

בספר מציע מרדכי בן-ארי אוסף עשיר של הפתעות מתמטיות, רובם ידועות פחות ממעגל 
\L{Feuerbach},
ועם סיבות מוצקות להכללתן. ראשית, למרות שהן נעדרות מספרי לימוד, אבני החן בספר נגישות עם רקע במתמטיקה של בית ספר תיכון בלבד (וסבלות, ונייר ועפרון, כי הנאה לא מגיעה בחינם). שנית, כאשר תוצאה מתמטית מאתגרת את מה שהנחנו, אנו באמת מופתעים (פרקים~%
\ref{c.collapse}, \ref{c.compass}).
באופן דומה אנו מופתעים מ: הוכחות נובנות (פרקים~%
~\ref{c.trisect}, \ref{c.square}),
הוכחה אלגברית של האפשרות לבנייה גיאומטרית (פרק~%
~\ref{c.heptadecagon}),
הוכחות המתבססות על נושאים לא קשורים לכאורה (פרקים~%
~\ref{c.five}, \ref{c.museum}),
הוכחה מוזרה באינדוקציה (פרק~%
~\ref{c.induction}),
דרכים חדשות להסתכל על תוצאה ידועה היטב (פרק~%
~\ref{c.quadratic}),
משפט שנראה שולי שהופך להיות הבסיס לתחום רחב במתמטיקה (פרק~%
~\ref{c.ramsey}),
מקורות בלתי צפויים להשראה (פרק~%
\ref{c.langford}),
מערכת אקסיומתית שנובעת מפעילות פנאי כגון אוריגמי (פרקים~%
~\ref{c.origami-axioms}-\ref{c.origami-constructions}).
אלו הסיבות השונות להכללת הפתעות מתמטית מהנות, יפות ובלתי נשכחות בספר נפלא זה.

עד כאן התייחסתי לצורה בה הספר מטפל בחלק הראשון של ההגדרת הפתעה, הסיבות הקוגניטיביות והרצניוליות לבלתי צפוי. בקשר להיבט השני, ההיבט הרגשי, הספר הוא מקרה מאיר של הטענה של מתמטיקאים לסיבה המרכזית לעסוק במתמטיקה: היא מרתקת! בנוסף, הם טוענים שמתמטיקה מעוררת גם הסקרנות האינטלקטואלית שלנו וגם הרגישות האסטטית, ושפתרון בעיות או הבנת מושג מספק גמול רוחני, שמפתה אותנו להמשיך לעבוד על בעיות ומושגים נוספים.

אומרים שתפקידו של פתח דבר הוא לספר לקוראים למה כדאי להם לקרוא את הספר. ניסיתי למלא תפקיד זה, אבל אני מאמין שתשובה מלאה יותר יגיע ממך הקורא, לאחר שתקראו ותחוו את מה שמשתמע ממקור המילה הפתעה: שישיג אותכם!

\bigskip

\begin{flushleft}
\textit{אברהם הרכבי}
\end{flushleft}

%%%%%%%%%%%%%%%%%%%%%%%%%%%%%%%%%%%%%%%%%%%%%%%%%%%%%%

\chapter*{הקדמה}
\thispagestyle{empty}

המאמר של 
\cite{toussaint} \L{Godfried Toussaint}
על "מחוגה מתמוטטת" עשה עלי רושם חזק. לעולם לא עלה על דעתי שהמחוגה המודרנית עם ציר חיכוך איננה אותה מחוגה שהיתה קיימת בימיו של אוקלידס. 
בספר זה אני מציג מבחר של נושאים מתמטיים שהם לא רק מעניינים, אלא שהפתיעו אותי כאשר נתקלתי בהם בפעם הראשונה. 

המתמטיקה הדרושה לקריאת הספר היא ברמה בית ספר תיכון, אבל זה לא אומר שהחומר פשוט. חלק מההוכחות הן ארוכות למדי ודרושה מהקורא נכונות להשקיעה ולהתמיד. הפרס הוא הבנה של נשואים מהיפים היותר במתמטיקה. הספר אינו ספר לימוד כי המגוון העשיר אל הנושאים לא מתאים לסילבוס. הוא כן מתאים לפעילויות העשרה של תלמידי תיכון, לסמינרים אוניברסיטאים ולמורים למתמטיקה.


הפרקים לא תלויים אחד בשני (פרט לפרק~%
\ref{c.origami-axioms}
על האקסיומות של אוריגמי שיש לקרוא אותו לפני פרקים~%
\ref{c.origami-cube}, \ref{c.origami-constructions},
הפרקים האחרים על אוריגמי).

\subsection*{מהי הפתעה?}

שלושה קריטריונים הנחו אותי בבחירת נושאים לספר:
\begin{itemize}
\item 
המשפט שהפתיע אותי. הפתיע במיוחד המשפטים על בנייה עם סרגל ומחוגה. העושר המתמטי של אוריגמי היה כמעט הלם: כאשר מורה למתמטיקה הציעה פרויקט בנושא סירבתי, כי פקפקתי באפשרות שקיימת מתמטיקה רצינית בתחום זה של אמנות. נושאים אחרים נכללו כי, למרות שידעתי אותם, הופתעתי מהאלגנטיות של ההוכחות ומהנגישות שלהן. בלט במיוחד ההוכחה 
\textbf{האלגברית}
של Gauss שניתן לבנות heptadecagon (מצולע משוכלל עם $17$ צלעות).

\item
הנושא אינו מופיע בספרי לימוד לבתי ספר תיכון או לאוניברסיטה. את המשפטים וההכחות מצאתי רק בספרים מתקדים או בספרות המחקר. קיימים מאמרי Wikipedia לרוב הנושאים, אבל חייבים לדעת איפה לחפשם ולעתים קרובות המאמרים לא נכנסים לפרטים.

\item
המשפטים וההוכחות נגישים עם ידע טוב במתמטיקה של בית ספר תיכון.
\end{itemize}
כל פרק מסתיים בסעיף 
\textbf{מה ההפתעה?}
המסביר את הבחירה של הנושא.

\subsection*{סקירה של התוכן}


פרק~%
\ref{c.collapse}
מביא את ההוכחה של אוקלידס שעבור כל בנייה עם מחוגה קבועה, קיימת בנייה שקולה עם "מחוגה מתמוטטת". לאורך השנים ניתנו הוכחות שגויות רבות שמבוססות על תרשימים שאינם נכונים בכל מצב. כדי להדגיש שאין לסמוך על תרשימים, הבאתי את "ההוכחה" המפורסמת שכל משולש שווה-שוקיים.

לאורך שנים רבות מתמטיקאים חיפשו לשווא בנייה שתחלק זווית שרירותית לשלושה חלקים שווים
\L{trisection}.
\L{Underwood Dudley}
חקר לעומק אנשים שהיקדשו את חייהם לחיפוש אחר בנייה. לרוב הבניות הן קירובים שממציאיהם טוענים לנכונותם. פרק~%
\ref{c.trisect}
מתחיל בהצגת שתי בניות ופיתוח הנוסחאות הטריגונומטריות המראות שמדובר בקירובים בלבד. כדי להראות שאין משמעות להגבלה לסרגל ומחוגה בלבד, נראה שניתן לחלק זווית לשלושה חלקים שווים עם כלים יותר משוכללים: ה-%
\L{neusis}
של
\L{Archimedes}
וה-%
\L{quadratrix}
של
\L{Hippias}.
בסוף הפרק נמצאת הוכחה שלא ניתן לחלק זווית שרירותית לשלושה חלקים עם סרגל ומחוגה.

לא ניתן לרבע מעגל עם סרגל ומחוגה (לבנות ריבוע עם שטח זהה למעגל נתון). הבנייה בלתי אפשרית כי לא ניתן לבנות את הערך של 
$\pi$.
פרק~%
\ref{c.square}
מביא שלוש בניות אלגנטיות של קירובים טובים ל-%
$\pi$,
אחת של
\L{Kochansky}
ושתיים של
\L{Ramanujan}.
בסוף הפרק נסביר איך לרבע מעגל באמצעות
\L{quadratrix}.

%%%%%%%%%%%%%%%%%%%%%%%%%%%%%%%%%%%%%%%%%%%%%%%%%%%%%%%%%%%%%%%

לפי משפט ארבעת-הצבעים ניתן לצבוע כל מפה במישור בארבעה צבעים כך ששתי ארצות שיש להן גבול משותף צבועות בצבעים שונים. ההוכחה של משפט זה מסובך ביותר, אבל ההוכחה של משפט חמשת-הצבעים פשוטה ואלגנטית (פרק%
~\ref{c.five}).
הפרק מביא גם את ה"הוכחה" של 
\L{Alfred Kempe}
לבעית ארבעת הצבים ואת ההדגמה של 
\L{Percy Heawood}
שההוכחה שגויה.

%%%%%%%%%%%%%%%%%%%%%%%%%%%%%%%%%%%%%%%%%%%%%%%%%%%%%%%%%%%%%%%

כמה שומרים נחוצים כדי לשמור על מוזיאון לאומנות כך שכל הקירות נמצאים תחת השגחה רצופה? ההוכחה בפרק%
~\ref{c.museum}
מתוחכמת כי היא משתמשת בצביעה של גרפים כדי לפתור בעיה שבמבט ראשון נראה כבעיה גיומטרית.

%%%%%%%%%%%%%%%%%%%%%%%%%%%%%%%%%%%%%%%%%%%%%%%%%%%%%%%%%%%%%%%

פרק~%
\ref{c.induction}
מביא משפטים פחות מוכרים שהוכחותיהם באינדוקציה. המשפטים הם בנושאים: מספרי 
\L{Fibonacci}, 
מספרי
\L{Fermat},
פונקציה 
$91$
של 
\L{McCarthy}
ובעיית
\L{Josephus}
(יוסף בן-מתתיהו).

%%%%%%%%%%%%%%%%%%%%%%%%%%%%%%%%%%%%%%%%%%%%%%%%%%%%%%%%%%%%%%%

פרק~%
\ref{c.quadratic}
עוסק בשיטה של
\L{Po-Shen Loh}
למציאת שורשים של משוואות ריבועיות. לשיטה חשיבות רבה בהוכחה האלגברית של
\L{Gauss}
לבניית 
\L{heptadecagon}.
בפרק כלולות שתי בניות גיאומטרויות לפתרון של בעיות אלגבריות. הפתרון של
\L{Khwarizmi}
למציאת שורשים של משוואות ריבועיות ובנייה ש-%
\L{Cardano}
השמתמש בו בפיתוח הנוסחה לשורשים של משוואות ממעלה שלוש.

%%%%%%%%%%%%%%%%%%%%%%%%%%%%%%%%%%%%%%%%%%%%%%%%%%%%%%%%%%%%%%%

תיאורית 
\L{Ramsey}
היא נושא בקומבניטוריקה שהמחקר בה פעיל מאוד. בתיאוריה מחפשים תבניות בקבוצות גדולות. פרק~%
\ref{c.ramsey}
מציג דוגמאות פשוטות של שלשות
\L{Schur},
שלשות פיתרגורס,
מספרי 
\L{Ramsey},
ובעייתו של
\L{van der Waerden}.
הוכחת המשפט על שלשות פיתגורס היא תוצאה חדשה שהשתמשה בתכנית מחשב שמבוססת על לוגיקה מתמטית. בסוף הפרק אנו סוטים מעט מהדרך הישרה כדי להציג את הידע של הבבלים על שלשות פיתגורס.

\L{C. Dudley Langford}
צפה יום אחד בבנו שסידר קוביות צבעוניות בסדר מעניין. 
פרק~%
\ref{c.langford}
מביא משפט שלו הקובע את התנאים בהם סידור זה אפשרי.

%%%%%%%%%%%%%%%%%%%%%%%%%%%%%%%%%%%%%%%%%%%%%%%%%%%%%%%%%%%%%%%

\newpage

בפרק%
~\ref{c.origami-axioms}
נציג את שבעת האקסיומות של אוריגמי ביחד חישובים מגיאומטריה אנאליטית של משוואות האקסיומות ואפיון הקפלים כמוקדים גיאומטריים.

פרק~%
\ref{c.origami-cube}
מביא את השיטה של
\L{Eduard Lill}
ואת הקיפול של
\L{Margharita P. Beloch}.
אני מציג את השיטה של
\L{Lill}
כקסם ולכן לא אפרט יותר כאן.

פרק~%
\ref{c.origami-constructions}
מראה שבאמצעות אוריגמי ניתן לבצע בניות שאינן אפשרויות עם סרגל ומחוגה: חלוקת זווית לשלושה לחלקים שווים, הכפלת קוביה ובניית 
\L{nonagon},
מצולע משוכלל עם תשע צלעות.

%%%%%%%%%%%%%%%%%%%%%%%%%%%%%%%%%%%%%%%%%%%%%%%%%%%%%%%%%%%%%%%

פרק~%
\ref{c.compass}
מביא את המשפט של 
\L{Lorenzo Mascheroni}
ו-%
\L{Georg Mohr} 
שכל בנייה על סרגל ומחוגה ניתן לבצע עם מחוגה בלבד.

הטענה המקבילה שניתן להסתפק בסרגל אינה נכונה כי עם סרגל לא ניתן לחשב ערכים עם שורש ריבועי.
\L{Jean-Victor Poncelet}
שיער ו-%
\L{Jakob Steiner}
הוכיח שאפשר להסתפק בסרגל בתנאי שקיים מעגל אחד אי-שם במישור (פרק~%
\ref{c.straightedge}).

%%%%%%%%%%%%%%%%%%%%%%%%%%%%%%%%%%%%%%%%%%%%%%%%%%%%%%%%%%%%%%%

האם שני משולשים עם אותו שטח ואותו היקף חייבים להיות חופפים? הטענה מתקבלת על הדעת אבל איננה נכונה, אולם מציאת זוגות לא-חופפים מחייבת מסע דרך הרבה אלגברה וגיאומטריה כפי שמתואר בפרק~%
\ref{c.congruent}.

%%%%%%%%%%%%%%%%%%%%%%%%%%%%%%%%%%%%%%%%%%%%%%%%%%%%%%%%%%%%%%%

פרק~%
\ref{c.heptadecagon}
מביא את ההישג המדהים של 
\L{Gauss}:
הוכחה שניתן להשתמש בסרגל ומחוגה כדי לבנות 
\L{heptadecagon}
(מצולע משוכלל עם $17$ צלעות). באמצעות טיעון מבריק על הסמטריה של שורשים של פולינומים, הוא מצא נוסחה המכילה רק את ארבעת פעולות החשבון ושורש ריבועי. 
\L{Gauss}
לא סיפק בנייה גיאומטרית ולכן הפרק מביא בנייה אלגנטית של 
\L{James Callagy}.
בסיום הפרק נמצאות בניות של מחומש משוכלל שמבוססות על השיטה של
\L{Gauss}.


%%%%%%%%%%%%%%%%%%%%%%%%%%%%%%%%%%%%%%%%%%%%%%%%%%%%%%%%%%%%%%%

 שהספר יהיה בלתי תלוי ככל האפשר בהוכחות של משפטים ונוסחאות אחרים, נספח%
~\ref{a.trig}
אוסף הוכחות של משפטים בגיאומטריה וטריגונומטריה שייתכן שאינם מוכרים לקורא.

%%%%%%%%%%%%%%%%%%%%%%%%%%%%%%%%%%%%%%%%%%%%%%%%%%%%%%%%%%%%%%%

\subsection*{סגנון}

\begin{itemize}
\item
הרקע הנדרש מהקורא הוא מתמטיקה ברמה של בית ספר תיכון הכולל:

\begin{itemize}
\item
אלגברה: פולינומים, חילוק של פולינומים, פולינומים 
\L{monic}
(שהמקדם של החזקה הגבוהה ביותר הוא $1$), משוואות ריבועיות, מכפלה של ביטויים מעריכיים 
$a^m\cdot a^n=a^{m+n}$.

\item
גיאומטריה אוקלידית: משולשים חופפים 
$\triangle ABC \cong \triangle DEF$
והקריטריונים לחפיפה, משולשים דומים
$\triangle ABC \sim \triangle DEF$
והיחסים בין הצלעות שלהם, מעגלים והזוויות ההיקפיות והמרכזיות שלהם.

\item
גיאומטריה אנלטית: המישור הקרטזי, חישוב אורכים ושיפועים של קטעי קו, נוסחת המעגל.

\item
טריגונומטריה: הפונקציות 
$\sin,\cos,\tan$
וההמרות ביניהן, זוויות במעגל היחידה, פנוקציות טריגונומטריות של זוויות לאחר שיקוף סביב ציר כגון
$\cos (180^\circ-\theta)=-\cos\theta$.
\end{itemize}

\item
כל טענה להוכחה נקראת 
\emph{משפט}
ואין ניסיון לסווג טענה כמשפט, למה או מסקנה.

\item
כאשר משפט מופיע לאחר בנייה, המשתנים המופיעים במשפט מתייחסים לנקודות, קווים וזוויות במסומנים באיור הנלווה לבנייה.

\item
השמות של מתמטיקאים ניתנים במלואם ללא מידע ביאוגרפי שניתן למצוא בקלות בויקיפדיה.

\item
הספר נכתב כדי שיהיה בלתי תלוי ככל האפשר במקורות אחרים. פה ושם נחוץ שימוש במושגים ומשפטים שניתנים ללא הוכחה. הסברים קצרים ניתנים בתוך מסגרות וניתן לדלג עליהם.

\item
אין תרגילים אבל הקורא השאפתן מוזמן לנסות להוכיח כל משפט לפני קריאת ההוכחה.

\item
ניתן להתעמק בבניות גיאמטריות באמצעות תכנה כגון גיאוגברה.

\item
$\overline{AB}$
מסמן גם שם של קטע קו וגם את אורכו.

\item
$\triangle ABC$
מסמן גם שם של משולש וגם את שטחו.
\end{itemize}

%%%%%%%%%%%%%%%%%%%%%%%%%%%%%%%%%%%%%%%%%%%%%%%%%%%%%%%%%%%%%%%

\subsection*{הבעת תודה}

הספר נכתב בעידודו של אברהם הרכבי שקיבל בברכה את הסגת הגבול לי בחינוך מתמטי. הוא גם התנדב לכתוב את פתח הדבר. אביטל אלבאום-כהן ורונית בן-בסט לוי היו נכונות תמיד לעזור לי ללמוד (מחדש) מתמטיקה של בית ספר תיכון. אוריה בן-לולו הכיר לי את המתמטיקה של אוריגמי ועזרה לי בכתיבת ההוכחות. אני מודה ל-%
\L{Michael Woltermann}
שהרשה לי להשתמש בעיבוד שלו לספרו של
\L{Heinrich D\"{o}rrie}.
ג'ייסון קופר, אברהם הרכבי,
\L{Richard Kruel}
והשופטים האנונימיים העירו הערות מועילות.

ברצוני להודות לצוות ב-%
\L{Springer}
עבור התמיכה והמקצועונות בתהליך ההוצאה לאור, במיוחד לעורך
\L{Richard Kruel}.

הספר פורסם באנגלית כ-%
\L{\textit{Mathematical Surprises}, Springer, 2022}
וניתן להורידו בחינם מ:\\
\L{\url{https://link.springer.com/book/10.1007/978-3-031-13566-8}}.

אני מודה למכון ויצמן למדע על מימון ההוצאה לאור.

\subsection*{קבצי המקור}
קובצי המקור של הספר ב-%
\L{\LaTeX{}}
(כולל קבצי המקור לאיורים ב-%
\L{Ti\textit{k}Z}%
)
זמינים ב:
\begin{center}
\L{\url{https://github.com/motib/surprises}}
\end{center}

\medskip

\begin{flushleft}
מוטי בן-ארי
\\
רחובות
2022
\end{flushleft}

\tableofcontents
