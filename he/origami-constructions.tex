% !TeX root = surprises.tex

%%%%%%%%%%%%%%%%%%%%%%%%%%%%%%%%%%%%%%%%%%%%%%%%%%%%%%%%%%%%%%%%

\chapter{בניות גיאומטריות באוריגמי}\label{c.origami-constructions}

\section{הבניה של 
\L{Abe}
לחלוקת זווית לשלושה חלקים%
}\label{s.trisection-abe}

נתונה זווית חדה
$\angle PQR$,
בנו את הקו
$p$
ניצב ל-%
$\overline{QR}$
ב-%
$Q$
והקו
$q$
ניצב ל-%
$p$
ב-% 
$A$
כך שהוא חותך את
$\overline{PQ}$.
בנו את הקו
$r$,
האנך האמצעי של
$\overline{AQ}$
שחותך אותו בנקודה
$B$.
לפי אקסיומה 
$6$
בנו קיפול
$l$
המניח את 
$A$
על
$\overline{PQ}$ 
בנקודה
$A'$
ומניח את
$Q$
על
$r$
בנקודה
$Q'$.
סמנו ב-%
$B'$
את השיקוף של 
$B$
מסביב ל-%
$l$.
בנו את הקווים
$\overline{QB'}$
ו-%
$\overline{QQ'}$.


\begin{center}
\selectlanguage{english}
\begin{tikzpicture}[scale=.8]

% Place points P, Q, R
\coordinate (P) at (60:10cm); %(5,8.67);
\coordinate (Q) at (0,0);
\coordinate (R) at (10,0);
\fill (P) circle (2pt) node[below right] {$P$};
\fill (Q) circle (2pt) node[left] {$Q$};
\fill (R) circle (2pt) node[right] {$R$};

% Draw PQR
\draw [very thick] (P)  -- (Q) -- (R);

% Draw perpendicular to QR
\draw [thick] (Q) -- node[left,very near end] {$p$} +(0,11);

% Draw parallel to QR and parallel halfway
\coordinate (A) at (0,5);
\coordinate (B) at (0,2.5);
\draw [thick] (A) -- node[above,very near end] {$q$} +(10,0);
\draw [thick] (B) -- node[above,very near end] {$r$} +(10,0);
\fill (A) circle (2pt) node[left] {$A$};
\fill (B) circle (2pt) node[left] {$B$};
\path (Q) -- node[left] {$a$} (B) -- node[left] {$a$} (A);
\draw (A) rectangle +(8pt,8pt);
\draw (B) rectangle +(8pt,8pt);

% Tangent line y = -2.75x + 10.69

% Draw fold
\coordinate (D) at (0,10.69);
\coordinate (fold-x) at (3.89,0);
\coordinate (AP) at (3.65,6.33);
\coordinate (QP) at (6.87,2.5);
\coordinate (BP) at (5.26,4.42);
\fill (D) circle (2pt) node[left] {$D$};
\fill (AP) circle (2pt) node[above,yshift=6pt] {$A'$};
\fill (QP) circle (2pt) node[above,yshift=6pt] {$Q'$};
\fill (BP) circle (2pt) node[above,xshift=2pt,yshift=2pt] {$B'$};
\draw [very thick,dashed] (D) -- node[left,near start] {$l$} (fold-x);

% Draw line of reflections
\draw [very thick, dotted] (D) -- (QP);

% Draw trisecting lines
\draw [very thick,dotted] (Q) -- ($(Q)!1.3!(QP)$);
\draw [very thick,dotted] (Q) -- ($(Q)!1.3!(BP)$);

% Complete triangle
\draw [very thick,dotted] (A) -- (QP);

% Draw fold arrows
\draw[thick,dotted,->,bend left=40]
  ($(A)+(4pt,10pt)$) to ($(AP)+(-4pt,0pt)$);
\draw[thick,dotted,->,bend right=40]
  ($(Q)+(2pt,-4pt)$) to ($(QP)+(0pt,-4pt)$);

\end{tikzpicture}
\end{center}

טיעון:
$\angle PQB'=\angle B'QQ'=\angle Q'QR=\disfrac{1}{3}\angle PQR$.

%\newpage

\textbf{הוכחה ראשונה}


הנקודות
$A', B', Q'$
הן שיקופים סביב אותו קו 
$l$
של הנקודות
$A,B,Q$
הנמצאות על קו אחד
$\overline{DQ}$,
ולכן גם הן נמצאות על קטע קו אחד
$\overline{DQ'}$.
לפי הבניה,
$\overline{AB}=\overline{BQ}$,
$\angle ABQ'=\angle QBQ'=90^\circ$,
$\overline{BQ'}$
הוא צלע משותף, ולכן
$\triangle ABQ'\cong \triangle QBQ'$ 
לפי צלע-זווית-צלע. מכאן ש-%
$\angle AQ'B=\angle QQ'B=\alpha$
כי
$\overline{Q'B}$
הוא האנך האמצעי של המשולש שווי-שוקיים
$\triangle AQ'Q$
)איור~
\ref{f.first-proof}(.

\begin{figure}[tb]
\begin{center}
\selectlanguage{english}
\begin{tikzpicture}[scale=.8]

% Place points P, Q, R
\coordinate (P) at (60:10cm);
\coordinate (Q) at (0,0);
\coordinate (R) at (10,0);
\fill (P) circle (2pt) node[below right] {$P$};
\fill (Q) circle (2pt) node[left,xshift=-4pt] {$Q$};
\fill (R) circle (2pt) node[right] {$R$};

% Draw PQR
\draw [very thick] (Q) -- (R);

% Draw perpendicular to QR
\draw [thick] (Q) -- node[left,very near end] {$p$} +(0,11);

% Draw parallel to QR and parallel halfway
\coordinate (A) at (0,5);
\coordinate (B) at (0,2.5);
\draw [thick] (A) -- node[above,very near end] {$q$} +(10,0);
\draw [thick] (B) -- node[above,very near end] {$r$} +(10,0);
\fill (A) circle (2pt) node[left,xshift=-4pt] {$A$};
\fill (B) circle (2pt) node[left,xshift=-4pt] {$B$};
\path (Q) -- node[left,xshift=-4pt] {$a$} (B) -- node[left,xshift=-4pt] {$a$} (A);
\draw (A) rectangle +(8pt,8pt);
\draw (B) rectangle +(8pt,8pt);

% Tangent line y = -2.75x + 10.69

% Draw fold
\coordinate (D) at (0,10.69);
\coordinate (fold-x) at (3.89,0);
\coordinate (AP) at (3.65,6.33);
\coordinate (QP) at (6.87,2.5);
\coordinate (BP) at (5.26,4.42);
\fill (D) circle (2pt) node[left] {$D$};
\fill (AP) circle (2pt) node[above,yshift=6pt] {$A'$};
\fill (QP) circle (2pt) node[above,xshift=2pt,yshift=6pt] {$Q'$};
\fill (BP) circle (2pt) node[above,xshift=4pt,yshift=2pt] {$B'$};
\draw [very thick,dashed] (D) -- node[left,near start] {$l$} (fold-x);
	
% Draw line of reflections
\draw [very thick, dotted] (D) -- (AP);

% Draw trisecting lines
\draw [very thick,dotted] (Q) -- ($(Q)!1.3!(BP)$);

\draw [very thick,loosely dash dot,red] (Q) -- (QP);
\draw [very thick,loosely dash dot,red] (QP) -- (AP);
\draw [very thick,loosely dash dot,red] (AP) -- (Q);
\draw [very thick,loosely dash dot dot,blue] ($(Q)+(0,-4pt)$) -- ($(QP)+(0,-4pt)$);
\draw [very thick,dash dot dot,blue] ($(QP)+(0,-4pt)$) -- ($(A)+(0,-4pt)$);
\draw [very thick,dash dot dot,blue] ($(A)+(-4pt,0)$) -- ($(Q)+(-4pt,0)$);

\draw [thick,dotted] (A) -- (AP);

\node[left,xshift=-40pt,yshift=7pt] at (QP) {$\alpha$};
\node[left,xshift=-40pt,yshift=-6pt] at (QP) {$\alpha$};
\node[right,xshift=40pt,yshift=6pt] at (Q) {$\alpha$};
\node[right,xshift=40pt,yshift=28pt] at (Q) {$\alpha$};
\node[right,xshift=30pt,yshift=42pt] at (Q) {$\alpha$};

\end{tikzpicture}
\end{center}
\caption{הוכחה ראשונה}\label{f.first-proof}
\end{figure}
$\triangle A'QQ'$
הוא השיקוף של
$\triangle AQ'Q$
ולכן
$\triangle A'QQ'\cong \triangle AQ'Q$.
מכאן שגם
$\triangle A'QQ'$
הוא משולש שווה-שוקיים.
$\overline{QB'}$ 
הוא השיקוף של 
$\overline{Q'B}$
כך ש-%
$\angle A'QB'=\angle Q'QB'=\alpha$.

לפי זוויות מתחלפות
$\angle Q'QR=\angle QQ'B=\alpha$.
ביחד:
\[
\angle A'QB'=\angle Q'QB'=\angle Q'QR=\alpha\,.
\]

\textbf{הוכחה שנייה}


ראו איור~%
\ref{f.second-proof}.
הקו
$l$
הוא קיפול ולכן הוא האנך האמצעי של
$\overline{QQ'}$.
סמנו ב-%
$U$
את נקודת החיתוך של
$l$
עם
$\overline{QQ'}$,
וסמנו ב-%
$V$
את נקודת החיתוך שלו עם
$\overline{QB'}$.
$\triangle VUQ\cong \triangle VUQ'$
לפי צלע-זווית-צלע כי
$\overline{VU}$
הוא צלע משותף, הזוויות ב-%
$U$
הן זוויות ישרות, ו-%
$\overline{QU}=\overline{Q'U}=b$.
מכאן ש-%
$\angle VQU=\angle VQ'U=\alpha$
ו-%
$\angle Q'QR=\angle VQ'U=\alpha$
לפי זוויות מתחלפות.

כמו בהוכחה הראשונה, 
$A', B', Q'$
הן כולן שיקופים סביב
$l$,
לכן הן כולן נמצאות על קטע קו אחד
$\overline{DQ'}$,
ו-%
$\overline{A'B'}=\overline{AB}=\overline{BQ}=\overline{B'Q'}=a$.
מכאן ש-%
$\triangle A'B'Q\cong\triangle Q'B'Q$
ו-%
$\angle A'QB'=\angle Q'QB'=\alpha$.
\begin{figure}[tb]
\begin{center}
\selectlanguage{english}
\begin{tikzpicture}[scale=.8]

% Place points P, Q, R
\coordinate (P) at (60:10cm); %(5,8.67);
\coordinate (Q) at (0,0);
\coordinate (R) at (10,0);
\fill (P) circle (2pt) node[below right] {$P$};
\fill (Q) circle (2pt) node[left] {$Q$};
\fill (R) circle (2pt) node[right] {$R$};

% Draw PQR
\draw [very thick] (P)  -- (Q) -- (R);

% Draw perpendicular to QR
\draw [thick] (Q) -- node[left,very near end] {$p$} +(0,11);

% Draw parallel to QR and parallel halfway
\coordinate (A) at (0,5);
\coordinate (B) at (0,2.5);
\draw [thick] (A) -- node[above,very near end] {$q$} +(10,0);
\draw [thick] (B) -- node[above,very near end] {$r$} +(10,0);
\fill (A) circle (2pt) node[left] {$A$};
\fill (B) circle (2pt) node[left] {$B$};
\path (Q) -- node[left] {$a$} (B) -- node[left] {$a$} (A);
\draw (A) rectangle +(8pt,8pt);
\draw (B) rectangle +(8pt,8pt);

% Tangent line y = -2.75x + 10.69

% Draw fold
\coordinate (D) at (0,10.69);
\coordinate (fold-x) at (3.89,0);
\coordinate (AP) at (3.65,6.33);
\coordinate (QP) at (6.87,2.5);
\coordinate (BP) at (5.26,4.42);
\fill (D) circle (2pt) node[left] {$D$};
\fill (AP) circle (2pt) node[above,yshift=6pt] {$A'$};
\fill (QP) circle (2pt) node[above,yshift=6pt] {$Q'$};
\fill (BP) circle (2pt) node[above,xshift=2pt,yshift=2pt] {$B'$};
\draw [very thick,dashed,name path=fold] (D) -- node[left,near start] {$l$} (fold-x);

% Draw line of reflections
\draw [very thick, dotted] (D) -- (QP);

% Draw trisecting lines
\draw [very thick,dotted,name path=Qr] (Q) -- ($(Q)!1.3!(QP)$);
\draw [very thick,dotted,name path=Qq] (Q) -- ($(Q)!1.3!(BP)$);

% Draw indications of right angles
\draw[rotate=-140] (BP) rectangle +(8pt,8pt);
\path [name intersections = {of = fold and Qr, by = {U}}];
\fill (U) circle (2pt) node[above left,xshift=-2pt,yshift=-2pt] {$U$};
\draw[rotate=20] (U) rectangle +(8pt,8pt);
\path [name intersections = {of = fold and Qq, by = {V}}];
\fill (V) circle (2pt) node[above left,xshift=-2pt,yshift=-2pt] {$V$};

\path (Q) -- node[below,near end] {$b$} (U);
\path (U) -- node[below] {$b$} (QP);

\node[left,xshift=-40pt,yshift=-6pt] at (QP) {$\alpha$};
\node[right,xshift=40pt,yshift=6pt] at (Q) {$\alpha$};
\node[right,xshift=40pt,yshift=28pt] at (Q) {$\alpha$};
\node[right,xshift=30pt,yshift=42pt] at (Q) {$\alpha$};
\end{tikzpicture}
\end{center}
\caption{הוכחה שנייה}\label{f.second-proof}
\end{figure}

\newpage

\section{הבניה של
\L{Martin}
לחלוקת זווית לשלושה חלקים%
}\label{s.trisection-martin}
\begin{figure}
\begin{center}
\selectlanguage{english}
\begin{tikzpicture}[scale=.9]

% Place points P, Q, R
\coordinate (P) at (60:10cm); %(5,8.67);
\coordinate (Q) at (0,0);
\coordinate (R) at (10,0);
\fill (P) circle (2pt) node[below right] {$P$};
\fill (Q) circle (2pt) node[above left] {$Q$};
\fill (R) circle (2pt) node[right] {$R$};

% Draw PQR
\draw [very thick] (R)  -- (Q);
\draw [very thick,name path=pq] (Q) -- (P);

% M is the midpoint of PQ
\coordinate (M) at (2.5, 4.33);
\fill (M) circle (2pt) node[above left,xshift=2pt] {$M$};
\draw [rotate=-90] (M) rectangle +(8pt,8pt);

% Drop a perpendicular from M to QR and extend the line upwards
% This is the given line p
\coordinate (pQR) at (M |- Q);
\draw [thick,name path=p] (pQR) --
   node[left, very near end,yshift=28pt] {$p$}
   ($(pQR)!2!(M)$);
\draw (pQR) rectangle +(8pt,8pt);

% Construct q perpendicular to p through M
\draw [thick,name path=q] ($(M)+(-2,0)$) --
   node[above, very near start,xshift=-30pt] {$q$}
   ($(M)+(10,0)$);

% Construct the fold line t
% Its equation is y = -2.75x + 18.51, as obtained from Geogebra
\coordinate (t1) at (6.7,.085);
\coordinate (t2) at (3.5,8.89);
\draw [very thick,dashed,name path=t] (t1) --
   node[very near end,left] {$l$}
   (t2);

% Construct a perpendicular to t through P
\coordinate (perp-p) at ($(t1)!(P)!(t2)$);
\path [name path=perp-p] (P) -- ($(P)!2.5!(perp-p)$);

% Get its intersection with t denoted Pt
% and its intersection with p named PP
\path [name intersections = {of = t and perp-p, by = {Pt}}];
\path [name intersections = {of = p and perp-p, by = {PP}}];
\fill (PP) circle(2pt) node[left] {$P'$};
\draw [rotate=22] (Pt) rectangle +(8pt,8pt);

% Draw PT
\draw [very thick,dotted] (P) -- (PP);

% Construct a perpendicular to t through Q
\coordinate (perp-q) at ($(t1)!(Q)!(t2)$);
\path[name path=perp-q] (Q) -- ($(Q)!2.1!(perp-q)$);

% Get its intersection with t denoted V
% and its intersection with q denoted S=Q'
\path [name intersections = {of = t and perp-q, by = {V}}];
\path [name intersections = {of = q and perp-q, by = {QP}}];
\fill (QP) circle(2pt) node[above,yshift=4pt] {$Q'$};
\fill (V) circle(2pt) node[above left,xshift=-4pt,yshift=-2pt] {$V$};
\draw [rotate=22] (V) rectangle +(8pt,8pt);

% Draw Q QP
\draw [very thick,dotted,name path=qs] (Q) -- (QP);

% Get the intersection of QS with p denoted U
\path [name intersections = {of = p and qs, by = {U}}];
\fill (U) circle(2pt) node[above left] {$U$};

% Draw PP QP
\draw [very thick,dotted,name path=ts] (PP) -- (QP);

% Get its intersection with QP denoted W
\path [name intersections = {of = ts and pq, by = {W}}];
\fill (W) circle(2pt) node[right,xshift=4pt,yshift=4pt] {$W$};

% Label line segments
\path (P) -- node[left] {$a$} (M);
\path (M) -- node[left]  {$a$} (Q);
\path (PP) -- node[left]  {$b$} (M);
\path (M) -- node[right] {$b$} (U);
\path (Q) -- node[below,near end] {$c$} (V);
\path (V) -- node[below] {$c$} (QP);

% Label angles
\node [xshift=5pt,yshift=20pt]        at (M) {$\gamma$};
\node [xshift=-5pt,yshift=-20pt]      at (M) {$\gamma$};
\node [xshift=18pt,yshift=15pt]       at (Q) {$\beta$};
\node [xshift=-18pt,yshift=-15pt]     at (P) {$\beta$};
\node [left,xshift=-30pt,yshift=7pt]  at (QP) {$\alpha$};
\node [left,xshift=-30pt,yshift=-7pt] at (QP) {$\alpha$};
\node [right,xshift=34pt,yshift=7pt]  at (Q) {$\alpha$};
\end{tikzpicture}
\end{center}
\caption{בניה של \L{Martin}}\label{f.martin}
\end{figure}

ראו איור~%
\ref{f.martin}.
נתונה זווית חדה
$\angle PQR$
תהי
$M$
נקודת האמצע של
$\overline{PQ}$.
בנו
$p$
ניצב ל-%
$\overline{QR}$
העובר דרך
$M$
ובנו
$q$
ניצב ל-%
$p$
העובר דרך 
$M$.
$q$
מקביל ל-%
$\overline{QR}$.
לפי אקסיומה $6$ בנו קיפול
$l$
המניח את
$P$
ב-%
$P'$
על
$p$,
ומניח את
$Q$
ב-%
$Q'$
על
$q$.
ייתכן שקיים מספר קיפולים מתאימים; בחרו את הקיפול החותך את
$\overline{PM}$.
בנו את קטעי הקו
$\overline{PP'}$
ו-%
$\overline{QQ'}$.
סמנו ב-%
$U$
את נקודת החיתוך של
$\overline{QQ'}$
עם
$p$,
וסמנו ב-%
$V$
את נקודת החיתוך שלו עם
$l$.
סמנו ב-%
$W$
את החיתוכים של 
$\overline{PQ}$
ו-%
$\overline{P'Q'}$
עם
$l$.\footnote{%
לא ברור מאליו ש-%
$\overline{PQ}$
ו-%
$\overline{P'Q'}$
חותכים את
$l$
באותה נקודה. 
$\triangle PP'W\sim\triangle QQ'W$
כך שהגבהים מחלקים את הזוויות 
$\angle PWP', \angle QWQ'$
בצורה דומה וחייבים להיות על אותו קו.%
}

%\textbf{הוכחה}

$\triangle QMU\cong \triangle PMP'$
לפי זווית-צלע-זווית:
$\angle P'PM=\angle UQM=\beta$
לפי זוויות מתחלפות,
$\overline{QM}=\overline{MP}=a$
כי 
$M$
היא נקודת האמצע של
$\overline{PQ}$,
$\angle QMU=\angle PMP'$
הן זוויות קודקודיות. מכאן ש-%
$\overline{P'M}=\overline{MU}=b$.

$\triangle P'MQ'\cong\triangle UMQ'$
לפי צלע-זוויות-צלע: הראנו ש-%
$\overline{P'M}=\overline{MU}=b$,
הזוויות ב-%
$M$
הן זוויות ישרות,
$\overline{MQ'}$
הוא צלע משותף. הגובה של המשולש שווה-שוקיים 
$\triangle P'Q'U$ 
הוא חוצה הזווית
$\angle P'Q'U$
ולכן
$\angle P'Q'M=\angle UQ'M=\alpha$.


$\triangle QWV\cong\triangle Q'WV$
לפי צלע-זווית-צלע:
$\overline{QV}=\overline{VQ'}=c$
והזוויות ב-%
$V$
ישרות כי הקיפול הוא האנך אמצעי של
$\overline{QQ'}$,
$\overline{VW}$
הוא צלע משותף. מכאן ש:
\erh{0pt}
\begin{equationarray*}{rcl}
\angle WQV&=&\beta=\angle WQ'V=2\alpha\\
\angle PQR &=& \beta + \alpha = 3\alpha\,,
\end{equationarray*}
%ולכן
%$\angle Q'QR$
%היא שליש מ-%
%$\angle PQR$.


%%%%%%%%%%%%%%%%%%%%%%%%%%%%%%%%%%%%%%%%%%%%%%%%%%%%%%%%%%%%%%%

\section{%
הבניה של 
\L{Messer}
להכפלת קוביה}%
\label{s.cube-messer}

לקוביה בנפח 
$V$
צלעות באורך
$\sqrt[3]{V}$.
נפח קוביה שנפחה פי שניים הוא
 $2 V$,
 כך שיש לבנות קטע קו שאורכו
$\sqrt[3]{2 V}=\sqrt[3]{2} \sqrt[3]{V}$.
אם נוכל לבנות קטע קו באורך
$\sqrt[3]{2}$,
נוכל להכפיל באורך הנתון
$\sqrt[3]{V}$
כדי להכפיל את נפח הקוביה.

קחו דף נייר שהוא ריבוע וקפלו לחצי כדי למצוא את הנקודת
$I=(0,1/2)$,
$Q=(1,1/2)$
)איור~%
\ref{f.lang}(.
בנו את קטעי הקו
$\overline{AC}$
ו-%
$\overline{BQ}$.
\begin{figure}
\begin{center}
\selectlanguage{english}
\begin{tikzpicture}[scale=.5]
% Draw square
\coordinate (A) at (0,12);
\coordinate (B) at (0,0);
\coordinate (C) at (12,0);
\coordinate (D) at (12,12);

\fill (A) circle (2pt) node[left]  {$A=(0,1)$};
\fill (B) circle (2pt) node[left]  {$B=(0,0)$};
\fill (C) circle (2pt) node[right] {$C=(1,0)$};
\fill (D) circle (2pt) node[right] {$D=(1,1)$};

\draw [thick] (A)  -- (B) -- (C) -- (D) -- cycle;

% Divide a side in half

\coordinate (M)  at (0,6);
\coordinate (N) at (12,6);
\fill (M) circle (2pt) node[left] {$I=(0,1/2)$};
\fill (N) circle (2pt) node[right] {$Q=(1,1/2)$};
\draw [thick,dashed] (M) -- (N);


\draw [thick,dotted,name path=ac] (A) -- 
   node[near start,above,xshift=24pt] {$y=1-x$} (C);
\draw [thick,dotted,name path=be2] (B) -- 
   node[near start,above,xshift=-12pt,yshift=-2pt] {$y=\disfrac{1}{2}x$} (N);

\path [name intersections = {of = ac and be2, by = {I}}];
\fill (I) circle (2pt) 
   node[below,xshift=-6pt,yshift=-8pt] {$K=$}
   node[below,xshift=-6pt,yshift=-20pt] {$(2/3,1/3)$};

\coordinate (E)  at (0,4);
\coordinate (F) at (12,4);
\fill (E) circle (2pt) node[left] {$E=(0,1/3)$};
\fill (F) circle (2pt) node[right] {$F=(1,1/3)$};
\draw [thick,dashed] (E) -- (F);

\coordinate (G)  at (0,8);
\coordinate (H) at (12,8);
\fill (G) circle (2pt) node[left] {$G=(0,2/3)$};
\fill (H) circle (2pt) node[right] {$H=(1,2/3)$};
\draw [very thick,dotted] (G) -- (H);
\end{tikzpicture}
\end{center}
\caption{בניית קטע קו באורך $1/3$}\label{f.lang}
\end{figure}
אפשר לחשב את הקואורדינטות של נקודה החיתוך 
$K=(2/3,1/3)$
על ידי פתרון המשוואות של הקטעים הללו:
\erh{4pt}
\begin{equationarray*}{rcl}
y&=&1-x\\
y&=&\disfrac{1}{2}x\,.
\end{equationarray*}
בנו את הקו
$\overline{EF}$
ניצב ל-%
$\overline{AB}$
כך שהוא עובר דרך 
$K$,
ובנו את 
$\overline{GH}$,
השיקוף של
$\overline{BC}$
סביב
$\overline{EF}$.
נסמן צלע של הריבוע 
$a+1$
ונוכיח ש-%
$a=\sqrt[3]{2}$
)איור~%
\ref{f.doubling}(

נשמתש באקסיומה $6$ כדי להניח את 
$C$
ב-%
$C'$
על
$\overline{AB}$,
ולהניח את
$F$
ב-%
$F'$
על
$\overline{GH}$.
סמנו את נקודת החיתוך של הקיפול עם
$\overline{BC}$ 
ב-%
$Q$,
וסמנו את אורכו של
$\overline{BQ}$
ב-%
$b$.
האורך של קטע הקו 
$\overline{QC}$
הוא
$(a+1)-b$.


לאחר ביצוע הקיפול, קטע הקו
$\overline{QC,}$
הוא שיקוף של קטע הקו
$\overline{QC}$
מאותו אורך, וקטע הקו
$\overline{C'F'}$
הוא שיקוף של קטע הקו
$\overline{CF}$
באותו אורך. מסימוני האורכים על
$\overline{AB}$
אפשר לראות שאורכו של
$\overline{GC'}$
הוא:
\erh{0pt}
\begin{equationarray}{c}
%\selectlanguage{english}
a-\disfrac{a+1}{3}=\disfrac{2a-1}{3}\,.\label{eq.one-third}
\end{equationarray}
לבסוף,
$\angle FCQ$
היא זווית ישרה, לכן גם השיקוף
$\angle F'C'Q$ 
הוא זווית ישרה.


\begin{figure}[H]
\begin{center}
\selectlanguage{english}
\begin{tikzpicture}[scale=.6]
% Draw and label square
\coordinate (A) at (0,12);
\coordinate (B) at (0,0);
\coordinate (C) at (12,0);
\coordinate (D) at (12,12);
\fill (A) circle (2pt) node[left]  {$A$};
\fill (B) circle (2pt) node[left]  {$B$};
\fill (C) circle (2pt) node[right] {$C$};
\fill (D) circle (2pt) node[right] {$D$};
\draw (B) rectangle +(12pt,12pt);
\draw[rotate=90] (C) rectangle +(12pt,12pt);
\draw [thick] (A)  -- (B) -- (C) -- (D) -- cycle;

% Draw line one-third from botton
\coordinate (E)  at (0,4);
\coordinate (F) at (12,4);
\fill (E) circle (2pt) node[left] {$E$};
\fill (F) circle (2pt) node[right] {$F$};
\draw [very thick,dotted,name path=ef] (E) -- (F);

% Draw line two-thirds from bottom
\coordinate (G)  at (0,8);
\coordinate (H) at (12,8);
\fill (G) circle (2pt) node[left] {$G$};
\fill (H) circle (2pt) node[right] {$H$};
\draw[rotate=-90] (G) rectangle +(12pt,12pt);
\draw [very thick,dotted] (G) -- (H);

% Draw reflections of C and F
\coordinate (CP) at (0,5.31);
\coordinate (FP) at (2.96,8);
\fill (CP) circle (2pt)
  node[left] {$C'$}
  node[above right,yshift=8pt] {$\alpha$}
  node[below right,xshift=-2pt,yshift=-12pt] {$\alpha'$};
\fill (FP) circle (2pt)
  node[above] {$F'$}
  node[below left,xshift=-8pt] {$\alpha'$};
\draw[rotate=-50] (CP) rectangle +(12pt,12pt);
\draw[very thick,dotted] (CP) -- (FP);

% Draw fold and fold arrows
% Tangent is y = 2.26x - 10.9
% Crosses x axis at (4.83,0)
\coordinate (Q) at (4.83,0);
\fill (Q) circle (2pt)
    node[below] {$Q$}
    node[above left,xshift=-8pt] {$\alpha$};
\draw [very thick,dashed,name path=jd] (Q) -- node[very near end,left] {$l$} (10,12);
\draw[thick,dotted,bend right=40,->] (C) to ($(CP)+(4pt,0)$);
\draw[thick,dotted,bend right=40,->] (F) to ($(FP)+(4pt,4pt)$);

% Draw hypotenuses of right triangles
\draw[very thick,dotted] (CP) -- (Q);
\path (Q)  -- (C);

% Labels on BC and hypotenuses
\path (CP) -- node[right] {$(a+1)-b$} (Q);
\path (Q)  -- node[below] {$(a+1)-b$} (C);
\path (B)  -- node[below] {$b$} (Q);
\path (C)  -- node[right] {$\disfrac{a+1}{3}$} (F);
\path (CP) -- node[right,xshift=10pt] {$\disfrac{a+1}{3}$} (FP);

% Labels on AB
\draw[<->] ($(A)+(-1,0)$)    --
  node[fill=white] {$a$} ($(CP)+(-1,0)$);
\draw[<->] ($(CP)+(-1,0)$)   --
  node[fill=white] {$1$} ($(B)+(-1,0)$);
\draw[<->] ($(CP)+(-2.5,0)$) --
  node[fill=white] {$a-\disfrac{a+1}{3}$} ($(G)+(-2.5,0)$);
\draw[<->] ($(A)+(-2.5,0)$) --
  node[fill=white] {$\disfrac{a+1}{3}$} ($(G)+(-2.5,0)$);
\end{tikzpicture}
\end{center}
\caption{הכלפת קוביה}\label{f.doubling}
\end{figure}

\vspace{-3ex}

$\triangle C'BQ$
הוא משולש ישר-זווית ולפי משפט פיתגורס:
\erh{8pt}
\begin{equationarray*}{rcl}
1^2 + b^2 &=& ((a+1)-b)^2\\
%&=& a^2+2a+1 - 2(a+1)b + b^2\\
%a^2+2a - 2(a+1)b&=&0\\
b&=&\disfrac{a^2+2a}{2(a+1)}\,.
\end{equationarray*}

\vspace{-3ex}

$\angle GC'F' + \angle F'C'Q + \angle QC'B = 180^\circ$
כי הם מרכיבים את הקו
$\overline{GB}$.
נסמן
$\alpha=\angle GC'F'$:
\[
\angle QC'B=180^\circ - \angle F'C'Q - \angle GC'F'= 90^\circ -\alpha\,.
\]
נסמן
$\alpha'=90^\circ-\alpha$.
$\triangle C'BQ$, $\triangle F'GC'$
הם משולשים ישר-זווית, ולכן 
$\angle C'QB=\alpha$
ו-%
$\angle C'F'G=\alpha'$.
מכאן שהמשולשים דומים וממשוואה~%
\ref{eq.one-third}
מתקבלת:
\[
\disfrac{b}{(a+1)-b}=\disfrac{\disfrac{2a-1}{3}}{\disfrac{a+1}{3}}\,.
\]
נציב עבור
$b$:
\erh{12pt}
\begin{equationarray*}{rcl}
\disfrac{\disfrac{a^2+2a}{2(a+1)}}{(a+1)-\disfrac{a^2+2a}{2(a+1)}}&=&\disfrac{2a-1}{a+1}\\
%\disfrac{a^2+2a}{(a+1)\cdot 2(a+1)-(a^2+2a)}&=&\disfrac{2a-1}{a+1}\\
\disfrac{a^2+2a}{a^2+2a +2}&=&\disfrac{2a-1}{a+1}\,.
%a^3+3a^2+2a&=&(2a-1)(a^2+2a+2)\,.
%&=&2a^3+3a^2+2a-2\,.
\end{equationarray*}
נפשט ונקבל
$a^3=2$
ו-%
$a=\sqrt[3]{2}$.


\section{הבניה של
\L{Beloch}
להכפלת קוביה%
}\label{s.cube-beloch}

הבניה מוצגת באיור~%
\ref{f.beloch-cube1}.
נסמן את הנקודה
$(-1,0)$
ב-%
$A$
ואת הנקודה
$(0,-2)$
ב-%
$B$.
נסמן ב-%
$p$ 
את הקו 
$x=1$
וב-%
$q$
את הקו
$y=2$.
לפי אקסיומה $6$ ניתן לבנות קיפול 
$l$
המניח את
$A$
ב-%
$A'$
על 
$p$,
והמניח את
$B$
ב-%
$B'$
על
$q$.
נסמן ב-%
$Y$
את נקודת החיתוך של הקיפול עם ציר ה-%
$y$,
ונסמן ב-%
$X$
את נקודת החיתוך של הקיפול עם ציר ה-%
$x$.

\begin{figure}[tbh]
\begin{center}
\selectlanguage{english}
\begin{tikzpicture}[scale=.9]
% Draw and label square
\coordinate (O) at (0,0);
\coordinate (A) at (-2,0);
\coordinate (B) at (0,-4);
\fill (O) circle (2pt)
  node[below left,xshift=-7pt] {$O$}
  node[below left,yshift=-12pt] {$(0,0)$};
\fill (A) circle (2pt)
  node[above left,xshift=-7pt] {$A$}
  node[below left,xshift=2pt,yshift=0pt] {$(-1,0)$}
  node[above right,xshift=10pt] {$\alpha$};
\fill (B) circle (2pt)
  node[left,xshift=-12pt] {$B$}
  node[left,yshift=-12pt] {$(0,-2)$}
  node[above right,yshift=12pt] {$\alpha'$};

\draw[thick] (0,-4.5) --  node[very near end,above left,yshift=12pt] {$y$-\R{ציר}} +(0,10);
\draw[thick] (-5,0)   -- node[very near start,above left] {$x$-\R{ציר}} +(12,0);
\draw[very thick] (2,-4.5) -- node[very near start, right,yshift=-10pt] {$p\!:x=1$} +(0,10);
\draw[very thick] (-5,4) -- node[very near start, above,xshift=-16pt] {$q\!: y=2$} +(12,0);

\coordinate (AP) at (2,5);
\fill (AP) circle (2pt) node[above right] {$A'$};
\coordinate (BP) at (6.34,4);
\fill (BP) circle (2pt) node[above right] {$B'$};

% Tangent y = -0.8x + 1.26

% Exchanged X and Y 
\coordinate (X) at (0,2.52);
\coordinate (Y) at (3.15,0);
\fill (X) circle (2pt)
  node[right,xshift=4pt,yshift=2pt] {$Y$}
  node[below right,yshift=-14pt] {$\alpha$}
  node[below left,xshift=2pt,yshift=-12pt] {$\alpha'$};

\fill (Y) circle (2pt)
  node[above right,xshift=10pt] {$X$}
 node[below left,xshift=-10pt] {$\alpha$}
 node[above left,xshift=-13pt] {$\alpha'$};
\draw [very thick,dashed] ($(X)!-1.1!(Y)$) -- node[very near end,right,xshift=8pt] {$l$} ($(X)!2!(Y)$);

\draw [very thick,dotted] (A) -- (AP);
\draw [very thick,dotted] (B) -- (BP);

\draw[thick,dotted,bend left=40,->] (A) to ($(AP)+(-4pt,0)$);
\draw[thick,dotted,bend left=40,->] (B) to ($(BP)+(-6pt,-3pt)$);

\draw[rotate=-130] (X) rectangle +(10pt,10pt);
\draw[rotate=-130] (Y) rectangle +(10pt,10pt);

\end{tikzpicture}
\end{center}
\caption{הכפלת קוביה לפי \L{Beloch}}\label{f.beloch-cube1}
\end{figure}
%\textbf{הוכחה}
%
הקיפול הוא האנך האמצעי של
$\overline{AA'}$
ו-%
$\overline{BB'}$,
ולכן
$\overline{AA'}\|\overline{BB'}$.
לפי זוויות מתחלפות
$\angle XAY =\angle AXB=\alpha$.
לפי משולשים ישר-זווית, מתקבלים סימוני הזוויות האחרות,
ו-%
$\triangle AOY\!\sim \!\triangle YOX \!\sim\! \triangle XOB$.
ידוע אורכם של קטעי הקו
$\overline{OA}=1$, $\overline{OB}=2$,
ולכן:

%\newpage

\erh{10pt}
\begin{equationarray*}{rcl}
\disfrac{\overline{OY}}{\overline{OA}}&=&\disfrac{\overline{OX}}{\overline{OY}}=\disfrac{\overline{OB}}{\overline{OX}}\\
\disfrac{\overline{OY}}{1}&=&\disfrac{\overline{OX}}{\overline{OY}}=\disfrac{2}{\overline{OX}}\\
\overline{OY}^2&=&\overline{OX}\\
\overline{OY}\:\overline{OX}&=&2\\
\overline{OY}^3&=&2\\
\overline{OY}&=&\sqrt[3]{2}\,.
\end{equationarray*}

%\vspace{-5ex}

\newpage

%%%%%%%%%%%%%%%%%%%%%%%%%%%%%%%%%%%%%%%%%%%%%%%%%%%%%%%%%%%%%%%%
\section{בניית מתושע}\label{s.nonagon}

%לפי משפט
%\L{Gauss-Wantzel}
ניתן לבנות באמצעות סרגל ומחוגה מצולע משוכלל שמספר צלעותיו הוא:
\[
n=2^k\cdot F_1 \cdot \:\cdots\: \cdot F_m\,,
\]
כאשר ה-%
$F_i$
)אם הם קיימים( הם מספרי 
\L{Fermat}
ראשוניים שונים 
$F_m=2^{2^m}+1$.
חמישה מספרי 
\L{Fermat}
ראשוניים ידועים:
$F_0=3, F_1=5, F_2=17, F_3=257, F_4=65537$.%
\footnote{%
בגיל
$19$ \L{Gauss}
בנה מצולע משוכלל עם 
$17$
צלעות )ראו פרק~%
\ref{c.heptadecagon}(,
והישג זה שיכנע אותו להיות מתמטיקאי. מצולע משוכלל עם 
$257$
צלעות נבנה על ידי
\L{Magnus Georg Paucker}
ב-%
$1822$
ועל ידי
\L{Friedrich Julius Richelot}
ב-%
$1832$.
\L{Johann Gustav Hermes}
טען בשנת 
$1894$
שהוא בנה מצולע משוכלל עם 
$65537$.
צלעות. כתב היד שלו שמור באוניברסיטת
\L{G\"{o}ttigen}.%
} 
לכן לא ניתן לבנות
\textbf{מתושע},
מצולע משוכלל עם תשע צלעות.

באמצעות קיפולי אוריגמי ניתן לבנות מצולעים משוכללים עם:
\[
n=2^i\cdot 3^j \cdot p_1 \cdot \: \cdots\: \cdot p_m,
\]
צלעות כאשר ה-%
$p_i$
)אם הם קיימים( הם מספרים ראשוניים שונים מהצורה
$2^k\cdot 3^l+1$.
כאן נבנה מתושע תוך שימוש בשיטה של
\L{Lill}
והקיפול של
\L{Beloch}.

\subsection{המשוואה ממעלה שלוש עבור מתושע}

ניתן לבנות מצולע משוכלל עם 
$n$
צלעות על ידי בניית הזווית המרכזית
$360^\circ/n$.
עבור מתושע הזווית המרכזית היא
$360^\circ/9=40^\circ$:
\begin{center}
\selectlanguage{english}
\begin{tikzpicture}[scale=.7]
\coordinate (O) at (0,0);
\fill (O) circle (1.5pt);
\foreach \x/\name in {0/a,40/b,80/c,120/d,160/e,200/f,240/g,280/h,320/i} {
  \coordinate (\name) at ($(O)+(\x:3cm)$);
  \draw (O) -- (\name);
  \fill (\name) circle (1.5pt);
}
\draw (a) -- (b) -- (c) -- (d) -- (e) -- (f) -- (g) -- (h) -- (i) -- cycle;
\node[above right,xshift=12pt] at (O) {$40^\circ$};
\end{tikzpicture}
\end{center}

הזווית
$40^\circ$
היא שליש מ-%
$120^\circ$
שניתנת לבניה על ידי הצמדת זווית של 
$30^\circ$ 
לזווית של
$90^\circ$,
שתיהן זוויות שקל לבנות אותן. 
ראינו שניתן לחלק זווית לשלושה חלקים שווים באוריגמי, ולכן ניתן לבנות 
$40^\circ$
באוריגמי. כאן נביא בניה המבוססת על מציאת שורש של פולינום ממעלה שלוש.

על ידי שימוש בזהויות טריגונומטריות נקבל משוואה המקשרת את הקוסינוס של זווית לקוסינוס של שליש מהזווית:
\erh{2pt}
\begin{equationarray*}{rcl}
\cos 3\theta &=& \cos (2\theta +\theta)\\
&=& \cos 2\theta\cos\theta - \sin 2\theta\sin\theta\\
&=& (\cos^2\theta -\sin^2\theta)\cos\theta - (2\sin\theta\cos\theta)\sin\theta\\
&=&\cos\theta (\cos^2\theta - (1-\cos^2\theta)) -(2(1-\cos^2\theta))\\
&=&4\cos^3\theta -3\cos\theta\,.
\end{equationarray*}
אם נתון 
$a=\cos 3\theta$
ונצליח לפתור את המשוואה ממעלה שלוש
$4x^3-3x-a=0$,
נקבל בניה של קטע קו באורך הקוסינוס של הזווית המרכזית של המצולע. בהמשך נראה איך ניתן לבנות את המצולע עצמו.

עבור מתושע המשוואה היא
$4x^3-3x+\frac{1}{2}=0$
כי
$\cos 120^\circ=-\frac{1}{2}$.

נבנה מסלולים עבור המשוואה לפי השיטה של 
\L{Lill}:
\begin{center}
\selectlanguage{english}
\begin{tikzpicture}[scale=1.1]
% Draw help lines and axes
\draw[step=10mm,white!60!black] (-1,-4) grid (9,1);
\draw[thick] (-1,0) -- (9,0);
\draw[thick] (0,-4) -- (0,1);
\foreach \x in {1,...,9}
  \node at (\x-.3,.3) {\sm{\x}};
\foreach \y in {-3,...,1}
  \node at (-.3,\y-.3) {\sm{\y}};
  
% Points of first path
\coordinate (A) at (0,0);
\coordinate (B) at (4,0);
\coordinate (C) at (7,0);
\coordinate (D) at (7,-.5);
\foreach \x in {A,B,C,D}
  \fill (\x) circle(2pt);
\node[above left] at (A) {$P$};
\node[below right,xshift=12pt] at (A) {\sm{-37.4537^\circ}};
\node[below right] at (D) {$Q$};

% Draw first path
\draw[very thick,-{Stealth[scale=1.4,inset=2pt]}] 
  (A) -- node[below] {$a_3$} (B);
\draw[{Stealth[scale=1.4,inset=2pt,reversed]}-,very thick]
  (B) -- ($(B)+(0,.1)$);
\draw[name path=c,very thick,{Stealth[scale=1.4,inset=2pt]}-]
  (B) -- node[below] {$a_1$} (C);
\draw[very thick,-{Stealth[scale=1.4,inset=2pt]}]
  (C) -- node[right,yshift=-2pt] {$a_0$} (D);

% Draw extension of second segment of first path
\draw[very thick,loosely dotted,name path=b] 
  ($(B)+(0,-4)$) -- ($(B)+(0,1)$);

% Draw second path
\path[name path=one] (A) -- +(-37.4537:6cm);
\path [name intersections = {of = b and one, by = {R}}];
\fill (R) circle (2pt) node[below left] {$R$};
\draw[thick,dashed] (A) -- (R);

\path[name path=two] (R) -- +(52.5463:6cm);
\path [name intersections = {of = c and two, by = {S}}];
\fill (S) circle (2pt) node[above] {$S$};
\draw[thick,dashed] (R) -- (S);

\draw[thick,dashed] (S) -- (D);

% Draw right angle rectangles
\draw[thick,rotate=52.5463] (R) rectangle +(8pt,8pt);
\draw[thick,rotate=-127.4537] (S) rectangle +(8pt,8pt);
\end{tikzpicture}
\end{center}
המסלול השני מתחיל מ-%
$P$
בזווית 
$-37.4537^\circ$,
ולכן
$x=-\tan (-37.4537)^\circ=0.766044$
הוא שורש של
$4x^3-3x+\frac{1}{2}$.



\subsection{פתרון המשוואה על ידי הקיפול של
\L{Beloch}}

ניתן למצוא את השורש באמצעות הקיפול של 
\L{Beloch}
)איור~%
\ref{f.nonagon-beloch}(.
נמתח קו מקביל ל-%
$a_2$\footnote{
בגלל ש-%
$a_2=0$
הקו נמתח מקביל לקו שהיינו מציירים אם 
$a_2$
היה שונה מאפס.}
באותו מרחק 
$4$
מ-%
$a_2$
כמו המרחק מ-%
$a_2$
ל-%
$P$.
באופן דומה, נמתח קו מקביל ל-%
$a_1$
באותו מרחק 
$\frac{1}{2}$
מ-%
$a_1$
כמו המרחק מ-%
$a_1$
ל-%
$Q$.
$\overline{RS}$,
הקיפול של
\L{Beloch},
מניח בו-זמנית את
$P$
ב-%
$P'$
על הקו המקביל ל-%
$a_2$,
ואת
$Q$
ב-%
$Q'$
על הקו המקביל ל-%
$a_1$.
הקיפול בונה את הזווית
$\angle SPR=-37.4537^\circ$.
\begin{figure}[H]
\begin{center}
\selectlanguage{english}
\begin{tikzpicture}[scale=1.1]
% Draw help lines and axes
\draw[step=10mm,white!60!black] (-1,-7) grid (9,1);
\draw[thick] (-1,0) -- (9,0);
\draw[thick] (0,-7) -- (0,1);
\foreach \x in {1,...,9}
  \node at (\x-.3,.3) {\sm{\x}};
\foreach \y in {-6,...,1}
  \node at (-.3,\y-.3) {\sm{\y}};
  
% Points of first path
\coordinate (A) at (0,0);
\coordinate (B) at (4,0);
\coordinate (C) at (7,0);
\coordinate (D) at (7,-.5);
\foreach \x in {A,B,C,D}
  \fill (\x) circle(2pt);
\node[above right] at (A) {$P$};
\node[below right,xshift=12pt] at (A) {\sm{-37.4537^\circ}};
\node[below right] at (D) {$Q$};

% Draw first path
\draw[very thick,-{Stealth[scale=1.4,inset=2pt]}] 
  (A) -- node[below] {$a_3$} (B);
\draw[{Stealth[scale=1.4,inset=2pt,reversed]}-,very thick]
  (B) -- ($(B)+(0,.1)$);
\draw[name path=c,very thick,{Stealth[scale=1.4,inset=2pt]}-]
  (B) -- node[below] {$a_1$} (C);
\draw[very thick,-{Stealth[scale=1.4,inset=2pt]}]
  (C) -- node[right,yshift=-2pt] {$a_0$} (D);

% Draw extension of second segment of first path
\draw[very thick,loosely dotted,name path=b] 
  ($(B)+(0,-7)$) -- ($(B)+(0,1)$);

% Draw second path
\path[name path=one] (A) -- +(-37.4537:6cm);
\path [name intersections = {of = b and one, by = {R}}];
\fill (R) circle (2pt) node[below left] {$R$};
%\draw[thick] (A) -- (R);

\path[name path=two] (R) -- +(52.5463:6cm);
\path [name intersections = {of = c and two, by = {S}}];
\fill (S) circle (2pt) node[above] {$S$};
\draw[thick,dashed] (R) -- (S);

% Draw parallel lines
\draw[very thick,loosely dotted,name path=para-2] 
  (8,1) -- (8,-7);
\draw[very thick,loosely dotted,name path=para-1] 
  (-1,.5) -- (9,.5);

% Draw second segments of the folds
\path[name path=p-two] (A) -- +(-37.4537:11cm);
\path [name intersections = {of = para-2 and p-two, by = {PP}}];
\fill (PP) circle (2pt) node[below left] {$P'$};
\draw[ultra thick,dotted] (A) -- (PP);

\path[name path=p-one] (D) -- +(142.5463:2cm);
\path [name intersections = {of = para-1 and p-one, by = {QP}}];
\fill (QP) circle (2pt) node[above] {$Q'$};
\draw[ultra thick,dotted] (D) -- (QP);

% Draw right angle indications
\draw[thick,rotate=-37.4537] (R) rectangle +(8pt,8pt);
\draw[thick,rotate=-127.4537] (S) rectangle +(8pt,8pt);
\end{tikzpicture}
\end{center}
\caption{בנית מתושע באמצעות הקיפול של \L{Beloch}}\label{f.nonagon-beloch}
\end{figure}

\subsection{בניית הזווית המרכזית של המתושע}

$\cos\theta=0.766044$
הוא שורש של המשוואה. כדי לבנות את הזווית המרכזית אנו צריכים לבנות  את
$\cos^{-1} 0.766044=40^\circ$.
נבנה קטע קו באורך
$1$
וקטע קו הניצב לו באורך 
$0.766044$.
מהשלמת הבניה למשולש ישר-זווית מתקבלת הזווית
$37.4537^\circ$,
כי:
\[
\tan 37.4537^\circ = \disfrac{0.766044}{1}\,.
\]
\begin{center}
\selectlanguage{english}
\begin{tikzpicture}[scale=.8]
\draw (0,0) coordinate (A) -- node[below] {$1$} (4,0) coordinate (B);
\draw (B) -- node[right] {$0.766044$} 
  ($(B)+(0,0.766044*4)$) coordinate (C);
\draw (A) -- (C);
\draw[rotate=90] (B) rectangle +(8pt,8pt);
\fill (A) circle (1pt) node[above left] {$A$}
  node[above right,xshift=14pt] {$37.4537^\circ$};
\fill (B) circle (1pt) node[above right] {$B$};
\fill (C) circle (1pt) node[right] {$C$};
\end{tikzpicture}
\end{center}
נקפל את הצלע
$\overline{CB}$
מעל 
$\overline{AB}$
ונסמן את המקום עליו מונח הנקודה
$C$
ב-%
$D$.
נעתיק את קטע הקו
$\overline{AB}$ 
ימינה )ללא סיבוב( כך שהנקודה
$A$
מונחת על
$D$,
ונסמן את המקום עליו מונחת הנקודה 
$B$
ב-% 
$E$:
\begin{center}
\selectlanguage{english}
\begin{tikzpicture}[scale=.8]
\coordinate (A) at (0,0);
\coordinate (B) at (4,0);
\draw (B) -- node[right] {$0.766044$} 
  ($(B)+(0,0.766044*4)$) coordinate (C);
\draw (A) -- (C);
\draw[rotate=90] (B) rectangle +(8pt,8pt);
\fill (A) circle (1pt) node[above left] {$A$};
\fill (B) circle (1pt) node[above right] {$B$};
\fill (C) circle (1pt) node[right] {$C$};
\coordinate (D) at (0.233956*4,0);
\fill (D) circle (1pt) node[below left] {$D$};
\coordinate (E) at ($(D)+(4,0)$);
\draw (D) -- node[fill=white] {$0.766044$} (B);
\draw (B) -- (E);
\fill (E) circle(1pt) node[above right] {$E$};
\draw[very thick,dotted,->,bend right=50] ($(C)+(-.2,0)$) to ($(A)+(.94,.2)$);
\draw[<->] ($(D)+(0,-.6)$) -- node[fill=white] {$1$} ($(E)+(0,-.6)$);
\end{tikzpicture}
\end{center}

נקפל את 
$\overline{DE}$
מעל להמשך של 
$\overline{CB}$
כך שהנקודה
$E$
מונחת על הקו בנקודה
$F$:
\begin{center}
\selectlanguage{english}
\begin{tikzpicture}[scale=.8]
\coordinate (B) at (4,0);
\draw (B) -- ($(B)+(0,0.766044*4)$) coordinate (C);
\draw[rotate=90] (B) rectangle +(8pt,8pt);
\fill (B) circle (1pt) node[above right] {$B$};
\fill (C) circle (1pt) node[right] {$C$};
\coordinate (D) at (0.233956*4,0);
\fill (D) circle (1pt) node[above left] {$D$}
  node[above right,xshift=10pt,yshift=4pt] {$40^\circ$};
\coordinate (E) at ($(D)+(4,0)$);
\draw (D) -- node[fill=white] {$0.766044$} (B);
\draw (B) -- (E);
\fill (E) circle(1pt) node[right] {$E$};
\coordinate (F) at ($(B)+(0,4)$);
\draw[very thick,dotted,->,bend right=50] ($(E)+(.1,.2)$) to ($(F)+(.2,0)$);
\draw (B) -- ($(B)+(0,5)$);
\fill (F) circle (1pt) node[left] {$F$};
\draw (D) -- node[fill=white] {$1$} (F);
\draw[<->] ($(D)+(0,-.6)$) -- node[fill=white] {$1$} ($(E)+(0,-.6)$);
\end{tikzpicture}
\end{center}
נקבל:
\[
\angle BDF=\cos^{-1} \disfrac{0.766044}{1}=40^\circ\,.
\]

\subsection*{מקודות}

פרק זה מבוסס על
\L{\cite{alperin,lang,martin,newton}}.
