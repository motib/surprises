% !TeX root = surprises.tex

%%%%%%%%%%%%%%%%%%%%%%%%%%%%%%%%%%%%%%%%%%%%%%%%%%%%%%%%%%%%%%%%

\chapter{בניות גיאומטריות באוריגמי}\label{c.origami-constructions}

בפרק זה נראה שבניות באוריגמי חזקות יותר מבניות בסרגל ובמחוגה. נביא שתי בניות לחלוקת זווית לשלושה חלקים, הראשונה של היסאשי אייב
\L{(Hisashi Abe)}
(סעיף%
~\ref{s.abe-trisection})
והשנייה של ג'ורג' א' מרטין
\L{(George E. Martin)}
(סעיף%
~\ref{s.martin-trisection});
שתי בניות להכפלת קובייה, הראשונה של פיטר מסר
\L{(Peter Messer)}
(סעיף%
~\ref{s.messer})
והשנייה של מרגריטה פ' בלוץ'
\L{(Marghareta P. Beloch)}
(סעיף%
~\ref{s.cube2}).
הבנייה האחרונה היא של מתושע
\L{(nonagon)},
מצולע משוכלל בעל תשע צלעות (סעיף%
~\ref{s.nonagon}).

\section{הבנייה של אייב לחלוקת זווית לשלושה חלקים}\label{s.abe-trisection}

נתונה זווית חדה
$\angle PQR$,
נבנה את הישר
$p$
המאונך ל-%
$\overline{QR}$
ב-%
$Q$
ואת הישר
$q$
המאונך ל-%
$p$
ב-% 
$A$
כך שהוא חותך את
$\overline{PQ}$.
נבנה את הישר
$r$,
האנך האמצעי של
$\overline{AQ}$
שחותך אותו בנקודה
$B$.
לפי אקסיומה 
$6$
נבנה קיפול
$l$
המניח את 
$A$
על
$\overline{PQ}$ 
בנקודה
$A'$
ומניח את
$Q$
על
$r$
בנקודה
$Q'$.
תהי
$B'$
השיקוף של 
$B$
ביחס ל-%
$l$.
נבנה את הקטעים
$\overline{QB'}$
ו-%
$\overline{QQ'}$
(איור%
~\ref{f.abe1}).

\begin{figure}[tb]
\begin{center}
\begin{tikzpicture}[scale=.8]
% Place points P, Q, R
\coordinate (P) at (60:10cm); %(5,8.67);
\coordinate (Q) at (0,0);
\coordinate (R) at (10,0);
\fill (P) circle (2pt) node[below right] {$P$};
\fill (Q) circle (2pt) node[left] {$Q$};
\fill (R) circle (2pt) node[right] {$R$};

% Draw PQR
\draw [very thick] (P)  -- (Q) -- (R);

% Draw perpendicular to QR
\draw [thick] (Q) -- node[left,very near end] {$p$} +(0,11);

% Draw parallel to QR and parallel halfway
\coordinate (A) at (0,5);
\coordinate (B) at (0,2.5);
\draw [thick] (A) -- node[above,very near end] {$q$} +(10,0);
\draw [thick] (B) -- node[above,very near end] {$r$} +(10,0);
\fill (A) circle (2pt) node[left] {$A$};
\fill (B) circle (2pt) node[left] {$B$};
\path (Q) -- node[left] {$a$} (B) -- node[left] {$a$} (A);
\draw (A) rectangle +(8pt,8pt);
\draw (B) rectangle +(8pt,8pt);

% Tangent line y = -2.75x + 10.69

% Draw fold
\coordinate (D) at (0,10.69);
\coordinate (fold-x) at (3.89,0);
\coordinate (AP) at (3.65,6.33);
\coordinate (QP) at (6.87,2.5);
\coordinate (BP) at (5.26,4.42);
\fill (D) circle (2pt) node[left] {$D$};
\fill (AP) circle (2pt) node[above,yshift=6pt] {$A'$};
\fill (QP) circle (2pt) node[above,yshift=6pt] {$Q'$};
\fill (BP) circle (2pt) node[above,xshift=2pt,yshift=2pt] {$B'$};
\draw [very thick,dashed] (D) -- node[left,near start] {$l$} (fold-x);

% Draw line of reflections
\draw [very thick, dotted] (D) -- (QP);

% Draw trisecting lines
\draw [very thick,dotted] (Q) -- ($(Q)!1.3!(QP)$);
\draw [very thick,dotted] (Q) -- ($(Q)!1.3!(BP)$);

% Complete triangle
\draw [very thick,dotted] (A) -- (QP);

% Draw fold arrows
\draw[thick,dotted,->,bend left=40]
  ($(A)+(4pt,10pt)$) to ($(AP)+(-4pt,0pt)$);
\draw[thick,dotted,->,bend right=40]
  ($(Q)+(2pt,-4pt)$) to ($(QP)+(0pt,-4pt)$);

\end{tikzpicture}
\selectlanguage{hebrew}
\caption{הבנייה של אייב
לחלוקת זווית לשלושה חלקים}
\label{f.abe1}
\end{center}
\end{figure}

\begin{theorem}
$\angle PQB'=\angle B'QQ'=\angle Q'QR=\angle PQR/3$.
\end{theorem}
\begin{proof}(1)
הנקודות
$A', B', Q'$
הן שיקופים ביחס לאותו ישר 
$l$
של הנקודות
$A,B,Q$
הנמצאות על ישר אחד
$\overline{DQ}$,
ולכן הן נמצאות על קטע אחד
$\overline{DQ'}$.
לפי הבנייה,
$\overline{AB}=\overline{BQ}$,
$\angle ABQ'=\angle QBQ'=90^\circ$,
$\overline{BQ'}$
היא צלע משותפת, ולכן
$\triangle ABQ'\cong \triangle QBQ'$ 
לפי צלע,זווית, צלע. מכאן ש-%
$\angle AQ'B=\angle BQ'Q=\alpha$
ו-%
$\triangle AQ'Q$
שווה-שוקיים (איור~
\ref{f.first-proof}).

\begin{figure}[tb]
\begin{center}
\begin{tikzpicture}[scale=.8]

% Place points P, Q, R
\coordinate (P) at (60:10cm);
\coordinate (Q) at (0,0);
\coordinate (R) at (10,0);
\node[below right] at (P) {$P$};
\node[left,xshift=-4pt] at (Q) {$Q$};
\node[right] at (R) {$R$};

% Draw PQR
\draw (Q) -- (R);

% Draw perpendicular to QR
\draw (Q) -- node[left,very near end] {$p$} +(0,11);

% Draw parallel to QR and parallel halfway
\coordinate (A) at (0,5);
\coordinate (B) at (0,2.5);
\draw (A) -- node[above,very near end] {$q$} +(10,0);
\draw[name path=Br] (B) -- node[above,very near end] {$r$} +(10,0);
\node[left,xshift=-4pt] at (A) {$A$};
\node[left,xshift=-4pt] at (B) {$B$};
\path (Q) -- node[left,xshift=-4pt] {$a$} (B) -- node[left,xshift=-4pt] {$a$} (A);
\draw (A) rectangle +(8pt,8pt);
\draw (B) rectangle +(8pt,8pt);
\draw (Q) rectangle +(8pt,8pt);

% Tangent line y = -2.75x + 10.69

% Draw fold
\coordinate (D) at (0,10.69);
\coordinate (fold-x) at (3.89,0);
\coordinate (AP) at (3.65,6.33);
\coordinate (QP) at (6.87,2.5);
\coordinate (BP) at (5.26,4.42);
\node[left] at (D) {$D$};
\node[above,yshift=6pt] at (AP)  {$A'$};
\node[above,xshift=2pt,yshift=6pt] at (QP) {$Q'$};
\node[right,xshift=4pt,yshift=2pt] at (BP) {$B'$};
\draw[name path=fold,very thick,dashed] (D) -- node[left,near start] {$l$}  ($(D)!1.03!(fold-x)$);
	
% Draw line of reflections
\draw (D) -- (AP);

% Draw trisecting lines
\draw (Q) -- ($(Q)!1.3!(BP)$);

\draw [very thick,loosely dash dot,red] (Q) -- (QP);
\draw [very thick,loosely dash dot,red] (QP) -- (AP);
\draw [very thick,loosely dash dot,red] (AP) -- (Q);
\draw [very thick,loosely dash dot dot,blue] ($(Q)+(0,-4pt)$) -- ($(QP)+(0,-4pt)$);
\draw [very thick,dash dot dot,blue] ($(QP)+(0,-4pt)$) -- ($(A)+(0,-4pt)$);
\draw [very thick,dash dot dot,blue] ($(A)+(-4pt,0)$) -- ($(Q)+(-4pt,0)$);

\draw (A) -- (AP);

\node[left,xshift=-40pt,yshift=7pt] at (QP) {$\alpha$};
\node[left,xshift=-40pt,yshift=-6pt] at (QP) {$\alpha$};
\node[right,xshift=40pt,yshift=6pt] at (Q) {$\alpha$};
\node[right,xshift=40pt,yshift=28pt] at (Q) {$\alpha$};
\node[right,xshift=30pt,yshift=42pt] at (Q) {$\alpha$};

\draw (AP) -- (P);

\path[name path=Qr] (Q) -- (QP);
\path[name intersections = {of = fold and Qr, by = {U}}];
\node[above left,xshift=-2pt,yshift=-2pt] at (U) {$U$};
\draw[rotate=20] (U) rectangle +(8pt,8pt);
\path[name intersections = {of = fold and Br, by = {V}}];
\node[above left,xshift=-2pt,yshift=-2pt] at (V)  {$V$};
\end{tikzpicture}
\end{center}
\selectlanguage{hebrew}
\caption{%
הוכחות של הבנייה של אייב
\L
(נשתמש ב-%
$U,V$
בהוכחה השנייה)}
\label{f.first-proof}
\end{figure}
בגלל השיקוף
$\triangle AQ'Q\cong \triangle A'QQ'$,
ולכן גם
$\triangle A'QQ'$
הוא משולש שווה-שוקיים.
$\overline{QB'}$,
השיקוף של 
$\overline{Q'B}$,
הוא גובה לבסיס במשולש שווה-שוקיים ולכן
$\angle A'QB'=\angle Q'QB'=\alpha$.
לפי זוויות מתחלפות
$\angle Q'QR=\angle QQ'B=\alpha$.
ביחד:
\[
\triangle PQB'=\angle A'QB'=\angle B'QQ'=\angle Q'QR=\alpha\,.
\]
\end{proof}

\begin{proof}(2)
הקו
$l$
הוא קיפול ולכן הוא האנך האמצעי של
$\overline{QQ'}$.
נסמן ב-%
$U$
את נקודת החיתוך של
$l$
עם
$\overline{QQ'}$,
נסמן ב-%
$V$
את נקודת החיתוך שלו עם
$\overline{QB'}$
(איור%
~\ref{f.first-proof}).
$\triangle VUQ\cong \triangle VUQ'$
לפי צלע, זווית, צלע כי
$\overline{VU}$
צלע משותפת, הזוויות ב-%
$U$
הן זוויות ישרות, ו-%
$\overline{QU}=\overline{Q'U}$.
מכאן ש-%
$\angle VQU=\angle VQ'U=\alpha$
ו-%
$\angle Q'QR=\angle VQ'U=\alpha$
לפי זוויות מתחלפות.

כמו בהוכחה הראשונה, 
$A', B', Q'$
הן כולן שיקופים ביחס לישר
$l$,
ולכן כולן מונחות על קטע אחד
$\overline{DQ'}$,
ו-%
$\overline{A'B'}=\overline{AB}=\overline{BQ}=\overline{B'Q'}=a$.
מכאן ש-%
$\triangle A'B'Q\cong\triangle Q'B'Q$
לפי צלע, זווית, צלע ו-%
$\angle A'QB'=\angle Q'QB'=\alpha$.
\end{proof}

\section{הבנייה של מרטין
\L{\normalsize }
לחלוקת זווית לשלושה חלקים%
}\label{s.martin-trisection}
\begin{figure}[tb]
\begin{center}

\begin{tikzpicture}[scale=.9]

% Place points P, Q, R
\coordinate (P) at (60:10cm); %(5,8.67);
\coordinate (Q) at (0,0);
\coordinate (R) at (10,0);
\fill (P) circle (2pt) node[below right] {$P$};
\fill (Q) circle (2pt) node[above left] {$Q$};
\fill (R) circle (2pt) node[right] {$R$};

% Draw PQR
\draw (R)  -- (Q);
\draw [name path=pq] (Q) -- (P);

% M is the midpoint of PQ
\coordinate (M) at (2.5, 4.33);
\fill (M) circle (2pt) node[above left,xshift=2pt] {$M$};
\draw [rotate=-90] (M) rectangle +(8pt,8pt);

% Drop a perpendicular from M to QR and extend the line upwards
% This is the given line p
\coordinate (pQR) at (M |- Q);
\draw [thick,name path=p] (pQR) --
   node[left, very near end,yshift=28pt] {$p$}
   ($(pQR)!2!(M)$);
\draw (pQR) rectangle +(8pt,8pt);

% Construct q perpendicular to p through M
\draw [thick,name path=q] ($(M)+(-2,0)$) --
   node[above, very near start,xshift=-30pt] {$q$}
   ($(M)+(10,0)$);

% Construct the fold line t
% Its equation is y = -2.75x + 18.51, as obtained from Geogebra
\coordinate (t1) at (6.7,.085);
\coordinate (t2) at (3.5,8.89);
\draw [very thick,dashed,name path=t] (t1) --
   node[very near end,left] {$l$}
   (t2);

% Construct a perpendicular to t through P
\coordinate (perp-p) at ($(t1)!(P)!(t2)$);
\path [name path=perp-p] (P) -- ($(P)!2.5!(perp-p)$);

% Get its intersection with t denoted Pt
% and its intersection with p named PP
\path [name intersections = {of = t and perp-p, by = {Pt}}];
\path [name intersections = {of = p and perp-p, by = {PP}}];
\fill (PP) circle(2pt) node[left] {$P'$};
\draw [rotate=22] (Pt) rectangle +(8pt,8pt);

% Draw PT
\draw (P) -- (PP);

% Construct a perpendicular to t through Q
\coordinate (perp-q) at ($(t1)!(Q)!(t2)$);
\path[name path=perp-q] (Q) -- ($(Q)!2.1!(perp-q)$);

% Get its intersection with t denoted V
% and its intersection with q denoted S=Q'
\path [name intersections = {of = t and perp-q, by = {V}}];
\path [name intersections = {of = q and perp-q, by = {QP}}];
\fill (QP) circle(2pt) node[above,yshift=4pt] {$Q'$};
\fill (V) circle(2pt) node[above left,xshift=-4pt,yshift=-2pt] {$V$};
\draw [rotate=22] (V) rectangle +(8pt,8pt);

% Draw Q QP
\draw [name path=qs] (Q) -- (QP);

% Get the intersection of QS with p denoted U
\path [name intersections = {of = p and qs, by = {U}}];
\fill (U) circle(2pt) node[above left] {$U$};

% Draw PP QP
\draw [name path=ts] (PP) -- (QP);

% Get its intersection with QP denoted W
\path [name intersections = {of = ts and pq, by = {W}}];
\fill (W) circle(2pt) node[right,xshift=4pt,yshift=4pt] {$W$};

% Label line segments
\path (P) -- node[left] {$a$} (M);
\path (M) -- node[left]  {$a$} (Q);
\path (PP) -- node[left]  {$b$} (M);
\path (M) -- node[right] {$b$} (U);
\path (Q) -- node[below,near end] {$c$} (V);
\path (V) -- node[below] {$c$} (QP);

% Label angles
\node [xshift=5pt,yshift=20pt]        at (M) {$\gamma$};
\node [xshift=-5pt,yshift=-20pt]      at (M) {$\gamma$};
\node [xshift=18pt,yshift=15pt]       at (Q) {$\beta$};
\node [xshift=-18pt,yshift=-15pt]     at (P) {$\beta$};
\node [left,xshift=-30pt,yshift=7pt]  at (QP) {$\alpha$};
\node [left,xshift=-30pt,yshift=-7pt] at (QP) {$\alpha$};
\node [right,xshift=34pt,yshift=7pt]  at (Q) {$\alpha$};
\end{tikzpicture}
\end{center}
\selectlanguage{hebrew}
\caption{בנייה של Martin}\label{f.martin}
\end{figure}

\textbf{בנייה}
נתונה זווית חדה
$\angle PQR.$
תהי
$M$
נקודת האמצע של
$\overline{PQ}$.
בנו
$p$
ניצב ל-%
$\overline{QR}$
העובר דרך
$M$
ובנו
$q$
ניצב ל-%
$p$
העובר דרך 
$M$.
$q$
מקביל ל-%
$\overline{QR}$.
לפי אקסיומה $6$ בנו קיפול
$l$
המניח את
$P$
ב-%
$P'$
על
$p$,
ומניח את
$Q$
ב-%
$Q'$
על
$q$.
ייתכן שקיימים מספר קיפולים מתאימים; בחרו את הקיפול החותך את
$\overline{PM}$.
בנו את הקטעים
$\overline{PP'}$
ו-%
$\overline{QQ'}$
(איור~%
\ref{f.martin}).

\begin{theorem}
$\angle Q'QR = \angle PQR/3$.
\end{theorem}

\begin{proof}
סמנו ב-%
$U$
את נקודת החיתוך של
$\overline{QQ'}$
עם
$p$,
וסמנו ב-%
$V$
את נקודת החיתוך שלו עם
$l$.
סמנו ב-%
$W$
את החיתוכים של 
$\overline{PQ}$
ו-%
$\overline{P'Q'}$
עם
$l$.
לא ברור מאליו ש-%
$\overline{PQ}$
ו-%
$\overline{P'Q'}$
חותכים את
$l$
באותה נקודה. אבל
$\triangle PWP'\sim\triangle QWQ'$
ומכאן שהגבהים חוצים את שתי הזוויות הקודקודיות
$\angle PWP', \angle QWQ'$
והם חייבים להיות על ישר אחד.

$\triangle QMU\cong \triangle PMP'$
לפי זווית, צלע, זווית:
$\angle P'PM=\angle UQM=\beta$
לפי זוויות מתחלפות,
$\overline{QM}=\overline{MP}=a$
כי 
$M$
היא נקודת האמצע של
$\overline{PQ}$
ו-%
$\angle QMU=\angle PMP'=\gamma$
כי הן זוויות קודקודיות. מכאן ש-%
$\overline{P'M}=\overline{MU}=b$.

$\triangle P'MQ'\cong\triangle UMQ'$
לפי צלע, זוויות, צלע: הראינו ש-%
$\overline{P'M}=\overline{MU}=b$,
הזוויות ב-%
$M$
הן זוויות ישרות,
$\overline{MQ'}$
צלע משותפת. הגובה של המשולש שווה-השוקיים 
$\triangle P'Q'U$ 
הוא חוצה הזווית
$\angle P'Q'U$
ולכן
$\angle P'Q'M=\angle UQ'M=\alpha$.
בנוסף,
$\angle UQ'M=\angle Q'QR=\alpha$
לפי זוויות מתחלפות.

$\triangle QWV\cong\triangle Q'WV$
לפי צלע, זווית, צלע:
$\overline{QV}=\overline{VQ'}=c$
הזוויות ב-%
$V$
ישרות ו-%
$\overline{VW}$
צלע משותפת. מכאן ש:
\begin{eqn}
\angle WQV&=&\beta=\angle WQ'V=2\alpha\\
\angle PQR &=& \beta + \alpha = 3\alpha\,.
\end{eqn}
\end{proof}

%%%%%%%%%%%%%%%%%%%%%%%%%%%%%%%%%%%%%%%%%%%%%%%%%%%%%%%%%%%%%%%

\newpage

\section{הבנייה של מסר להכפלת קובייה}%
\label{s.messer}

לקובייה בנפח 
$V$
צלעות באורך
$\sqrt[3]{V}$.
בקובייה בעלת נפח כפול אורכי הצלעות הם
$\sqrt[3]{2 V}=\sqrt[3]{2} \sqrt[3]{V}$,
ולכן אם נוכל לבנות קטע באורך
$\sqrt[3]{2}$,
נוכל לכפול אותו באורך הנתון
$\sqrt[3]{V}$
כדי להכפיל את נפח הקובייה.

\textbf{בנייה:}
ניקח דף נייר שהוא ריבוע יחידה ונקפל אותו לחצי כדי למצוא את הנקודות
$I=(0,1/2)$,
$J=(1,1/2)$
(איור~%
\ref{f.lang}).
נבנה את הקטעים
$\overline{AC}$
ו-%
$\overline{BJ}$.
אפשר לחשב את שיעורי נקודת החיתוך 
$K=(2/3,1/3)$
על ידי פתרון המשוואות 
$y=1-x$
ו-%
$y=x/2$.

\begin{figure}[tb]
\begin{center}
\begin{tikzpicture}[scale=.55]
% Draw square
\coordinate (A) at (0,12);
\coordinate (B) at (0,0);
\coordinate (C) at (12,0);
\coordinate (D) at (12,12);

\node[left]  at (A) {$A=(0,1)$};
\node[left]  at (B) {$B=(0,0)$};
\node[right] at (C) {$C=(1,0)$};
\node[right] at (D) {$D=(1,1)$};

\draw [thick] (A)  -- (B) -- (C) -- (D) -- cycle;

% Divide a side in half

\coordinate (M)  at (0,6);
\coordinate (N) at (12,6);
\node[left] at (M) {$I=(0,1/2)$};
\node[right] at (N) {$J=(1,1/2)$};
\draw [thick,dashed] (M) -- (N);


\draw [very thick,dotted,name path=ac] (A) -- 
   node[near start,above,xshift=24pt] {$y=1-x$} (C);
\draw [very thick,dotted,name path=be2] (B) -- 
   node[near start,above,xshift=-12pt] {$y=x/2$} (N);

\path [name intersections = {of = ac and be2, by = {I}}];
\node[below,xshift=-6pt,yshift=-8pt] at (I) {$K=$};
\node[below,xshift=-6pt,yshift=-20pt] at (I) {$(2/3,1/3)$};

\coordinate (E)  at (0,4);
\coordinate (F) at (12,4);
\node[left] at (E) {$E=(0,1/3)$};
\node[right] at (F) {$F=(1,1/3)$};
\draw [thick,dashed] (E) -- (F);

\coordinate (G)  at (0,8);
\coordinate (H) at (12,8);
\node[left] at (G) {$G=(0,2/3)$};
\node[right] at (H) {$H=(1,2/3)$};
\draw (G) -- (H);
\end{tikzpicture}
\end{center}
\selectlanguage{hebrew}
\caption{בניית קטע באורך
 $1/3$}\label{f.lang}
\end{figure}
נבנה את הקטע
$\overline{EF}$
ניצב ל-%
$\overline{AB}$
דרך 
$K$,
ונבנה את 
$\overline{GH}$,
השיקוף של
$\overline{BC}$
ביחס ל-
$\overline{EF}$.
חילקנו את צלע הריבוע לקטעים באורך 
$1/3$.

נשתמש באקסיומה $6$ כדי להניח את 
$C$
ב-%
$C'$
על
$\overline{AB}$,
ולהניח את
$F$
ב-%
$F'$
על
$\overline{GH}$.
נסמן ב-%
$L$
את נקודת החיתוך של הקיפול עם
$\overline{BC}$ 
ונסמן 
ב-%
$b$
את אורכו של
$\overline{BL}$.
נשנה את סימון הצלע של הריבוע ל-%
$a+1$
כאשר
$a=\overline{AC'}$.
אורכו של
$\overline{LC}$
הוא
$(a+1)-b$
(איור~%
\ref{f.doubling}).
\begin{theorem}
$\overline{AC'}=\sqrt[3]{2}$.
\end{theorem}

\begin{proof}
לאחר ביצוע הקיפול, הקטע
$\overline{LC'}$
הוא שיקוף של הקטע 
$\overline{LC}$
באותו אורך, והקטע 
$\overline{C'F'}$
הוא שיקוף של הקטע
$\overline{CF}$.
מכאן ש:
\begin{equation}
\overline{GC'}=a-\disfrac{a+1}{3}=\disfrac{2a-1}{3}\,.\label{eq.one-third}
\end{equation}
$\angle FCL$
היא זווית ישרה, לכן גם
$\angle F'C'L$ 
היא זווית ישרה.
\begin{figure}[tb]
\begin{center}
\begin{tikzpicture}[scale=.65]
% Draw and label square
\coordinate (A) at (0,12);
\coordinate (B) at (0,0);
\coordinate (C) at (12,0);
\coordinate (D) at (12,12);
\node[left]  at (A) {$A$};
\node[left]  at (B) {$B$};
\node[right] at (C) {$C$};
\node[right] at (D) {$D$};
\draw (B) rectangle +(9pt,9pt);
\draw[rotate=90] (C) rectangle +(9pt,9pt);
\draw [thick] (A)  -- (B) -- (C) -- (D) -- cycle;

% Draw line one-third from botton
\coordinate (E)  at (0,4);
\coordinate (F) at (12,4);
\node[left] at (E) {$E$};
\node[right] at (F) {$F$};
\draw [name path=ef] (E) -- (F);

% Draw line two-thirds from bottom
\coordinate (G)  at (0,8);
\coordinate (H) at (12,8);
\node[left] at (G) {$G$};
\node[right] at (H) {$H$};
\draw[rotate=-90] (G) rectangle +(9pt,9pt);
\draw (G) -- (H);

% Draw reflections of C and F
\coordinate (CP) at (0,5.31);
\coordinate (FP) at (2.96,8);
\node[left] at (CP) {$C'$};
\node[above right,yshift=8pt] at (CP) {$\alpha$};
\node[below right,xshift=-2pt,yshift=-12pt] at (CP) {$\alpha'$};
\node[above] at (FP) {$F'$};
\node[below left,xshift=-8pt] at (FP) {$\alpha'$};
\draw[rotate=-50] (CP) rectangle +(9pt,9pt);
\draw (CP) -- (FP);

% Draw fold and fold arrows
% Tangent is y = 2.26x - 10.9
% Crosses x axis at (4.83,0)
\coordinate (J) at (4.83,0);
\node[below] at (J) {$L$};
\node[above left,xshift=-8pt] at (J) {$\alpha$};
\draw [very thick,dashed,name path=jd] (J) -- node[very near end,left] {$l$} (10,12);
\draw[thick,dotted,bend right=40,->] (C) to ($(CP)+(4pt,0)$);
\draw[thick,dotted,bend right=40,->] (F) to ($(FP)+(4pt,4pt)$);

% Draw hypotenuses of right triangles
\draw (CP) -- (J);
\path (J)  -- (C);

% Labels on BC and hypotenuses
\path (CP) -- node[right] {$(a+1)-b$} (J);
\path (J)  -- node[below] {$(a+1)-b$} (C);
\path (B)  -- node[below] {$b$} (J);
\path (C)  -- node[right] {$\displaystyle\frac{a+1}{3}$} (F);
\path (CP) -- node[right,xshift=10pt] {$\displaystyle\frac{a+1}{3}$} (FP);

% Labels on AB
\draw[<->] ($(A)+(-1,0)$)    --
  node[fill=white] {$a$} ($(CP)+(-1,0)$);
\draw[<->] ($(CP)+(-1,0)$)   --
  node[fill=white] {$1$} ($(B)+(-1,0)$);
\draw[<->] ($(CP)+(-2.5,0)$) --
  node[fill=white] {$\displaystyle\frac{2a-1}{3}$} ($(G)+(-2.5,0)$);
\draw[<->] ($(A)+(-2.5,0)$) --
  node[fill=white] {$\displaystyle\frac{a+1}{3}$} ($(G)+(-2.5,0)$);
\end{tikzpicture}
\end{center}
\selectlanguage{hebrew}
\caption{בניית
$\sqrt[3]{2}$}
\label{f.doubling}
\end{figure}

$\triangle C'BL$
הוא משולש ישר-זווית ולפי משפט פיתגורס:
\begin{eqnlabels}
1^2 + b^2 &=& ((a+1)-b)^2\\
b&=&\disfrac{a^2+2a}{2(a+1)}\label{eq.value-of-b}\,.
\end{eqnlabels}
$\angle GC'F' + \angle F'C'L + \angle LC'B = 180^\circ$
כי הם יוצרים את הישר
$\overline{GB}$.
נסמן
$\alpha=\angle GC'F'$:
\[
\angle LC'B=180^\circ - \angle F'C'L - \angle GC'F'= 90^\circ -\alpha\,,
\]
ונסמן
$\alpha'=90^\circ-\alpha$.
המשולשים
$\triangle C'LB$, $\triangle C'F'G$
הם משולשים ישרי-זווית, ולכן 
$\angle C'LB=\alpha$
ו-%
$\angle C'F'G=\alpha'$.
מכאן ש-%
$\triangle C'BL\sim\triangle F'GC'$
ו-
\[
\frac{\overline{BL}}{\overline{C'L}}=\frac{\overline{GC'}}{\overline{C'F'}}\,.
\]
ממשוואה~%
\ref{eq.one-third}
מתקבל:
\[
\disfrac{b}{(a+1)-b}=\disfrac{\disfrac{2a-1}{3}}{\disfrac{a+1}{3}}\,.
\]
נשתמש בערכו של
$b$
ממשוואה%
~\ref{eq.value-of-b}
ונקבל:
\begin{eqn}
\disfrac{\disfrac{a^2+2a}{2(a+1)}}{(a+1)-\disfrac{a^2+2a}{2(a+1)}}&=&\disfrac{2a-1}{a+1}\\
%\disfrac{a^2+2a}{(a+1)\cdot 2(a+1)-(a^2+2a)}&=&\disfrac{2a-1}{a+1}\\
\disfrac{a^2+2a}{a^2+2a +2}&=&\disfrac{2a-1}{a+1}\,.
%a^3+3a^2+2a&=&(2a-1)(a^2+2a+2)\,.
%&=&2a^3+3a^2+2a-2\,.
\end{eqn}
נפשט ונקבל
$a^3=2$
ו-%
$a=\sqrt[3]{2}$.
\end{proof}

%%%%%%%%%%%%%%%%%%%%%%%%%%%%%%%%%%%%%%%%%%%%%%%%%%%%%%%%%%%%%

\section{הבנייה של
בלוץ' להכפלת קובייה%
}\label{s.cube2}

הקיפול של בלוץ'
(אקסיומה~
$6$)
מסוגל לפתור משוואות ממעלה שלישית, ולכן סביר לשער שניתן להשתמש בו כדי להכפיל קובייה. הנה בנייה ישירה שמשתמשת בקיפול.

\textbf{הבנייה:}
נסמן את הנקודה
$(-1,0)$
ב-%
$A$
ואת הנקודה
$(0,-2)$
ב-%
$B$.
נסמן ב-%
$p$ 
את הישר 
$x=1$
וב-%
$q$
את הישר
$y=2$.
לפי אקסיומה $6$ ניתן לבנות קיפול 
$l$
המניח את
$A$
ב-%
$A'$
על 
$p$,
ואת
$B$
ב-%
$B'$
על
$q$.
נסמן ב-%
$Y$
את נקודת החיתוך של הקיפול עם ציר ה-%
$y$,
ונסמן ב-%
$X$
את נקודת החיתוך של הקיפול עם ציר ה-%
$x$
(איור%
~\ref{f.beloch-cube1}).

\begin{theorem}
$\overline{OY}=\sqrt[3]{2}$.
\end{theorem}

\begin{figure}[tb]
\begin{center}

\begin{tikzpicture}[scale=.9]
% Draw and label square
\coordinate (O) at (0,0);
\coordinate (A) at (-2,0);
\coordinate (B) at (0,-4);
\fill (O) circle (2pt)
  node[below left,xshift=-7pt] {$O$}
  node[below left,yshift=-12pt] {$(0,0)$};
\fill (A) circle (2pt)
  node[above left,xshift=-7pt] {$A$}
  node[below left,xshift=2pt,yshift=0pt] {$(-1,0)$}
  node[above right,xshift=10pt] {$\alpha$};
\fill (B) circle (2pt)
  node[left,xshift=-12pt] {$B$}
  node[left,yshift=-12pt] {$(0,-2)$}
  node[above right,yshift=12pt] {$\alpha'$};

\draw[thick] (0,-4.5) --  node[very near end,above left,yshift=12pt] {$y$\R{ציר-}} +(0,10);
\draw[thick] (-5,0)   -- node[very near start,above left] {$x$\R{ציר-}} +(12,0);
%\draw[thick] (0,-4.5) --  node[very near end,above left,yshift=12pt] {$y$-\R{ציר}} +(0,10);
%\draw[thick] (-5,0)   -- node[very near start,above left] {$x$-\R{ציר}} +(12,0);
\draw[very thick] (2,-4.5) -- node[very near start, right,yshift=-10pt] {$p\!:x=1$} +(0,10);
\draw[very thick] (-5,4) -- node[very near start, above,xshift=-16pt] {$q\!: y=2$} +(12,0);

\coordinate (AP) at (2,5);
\fill (AP) circle (2pt) node[above right] {$A'$};
\coordinate (BP) at (6.34,4);
\fill (BP) circle (2pt) node[above right] {$B'$};

% Tangent y = -0.8x + 1.26

% Exchanged X and Y 
\coordinate (X) at (0,2.52);
\coordinate (Y) at (3.15,0);
\fill (X) circle (2pt)
  node[right,xshift=4pt,yshift=2pt] {$Y$}
  node[below right,yshift=-14pt] {$\alpha$}
  node[below left,xshift=2pt,yshift=-12pt] {$\alpha'$};

\fill (Y) circle (2pt)
  node[above right,xshift=10pt] {$X$}
 node[below left,xshift=-10pt] {$\alpha$}
 node[above left,xshift=-13pt] {$\alpha'$};
\draw [very thick,dashed] ($(X)!-1.1!(Y)$) -- node[very near end,right,xshift=8pt] {$l$} ($(X)!2!(Y)$);

\draw [very thick,dotted] (A) -- (AP);
\draw [very thick,dotted] (B) -- (BP);

\draw[thick,dotted,bend left=40,->] (A) to ($(AP)+(-4pt,0)$);
\draw[thick,dotted,bend left=40,->] (B) to ($(BP)+(-6pt,-3pt)$);

\draw[rotate=-130] (X) rectangle +(10pt,10pt);
\draw[rotate=-130] (Y) rectangle +(10pt,10pt);

\end{tikzpicture}
\end{center}
\selectlanguage{hebrew}
\caption{הכפלת קובייה לפי בלוץ '}\label{f.beloch-cube1}
\end{figure}

\begin{proof}
הקיפול הוא האנך האמצעי של
$\overline{AA'}$
ושל
$\overline{BB'}$,
ולכן
$\overline{AA'}\|\overline{BB'}$.
$\angle XAY =\angle AXB=\alpha$
לפי זוויות מתחלפות. סימון הזוויות האחרות נובע מהתכונות של משולשים ישרי-זווית. ניתן להסיק ש-%
$\triangle AOY\sim \triangle YOX \sim \triangle XOB$
ונתון
$\overline{OA}=1$, $\overline{OB}=2$:
\begin{eqn}
\disfrac{\overline{OY}}{\overline{OA}}&=&\disfrac{\overline{OX}}{\overline{OY}}=\disfrac{\overline{OB}}{\overline{OX}}\\
\disfrac{\overline{OY}}{1}&=&\disfrac{\overline{OX}}{\overline{OY}}=\disfrac{2}{\overline{OX}}\,.
\end{eqn}
משני היחסים הראשונים נקבל
$\overline{OY}^2=OX$
ומהראשון והשלישי נקבל
$\overline{OY}\:\overline{OX}=2$.
מהצבת
$\overline{OX}$
נקבל
$\overline{OY}^3=2$
 ו-%
$\overline{OY}=\sqrt[3]{2}$.
\end{proof}

%%%%%%%%%%%%%%%%%%%%%%%%%%%%%%%%%%%%%%%%%%%%%%%%%%%%%%%%%%%%%%%%

\section{בניית מתושע}\label{s.nonagon}

ניתן לבנות מתושע משוכלל (מצולע משוכלל בעל תשע צלעות) על ידי פיתוח משוואה ממעלה שלישית עבור הזווית המרכזית שלו ופתרון המשוואה באמצעות השיטה של ליל
והקיפול של בלוץ'.
הזווית המרכזית היא
$\theta=360^\circ/9=40^\circ$.
לפי משפט%
~\ref{thm.triple-angle}:
\[
\cos 3\theta=4\cos^3 \theta -3\cos\theta\,.
\]
יהי 
$x=\cos 40^\circ$.
עבור המתושע
$4x^3-3x+(1/2)=0$
כי 
$\cos 3\cdot 40^\circ=\cos 120^\circ=-(1/2)$.
איור%
~\ref{f.nonagon2}
מציג את המסלולים עבור המשוואה לפי השיטה של ליל. 
\begin{figure}[tb]
\begin{center}
\begin{tikzpicture}[scale=.85]
% Draw help lines and axes
\draw[step=10mm,white!60!black] (-1,-4) grid (9,1);
\draw[thick] (-1,0) -- (9,0);
\draw[thick] (0,-4) -- (0,1);
\foreach \x in {1,...,9}
  \node at (\x-.3,.3) {\sm{\x}};
\foreach \y in {-3,...,1}
  \node at (-.3,\y-.3) {\sm{\y}};
  
% Points of first path
\coordinate (A) at (0,0);
\coordinate (B) at (4,0);
\coordinate (C) at (7,0);
\coordinate (D) at (7,-.5);
\node[above left] at (A) {$P$};
\node[below right,xshift=12pt] at (A) {$37.45^\circ$};
\node[below right] at (D) {$Q$};

% Draw first path
\draw[very thick,-{Stealth[scale=1.4,inset=2pt]}] 
  (A) -- node[below,xshift=6pt] {$a_3$} (B);
\draw[{Stealth[scale=1.4,inset=2pt,reversed]}-,very thick]
  (B) -- ($(B)+(0,.1)$);
\draw[name path=c,very thick,{Stealth[scale=1.4,inset=2pt]}-]
  (B) -- node[below] {$a_1$} (C);
\draw[very thick,-{Stealth[scale=1.4,inset=2pt]}]
  (C) -- node[right,yshift=-2pt] {$a_0$} (D);

% Draw extension of second segment of first path
\draw[thick,name path=b] 
  ($(B)+(0,-4)$) -- node[right] {$a_2$} ($(B)+(0,1)$);

% Draw second path
\path[name path=one] (A) -- +(-37.45:6cm);
\path [name intersections = {of = b and one, by = {R}}];
\node[below left] at (R) {$R$};
\draw[thick,dashed] (A) -- (R);

\path[name path=two] (R) -- +(52.5463:6cm);
\path [name intersections = {of = c and two, by = {S}}];
\node[above] at (S) {$S$};
\draw[thick,dashed] (R) -- (S);

\draw[thick,dashed] (S) -- (D);

% Draw right angle rectangles
\draw[thick,rotate=52.5463] (R) rectangle +(9pt,9pt);
\draw[thick,rotate=-127.4537] (S) rectangle +(9pt,9pt);
\end{tikzpicture}
\end{center}
\selectlanguage{hebrew}
\caption{השיטה של ליל
לבניית מתושע}
\label{f.nonagon2}
\end{figure}

המסלול השני מתחיל מ-%
$P$
בזווית 
$-37.45^\circ$
בערך. פנייה של
$90^\circ$
ב-%
$R$
ואז פנייה
$-90^\circ$
ב-%
$S$
גורמים למסלול לחתוך את המסלול הראשון בנקודת הקצה שלו
$Q$,
ולכן
$x=-\tan (-37.45)^\circ=0.766$
הוא שורש של
$4x^3-3x+(1/2)$.

ניתן למצוא את השורש באמצעות הקיפול של בלוץ'. 
נשרטט את הישר
$a_2'$
המקביל ל-%
$a_2$
ומרחקו מ-%
$a_2$
שווה למרחק בין
$a_2$
ל-%
$P$.
למרות שאורכו של
$a_2$ 
הוא אפס, עדיין יש לו כיוון (למעלה) ולכן ניתן לבנות ישר מקביל. באופן דומה, נשרטט ישר
$a_1'$
מקביל ל-%
$a_1$
שמרחקו מ-%
$a_1$
שווה למרחק בין
$a_1$
ל-%
$Q$.
$\overline{RS}$,
הקיפול של בלוץ', 
מניח בו-זמנית את
$P$
ב-%
$P'$
על
$a_2'$,
ואת
$Q$
ב-%
$Q'$
על
$a_1'$.
הקיפול בונה את הזווית
$\angle SPR=-37.45^\circ$
(איור%
~\ref{f.nonagon-beloch}).
\begin{figure}[tb]
\begin{center}
\begin{tikzpicture}[scale=.85]
% Draw help lines and axes
\draw[step=10mm,white!60!black] (-1,-7) grid (9,1);
\draw[thick] (-1,0) -- (9,0);
\draw[thick] (0,-7) -- (0,1);
\foreach \x in {1,...,9}
  \node at (\x-.3,.3) {\sm{\x}};
\foreach \y in {-6,...,1}
  \node at (-.3,\y-.3) {\sm{\y}};
  
% Points of first path
\coordinate (A) at (0,0);
\coordinate (B) at (4,0);
\coordinate (C) at (7,0);
\coordinate (D) at (7,-.5);
\node[above right] at (A) {$P$};
\node[below right] at (D) {$Q$};

% Draw first path
\draw[very thick,-{Stealth[scale=1.4,inset=2pt]}] 
  (A) -- node[below] {$a_3$} (B);
\draw[{Stealth[scale=1.4,inset=2pt,reversed]}-,very thick]
  (B) -- ($(B)+(0,.1)$);
\draw[name path=c,very thick,{Stealth[scale=1.4,inset=2pt]}-]
  (B) -- node[below] {$a_1$} (C);
\draw[very thick,-{Stealth[scale=1.4,inset=2pt]}]
  (C) -- node[right,yshift=-2pt] {$a_0$} (D);

% Draw extension of second segment of first path
\draw[very thick,loosely dotted,name path=b] 
  ($(B)+(0,-7)$) -- node[right,near end] {$a_2$} ($(B)+(0,1)$);

% Draw second path
\path[name path=one] (A) -- +(-37.45:6cm);
\path [name intersections = {of = b and one, by = {R}}];
\node[below left] at (R) {$R$};
\path[name path=two] (R) -- +(52.55:6cm);
\path [name intersections = {of = c and two, by = {S}}];
\node[above] at (S) {$S$};
\draw[very thick,dashed] (R) -- (S);

% Draw parallel lines
\draw[thick,name path=para-2] 
  (8,1) -- node[right,yshift=8pt] {$a_2'$} (8,-7);
\draw[thick,name path=para-1] 
  (-1,.5) -- node[right,xshift=44mm] {$a_1'$} (9,.5);

% Draw second segments of the folds
\path[name path=p-two] (A) -- +(-37.45:11cm);
\path [name intersections = {of = para-2 and p-two, by = {PP}}];
\node[below left] at (PP) {$P'$};
\draw[very thick,dotted] (A) -- (PP);

\path[name path=p-one] (D) -- +(142.55:2cm);
\path [name intersections = {of = para-1 and p-one, by = {QP}}];
\node[above] at (QP) {$Q'$};
\draw[very thick,dotted] (D) -- (QP);

% Draw right angle indications
\draw[thick,rotate=-37.45] (R) rectangle +(9pt,9pt);
\draw[thick,rotate=-127.4537] (S) rectangle +(9pt,9pt);
\end{tikzpicture}
\end{center}
\selectlanguage{hebrew}
\caption{בנית מתושע באמצעות הקיפול של Beloch}\label{f.nonagon-beloch}
\end{figure}

לפי השיטה של ליל,
$-\tan (-37.45^\circ)\approx 0.766$
ולכן
$\cos \theta \approx 0.766$
הוא השורש של המשוואה עבור הזווית המרכזית
$\theta$.

נסיים את בניית המתושע על ידי בניית
$\cos^{-1} 0.766\approx 40^\circ$:
\begin{itemize}
\item
במשולש ישר-זווית
$\triangle ABC$
עם
$\angle CAB\approx 37.45^\circ$
ו-%
$\overline{AB}=1$
הצלע הנגדית היא
$\overline{BC}\approx 0.766$
לפי הגדרת הטנגנס 
(איור~%
\ref{f.nonagon5-eq}).
\item
לפי אקסיומה 
$1$
נקפל את
$\overline{CB}$
על 
$\overline{AB}$
כך שהשיקוף של
$C$
הוא
$D$
ו-%
$\overline{DB}=0.766$.
\item
נמשיך את
$\overline{DB}$
ונבנה את
$E$
כך ש-%
$\overline{DE}=1$.
\item
לפי אקסיומה 
$1$
נקפל את
$\overline{DE}$
כדי לשקף את
$E$
ב-%
$F$
בהמשך של
$\overline{BC}$
(איור~%
\ref{f.nonagon5-central}).
אזי:
\[
\angle BDF=\cos^{-1} \frac{0.766}{1}\approx 40^\circ\,.
\]
\end{itemize}
\begin{figure}[tb]
\begin{center}
\begin{subfigure}{.4\textwidth}
\begin{tikzpicture}[scale=1]
\draw (0,0) coordinate (A) -- (4,0) coordinate (B);
\draw (B) -- node[right] {$0.766$} 
  ($(B)+(0,0.766*4)$) coordinate (C);
\draw (A) -- (C);
\draw[rotate=90] (B) rectangle +(8pt,8pt);
\node[above left] at (A) {$A$};
\node[above right] at (B) {$B$};
\node[right] at (C) {$C$};
\coordinate (D) at (0.234*4,0);
\node[below left] at (D)  {$D$};
\coordinate (E) at ($(D)+(4,0)$);
\draw (D) -- (B);
\draw (B) -- (E);
\node[above right] at (E) {$E$};
\draw[very thick,dotted,->,bend right=50] ($(C)+(-.2,0)$) to ($(A)+(.94,.4)$);
\draw[<->] ($(D)+(0,-1.2)$) -- node[fill=white] {$1$} ($(E)+(0,-1.2)$);
\draw[<->] ($(A)+(0,-.8)$) -- node[fill=white] {$1$} ($(B)+(0,-.8)$);
\node[above right,xshift=14pt] at (A) {$37.45^\circ$};
\vertex{D};
\vertex{E};
\end{tikzpicture}
\caption{הטנגנס הוא הפתרון של משוואת המתושע}
\label{f.nonagon5-eq}
\end{subfigure}
\hspace{3em}
\begin{subfigure}{.4\textwidth}
\begin{tikzpicture}[scale=1]
\coordinate (B) at (4,0);
\draw (B) -- ($(B)+(0,0.766*4)$) coordinate (C);
\draw[rotate=90] (B) rectangle +(8pt,8pt);
\node[above right] at (B) {$B$};
\node[right] at (C) {$C$};
\coordinate (D) at (0.234*4,0);
\node[above left] at (D) {$D$};
\node[above right,xshift=8pt,yshift=4pt] at (D) {$40^\circ$};
\coordinate (E) at ($(D)+(4,0)$);
\draw (D) -- node[fill=white] {$0.766$} (B);
\draw (B) -- (E);
\node[above right,xshift=4pt] at (E) {$E$};
\coordinate (F) at ($(B)+(0,4)$);
\draw[very thick,dotted,->,bend right=50] ($(E)+(.1,.2)$) to ($(F)+(.2,0)$);
\draw (B) -- (F);
\node[left] at (F) {$F$};
\draw (D) -- node[fill=white] {$1$} (F);
\draw[<->] ($(D)+(0,-.8)$) -- node[fill=white] {$1$} ($(E)+(0,-.8)$);
\vertex{C};
\vertex{E};
\coordinate (A) at (0,0) node [above left] {$A$};
\draw (A) -- (D);
\vertex{A};
\vertex{D};
\end{tikzpicture}
\caption{הקוסינוס של הזווית המרכזית של המתושע}\label{f.nonagon5-central}
\end{subfigure}
\end{center}
\end{figure}
\subsection*{מה ההפתעה?}

ראינו בפרקים%
~\ref{c.trisect}
ו-%
~\ref{c.square}
שבעזרת כלים כגון הנוסיס אפשר לבצע בניות שלא ניתן לבצען בעזרת סרגל ומחוגה. לכן, מפתיע שניתן לחלק זווית לשלושה חלקים ולהכפיל קובייה רק על ידי קיפולי נייר. רוג'ר אלפרין
\L{(Roger C. Alperin)}
פיתח הייררכייה של שיטות בנייה שכל אחת מהן חזקה מקודמתה.

\subsection*{מקודות}

פרק זה מבוסס על
\cite{alperin,lang,martin,newton}.
