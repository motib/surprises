% !TeX root = surprises.tex

\selectlanguage{hebrew}

\chapter{אינדוקציה}\label{c.induction}

%%%%%%%%%%%%%%%%%%%%%%%%%%%%%%%%%%%%%%%%%%%%%%%%%%%%%%%%%%%%%%%

האקסיומה של אינדוקציה מתמטית נמצאת בשימוש נרחב כשיטת הוכחה במתימטיה. פרק זה מציג הוכחות אינדוקטיביות של תוצאות שייתכן שהן לא מוכרות לקורא. נתחיל עם סקירה קצרה של אינדוקציה מתמטית (סעיף%
~\ref{s.induction-axiom}).
סעיף%
~\ref{s.induction-fibonacci}
מביא הוכחות של משפטים על מספרי
\L{Fibonacci}
המוכרים וסעיף%
~\ref{s.induction-fermat}
מביא הוכחות של משפטים על מספרי
\L{Fermat}.
בסעיף%
~\ref{s.induction-mccarthy}
נציג את פונקציה
$91$
של 
\L{John McCarthy}.
ההוכחה היא לא שגרתית כי היא משתמשת באינדוקציה על מספרים שלמים בסדר הפוך. הוכחת הנוסחה עבור הבעיה של
\L{Josephus}
(יוסף בן-מתתיהו) גם היא לא שגרתית כי היא משתמשת באינדוקציה כפולה על חלקים שונים של ביטוי (סעיף%
.~\ref{s.josephus}).

\section{האקסיומה של אינדוקציה מתמטית}\label{s.induction-axiom}

אינדוקציה מתמטית היא הדרך המובילה להוכיח משפטים עבור קבוצה לא חסומה של מספרים. נעיין במשוואות:
\[
1=1,\quad 1+2=3,\quad 1+2+3=6,\quad 1+2+3+4=10\,,
\]
נשים לב ש:
\[
1=(1\cdot 2)/2,\quad 3=(2\cdot 3)/2,\quad  6=(3\cdot 4)/2,\quad 10=(4\cdot 5)/2\,,
\]
ונשער שעבור כל המספרים שלמים
$n\geq 1$:
\[
\sum_{i=1}^n i = \frac{n(n+1)}{2}\,.
\]
עם מספיק סבלנות קל לבדוק את הנוסחה עבור כל ערך של
$n$,
אבל איך אפשר להוכיח עבור אינסוף מספרים שלמים חיוביים? כאן נכנסה אינדוקציה מתמטית.

\begin{axiom}
יהי
$P(n)$
תכונה (כגון משוואה, נוסחה או משפט), כאשר 
$n$
הוא מספר שלם חיובי. נניח שניתן:
\begin{itemize}
\item \textbf{טענת הבסיס}: 
להוכיח ש-%
$P(1)$
נכונה.
\item \textbf{צעד אינדוטיבי}:
עבור 
$m$
שרירותי, להוכיח ש-%
$P(m+1)$
נכונה בהנחה ש-%
$P(m)$
נכונה.
\end{itemize}
אזי
$P(n)$
נכונה עבור כל
$n\geq 1$.
ההנחה ש-%
$P(m)$
נכונה עבור 
$m$ 
שרירותי נקראת
\textbf{הנחת האינדוקציה}.
\end{axiom}
הנה דוגמה פשוטה עבור הוכחה באינדוקציה מתמטית.
\begin{theorem}\label{t.sum}
עבור
$n\geq 1$:
\[
\sum_{i=1}^n i = \frac{n(n+1)}{2}\,.
\]
\end{theorem}

\begin{proof} 
טענת הבסיס פשוטה:
\[
\sum_{i=1}^1 i = 1 =\frac{1(1+1)}{2}\,.
\]
הנחת האינדוקציה היא שמשוואה שלהלן נכונה עבור כל 
$m$:
\[
\sum_{i=1}^{m} i = \frac{m(m+1)}{2}\,.
\]
הצעד האינדוקטיבי הוא להוכיח את המשפט עבור
$m+1$:
\begin{eqn}
\sum_{i=1}^{m+1} i &=& \sum_{i=1}^m i + (m+1)\label{l.sum1}\\
&=&\frac{m(m+1)}{2} + (m+1)\label{l.sum2}
%&=&\frac{m(m+1) + 2(m+1)}{2}\label{l.sum3}\\
=\frac{(m+1)(m+2)}{2}\,.\label{l.sum4}
\end{eqn}
לפי האקסיומה של אינדוקציה מתמטית, עבור כל
$n\geq 1$:
\[
\sum_{i=1}^n i = \frac{n(n+1)}{2}\,.
\]
\end{proof}

הנחת האינדוקציה יכולה לבלבל כי נראה שאנחנו מניחים את מה שרוצים להוכיח. אין כאן הסקת מסקנות מעגלית כי ההנחה היא עבור תכונה של משהו קטן ומשתמשים בהנחה להוכיח תכונה עבור משהו גדול יותר.

אינדוקציה מתמטית היא אקסיומה שאי-אפשר להוכיח. פשוט צריכים לקבל אותה כמו שמקבלים אקסיומות אחרות כגון
$x+0=x$.
כמובן שתוכלו לדחות את האקסיומה אבל אז תצטרכו לדחות חלק גדול מהמתמטיקה המודרנית.
\begin{advanced}
אינדוקציה מתמטית היא כלל היסק שהוא אחד מאקסיומות
\L{Peano}
לפורמלזציה של המספרים הטבעיים. ניתן להוכיח את האקסיומה המאקסיומה אחרת כגון אקסיומה
\L{well ordering},
ולהיפך, אבל לא ניתן להוכיח אות מאקסיומות אחרות, פשוטות יותר, של 
\L{Peano}.
\end{advanced}

%%%%%%%%%%%%%%%%%%%%%%%%%%%%%%%%%%%%%%%%%%%%%%%%%%%%%%%%


\section{מספרי \L{\large Fibonacci}}\label{s.induction-fibonacci}

מספרי פיבונצ'י מוגדרים ברקורסיה:
\begin{eqn}
f_1 &=& 1\\
f_2 &=& 1\\
f_n &=& f_{n-1} + f_{n-2}, \;\;  n \geq 3 \;\; \textrm{\R{עבור}}\,.
\end{eqn}
שנים עשר מספרי פיבונצ'י הראשונים הם:
$
1, 1, 2, 3, 5, 8, 13, 21, 34, 55, 89, 144
$.
\begin{theorem}\label{thm.fib-div3}
כל מספר פיבונצ'י רביעי מתחלק ב-%
$3$.
\end{theorem}
\begin{example}
$f_4=3=3\cdot 1,\; f_8=21=3\cdot 7,\; f_{12}=144=3\cdot 48$.
\end{example}
\begin{proof}
טענת הבסיס מתקבלת באופן מיידי כי
$f_4=3$
מתחלק ב-%
$3$.
הנחת האינדוקציה היא ש-%
$f_{4n}$
מתחלק ב-%
$3$.
הצעד האינדוקטיבי הוא:
\begin{eqn}
f_{4(n+1)} &=& f_{4n+4}\\
&=& f_{4n+3}+f_{4n+2}\\
&=& (f_{4n+2}+f_{4n+1})+f_{4n+2}\\
&=& ((f_{4n+1}+f_{4n})+f_{4n+1})+f_{4n+2}\\
&=& ((f_{4n+1}+f_{4n})+f_{4n+1})+(f_{4n+1}+f_{4n})\\
&=& 3f_{4n+1}+2f_{4n}\,.
\end{eqn}
$3f_{4n+1}$
מתחלק ב-%
$3$
ולפי הנחת האינדוקציה גם
$f_{4n}$,
ולכן
$f_{4(n+1)}$
מתחלק ב-%
$3$.
\end{proof}

\begin{theorem}\label{thm.seven-fourths}
$f_n < \left(\disfrac{7}{4}\right)^n$.
\end{theorem}
\begin{proof}
טענות הבסיס:
$f_1=1<\left(\disfrac{7}{4}\right)^1$
ו-%
$f_2=1<\left(\disfrac{7}{4}\right)^2=\disfrac{49}{16}$.

הצעד האינדוקטיבי:
\begin{eqn}
f_{n+1}&=&f_n+f_{n-1}\\
&<&\left(\frac{7}{4}\right)^n + f_{n-1}\\
%&<&\left(\frac{7}{4}\right)^n + \left(\frac{7}{4}\right)^{n-1}\\
&=&\left(\frac{7}{4}\right)^{n-1}\cdot\left(\frac{7}{4}+1\right)\\
&<&\left(\frac{7}{4}\right)^{n-1}\cdot\left(\frac{7}{4}\right)^2\\
&=&\left(\frac{7}{4}\right)^{n+1},
\end{eqn}
בגלל ש:
\[
\left(\frac{7}{4}+1\right) = \frac{11}{4} = \frac{44}{16}<\frac{49}{16}=\left(\frac{7}{4}\right)^2.
\]
\end{proof}

\begin{theorem}[נוסחת \L{Binet}]
\begin{displaymath}
f_n = \frac{\phi^n - \bar{\phi}^n}{\sqrt{5}}, \;\;\;\;\;
\phi = \frac{1+\sqrt{5}}{2},\;\;\bar{\phi} = \frac{1-\sqrt{5}}{2}\,.
\end{displaymath}
\end{theorem}
\begin{proof}
נוכיח קודם ש-%
$\phi^2=\phi+1$:
\begin{eqn}
\phi^2 &=& \left(\frac{1+\sqrt{5}}{2}\right)^2\\
&=& \frac{1}{4} + \frac{2\sqrt{5}}{4} + \frac{5}{4}\\
&=& \frac{2}{4} + \frac{2\sqrt{5}}{4} + \frac{4}{4}\\
&=& \frac{1+2\sqrt{5}}{2} + 1\\
&=&\phi + 1\,.
\end{eqn}
באופן דומה אפשר להוכיח ש:
$\bar{\phi}^2=\bar{\phi}+1$.

טענת הבסיס של נוסחת
\L{Binet}
היא:
\[
\frac{\phi^1 - \bar{\phi}^1}{\sqrt{5}}=\frac{(1+\sqrt{5})/2-(1-\sqrt{5})/2}{\sqrt{5}}=\frac{2\sqrt{5}}{2\sqrt{5}}=1\,.
\]
נניח שהנחת האינדוקציה נכונה עבור כל
$k\leq n$.
הצעד האינדוקטיבי הוא:
\begin{eqn}
\phi^n - \bar{\phi}^n &=& \phi^2\,\phi^{n-2} - \bar{\phi}^2\,\bar{\phi}^{n-2}\\
&=&(\phi+1)\,\phi^{n-2} - (\bar{\phi}+1)\,\bar{\phi}^{n-2}\\
&=&(\phi^{n-1} - \bar{\phi}^{n-1}) + (\phi^{n-2} - \bar{\phi}^{n-2})\\
&=&\sqrt{5}f_{n-1} + \sqrt{5}f_{n-2}\\
\frac{\phi^n - \bar{\phi}^n}{\sqrt{5}} &=& f_{n-1} + f_{n-2} = f_n\,.
\end{eqn}
\end{proof}
\begin{theorem}\label{eq.fibo:combinations}
\[
f_n = {n \choose 0} + {n-1 \choose 1} + {n-2 \choose 2} + \cdots.
\]
\end{theorem}
\begin{proof}
נוכיח תחילה את הנוסחה של 
\L{Pascal]}:
\[
{n \choose k} + {n \choose k+1} = {n+1 \choose k+1}.
\]
\begin{eqn}
{n \choose k} + {n \choose k+1} &=& \frac{n!}{k!(n-k)!} + \frac{n!}{(k+1)!(n-(k+1))!}\\
&=& \frac{(k+1)n!}{(k+1)!(n-k)!} + \frac{n! (n-k)}{(k+1)!(n-k)!}\\
%&=&\frac{n![(k+1)+(n-k)]}{(k+1)!(n-k)!}\\
&=&\frac{n!(n+1)}{(k+1)!(n-k)!}\\
&=&\frac{(n+1)!}{(k+1)!((n+1)-(k+1))!}\\
&=&{n+1 \choose k+1}\,.
\end{eqn}
נשתמש גם בשוויון
$\displaystyle{k\choose 0} = \frac{k!}{0!(k-0)!} = 1$
עבור כל
$k\geq 1$.

עכשיו אפשר להוכיח את המשפט. טענת הבסיס:
\[
f_1 = 1 = {1 \choose 0} = \frac{1!}{0!(1-0)!}\,.
\]
הצעד האינדוקטיבי הוא:
\begin{eqn}
f_{n-1} + f_{n-2} &=& {n-1 \choose 0} + {n-2 \choose 1} + {n-3 \choose 2} + {n-4 \choose 3} + \cdots\\
&&\hspace{54pt}{n-2 \choose 0} + {n-3 \choose 1} + {n-4 \choose 2} + \cdots\\
&=&{n-1 \choose 0} + {n-1 \choose 1} + {n-2 \choose 2} + {n-3 \choose 3} + \cdots\\
&=&{n \choose 0}\hspace{20pt} + {n-1 \choose 1} + {n-2 \choose 2} + {n-3 \choose 3} + \cdots.
\end{eqn}
\end{proof}

%%%%%%%%%%%%%%%%%%%%%%%%%%%%%%%%%%%%%%%%%%%%%%%%%%%%%%%%%%

\section{מספרי \L{\large Fermat}}\label{s.induction-fermat}

\begin{definition}
מספר פרמה הוא מספר שלם שערכו
$2^{2^{n}}+1$
עבור 
$n\geq 0$.
\end{definition}
חמשת מספרי פרמה הראשונים הם מספרים ראשוניים:
\[
F_0=3,\quad F_1=5,\quad F_2=17,\quad F_3=257,\quad F_4=65537\,.
\]
במאה השבע עשרה המתמיטאי
\L{Pierre de Fermat}
שיער שכל מספרי פרמה הם ראשוניים, אבל כעבור כמאה שנים
\L{Leonhard Euler}
הראה ש:
\[
2^{2^5}+1 = 2^{32}+1 = 4294967297 = 641 \times 6700417\,.
\]
מספרי פרמה גדלים מאוד מהר ככל ש-%
$n$
גדל. ידוע שמספרי פרמה אינם ראשוניים עבור
$5\leq n \leq 32$,
אבל הפירוק לגורמים של חלק מהמספרים הללו עדיין לא ידוע. 
\begin{theorem}
עבור
$n\geq 2$,
הספרה האחרונה של
$F_n$
היא
$7$.
\end{theorem}
\begin{proof}
טענת הבסיס:
$F_2=2^{2^2}+1=17$.
הנחת האינדוקציה היא ש-%
$F_n=10k_n+7$
עבור
$k\geq 1$.
הצעד האינדוקטיבי הוא:
\begin{eqn}
F_{n+1}&=&2^{2^{n+1}}+1=\left(2^{2^{n}}\right)^2+1\\
&=&\left(\left(2^{2^{n}}+1\right)-1\right)^2+1\\
&=&\left(2^{2^{n}}+1\right)^2
-2\cdot\left(2^{2^{n}}+1\right)+1+1\\
&=&(10k_n+7)^2-2(10k_n+7)+2\\
%&=&(10k_n+7-1)^2+1=(10k_n+6)^2+1\\
&=&100k_n^2+120k_n+37\\
&=&10(10k_n^2+12k_n+3)+7\\
&=&10k_{n+1}+7\,.
\end{eqn}
\end{proof}
\begin{theorem}\label{thm.fermat}
עבור כל
$n\geq 1$, $\displaystyle F_n = \prod_{k=0}^{n-1} F_k + 2$.
\end{theorem}
\begin{proof}
טענת הבסיס:
\[
5=F_1=\prod_{k=0}^{0} F_k + 2=F_0+2=3+2\,.
\]
הצעד האינדוקטיבי:
\begin{eqn}
\displaystyle\prod_{k=0}^{n}F_k&=&\left(\displaystyle\prod_{k=0}^{n-1}F_k\right) F_n \\
&=& (F_n-2)F_n\\
&=& F_n^2-2F_n\\
&=& \left(2^{2^n}+1\right)^2-2\cdot \left(2^{2^n}+1\right)\\
&=& 2^{2^{n+1}}-1= (2^{2^{n+1}}+1)-2\\
&=&F_{n+1}-2\\
F_{n+1}&=&\displaystyle\prod_{k=0}^{n}F_k + 2\,.
\end{eqn}
\end{proof}

%%%%%%%%%%%%%%%%%%%%%%%%%%%%%%%%%%%%%%%%%%%%%%%%%%%%%%%%%%%

\section{פונקציה $91$ של
\L{\large McCarthy}}\label{s.induction-mccarthy}

אינדוקציה מתקשר אצלנו עם הוכחות של תכונות המוגדרות על קבוצת המספרים השלמים החיוביים. כאן נביא הוכחה אינדוקטיבית המבוססת על יחס מוזר כאשר מספרים גודלים הם קטנים ממספרים קטנים. האינדוקציה מצליחה כי התכונה היחידה שנדרשת מהקבוצה היא שקיים סדר לפי פעולה יחס.

נעיין בפונקציה הרקורסיבית שלהלן המוגדר עם מספרים שלמים:
\[
f(x) = \textrm{if}\;\; x > 100 \;\;\textrm{then}\;\; x - 10 \;\;\textrm{else}\;\; f(f(x+11))\,.
\]
עבור מספרים גדולים מ-%
$100$
חישוב הפונקציה פשוטה ביותר:
\[
f(101) = 91, \;\; f(102) = 92,\;\; f(103) = 93,\;\; f(104) = 94\,.
\]
מה עם מספרים קטנים או שווים ל-%
$100$?
נחשב את
$f(x)$
עבור מספרים מסויימים כאשר החישוב בכל שורה מסתמכת על השורות הקודמות:
\begin{eqn}
f(100) &=& f(f(100+11)) = f(f(111)) = f(101) = 91\\
f(99) &=& f(f(99+11)) = f(f(110)) = f(100) = 91\\
f(98) &=& f(f(98+11)) = f(f(109)) = f(99) = 91\\
&\cdots&\\
f(91) &=& f(f(91+11)) = f(f(102)) = f(92)\\
&& \quad = f(f(103)) = f(93) = \cdots =f(98) = 91\\
f(90) &=& f(f(90+11)) = f(f(101)) = f(91) = 91\\
f(89) &=& f(f(89+11)) = f(f(100)) = f(91) = 91\,.
\end{eqn}
נגדיר את הפונקציה
$g$:
\[
g(x) = \textrm{if}\;\; x > 100 \;\;\textrm{then}\;\; x - 10 \;\;\textrm{else}\;\; 91\,.
\]
\begin{theorem}
עבור כל 
$x$, $f(x)=g(x)$.
\end{theorem}
\begin{proof}
ההוכחה באינדוקציה מעל קבוצת המספרים
$S=\{x\,|\,x\leq 101\}$
כאשר היחס
$\prec$
מוגדר על ידי:
\[
x \prec y \;\; \textrm{iff}\;\; y < x\,.
\]
בצד הימני 
$<$
הוא היחס הרגיל מעל למספרים שלמים. סדר המספרים לפי
$\prec$
הוא:
\[
101 \prec 100 \prec 99 \prec 98 \prec 97 \prec \cdots\,.
\]
יש שלושה מקרים בהוכחה. נשמתמש בתוצאות של החישובים לעיל:

\noindent\textbf{מקרה 1}  $x > 100$.
ההוכחה מיידית מההגדרות של 
$f$
ו-
$g$.

\noindent\textbf{מקרה 2} 
$90\leq x \leq 100$.

\noindent{}%
טענת הבסיס היא:
\[
f(100)  = 91 = g(100)\,,
\]
כי הראנו ש-%
$f(100)=91$
ולפי ההגדרה
$g(100)=91$.

הנחת האינדוקציה היא
$f(y) = g(y)$
עבור
$y\prec x$,
והצעד האינדוקטיבי הוא:
\begin{eqnlabels}
f(x) &=& f(f(x+11))\label{m91-1}\\
&=& f(x+11-10)= f(x+1)\label{m91-3}\\
&=& g(x+1)\label{m91-4}\\
&=& 91\label{m91-5}\\
&=& g(x)\label{m91-6}\,.
\end{eqnlabels}
משוואה%
~\ref{m91-1}
נכונה מההגדרה של
$f$
כי
$x\leq 100$.
השוויון בין משוואה%
~\ref{m91-1}
לבין משוואה%
~\ref{m91-3}
נכון מההגדרה של
$f$,
כי
$x \geq 90$
ולכן
$x+11 > 100$.
השוויון בין משוואה%
~\ref{m91-3}
ומשוואה%
~\ref{m91-4}
נובע מהנחת האינדוקציה
$x\leq 100$
ולכן
$x+1\leq 101$
שממנו אפשר להסיק ש-%
$x+1\in S$
ו-%
$x+1\prec x$.
השוויון בין המשוואות%
~\ref{m91-4}, \ref{m91-5}, \ref{m91-6}
נכון מההגדרה של 
$g$
ו-%
$x+1 \leq 101$
ולכן
$x\leq 100$.

\noindent\textbf{מקרה 3} $x< 90$.

טענת הבסיס היא
$f(89) = f(f(100)) = f(91) = 91 = g(89)$
לפי ההגדרה של
$g$
כי
$89\leq 100$.

הנחת האינדוקציה היא
$f(y) = g(y)$
עבור
$y\prec x$
והצעד האינדוקטיבי הוא:
\begin{eqnlabels}
f(x) &=& f(f(x+11))\label{m91a}\\
&=& f(g(x+11))\label{m91b}\\
&=& f(91)\label{m91c}\\
&=& 91\label{m91d}\\
&=& g(x)\,.
\end{eqnlabels}
משוואה%
~\ref{m91a}
נכונה לפי ההגדרה של
$f$
ו-%
$x<90\leq 100$.
השוויון בין המשוואות
\ref{m91a}
ו-%
\ref{m91b}
נובע מהנחת האינדוקציה
$x < 90$
ולכן
$x+11< 101$
שממנו נובע
$x+11\in S$
ו-%
$x+11 \prec x$.
השוויון בין המשוואות%
~\ref{m91b}
ו-%
\ref{m91c}
נכון לפי ההגדרה של
$g$
ו-%
$x+11 < 101$.
לבסוף, כבר הוכחנו ש-%
$f(91)=91$
ולפי ההגדרה
$g(x)=91$
עבור
$x<90$.
\end{proof}

%%%%%%%%%%%%%%%%%%%%%%%%%%%%%%%%%%%%%%%%%%%%%%%%%%%%%%%%

\section{בעיית \L{\large Josephus}}\label{s.josephus}

יוסף בן מתתיהו 
\L{(Titus Flavius Josephus)}
היה מפקד העיר יודפת בזמן המרד הגדול נגד הרומאים. הכוח העצום של הצבא הרומי מחץ את הגנת העיר ויוסף מצא מקלט במערה עם חלק מאנשיו שהעדיפו להתאבד ולא להיהרג או ליפול בשבי הרומאים. לפי מה שיוסף סיפר הוא מצא דרך להציל את עצמו, נשבה והפך למשקיף עם הרומאים ואחר כך כתב היסטוריה של המרד. נציג את הבעיה הקרויה על שמו כבעיה מתמטית מופשטת.
\begin{definition}[בעיית \L{Josephus}]
נסדר את המספרים
$1,\ldots,n\!+\!1$
במעגל. נמחק כל מספר ה-%
$q$
מסביב למעגל
$q, 2q, 3q, \ldots$
(מודולו
$n\!+\!1$)
עד שרק מספר אחד 
$m$
נשאר.
$J(n+1,q)=m$
הוא
\textbf{מספר \L{Josephus}}
עבור
$n+1$
ו-%
$q$.
\end{definition}
\begin{example}
יהי
$n+1=41$
ו-%
$q=3$.
נסדר את המספרים במעגל:
\[
\begin{array}{rrrrrrrrrrrrrrrrrrrrrrr}
\rightarrow\!&\!1\!&\!2\!&\!3\!&\!4\!&\!5\!&\!6\!&\!7\!&\!8\!&\!9\!&\!10\!&\!
           11\!&\!12\!&\!13\!&\!14\!&\!15\!&\!16\!&\!17\!&\!18\!&\!19\!&\!20\!&\!21\!&\!
\downarrow\\
\uparrow\quad\!&\!41\!&\!40\!&\!39\!&\!38\!&\!37\!&\!36\!&\!35\!&\!34\!&\!33\!&\!
32\!&\!31\!&\!30\!&\!29\!&\!28\!&\!27\!&\!26\!&\!25\!&\!24\!&\!23\!&\!22\!&\!
&\leftarrow
\end{array}
\]
תוצאת הסבב הראשון של המחיקות היא:
\[
\begin{array}{rrrrrrrrrrrrrrrrrrrrrrr}
\rightarrow\!&\!1\!&\!2\!&\!\not\! 3\!&\!4\!&\!5\!&\!\not\! 6\!&\!7\!&\!8\!&\!\not\! 9\!&\!10\!&\!
           11\!&\!\not\!\! 12\!&\!13\!&\!14\!&\!\not\!\! 15\!&\!16\!&\!17\!&\!\not\!\! 18\!&\!19\!&\!20\!&\!\not\!\! 21\!&\!
\downarrow\\
	\uparrow\quad\!&\!41\!&\!40\!&\!\not\!\! 39\!&\!38\!&\!37\!&\!\not\!\! 36\!&\!35\!&\!34\!&\!\not\!\! 33\!&\!
32\!&\!31\!&\!\not\!\! 30\!&\!29\!&\!28\!&\!\not\!\! 27\!&\!26\!&\!25\!&\!\not\!\! 24\!&\!23\!&\!22\!&\!
\!&\!\leftarrow
\end{array}
\]
לאחר השמטת המספרים המחוקים נקבל:
\[
\begin{array}{rrrrrrrrrrrrrrrrrrrrrrrrrrrr}
1&2&4&5&7&8&10&11&13&14&16&17&19&20\\
22&23&25&26&28&29&31&32&34&35&37&38&40&41
\end{array}
\]
תוצאת הסבב השני של המחיקות היא (כאשר מתחילים מהמחיקה האחרונה
$39$):
\[
\begin{array}{rrrrrrrrrrrrrrrrrrrrrrrrrrrr}
\not\!\! 1&2&4&\not\!\! 5&7&8&\not\!\! 10&11&13&\not\!\! 14&16&17&\not\!\! 19&20\\
22&\not\!\! 23&25&26&\not\!\! 28&29&31&\not\!\! 32&34&35&\not\!\! 37&38&40&\not\!\! 41
\end{array}
\]
נמשיך למחוק כל מספר שלישי עד שרק אחד נשאר:
\[
\begin{array}{rrrrrrrrrrrrrrrrrr}
2&4&\not\!7&8&11&\not\!\!13&16&17&\not\!\!20&22&25&\not\!\!26&29&31&\not\!\!34&35&38&\not\!\!40
\\
2&4&\not\!8&11&16&\not\!\!17&22&25&\not\!\!29&31&35&\not\!\!38
\\
2&4&\not\!\!11&16&22&\not\!\!25&31&35
\\
\not\!2&4&16&\not\!\!22&31&35
\\
\not\!4&16&31&\not\!\!35
\\
\not\!\!16&31
\\
31
\end{array}
\]
מכאן ש-%
$J(41,3)=31$.
\end{example}
הקורא מוזמן לבצע את החישוב עבור מחיקת כל מספר שביעי ממעגל של 
$40$
ולבדוק שהמספר האחרון הוא
$30$.
\begin{theorem}\label{thm.jo1}
$J(n+1,q)=(J(n,q)+q) \pmod {n+1}$.
\end{theorem}

\begin{proof}
המספר הראשון שנמחק בסבב הראשון הוא מספר ה-%
$q$
והמספרים שנשארים לאחר המחיקת הם 
$n$
המספרים:
\[
\begin{array}{rrrrrrrr}
\;1&\;2&\;\ldots&\;q-1&\;q+1&\;\ldots&\;n&\;n+1 \pmod {n+1}\,.
\end{array}
\]
נמשיך ונחפש את המחיקה הבאה שמתחילה עם
$q+1$.
מיפוי של
$1,\ldots,n$
אל סדרה זו נותן:
\begin{small}
\[
\begin{array}{cccccccccc}
1&\;\; 2&\ldots& n-q&\;\; n+1-q&\;\; n+2-q&\ldots&n-1&\;\; n& \!\!\!\!\!\!\!\!\pmod {n\!+\!1}\\
\downarrow&\;\; \downarrow&&\downarrow&\;\; \downarrow&\;\; \downarrow&&\downarrow&\;\; \downarrow\\
q+1&\;\; q+2&\ldots&n&\;\; n+1&\;\; 1&\ldots&q-2&\;\; q-1& \!\!\!\!\!\pmod {n\!+\!1}\,.
\end{array}
\]
\end{small}
זיכרו שכל החישובים הם מודולו
$n+1$:
\[
\begin{array}{lclcl}
(n+2-q)+q&=& (n+1)+1&=& 1 \quad\;\;\pmod {n+1}\\
(n)+q&= &(n+1)-1+q&= &q-1\pmod {n+1}\,.
\end{array}
\]
זאת בעיה 
\L{Josephus}
עבור
$n$
מספרים, פרט לעובדה שהמספירם מוזזים ב-%
$q$.
מכאן ש:
\[
J(n+1,q)=(J(n,q)+q) \pmod {n+1}\,.
\vspace{-4ex}
\]
\end{proof}

\begin{theorem}\label{lem.jo}
עבור
$n\geq 1$
קיימים מספרים 
$a\geq 0, 0\leq t < 2^a$
כך ש-%
$n=2^a+t$.
\end{theorem}
\begin{proof}
נוכיח את המשפט באמצעות אלגוריתם החילוק עם המחלקים 
$2^0, 2^1, 2^2, 2^4,\ldots$,
אבל קל יותר לראות מהייצוג הבינרי של
$n$.
קיימים
$a$
ו-%
$b_{a-1},b_{a-2},\ldots,b_{1},b_{0}$,
כך שעבור כל
$i$, $b_i=0$
או
$b_i=1$,
ניתן לבטא את 
$n$
כ:
\begin{eqn}
n&=&2^a+b_{a-1}2^{a-1}+\cdots+b_{0}2^{0}\\
n&=&2^a+(b_{a-1}2^{a-1}+\cdots+b_{0}2^{0})\\
n&=&2^a+t,\quad \textrm{כאשר}\; t\leq 2^a-1\,.
\end{eqn}
\end{proof}
כעת נוכיח שקיים ביטוי סגור פשוט עבור
$J(n,2)$. 
\begin{theorem}\label{thm.jo2}
עבור
$n=2^a+t$, $a\geq 0, 0\leq t < 2^a$, $J(n,2)=2t+1$.
\end{theorem}

\begin{proof}
לפי משפט%
~\ref{lem.jo}
ניתן לבטא את
$n$
כפי שרשום שם. ההוכחה ש-%
$J(n,2)=2t+1$
היא על ידי אינדוקציה כפולה, תחילה על 
$a$
אחר כך על
$t$.

\textbf{אינדוקציה ראשונה}

טענת בסיס: נניח ש-%
$t=0$
כך ש-%
$n=2^a$.
יהי
$a=1$
כך ששני המספרים הראשונים במעגל הם
$1,2$. 
אבל
$q=2$
ולכן המספר השני יימחק והמספר שנשאר הוא
$1$
ומכאן ש-%
$J(2^1,2)=1$.

הנחת האינדוקציה היא 
$J(2^a,2)=1$.
מהו
$J(2^{a+1},2)$?
בסבב הראשון מוחקים את כל המספרים הזוגיים:
\[
\begin{array}{rrrrrrrrrrrrrrrrrrrrrrrrrrrr}
1&\quad\not\! 2&\quad3&\quad\not\! 4& \quad\ldots&\quad 2^{a+1}\!-\!1&\quad \not\! 2^{a+1}\,.
\end{array}
\]
כעת נשארו 
$2^a$
מספרים:
\[
\begin{array}{rrrrrrrrrrrrrrrrrrrrrrrrrrrr}
1&\quad3&\quad\ldots&\quad 2^{a+1}\!-\!1\,.
\end{array}
\]
לפי הנחת האינדוקציה
$J(2^{a+1},2)=J(2^a,2)=1$
ולכן לפי אינדוקציה
$J(n,2)=1$
כאשר
$n=2^a+0$.

\textbf{אינדוקציה שניה}

הוכחנו ש-%
$J(2^a+0,2)=2\cdot 0 +1$, 
טענת הבסיס של האינדוקציה השניה.

הנחת האינדוקציה היא
$J(2^a+t,2)=2t+1$.
לפי משפט%
~\ref{thm.jo1}:
\[
J(2^a+(t+1),2)=J(2^a+t,2)+2=2t+1+2=2(t+1)+1\,.
\vspace{-5ex}
\]
\end{proof}

קיים אלגוריתם פשוט לחישוב
$J(n,2)$
שמבוסס על משפטים%
~\ref{lem.jo}
ו%
~\ref{thm.jo2}.
מההוכחה של משפט%
~\ref{lem.jo}:
\[
n=2^a+t=2^a+(b_{a-1}2^{a-1}+\cdots+b_{0}2^{0})\,,
\]
כך ש-%
$t=b_{a-1}2^{a-1}+\cdots+b_{0}2^{0}$.
פשוט נכפיל ב-%
$2$
(על ידי הזזה שמאלה של ספרה אחת) ונוסיף
$1$.
לדוגמה, 
$n=41=2^5+2^3+2^0=101001$
ולכן
$J(41,2)=2t+1$
ובסימון בינרי:
\begin{eqn}
41&=&101001\\
9&=&01001\\
2t+1&=&10011=16+2+1=19\,.
\end{eqn}
הקורא מוסמן לבדוק את התוצאה על ידי מחיקת כל מספר שני במעגל
$1,\ldots,41$.

קיים ביטוי סגור עבור 
$J(n,3)$
אבל הוא מאוד מסובך.

%%%%%%%%%%%%%%%%%%%%%%%%%%%%%%%%%%%%%%%%%%%%%%%%%%%%%%%%

\subsection*{מה ההפתעה?}

אינדוקציה היא אחת ששיטות ההוכחה החשובות ביותר במתמטיקה מודרנית. מספרי
\L{Fibonacci}
מאוד ידועים ומספרי 
\L{Fermat}
קלים להבנה. הופתעתי לגלות כל כך הרבה נוסחאות שלא היכרתי (כגון משפטים%
~\ref{thm.fib-div3}
ו-%
~\ref{thm.seven-fourths})
שניתנות להוכחה באינדוקציה. פונקציה
$91$
של
\L{McCarthy}
התגלתה בהקשר של מדעי המחשב למרות שהיא פונקציה מתמטית. מה שמפתיעה איננה הפונקציה עצמה אלא האינדוקציה המוזרה כאשר 
$98\prec 97$. 
ההפתעה בבעיית
\L{Josephus}
היא באינדוקציה הדו-כיוונית בהוכחה.

\subsection*{מקורות}

ניתן למצוא הצגה נרחבת של אינדוקציה ב-%
\L{\cite{gunderson}}.
ההוכחה של פונקציה 
$91$
של
\L{McCarthy}
נמצאת ב-%
\L{\cite{manna}}
שמייחס אותה ל-%
\L{Rod M. Burstall}.
ההצגה של בעיית 
\L{Josephus}
מבוססת על פרק~
$17$
של
\L{\cite{gunderson}}
שגם מביא את הרקע ההיסטורי ובעיות מעניינות אחרות כגון הילדים המרוחים בבוץ, המטבע המזוייפת, והאגורות בקופסה. חומר נוסף על בעיית
\L{Josephus}
ניתן למצוא ב-%
\L{\cite{schumer,wiki:josephus}}.
