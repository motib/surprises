% !TeX root = surprises.tex

\selectlanguage{hebrew}


\chapter{הוכחות מעניינות באינדוקציה}
\label{c.induction}

פרק זה מביא הוכחות באינדוקציה למספר משפטים בעיקר מתורת המספרים שאולי אינם מוכרים לרבים מהקוראים. 

\selectlanguage{hebrew}


\section{מספרי \L{Fibonacci}}

מספרי פיבונצ'י מוגדרים ברקורסיה:

\begin{eqnarray*}
f_1 &=& 1\\
f_2 &=& 1\\
f_n &=& f_{n-1} + f_{n-2}, \;\;  n \geq 3 \;\; \textrm{\R{עבור}}\,.
\end{eqnarray*}
שנים עשר מספרי פיבונצ'י הראשונים הם:
$
1, 1, 2, 3, 5, 8, 13, 21, 34, 55, 89, 144
$.
\begin{theorem}\mbox{}\\
כל מספר פיבונצ'י רביעי מתחלק ב-%
$3$.
\end{theorem}
\textbf{דוגמאות:}
$f_4=3=3\cdot 1,\; f_8=21=3\cdot 7,\; f_{12}=144=3\cdot 48$.

\textbf{הוכחה:}
טענת הבסיס מתקבלת באופן מיידי כי
$f_4=3$
מתחלק ב-%
$3$.
הנחת האינדוקציה היא ש-%
$f_{4n}$
מתחלק ב-%
$3$.
הצעד האינדוקטיבי הוא:

\begin{eqnarray*}
f_{4(n+1)} &=& f_{4n+4}\\
&=& f_{4n+3}+f_{4n+2}\\
&=& (f_{4n+2}+f_{4n+1})+f_{4n+2}\\
&=& ((f_{4n+1}+f_{4n})+f_{4n+1})+f_{4n+2}\\
&=& ((f_{4n+1}+f_{4n})+f_{4n+1})+(f_{4n+1}+f_{4n})\\
&=& 3f_{4n+1}+2f_{4n}\,.
\end{eqnarray*}



$3f_{4n+1}$
מתחלק ב-%
$3$
ולפי הנחת האינדוקציה גם
$f_{4n}$,
ולכן
$f_{4(n+1)}$
מתחלק ב-%
$3$.\qed


%

\begin{theorem}\mbox{}\\\mbox{}\\
$f_n < \left(\disfrac{7}{4}\right)^n$.
\end{theorem}
\textbf{הוכחה:}
טענות הבסיס:
$f_1=1<\left(\disfrac{7}{4}\right)^1$
ו-%
$f_2=1<\left(\disfrac{7}{4}\right)^2=\disfrac{49}{16}$.



הצעד האינדוקטיבי:




\begin{eqnarray*}
f_{n+1}&=&f_n+f_{n-1}\\
&<&\left(\frac{7}{4}\right)^n + f_{n-1}\\
%&<&\left(\frac{7}{4}\right)^n + \left(\frac{7}{4}\right)^{n-1}\\
&=&\left(\frac{7}{4}\right)^{n-1}\cdot\left(\frac{7}{4}+1\right)\\
&<&\left(\frac{7}{4}\right)^{n-1}\cdot\left(\frac{7}{4}\right)^2\\
&=&\left(\frac{7}{4}\right)^{n+1},
\end{eqnarray*}
בגלל ש-%
\[
\left(\frac{7}{4}+1\right) = \frac{11}{4} = \frac{44}{16}<\frac{49}{16}=\left(\frac{7}{4}\right)^2.
\]
\qed

\begin{theorem}[\L{Binet}]
\begin{displaymath}
f_n = \frac{\phi^n - \bar{\phi}^n}{\sqrt{5}}, \;\;\;\;\;
\phi = \frac{1+\sqrt{5}}{2},\;\;\bar{\phi} = \frac{1-\sqrt{5}}{2}\,.
\end{displaymath}
\end{theorem}



\textbf{הוכחה:}
נוכיח קודם ש-%
$\phi^2=\phi+1$:

\begin{eqnarray*}
\phi^2 &=& \left(\frac{1+\sqrt{5}}{2}\right)^2\\
&=& \frac{1}{4} + \frac{2\sqrt{5}}{4} + \frac{5}{4}\\
&=& \frac{2}{4} + \frac{2\sqrt{5}}{4} + \frac{4}{4}\\
&=& \frac{1+2\sqrt{5}}{2} + 1\\
&=&\phi + 1\,.
\end{eqnarray*}



באופן דומה אפשר להוכיח ש:
$\bar{\phi}^2=\bar{\phi}+1$.

כעת נוכיח את המשפט. טענת הבסיס:
\[
\frac{\phi^1 - \bar{\phi}^1}{\sqrt{5}}=\frac{(1+\sqrt{5})/2-(1-\sqrt{5})/2}{\sqrt{5}}=\frac{2\sqrt{5}}{2\sqrt{5}}=1\,.
\]




הצעד האינדוקטיבי:

\begin{eqnarray*}
\phi^n - \bar{\phi}^n &=& \phi^2\,\phi^{n-2} - \bar{\phi}^2\,\bar{\phi}^{n-2}\\
&=&(\phi+1)\,\phi^{n-2} - (\bar{\phi}+1)\,\bar{\phi}^{n-2}\\
&=&(\phi^{n-1} - \bar{\phi}^{n-1}) + (\phi^{n-2} - \bar{\phi}^{n-2})\\
&=&\sqrt{5}f_{n-1} + \sqrt{5}f_{n-2}\,,
\end{eqnarray*}


לכן:
\[
\frac{\phi^n - \bar{\phi}^n}{\sqrt{5}} = f_{n-1} + f_{n-2} = f_n\,.
\]
\begin{theorem}\label{eq.fibo:combinations}
\[
f_n = {n \choose 0} + {n-1 \choose 1} + {n-2 \choose 2} + \cdots.
\]
\end{theorem}



כדי להוכיח את המשפט, נוכיח תחילה משפט עזר:
\begin{theorem}[\L{Pascal}]
\[
{n \choose k} + {n \choose k+1} = {n+1 \choose k+1}.
\]
\end{theorem}

\textbf{הוכחה:}

\begin{eqnarray*}
{n \choose k} + {n \choose k+1} &=& \frac{n!}{k!(n-k)!} + \frac{n!}{(k+1)!(n-(k+1))!}\\
&=& \frac{(k+1)n!}{(k+1)!(n-k)!} + \frac{n! (n-k)}{(k+1)!(n-k)!}\\
%&=&\frac{n![(k+1)+(n-k)]}{(k+1)!(n-k)!}\\
&=&\frac{n!(n+1)}{(k+1)!(n-k)!}\\
&=&\frac{(n+1)!}{(k+1)!((n+1)-(k+1))!}\\
&=&{n+1 \choose k+1}\,.
\end{eqnarray*}

עכשיו אפשר להוכיח את משפט~%
\ref{eq.fibo:combinations}.
טענת הבסיס:
\[
f_1 = 1 = {1 \choose 0} = \frac{1!}{0!(1-0)!}\,.
\]
הצעד האינדוקטיבי:

\begin{eqnarray*}
f_{n-1} + f_{n-2} &=& {n-1 \choose 0} + {n-2 \choose 1} + {n-3 \choose 2} + {n-4 \choose 3} + \cdots\\
&&\hspace{54pt}{n-2 \choose 0} + {n-3 \choose 1} + {n-4 \choose 2} + \cdots\\
&=&{n-1 \choose 0} + {n-1 \choose 1} + {n-2 \choose 2} + {n-3 \choose 3} + \cdots\\
&=&{n \choose 0}\hspace{20pt} + {n-1 \choose 1} + {n-2 \choose 2} + {n-3 \choose 3} + \cdots.
\end{eqnarray*}
השוויון האחרון משתמש בעובדה ש:
\[
{k \choose 0} = \frac{k!}{0!(k-0)!} = 1\,.
\]

%%%%%%%%%%%%%%%%%%%%%%%%%%%%%%%%%%%%%%%%%%%%%%%%%%%%%%%%%%

\selectlanguage{hebrew}


\section{מספרי \L{Fermat}}


\textbf{הגדרה:}
מספר פרמה הוא מספר שערכו
$2^{2^{n}}+1$
עבור 
$n\geq 0$.

חמשת מספרי  פרמה הראשונים הם:
\begin{center}

%\renewcommand{\arraystretch}{1}

\begin{tabular}{|c|c|c|c|c|c|}
\hline
$n$ & $0$ & $1$ & $2$ & $3$ & $4$ \\\hline
$2^{2^{n}}+1$ & $3$ & $5$ & $17$ & $257$ & $65537$ \\\hline
\end{tabular}
\end{center}
כל המספרים הללו ראשוניים ו-%
\L{Pierre de Fermat}
שיער שכל מספרי פרמה הם ראשוניים. כעבור כמאה שנים
\L{Leonhard Euler}
הראה ש:
\[
2^{2^5}+1 = 2^{32}+1 = 4294967297 = 641 \times 6700417\,.
\]
ידוע שמספרי פרמה אינם ראשוניים עבור
$5\leq n \leq 32$,
אבל הפירוק לגורמים של חלק מהמספרים הללו עדיין לא ידוע. 
הנה שני משפטים מעניינים על מספרי פרמה:

\begin{theorem}
עבור כל
$n\geq 2$,
הספרה האחרונה של
$F_n$
היא
$7$.
\end{theorem}
\textbf{הוכחה:}
טענת הבסיס:
$F_2=2^{2^2}+1=17$.

הצעד האינדוקטיבי: נניח ש-%
$F_n=10k_n+7$
עבור
$k\geq 1$.
אזי:

\begin{eqnarray*}
F_{n+1}&=&2^{2^{n+1}}+1=\left(2^{2^{n}}\right)^2+1\\
&=&\left(\left(2^{2^{n}}+1\right)-1\right)^2+1\\
&=&\left(2^{2^{n}}+1\right)^2
-2\cdot\left(2^{2^{n}}+1\right)+1+1\\
&=&(10k_n+7)^2-2(10k_n+7)+2\\
%&=&(10k_n+7-1)^2+1=(10k_n+6)^2+1\\
&=&100k_n^2+120k_n+37\\
&=&10(10k_n^2+12k_n+3)+7\\
&=&10k_{n+1}+7\,.
\end{eqnarray*}



\begin{theorem}\label{thm.fermat}
עבור כל
$n\geq 1$, $\displaystyle F_n = \prod_{k=0}^{n-1} F_k + 2$.
\end{theorem}
\textbf{הוכחה:}
טענת הבסיס:
\[
5=F_1=\prod_{k=0}^{0} F_k + 2=F_0+2=3+2\,.
\]
הצעד האינדוקטיבי:

\begin{eqnarray*}
\displaystyle\prod_{k=0}^{n}F_k&=&\left(\displaystyle\prod_{k=0}^{n-1}F_k\right) F_n \\
&=& (F_n-2)F_n\\
&=& F_n^2-2F_n\\
&=& \left(2^{2^n}+1\right)^2-2\cdot \left(2^{2^n}+1\right)\\
&=& 2^{2^{n+1}}-1= (2^{2^{n+1}}+1)-2\\
&=&F_{n+1}-2\\
F_{n+1}&=&\displaystyle\prod_{k=0}^{n}F_k + 2\,.
\end{eqnarray*}

%

%%%%%%%%%%%%%%%%%%%%%%%%%%%%%%%%%%%%%%%%%%%%%%%%%%%%%%%%%%%
\selectlanguage{hebrew}


\section{פונקציה $91$ של
\L{McCarthy}}

בסעיף זה נוכיח באינדוקציה תכונה מוזרה של פונקציה רקורסיבית המוגדרת עבור כל המספרים השלמים. ממציא הפונקציה הוא
\L{John McCarthy}.
הגדרתה היא:
\[
f(x) = \textrm{if}\;\; x > 100 \;\;\textrm{then}\;\; x - 10 \;\;\textrm{else}\;\; f(f(x+11))\,.
\]
אם 
$x$
גדול מ-%
$100$,
ערכה של הפונצקיה היא 
$x-10$.
אחרת, חשב את 
$f(x+11)$.
אחר כך, נחשב את הפונקציה עבור התוצאה שהתקבלה.


עבור מספרים גדולים מ-%
$100$,
חישוב הפונקציה פשוטה ביותר:
\[
f(101) = 91, \;\; f(102) = 92,\;\; f(103) = 93,\;\; f(104) = 94\,.
\]
נחשב את ערך הפונקציה עבור קומץ מספרים פחות או שווים  ל-%
$100$:
\begin{eqnarray*}
f(100) &=& f(f(100+11)) = f(f(111)) = f(101) = 91\\
f(99) &=& f(f(99+11)) = f(f(110)) = f(100) = 91\\
f(91) &=& f(f(91+11)) = f(f(102)) = f(92) = f(f(103)) = f(93) = \cdots = 91\\
f(89) &=& f(f(89+11)) = f(f(100)) = f(f(111)) = f(101) = 91\,.
\end{eqnarray*}

\begin{theorem}\mbox{}\\
הפונקציה
$f(x)$
שקולה ל:
\[
g(x) = \textrm{if}\;\; x > 100 \;\;\textrm{then}\;\; x - 10 \;\;\textrm{else}\;\; 91\,.
\]
\end{theorem}

\textbf{הוכחה:}
ההוכחה באינדוקציה מעל קבוצת המספרים:
\[
S=\{x\,|\,x\leq 101\}
\]
היחס "פחות מ"
$\prec$
מוגדר כך:
\[
x \prec y \;\; \textrm{iff}\;\; y < x\,,
\]
בצד הימני 
$<$
הוא היחס הרגיל מעל למספרים שלמים. סדר המספרים לפי
$\prec$
הוא:
\[
101 \prec 100 \prec 99 \prec 98 \prec 97 \prec \cdots\,.
\]
נוכיח את המשפט באינדוקציה על הקבוצה
$S$
עם האופרטור
$\prec$.

\noindent\textbf{מקרה 1}  $x > 100$.
ההוכחה מיידית מההגדרות של 
$f$
ו-
$g$.

\noindent\textbf{מקרה 2} 
$90\leq x \leq 100$.

\noindent{}%
טענת הבסיס:
\[
f(100) = f(f(100+11)) = f(f(111)) = f(101) = 91 = g(100)\,,
\]
הנחת האינדוקציה היא
$f(y) = g(y)$
עבור
$y\prec x$.

\noindent{}%
הצעד האינדוקטיבי:

\begin{eqnarray}
f(x) &=& f(f(x+11))\label{m91-1}\\
&=& f(x+11-10)= f(x+1)\label{m91-3}\\
&=& g(x+1)\label{m91-4}\\
&=& 91\label{m91-5}\\
&=& g(x)\label{m91-6}\,.
\end{eqnarray}

משוואה%
~\ref{m91-1}
נכונה מההגדרה של
$f$
כי
$x\leq 100$.
השוויון בין משוואה%
~\ref{m91-1}
לבין משוואה%
~\ref{m91-3}
נכון מההגדרה של
$f$
במקרה זה כי
$x \geq 90$
ולכן
$x+11 > 100$.
השוויון בין משוואה%
~\ref{m91-3}
ומשוואה%
~\ref{m91-4}
נובע מהנחת האינדוקציה:
\[
x\leq 100 \Rightarrow x+1 \leq 101 \Rightarrow x+1\in S \Rightarrow x+1\prec x\,.
\]
השוויון בין המשוואות%
~\ref{m91-4}, \ref{m91-5}, \ref{m91-6}
נכון מההגדרה של 
$g$
ו-%
$x+1 \leq 101$.

\noindent\textbf{מקרה 3} $x< 90$.

\noindent{}%
טענת הבסיס:
\[
f(89) = f(f(100)) = f(f(f(111))) = f(f(101)) = f(91) = 91 = g(89)\,,
\]
לפי ההגדרה של
$g$
כי
$89\leq 00$.

הנחת האינדוקציה היא
$f(y) = g(y)$
עבור
$y\prec x$.

הצעד האינדוקטיבי:

\begin{eqnarray}
f(x) &=& f(f(x+11))\label{m91a}\\
&=& f(g(x+11))\label{m91b}\\
&=& f(91)\label{m91c}\\
&=& 91\label{m91d}\\
&=& g(x)\,.
\end{eqnarray}



משוואה%
~\ref{m91a}
נכונה לפי ההגדרה של
$f$
ו-%
$x<90\leq 100$.
השוויון בין המשוואות
\ref{m91a}
ו-%
\ref{m91b}
נובע מהנחת האינדוקציה:
\[
x < 90 \Rightarrow x+11< 101 \Rightarrow x+11\in S \Rightarrow x+11 \prec x\,.
\]
השוויון בין המשוואות%
~\ref{m91b}
ו-%
\ref{m91c}
נכון לפי ההגדרה של
$g$
ו-%
$x+11 < 101$.
לבסוף, כבר הוכחנו ש-%
$f(91)=91$
ולפי ההגדרה
$g(x)=91$
עבור
$x<90$.

\selectlanguage{hebrew}
\subsection*{מקורות}

לדיון נרחב על הוכחה באינדוקציה ראו
\L{\cite{ben-ari:induction}}.
הספר הארוך של
\L{Gunderson}
\L{\cite{gunderson}}
מוקדש כולו לאינדוקציה.
ההוכחה למשפט~%
\ref{thm.fermat}
לקוחה מ-%
\L{\cite{thebook}}.
ההוכחה של הפונקציה של 
\L{McCarthy}
מבוססת על
\L{\cite{manna}}
ומיוחסת ל-%
\L{Rod M. Burstall}.
