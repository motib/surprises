% !TeX Program=XeLaTeX
% !TeX root = surprises.tex

\selectlanguage{hebrew}

\chapter{אינדוקציה}\label{c.induction}

%%%%%%%%%%%%%%%%%%%%%%%%%%%%%%%%%%%%%%%%%%%%%%%%%%%%%%%%%%%%%%%

האקסיומה של אינדוקציה מתמטית נמצאת בשימוש נרחב כשיטת הוכחה במתמטיקה. פרק זה מציג הוכחות אינדוקטיביות של תוצאות שייתכן שאינן מוכרות לקורא. נתחיל בסקירה קצרה של אינדוקציה מתמטית (סעיף%
~\ref{s.induction-axiom}).
סעיף%
~\ref{s.induction-fibonacci}
מביא הוכחות של משפטים על מספרי פיבונאצ'י
\L{(Fibonacci)}
המוכרים וסעיף%
~\ref{s.induction-fermat}
מביא הוכחות של משפטים על מספרי פרמה
\L{(Fermat)}.
בסעיף%
~\ref{s.induction-mccarthy}
נציג את פונקציה
$91$
של ג'ון מקארתי
\L{(John McCarthy)}.
ההוכחה אינה שגרתית כי היא משתמשת באינדוקציה על מספרים שלמים בסדר הפוך. הוכחת הנוסחה עבור הבעיה של יוסף בן-מתתיהו
\L{(Josephus)}
גם היא אינה שגרתית כי היא משתמשת באינדוקציה כפולה על חלקים שונים של ביטוי (סעיף%
.~\ref{s.josephus}).

\section{האקסיומה של אינדוקציה מתמטית}\label{s.induction-axiom}

אינדוקציה מתמטית היא הדרך המובילה להוכחת משפטים עבור קבוצה לא חסומה של מספרים. נעיין בשוויונות:
\[
1=1,\quad 1+2=3,\quad 1+2+3=6,\quad 1+2+3+4=10\,,
\]
נשים לב ש:
\[
1=(1\cdot 2)/2,\quad 3=(2\cdot 3)/2,\quad  6=(3\cdot 4)/2,\quad 10=(4\cdot 5)/2\,,
\]
ונשער שעבור כל המספרים שלמים
$n\geq 1$:
\[
\sum_{i=1}^n i = \frac{n(n+1)}{2}\,.
\]
עם מספיק סבלנות קל לבדוק את הנוסחה עבור כל ערך מסוים של
$n$,
אבל איך אפשר להוכיח עבור אינסוף המספרים השלמים החיוביים? כאן נכנסת אינדוקציה מתמטית.

\begin{axiom}
תהי
$P(n)$
תכונה (כגון משוואה, נוסחה או משפט), כאשר 
$n$
הוא מספר שלם חיובי. נניח שניתן:
\begin{itemize}
\item \textbf{טענת הבסיס}: 
להוכיח ש-%
$P(1)$
נכונה,
\item \textbf{צעד אינדוקטיבי}:
עבור 
$m$
שרירותי, להוכיח ש-%
$P(m+1)$
נכונה בהנחה ש-%
$P(m)$
נכונה,
\end{itemize}
אז
$P(n)$
נכונה עבור כל
$n\geq 1$.
ההנחה ש-%
$P(m)$
נכונה עבור 
$m$ 
שרירותי נקראת
\textbf{הנחת האינדוקציה}.
\end{axiom}

הנה דוגמה פשוטה עבור הוכחה באינדוקציה מתמטית.
\begin{theorem}\label{t.sum}
עבור
$n\geq 1$:
\[
\sum_{i=1}^n i = \frac{n(n+1)}{2}\,.
\]
\end{theorem}

\begin{proof} 
טענת הבסיס פשוטה:
\[
\sum_{i=1}^1 i = 1 =\frac{1(1+1)}{2}\,.
\]
הנחת האינדוקציה היא שמשוואה שלהלן נכונה עבור  
$m$:
\[
\sum_{i=1}^{m} i = \frac{m(m+1)}{2}\,.
\]
הצעד האינדוקטיבי הוא להוכיח את המשפט עבור
$m+1$:
\begin{eqn}
\sum_{i=1}^{m+1} i &=& \sum_{i=1}^m i + (m+1)\label{l.sum1}\\
&=&\frac{m(m+1)}{2} + (m+1)\label{l.sum2}
%&=&\frac{m(m+1) + 2(m+1)}{2}\label{l.sum3}\\
=\frac{(m+1)(m+2)}{2}\,.\label{l.sum4}
\end{eqn}
לפי אקסיומת האינדוקציה המתמטית, עבור כל
$n\geq 1$:
\[
\sum_{i=1}^n i = \frac{n(n+1)}{2}\,.
\]
\end{proof}

הנחת האינדוקציה עלולה לבלבל ,כי נראה שאנחנו מניחים את מה שרוצים להוכיח. אין כאן הסקת מסקנות מעגלית כי ההנחה היא עבור תכונה של משהו קטן ומשתמשים בהנחה כדי להוכיח תכונה עבור משהו גדול יותר.

אינדוקציה מתמטית היא אקסיומה שאי-אפשר להוכיח. פשוט צריכים לקבל אותה כמו שמקבלים אקסיומות אחרות כגון
$x+0=x$.
כמובן שתוכלו לדחות את האקסיומה, אבל אז תצטרכו לדחות חלק גדול מהמתמטיקה המודרנית.
\begin{advanced}
אינדוקציה מתמטית היא כלל היסק שהוא אחד מאקסיומות פאנו
\L{(Peano)}
לפורמליזציה של המספרים הטבעיים. ניתן להוכיח את האקסיומה בעזרת אקסיומה אחרת כגון עקרון הסדר הטוב
ולהפך, אבל לא ניתן להוכיח אותה בעזרת אקסיומות אחרות, פשוטות יותר, של פאנו. 
\end{advanced}

%%%%%%%%%%%%%%%%%%%%%%%%%%%%%%%%%%%%%%%%%%%%%%%%%%%%%%%%


\section{מספרי פיבונאצ'י }\label{s.induction-fibonacci}

מספרי פיבונטצ'י מוגדרים ברקורסיה:
\begin{eqn}
f_1 &=& 1\\
f_2 &=& 1\\
f_n &=& f_{n-1} + f_{n-2}, \;\;  n \geq 3 \;\; \textrm{\R{עבור}}\,.
\end{eqn}
שנים עשר מספרי פיבונאצ'י הראשונים הם:
$
1, 1, 2, 3, 5, 8, 13, 21, 34, 55, 89, 144
$.
\begin{theorem}\label{thm.fib-div3}
כל מספר פיבונאצ'י רביעי מתחלק ב-%
$3$.
\end{theorem}
\begin{example}
$f_4=3=3\cdot 1,\; f_8=21=3\cdot 7,\; f_{12}=144=3\cdot 48$.
\end{example}
\begin{proof}
טענת הבסיס מתקבלת באופן מיידי כי
$f_4=3$
מתחלק ב-%
$3$.
הנחת האינדוקציה היא ש-%
$f_{4n}$
מתחלק ב-%
$3$.
הצעד האינדוקטיבי הוא:
\begin{eqn}
f_{4(n+1)} &=& f_{4n+4}\\
&=& f_{4n+3}+f_{4n+2}\\
&=& (f_{4n+2}+f_{4n+1})+f_{4n+2}\\
&=& ((f_{4n+1}+f_{4n})+f_{4n+1})+f_{4n+2}\\
&=& ((f_{4n+1}+f_{4n})+f_{4n+1})+(f_{4n+1}+f_{4n})\\
&=& 3f_{4n+1}+2f_{4n}\,.
\end{eqn}
$3f_{4n+1}$
מתחלק ב-%
$3$
ולפי הנחת האינדוקציה גם
$f_{4n}$,
ולכן
$f_{4(n+1)}$
מתחלק ב-%
$3$.
\end{proof}

\begin{theorem}\label{thm.seven-fourths}
$f_n < \left(\disfrac{7}{4}\right)^n$.
\end{theorem}
\begin{proof}
טענות הבסיס:
$f_1=1<\left(\disfrac{7}{4}\right)^1$
ו-%
$f_2=1<\left(\disfrac{7}{4}\right)^2=\disfrac{49}{16}$.

הצעד האינדוקטיבי:
\begin{eqn}
f_{n+1}&=&f_n+f_{n-1}\\
&<&\left(\frac{7}{4}\right)^n + f_{n-1}\\
%&<&\left(\frac{7}{4}\right)^n + \left(\frac{7}{4}\right)^{n-1}\\
&=&\left(\frac{7}{4}\right)^{n-1}\cdot\left(\frac{7}{4}+1\right)\\
&<&\left(\frac{7}{4}\right)^{n-1}\cdot\left(\frac{7}{4}\right)^2\\
&=&\left(\frac{7}{4}\right)^{n+1},
\end{eqn}
מכיוון ש:
\[
\left(\frac{7}{4}+1\right) = \frac{11}{4} = \frac{44}{16}<\frac{49}{16}=\left(\frac{7}{4}\right)^2.
\]
\end{proof}

כדי להוכיח 
$f_{n+1}$,
הנחנו שהטענה נכונה לא רק עבור
$f_{n}$
אלא גם עבור
$f_{n-1}$.
ב-%
\textbf{אינדוקציה שלמה}
אפשר להניח שהטענה נכונה עבור כל
$m\leq n$.
ניתן להוכיח שאינדוקציה שלמה שקולה לאינדוקציה מתמטית.

\begin{theorem}[ נוסחת בינה \L{(Binet)}]
\begin{displaymath}
\phi = \frac{1+\sqrt{5}}{2},\;\;\bar{\phi} = \frac{1-\sqrt{5}}{2}
\quad\textrm{\R{כאשר}}\quad
f_n = \frac{\phi^n - \bar{\phi}^n}{\sqrt{5}}\,.
\end{displaymath}
\end{theorem}
\begin{proof}
ראשית נוכיח ש-%
$\phi^2=\phi+1$:
\begin{eqn}
\phi^2 &=& \left(\frac{1+\sqrt{5}}{2}\right)^2\\
&=& \frac{1}{4} + \frac{2\sqrt{5}}{4} + \frac{5}{4}\\
&=& \frac{2}{4} + \frac{2\sqrt{5}}{4} + \frac{4}{4}\\
&=& \frac{1+\sqrt{5}}{2} + 1\\
&=&\phi + 1\,.
\end{eqn}
באופן דומה אפשר להוכיח ש:
$\bar{\phi}^2=\bar{\phi}+1$.

טענת הבסיס של נוסחת בינה
היא:
\[
\frac{\phi^1 - \bar{\phi}^1}{\sqrt{5}}=\frac{(1+\sqrt{5})/2-(1-\sqrt{5})/2}{\sqrt{5}}=\frac{2\sqrt{5}}{2\sqrt{5}}=1\,.
\]
נניח שהנחת האינדוקציה נכונה עבור כל
$k\leq n$.
הצעד האינדוקטיבי הוא:
\begin{eqn}
\phi^n - \bar{\phi}^n &=& \phi^2\,\phi^{n-2} - \bar{\phi}^2\,\bar{\phi}^{n-2}\\
&=&(\phi+1)\,\phi^{n-2} - (\bar{\phi}+1)\,\bar{\phi}^{n-2}\\
&=&(\phi^{n-1} - \bar{\phi}^{n-1}) + (\phi^{n-2} - \bar{\phi}^{n-2})\\
&=&\sqrt{5}f_{n-1} + \sqrt{5}f_{n-2}\\
\frac{\phi^n - \bar{\phi}^n}{\sqrt{5}} &=& f_{n-1} + f_{n-2} = f_n\,.
\end{eqn}
\end{proof}

\newpage

\begin{theorem}\label{eq.fibo:combinations}
\[
f_n = {n \choose 0} + {n-1 \choose 1} + {n-2 \choose 2} + \cdots.
\]
\end{theorem}
\begin{proof}
נוכיח תחילה את נוסחת פסקל
\L{(Pascal)}:
\[
{n \choose k} + {n \choose k+1} = {n+1 \choose k+1}.
\]
\begin{eqn}
{n \choose k} + {n \choose k+1} &=& \frac{n!}{k!(n-k)!} + \frac{n!}{(k+1)!(n-(k+1))!}\\
&=& \frac{(k+1)n!}{(k+1)!(n-k)!} + \frac{n! (n-k)}{(k+1)!(n-k)!}\\
%&=&\frac{n![(k+1)+(n-k)]}{(k+1)!(n-k)!}\\
&=&\frac{n!(n+1)}{(k+1)!(n-k)!}\\
&=&\frac{(n+1)!}{(k+1)!((n+1)-(k+1))!}\\
&=&{n+1 \choose k+1}\,.
\end{eqn}
נשתמש גם בשוויון
$\displaystyle{k\choose 0} = \frac{k!}{0!(k-0)!} = 1$
עבור כל
$k\geq 1$.

עכשיו אפשר להוכיח את המשפט. טענת הבסיס:
\[
f_1 = 1 = {1 \choose 0} = \frac{1!}{0!(1-0)!}\,.
\]
הצעד האינדוקטיבי הוא:
\begin{eqn}
f_{n-1} + f_{n-2} &=& {n-1 \choose 0} + {n-2 \choose 1} + {n-3 \choose 2} + {n-4 \choose 3} + \cdots\\
&&\hspace{54pt}{n-2 \choose 0} + {n-3 \choose 1} + {n-4 \choose 2} + \cdots\\
&=&{n-1 \choose 0} + {n-1 \choose 1} + {n-2 \choose 2} + {n-3 \choose 3} + \cdots\\
&=&{n \choose 0}\hspace{20pt} + {n-1 \choose 1} + {n-2 \choose 2} + {n-3 \choose 3} + \cdots.
\end{eqn}
\end{proof}

%%%%%%%%%%%%%%%%%%%%%%%%%%%%%%%%%%%%%%%%%%%%%%%%%%%%%%%%%%
\newpage

\section{ מספרי פרמה}\label{s.induction-fermat}

\begin{definition}
מספר פרמה הוא מספר שלם שערכו
$2^{2^{n}}+1$
עבור 
$n\geq 0$.
\end{definition}
אמנם הגדרנו שטענת הבסיס היא מ-%
$n=1$
אבל ניתן להשתמש בכל ערך שלם, והטענה נכונה מאותו ערך והלאה.

חמשת מספרי פרמה הראשונים הם מספרים ראשוניים:
\[
F_0=3,\quad F_1=5,\quad F_2=17,\quad F_3=257,\quad F_4=65537\,.
\]
במאה ה-17 שיער המתמטיקאי פייר דה פרמה 
שכל מספרי פרמה הם ראשוניים, אבל כעבור כמאה שנים הראה לאונרד אוילר 
ש:
\[
2^{2^5}+1 = 2^{32}+1 = 4294967297 = 641 \times 6700417\,.
\]
מספרי פרמה גדלים מהר מאוד ככל ש-%
$n$
גדל. ידוע שמספרי פרמה אינם ראשוניים עבור
$5\leq n \leq 32$,
אבל הפירוק לגורמים של חלק מהמספרים הללו עדיין לא ידוע. 
\begin{theorem}
עבור
$n\geq 2$,
הספרה האחרונה של
$F_n$
היא
$7$.
\end{theorem}
\begin{proof}
טענת הבסיס:
$F_2=2^{2^2}+1=17$.
הנחת האינדוקציה היא ש-%
$F_n=10k_n+7$
עבור
$k\geq 1$.
הצעד האינדוקטיבי הוא:
\begin{eqn}
F_{n+1}&=&2^{2^{n+1}}+1=\left(2^{2^{n}}\right)^2+1\\
&=&\left(\left(2^{2^{n}}+1\right)-1\right)^2+1\\
&=&\left(2^{2^{n}}+1\right)^2
-2\cdot\left(2^{2^{n}}+1\right)+1+1\\
&=&(10k_n+7)^2-2(10k_n+7)+2\\
%&=&(10k_n+7-1)^2+1=(10k_n+6)^2+1\\
&=&100k_n^2+120k_n+37\\
&=&10(10k_n^2+12k_n+3)+7\\
&=&10k_{n+1}+7\,.
\end{eqn}
\end{proof}
\begin{theorem}\label{thm.fermat}
עבור כל
$n\geq 1$, $\displaystyle F_n = \prod_{k=0}^{n-1} F_k + 2$.
\end{theorem}
\begin{proof}
טענת הבסיס:
\[
5=F_1=\prod_{k=0}^{0} F_k + 2=F_0+2=3+2\,.
\]
הצעד האינדוקטיבי:
\begin{eqn}
\displaystyle\prod_{k=0}^{n}F_k&=&\left(\displaystyle\prod_{k=0}^{n-1}F_k\right) F_n \\
&=& (F_n-2)F_n\\
&=& F_n^2-2F_n\\
&=& \left(2^{2^n}+1\right)^2-2\cdot \left(2^{2^n}+1\right)\\
&=& 2^{2^{n+1}}-1= (2^{2^{n+1}}+1)-2\\
&=&F_{n+1}-2\\
F_{n+1}&=&\displaystyle\prod_{k=0}^{n}F_k + 2\,.
\end{eqn}
\end{proof}

%%%%%%%%%%%%%%%%%%%%%%%%%%%%%%%%%%%%%%%%%%%%%%%%%%%%%%%%%%%

\section{פונקציה $91$ של מקארתי
\L{\normalsize (McCarthy)}}\label{s.induction-mccarthy}

אינדוקציה מתקשרת אצלנו להוכחות של תכונות המוגדרות על קבוצת המספרים השלמים החיוביים. כאן נביא הוכחה אינדוקטיבית המבוססת על יחס מוזר כאשר מספרים גודלים הם קטנים ממספרים קטנים. האינדוקציה מצליחה, כי התכונה היחידה הנדרשת מהקבוצה היא שקיים סדר לפי פעולה יחס.

נתבונן בפונקציה הרקורסיבית שלהלן המוגדרת על המספרים השלמים:
\[
f(x) = 
\left\{
\begin{array}{l@{\hspace{2em}}r}
x-10 & x > 100\;\;\textrm{\R{אם}}\\
f(f(x+11)) & \textrm{\R{אחרת}}
\end{array}
\right.
\]

עבור מספרים גדולים מ-%
$100$
חישוב הפונקציה פשוט ביותר:
\[
f(101) = 91, \;\; f(102) = 92,\;\; f(103) = 93,\;\; f(104) = 94\,.
\]
מה עם מספרים קטנים או שווים ל-%
$100$?
נחשב את
$f(x)$
עבור מספרים מסוימים כאשר החישוב בכל שורה מסתמך על השורות הקודמות:
\begin{eqn}
f(100) &=& f(f(100+11)) = f(f(111)) = f(101) = 91\\
f(99) &=& f(f(99+11)) = f(f(110)) = f(100) = 91\\
f(98) &=& f(f(98+11)) = f(f(109)) = f(99) = 91\\
&\cdots&\\
f(91) &=& f(f(91+11)) = f(f(102)) = f(92)\\
&& \quad = f(f(103)) = f(93) = \cdots =f(98) = 91\\
f(90) &=& f(f(90+11)) = f(f(101)) = f(91) = 91\\
f(89) &=& f(f(89+11)) = f(f(100)) = f(91) = 91\,.
\end{eqn}
נגדיר את הפונקציה
$g$:
\[
g(x) = 
\left\{
\begin{array}{l@{\hspace{2em}}r}
x-10 & x > 100\;\;\textrm{\R{אם}}\\
91&\textrm{\R{אחרת}}
\end{array}
\right.
\]
\begin{theorem}
עבור כל 
$x$, $f(x)=g(x)$.
\end{theorem}
\begin{proof}
ההוכחה באינדוקציה מעל קבוצת המספרים
$S=\{x\,|\,x\leq 101\}$
כאשר היחס
$\prec$
מוגדר על ידי:
\[
x \prec y \;\; \textrm{\R{או"א}} \;\; y < x\,.
\]
בצד הימני 
$<$
הוא היחס הרגיל מעל למספרים שלמים. סדר המספרים לפי
$\prec$
הוא:
\[
101 \prec 100 \prec 99 \prec 98 \prec 97 \prec \cdots\,.
\]
נפצל לשלושה מקרים. נשתמש בתוצאות של החישובים לעיל:

\noindent\textbf{מקרה 1}  $x > 100$.
ההוכחה מיידית מההגדרות של 
$f$
ו-
$g$.

\noindent\textbf{מקרה 2} 
$90\leq x \leq 100$.

\noindent{}%
טענת הבסיס היא:
\[
f(100)  = 91 = g(100)\,,
\]
כי הראינו ש-%
$f(100)=91$
ולפי ההגדרה
$g(100)=91$.

הנחת האינדוקציה היא
$f(y) = g(y)$
עבור
$y\prec x$,
והצעד האינדוקטיבי הוא:
\begin{eqnlabels}
f(x) &=& f(f(x+11))\label{m91-1}\\
&=& f(x+11-10)= f(x+1)\label{m91-3}\\
&=& g(x+1)\label{m91-4}\\
&=& 91\label{m91-5}\\
&=& g(x)\label{m91-6}\,.
\end{eqnlabels}
השוויון%
~\ref{m91-1}
מתקיים לפי ההגדרה של
$f$
כי
$x\leq 100$.
השוויון בין %
~\ref{m91-1}
ל-%
~\ref{m91-3}
מתקיים לפי ההגדרה של
$f$,
כי
$x \geq 90$
ולכן
$x+11 > 100$.
השוויון בין %
~\ref{m91-3}
ו-%
~\ref{m91-4}
נובע מהנחת האינדוקציה
$x\leq 100$
ולכן
$x+1\leq 101$
ומכאן אפשר להסיק ש-%
$x+1\in S$
ו-%
$x+1\prec x$.
השוויון בין %
~\ref{m91-4}, \ref{m91-5}, \ref{m91-6}
מתקיים לפי ההגדרה של 
$g$
ו-%
$x+1 \leq 101$
ולכן
$x\leq 100$.

\noindent\textbf{מקרה 3} $x< 90$.

טענת הבסיס היא
$f(89) = f(f(100)) = f(91) = 91 = g(89)$
לפי ההגדרה של
$g$
כי
$89\leq 100$.

הנחת האינדוקציה היא
$f(y) = g(y)$
עבור
$y\prec x$
והצעד האינדוקטיבי הוא:
\begin{eqnlabels}
f(x) &=& f(f(x+11))\label{m91a}\\
&=& f(g(x+11))\label{m91b}\\
&=& f(91)\label{m91c}\\
&=& 91\label{m91d}\\
&=& g(x)\,.
\end{eqnlabels}
השוויון%
~\ref{m91a}
מתקיים לפי ההגדרה של
$f$
ו-%
$x<90\leq 100$.
השוויון בין 
\ref{m91a}
ו-%
\ref{m91b}
נובע מהנחת האינדוקציה,
$x < 90$
ולכן
$x+11< 101$
שממנו נובע
$x+11\in S$
ו-%
$x+11 \prec x$.
השוויון בין %
~\ref{m91b}
ו-%
\ref{m91c}
מתקיים לפי ההגדרה של
$g$
ו-%
$x+11 < 101$.
לבסוף, כבר הוכחנו ש-%
$f(91)=91$
ולפי ההגדרה
$g(x)=91$
עבור
$x<90$.
\end{proof}

%%%%%%%%%%%%%%%%%%%%%%%%%%%%%%%%%%%%%%%%%%%%%%%%%%%%%%%%

\section{ בעיית יוספוס \L{\normalsize}}\label{s.josephus}

יוסף בן מתתיהו 
\L{(Titus Flavius Josephus)}
היה מפקד העיר יודפת בזמן המרד הגדול נגד הרומאים. הכוח העצום של הצבא הרומי מחץ את הגנת העיר ויוסף מצא מקלט במערה עם חלק מאנשיו שהעדיפו להתאבד ולא להיהרג או ליפול בשבי הרומאים. לפי מה שסיפר יוסף הוא מצא דרך להציל את עצמו, נשבה והפך למשקיף עם הרומאים ואחר כך כתב את ההיסטוריה של המרד. נציג את הבעיה הקרויה על שמו כבעיה מתמטית מופשטת.
\begin{definition}[בעיית יוספוס]
נסדר את המספרים
$1,\ldots,n\!+\!1$
במעגל. נמחק כל מספר ה-%
$q$
מסביב למעגל
$q, 2q, 3q, \ldots$
(מודולו
$n\!+\!1$)
עד שיישאר רק מספר אחד 
$m$
.
$J(n+1,q)=m$
הוא
\textbf{מספר יוסיפוס \L{(Josephus)}}
עבור
$n+1$
ו-%
$q$.
\end{definition}
\begin{example}
יהיו
$n+1=41$
ו-%
$q=3$.
נסדר את המספרים במעגל:
\[
\begin{array}{rrrrrrrrrrrrrrrrrrrrrrr}
\rightarrow\!&\!1\!&\!2\!&\!3\!&\!4\!&\!5\!&\!6\!&\!7\!&\!8\!&\!9\!&\!10\!&\!
           11\!&\!12\!&\!13\!&\!14\!&\!15\!&\!16\!&\!17\!&\!18\!&\!19\!&\!20\!&\!21\!&\!
\downarrow\\
\uparrow\quad\!&\!41\!&\!40\!&\!39\!&\!38\!&\!37\!&\!36\!&\!35\!&\!34\!&\!33\!&\!
32\!&\!31\!&\!30\!&\!29\!&\!28\!&\!27\!&\!26\!&\!25\!&\!24\!&\!23\!&\!22\!&\!
&\leftarrow
\end{array}
\]
תוצאת הסבב הראשון של המחיקות היא:
\[
\begin{array}{rrrrrrrrrrrrrrrrrrrrrrr}
\rightarrow\!&\!1\!&\!2\!&\!\not\! 3\!&\!4\!&\!5\!&\!\not\! 6\!&\!7\!&\!8\!&\!\not\! 9\!&\!10\!&\!
           11\!&\!\not\!\! 12\!&\!13\!&\!14\!&\!\not\!\! 15\!&\!16\!&\!17\!&\!\not\!\! 18\!&\!19\!&\!20\!&\!\not\!\! 21\!&\!
\downarrow\\
	\uparrow\quad\!&\!41\!&\!40\!&\!\not\!\! 39\!&\!38\!&\!37\!&\!\not\!\! 36\!&\!35\!&\!34\!&\!\not\!\! 33\!&\!
32\!&\!31\!&\!\not\!\! 30\!&\!29\!&\!28\!&\!\not\!\! 27\!&\!26\!&\!25\!&\!\not\!\! 24\!&\!23\!&\!22\!&\!
\!&\!\leftarrow
\end{array}
\]
לאחר השמטת המספרים שנמחקו נקבל:
\[
\begin{array}{rrrrrrrrrrrrrrrrrrrrrrrrrrrr}
\rightarrow\!&1&2&4&5&7&8&10&11&13&14&16&17&19&20&\!\downarrow\\
\uparrow\quad\!&41&40&38&37&35&34&32&31&29&28&26&25&23&22&\!\leftarrow
\end{array}
\]
תוצאת הסבב השני של המחיקות (כאשר מתחילים מהמחיקה האחרונה
$39$)
היא:
\[
\begin{array}{rrrrrrrrrrrrrrrrrrrrrrrrrrrr}
\rightarrow\!&\not\!\!1&2&4&\not\!\!5&7&8&\not\!\!10&11&13&\not\!\!14&16&17&\not\!\!19&20&\!\downarrow\\
\uparrow\quad\!&\not\!\!41&40&38&\not\!\!37&35&34&\not\!\!32&31&29&\not\!\!28&26&25&\not\!\!23&22&\!\leftarrow
\end{array}
\]
נמשיך למחוק כל מספר שלישי עד שרק אחד נשאר:
\[
\begin{array}{rrrrrrrrrrrrrrrrrr}
2&4&\not\!7&8&11&\not\!\!13&16&17&\not\!\!20&22&25&\not\!\!26&29&31&\not\!\!34&35&38&\not\!\!40
\\
2&4&\not\!8&11&16&\not\!\!17&22&25&\not\!\!29&31&35&\not\!\!38
\\
2&4&\not\!\!11&16&22&\not\!\!25&31&35
\\
\not\!2&4&16&\not\!\!22&31&35
\\
\not\!4&16&31&\not\!\!35
\\
\not\!\!16&31
\\
31
\end{array}
\]
מכאן ש-%
$J(41,3)=31$.
\end{example}
הקורא מוזמן לבצע את החישוב עבור מחיקת כל מספר שביעי במעגל של 
$40$
ולבדוק שהמספר האחרון הוא
$30$.

\begin{theorem}\label{thm.jo1}
$J(n+1,q)=(J(n,q)+q) \pmod {n+1}$.
\end{theorem}

\begin{proof}
המספר הראשון שנמחק בסבב הראשון הוא מספר ה-%
$q$
והמספרים שנשארים לאחר המחיקה הם 
$n$
המספרים:
\[
\begin{array}{rrrrrrrr}
\;1&\;2&\;\ldots&\;q-1&\;q+1&\;\ldots&\;n&\;n+1 \pmod {n+1}\,.
\end{array}
\]
נמשיך ונחפש את המחיקה הבאה שמתחילה ב-
$q+1$.
העתקה של
$1,\ldots,n$
לסדרה זו נותנת:
\begin{small}
\[
\begin{array}{cccccccccc}
1&\;\; 2&\ldots& n-q&\;\; n+1-q&\;\; n+2-q&\ldots&n-1&\;\; n& \!\!\!\!\!\!\!\!\pmod {n\!+\!1}\\
\downarrow&\;\; \downarrow&&\downarrow&\;\; \downarrow&\;\; \downarrow&&\downarrow&\;\; \downarrow\\
q+1&\;\; q+2&\ldots&n&\;\; n+1&\;\; 1&\ldots&q-2&\;\; q-1& \!\!\!\!\!\pmod {n\!+\!1}\,.
\end{array}
\]
\end{small}
זכרו שכל החישובים הם מודולו
$n+1$:
\[
\begin{array}{lclcl}
(n+2-q)+q&=& (n+1)+1&=& 1 \quad\;\;\pmod {n+1}\\
(n)+q&= &(n+1)-1+q&= &q-1\pmod {n+1}\,.
\end{array}
\]
זאת בעיית יוספוס 
עבור
$n$
מספרים, פרט לעובדה שהמספרים מוזזים ב-%
$q$.
מכאן ש:
\[
J(n+1,q)=(J(n,q)+q) \pmod {n+1}\,.
\vspace{-4ex}
\]
\end{proof}

\begin{theorem}\label{lem.jo}
עבור
$n\geq 1$
קיימים מספרים 
$a\geq 0, 0\leq t < 2^a$
כך ש-%
$n=2^a+t$.
\end{theorem}
\begin{proof}
ניתן להוכיח את המשפט באמצעות אלגוריתם החילוק עם המחלקים 
$2^0, 2^1, 2^2, 2^4,\ldots$,
אבל קל יותר לראות זאת מהייצוג הבינארי של
$n$.
קיימים
$a$
ו-%
$b_{a-1},b_{a-2},\ldots,b_{1},b_{0}$,
כך שעבור כל
$i$, $b_i=0$
או
$b_i=1$,
ניתן לבטא את 
$n$
כ:
\begin{eqn}
n&=&2^a+b_{a-1}2^{a-1}+\cdots+b_{0}2^{0}\\
n&=&2^a+(b_{a-1}2^{a-1}+\cdots+b_{0}2^{0})\\
n&=&2^a+t,\quad \textrm{\R{כאשר}}\; t\leq 2^a-1\,.
\end{eqn}
\end{proof}
כעת נוכיח שקיים ביטוי סגור פשוט עבור
$J(n,2)$. 
\begin{theorem}\label{thm.jo2}
עבור
$n=2^a+t$, $a\geq 0, 0\leq t < 2^a$, $J(n,2)=2t+1$.
\end{theorem}

\begin{proof}
לפי משפט%
~\ref{lem.jo}
ניתן לבטא את
$n$
כפי שרשום שם. ההוכחה ש-%
$J(n,2)=2t+1$
היא על ידי אינדוקציה כפולה, תחילה על 
$a$
ואחר כך על
$t$.

\textbf{אינדוקציה ראשונה}

טענת בסיס: נניח ש-%
$t=0$
כך ש-%
$n=2^a$.
יהי
$a=1$
כך ששני המספרים הראשונים במעגל הם
$1,2$. 
אבל
$q=2$
ולכן המספר השני יימחק והמספר שנשאר הוא
$1$
ומכאן ש-%
$J(2^1,2)=1$.

הנחת האינדוקציה היא 
$J(2^a,2)=1$.
מהו
$J(2^{a+1},2)$?
בסבב הראשון מוחקים את כל המספרים הזוגיים:
\[
\begin{array}{rrrrrrrrrrrrrrrrrrrrrrrrrrrr}
1&\quad\not\! 2&\quad3&\quad\not\! 4& \quad\ldots&\quad 2^{a+1}\!-\!1&\quad \not\! 2^{a+1}\,.
\end{array}
\]
כעת נשארו 
$2^a$
מספרים:
\[
\begin{array}{rrrrrrrrrrrrrrrrrrrrrrrrrrrr}
1&\quad3&\quad\ldots&\quad 2^{a+1}\!-\!1\,.
\end{array}
\]
לפי הנחת האינדוקציה
$J(2^{a+1},2)=J(2^a,2)=1$
ולכן באינדוקציה
$J(n,2)=1$
כאשר
$n=2^a+0$.

\textbf{אינדוקציה שנייה}

הוכחנו ש-%
$J(2^a+0,2)=2\cdot 0 +1$, 
טענת הבסיס של האינדוקציה השנייה.

הנחת האינדוקציה היא
$J(2^a+t,2)=2t+1$.
לפי משפט%
~\ref{thm.jo1}:
\[
J(2^a+(t+1),2)=J(2^a+t,2)+2=2t+1+2=2(t+1)+1\,.
\vspace{-5ex}
\]
\end{proof}

קיים אלגוריתם פשוט לחישוב
$J(n,2)$
המבוסס על משפטים%
~\ref{lem.jo}
ו-%
~\ref{thm.jo2}.
מהוכחת משפט%
~\ref{lem.jo}:
\[
n=2^a+t=2^a+(b_{a-1}2^{a-1}+\cdots+b_{0}2^{0})\,,
\]
כך ש-%
$t=b_{a-1}2^{a-1}+\cdots+b_{0}2^{0}$.
פשוט נכפול ב-%
$2$
(על ידי הזזה שמאלה של ספרה אחת) ונוסיף
$1$.
לדוגמה, 
$n=41=2^5+2^3+2^0=101001$
ולכן
$J(41,2)=2t+1$
ובסימון בינארי:
\begin{eqn}
41&=&101001\\
9&=&001001\\
2t+1&=&010011=16+2+1=19\,.
\end{eqn}
הקורא מוזמן לבדוק את התוצאה על ידי מחיקת כל מספר שני במעגל
$1,\ldots,41$.

קיים ביטוי עבור 
$J(n,3)$
אבל הוא מסובך מאוד.

%%%%%%%%%%%%%%%%%%%%%%%%%%%%%%%%%%%%%%%%%%%%%%%%%%%%%%%%

\subsection*{מה ההפתעה?}

אינדוקציה היא אחת משיטות ההוכחה החשובות ביותר במתמטיקה מודרנית. מספרי פיבונאצ'י
מוכרים מאוד ומספרי פרמה 
קלים להבנה. הופתעתי לגלות נוסחאות רבות כל כך שלא הכרתי (כגון משפטים%
~\ref{thm.fib-div3}
ו-%
\ref{thm.seven-fourths})
הניתנות להוכחה באינדוקציה. פונקציה
$91$
של מקארתי
התגלתה בהקשר של מדעי המחשב למרות שהיא פונקציה מתמטית. מה שמפתיע איננה הפונקציה עצמה אלא האינדוקציה המוזרה כאשר 
$98\prec 97$. 
ההפתעה בבעיית יוספוס
היא באינדוקציה הדו-כיוונית בהוכחה.

\subsection*{מקורות}

ניתן למצוא הצגה נרחבת של אינדוקציה ב-%
\cite{gunderson}.
ההוכחה של פונקציה 
$91$
של מקארתי
נמצאת ב-%
\cite{manna}
שמייחס אותה לבורסטל
\L{(Rod M. Burstall)}.
ההצגה של בעיית יוספוס
מבוססת על פרק~%
$17$
של
\cite{gunderson}
שגם מביא את הרקע ההיסטורי ובעיות מעניינות אחרות, כגון: הילדים המרוחים בבוץ, המטבע המזויף והאגורות בקופסה. חומר נוסף על בעיית יוספוס
ניתן למצוא ב-%
\cite{schumer,wiki:josephus}.
