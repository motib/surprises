% !TeX root = surprises.tex

\chapter{La cuadratura del círculo}\label{c.square}

%%%%%%%%%%%%%%%%%%%%%%%%%%%%%%%%%%%%%%%%%%%%%%%%%%%%%%%%%%%%%%%

La cuadratura del círculo, la construcción de un cuadrado con la misma área que un círculo dado, es uno de los tres problemas de construcción que los griegos se plantearon y no pudieron resolver. A diferencia de la trisección del ángulo y la duplicación del cubo, en las que la imposibilidad se deriva de las propiedades de las raíces de los polinomios, la imposibilidad de cuadrar el círculo se deriva de la trascendentalidad de $\pi$: no es la raíz de ningún polinomio con coeficientes racionales. Este es un teorema difícil que fue demostrado en 1882 por Carl von Lindemann.

Aproximaciones a $\pi\approx 3.14159265359$ se han conocido desde la antigüedad. Algunas aproximaciones sencillas pero razonablemente exactas son:
\[
\displaystyle\frac{22}{7}\approx 3.142857,\quad \displaystyle\frac{333}{106}\approx 3.141509,\quad \displaystyle\frac{355}{113}\approx 3.141593\,.
\]

Presentamos tres construcciones mediante regla y compás de aproximaciones a $\pi$. Una es por Adam Kocha\'{n}ski (Sección.~\ref{s.square-kochanski}) y dos son de Ramanujan (Secciones.~\ref{s.square-ramanujan-first}, \ref{s.square-ramanujan-second}). Sección~\ref{s.square-quad} cómo cuadrar el círculo utilizando la cuadratriz.

La siguiente tabla muestra las fórmulas de las longitudes que se construyen, sus valores aproximados, la diferencia entre estos valores y el valor de $\pi$ y el error en metros que resulta si se utiliza la aproximación para calcular la circunferencia de la tierra dado que su radio es de $6378$ km.
\[
\begin{array}{l@{\hspace{5mm}}c@{\hspace{5mm}}c@{\hspace{5mm}}c@{\hspace{5mm}}r}
\hline\noalign{\smallskip}
\textrm{Construcción} & \textrm{Fórmula} &\textrm{Valor} & \textrm{Diferencia} & \textrm{Error (m)}\\\hline\noalign{\medskip}
\pi & -& 3.14159265359 & - & -\\\noalign{\medskip}
\textrm{Kocha\'{n}ski} & \sqrt{\displaystyle\frac{40}{3}-2\sqrt{3}}&
  3.14153333871 & 5.93 \times 10^{-5} & 757\\\noalign{\medskip}
\textrm{Ramanujan}\; 1 & \displaystyle\frac{355}{113} &
  3.14159292035 &2.67  \times 10^{-7}&3.4\\\noalign{\medskip}
\textrm{Ramanujan}\; 2 &\left(9^2+\displaystyle\frac{19^2}{22}\right)^{1/4}&
  3.14159265258 & 1.01 \times 10^{-9}& 0.013\\
\hline\noalign{\smallskip}
\end{array}
\]

%%%%%%%%%%%%%%%%%%%%%%%%%%%%%%%%%%%%%%%%%%%%%%%%%%%%%%%%%%%%%%%%



\section{Construcción de Kocha\'{n}ski}\label{s.square-kochanski}
\index{Kocha\'{n}ski, Adam}\index{Squaring a circle!approximation}
\textbf{Construcción (Fig.~\ref{f.square-kochansky}):}
\begin{enumerate}
\item Construir una circunferencia unitaria centrada en $O$, que $\overline{AB}$ sea un diámetro y construir una tangente a la circunferencia en $A$.
\item Construir una circunferencia unitaria centrada en $A$ y denotar su intersección con la primera circunferencia por $C$. Construir una circunferencia unitaria centrada en $C$ y denotar su intersección con la segunda circunferencia por $D$. 
\item Construir $\overline{OD}$ y denotar su intersección con la tangente por $E$.
\item A partir de $E$ construir $F,G,H$, cada uno a distancia $1$ del punto anterior.
\item Construir $\overline{BH}$.
\end{enumerate}

\begin{figure}[h]
\begin{center}
\begin{tikzpicture}[scale=.55]
% Scale at 4

% Coordinates of circle
\coordinate (O) at (0,0);
\coordinate (A) at (0,-4);
\coordinate (B) at (0,4);
\node[above right] at (O) {$O$};
\vertex{O};
\node[below right] at (A) {$A$};
\node[above right] at (B) {$B$};
\draw (A) rectangle +(14pt,14pt);

% Draw circle and diameter
\node [draw,circle through=(A),name path=circle] at (O) {};
\draw (A) --  node[right] {$1$} (O) -- node[right] {$1$} (B);

% Draw tangent at A
\draw[name path=tangent] ($(A)+(-4.5,0)$) -- ($(A)+(10.5,0)$);

% Draw arc centered at A which intersects circle at C
\draw[name path=Aarc] (O)
   arc [start angle=90,end angle=220,radius=4];
\path[name intersections={of=circle and Aarc,by=C}];
\node[left,xshift=-4pt] at (C) {$C$};

% Draw arc centered at C which intersects the first arc at D
\draw[name path=Carc] ($(C)+(200:4)$)
   arc [start angle=200,end angle=310,radius=4];
\path[name intersections={of=Carc and Aarc,by=D}];
\node[below left] at (D) {$D$};

% Draw O--D which intersects the tangent at E
\draw[name path=OD] (O) -- (D);
\path[name intersections={of=tangent and OD,by=E}];
\node[above left] at (E) {$E$};

% Find point H at length 3 from E
\coordinate (F) at ($(E)+(4,0)$);
\vertex{F};
\node[above right,xshift=6pt] at (F) {$F$};
\coordinate (G) at ($(F)+(4,0)$);
\vertex{G};
\node[above] at (G) {$G$};
\coordinate (H) at ($(G)+(4,0)$);
\node[above,xshift=2pt] at (H) {$H$};

% Draw BH of length approximately pi
\draw (B) -- (H);

\draw[<->] ($(E)+(0,-.8)$) -- node[fill=white] {$1$} ($(F)+(0,-.8)$);
\draw[<->] ($(F)+(0,-.8)$) -- node[fill=white] {$1$} ($(G)+(0,-.8)$);
\draw[<->] ($(G)+(0,-.8)$) -- node[fill=white] {$1$} ($(H)+(0,-.8)$);

\end{tikzpicture}
\end{center}
\caption{La aproximación de Kocha\'{n}ski a $\pi$}\label{f.square-kochansky}
\end{figure}

\begin{figure}[t]
\begin{center}
\begin{tikzpicture}[rotate=0,scale=1]
% Scale at 4

\clip (-4.5,-6.5) rectangle +(5,7);
% Coordinates of circle
\coordinate (O) at (0,0);
\coordinate (A) at (0,-4);
\coordinate (B) at (0,4);

% Draw circle
\node [circle through=(A),name path=circle] at (O) {};
\draw ($(O)+(200:4)$) arc [start angle=200,end angle=280,radius=4];

% Draw tangent at A
\draw[name path=tangent] ($(A)+(-4.5,0)$) -- ($(A)+(10.5,0)$);
\draw (A) rectangle +(6pt,6pt);

% Draw arc centered at O which intersects circle at C
\draw[name path=Aarc] (O)
   arc [start angle=90,end angle=220,radius=4];
\path[name intersections={of=circle and Aarc,by=C}];

% Draw arc centered at C which intersects the first arc at D
\draw[name path=Carc] ($(C)+(260:4)$)
   arc [start angle=260,end angle=280,radius=4];
\path[name intersections={of=Carc and Aarc,by=D}];

% Draw O--D which intersects the tangent at E
\draw[name path=OD] (O) -- (D);
\path[name intersections={of=tangent and OD,by=E}];

% Find point H at length 3 from E
\coordinate (F) at ($(E)+(4,0)$);
\coordinate (G) at ($(F)+(4,0)$);
\coordinate (H) at ($(G)+(4,0)$);

% Draw BH of length approximately pi
\draw (B) -- (H);

\draw[very thick,dashed] (A) -- (O) -- (C) -- (D) -- cycle;
\draw[very thick,dashed,name path=AC] (A) -- (C);

\node[above left,yshift=10pt,xshift=2pt] at (A) {$60^\circ$};
\node[left,yshift=10pt,xshift=-20pt] at (A) {$30^\circ$};

\path[name intersections={of=AC and OD,by=K}];
\draw[rotate=-30] (K) rectangle +(6pt,6pt);

\path (A) -- node[above,xshift=2pt,yshift=2pt] {$1/2$} (K);
\path (A) -- node[below,xshift=-10pt] {$1/\sqrt{3}$} (E);

\node[above right] at (O) {$O$};
\node[below right] at (A) {$A$};
\node[above right] at (B) {$B$};
\node[left,xshift=-4pt] at (C) {$C$};
\node[below left] at (D) {$D$};
\node[above left] at (E) {$E$};
\node[above] at (H) {$H$};
\node[above,yshift=4pt] at (K) {$K$};
\end{tikzpicture}
\end{center}
\caption[Detalle de la construcción de Kocha\'{n}ski]{Detalle de la construcción de Kocha\'{n}ski}\label{f.square-kochansky-detail}
\end{figure}

\begin{theorem}
$\quad\overline{BH}=\sqrt{\displaystyle\frac{40}{3}-2\sqrt{3}}\approx \pi$.
\end{theorem}
\begin{proof}
La figura ~\ref{f.square-kochansky-detail} es un extracto ampliado de la figura ~\ref{f.square-kochansky}, donde se han añadido segmentos de línea discontinua. Como todos los círculos son círculos unitarios, las longitudes de las líneas discontinuas son $1$. Se deduce que $\overline{AOCD}$ es un rombo por lo que sus diagonales son perpendiculares y se bisecan entre sí en el punto marcado con $K$. $\overline{AK}=1/2$.

La diagonal $\overline{AC}$ forma dos triángulos equiláteros $\triangle OAC, \triangle DAC$ por lo que $\angle OAC=60^\circ$. Como la tangente forma un ángulo recto con el radio $\overline{OA}$, $\angle KAE=30^\circ$. Ahora bien:
\begin{displaymath}
\renewcommand{\arraystretch}{1.5}
\begin{array}{lcl}
\displaystyle\frac{1/2}{\overline{EA}}&=&
\cos 30^\circ=\displaystyle\frac{\sqrt{3}}{2}\\
\overline{EA}&=&\displaystyle\frac{1}{\sqrt{3}}\\
\overline{AH}&=&3-\overline{EA}=\left(3-\displaystyle\frac{1}{\sqrt{3}}\right)
=\displaystyle\frac{3\sqrt{3}-1}{\sqrt{3}}\,.
\end{array}
\end{displaymath}

$\triangle ABH$ es un triángulo rectángulo y $\overline{AH}=3-\overline{EA}$, así que por el Teorema de Pitágoras:
\begin{displaymath}
\renewcommand{\arraystretch}{1.5}
\begin{array}{lcl}
\overline{BH}^2&=&\overline{AB}^2+\overline{AH}^2\\
&=&4+\displaystyle\frac{27 -6\sqrt{3}+1}{3}=\frac{40}{3}-2\sqrt{3}\\
\overline{BH}&=&\rule{0pt}{25pt}\sqrt{\displaystyle\frac{40}{3}-2\sqrt{3}}\approx 3.141533387\approx \pi\,.
\end{array}
\end{displaymath}
\end{proof}

%%%%%%%%%%%%%%%%%%%%%%%%%%%%%%%%%%%%%%%%%%%%%%%%%%%%%%%%%%%%%%



\section{Primera construcción de Ramanujan}\label{s.square-ramanujan-first}
\index{Ramanujan}\index{Squaring a circle!approximation}

\textbf{Construcción (Fig.~\ref{f.square-rama1a}):}
\begin{enumerate}

\item Construir una circunferencia unitaria centrada en $O$ y que $\overline{PR}$ sea un diámetro.

\item Construir el punto $H$ que biseca $\overline{PO}$ y el punto $T$ que triseca $\overline{RO}$ (Thm.~\ref{thm.trisect-constructible}).

\item Construir la perpendicular a $T$ que corta a la circunferencia en $Q$.

\item Construir las cuerdas $\overline{RS}=\overline{QT}$ y $\overline{PS}$.

\item Construir una recta paralela a $\overline{RS}$ desde $T$ que corte a $\overline{PS}$ en $N$.

\item Construir una recta paralela a $\overline{RS}$ desde $O$ que corte a $\overline{PS}$ en $M$.

\item Construir la cuerda $\overline{PK}=\overline{PM}$.

\item Construir la tangente a $P$ de longitud $\overline{PL}=\overline{MN}$.

\item Conectar los puntos $K,L,R$.

\item Encontrar el punto $C$ tal que $\overline{RC}$ sea igual a $\overline{RH}$.

\item Construir $\overline{CD}$ paralela a $\overline{KL}$ que corte a $\overline{LR}$ en $D$. 
\end{enumerate}

\begin{figure}[b]
\begin{center}
\begin{tikzpicture}[scale=.95,align=left]
\clip (-6,-5.1) rectangle +(11.5,10.2);
% Draw circle and horizontal diameter
\draw[name path=circle] (0,0)  coordinate (o) node[below] {$O$} circle[radius=5cm];
\draw (-5,0) coordinate (p) node[left] {$P$} -- (5,0) coordinate (r) node[right] {$R$};
\vertex{o};
\coordinate (h) at (-2.5,0);
\node[below] at (h) {$H$};
\vertex{h};
\coordinate (t) at (10/3,0);
\node[below left] at (t) {$T$};
\path (p) -- node[above,xshift=10pt] {\sm{1/2}} (h) -- node[above] {\sm{1/2}} (o) -- node[above] {\sm{2/3}} (t) -- node[above] {\sm{1/3}} (r);

% Draw perpendicular TQ
\path[name path=tq] (t) -- +(0,5);
\path[name intersections={of=tq and circle,by=q}];
\draw (t) -- (q) node[above] {$Q$};

% Draw chord RS and line PS
\path[name path=tq] (t) -- +(0,5);
\path[name intersections={of=tq and circle,by=q}];
\path[name path=rcirc] (r) let \p1 = ($ (t) - (q) $) in circle ({veclen(\x1,\y1)});
\path[name intersections={of=rcirc and circle,by=s}];
\draw (r) -- (s) node[above right] at (s) {$S$};
\draw[name path=ps] (p) -- (s);

% Draw TN
\path[name path=tn] (t) -- +($(s)-(r)$);
\path[name intersections={of=ps and tn,by=n}];
\draw (t) -- (n) node[above] {$N$};


% Draw OM
\path[name path=om] (o) -- +($(s)-(r)$);
\path[name intersections={of=ps and om,by=m}];
\draw (o) -- (m) node[above left] {$M$};
\path (p) -- (m);
\path (m) -- (n);

% Draw chord PK
\draw (p) -- +(-62.3:4.64) coordinate (k) node[below left] {$K$};

% Draw tangent PL
\draw let \p1 = ($ (m) - (n) $), \n1 = {veclen(\x1,\y1)} in (p) -- (-5,-\n1) coordinate (l) node[left] {$L$};

% Connect L and K to R
\draw (r) -- (l) -- (k) -- cycle;

% Find point C on RK
\coordinate (c) at ($(r)!7.5cm!(k)$);
\path (r) -- (c);
\node[below] at (c) {$C$};

% Draw CD
\path[name path=cd] (c) -- +($(l)-(k)$);
\path[name path=lr] (l) -- (r);
\path[name intersections={of=cd and lr,by=d}];
\draw (c) -- (d);
\node[above,xshift=2pt] at (d) {$D$};
\path (r) -- (d);

\draw[rotate=90] (t) rectangle +(7pt,7pt);
\draw[rotate=-90] (p) rectangle +(7pt,7pt);
\draw[rotate=-160] (s) rectangle +(7pt,7pt);
\draw[rotate=-160] (n) rectangle +(7pt,7pt);
\draw[rotate=-160] (m) rectangle +(7pt,7pt);
\draw[rotate=28] (k) rectangle +(7pt,7pt);
\end{tikzpicture}
\end{center}
\caption{Primera construcción de Ramanujan}\label{f.square-rama1a}
\end{figure}

\begin{theorem}
$\quad\overline{RD}^2=\displaystyle\frac{355}{113}\approx \pi$.
\end{theorem}


\begin{proof}
$\overline{RS}=\overline{QT}$ por construcción y por el Teorema de Pitágoras para $\triangle QOT$:
\[
\overline{RS} =\overline{QT} =\sqrt{1^2-\left(\frac{2}{3}\right)^2}=\frac{\sqrt{5}}{3}\,.
\]
$\angle PSR$ está subtendido por un diámetro por lo que $\triangle PSR$ es un triángulo rectángulo. Por el teorema de Pitágoras:
\[
\overline{PS} = \sqrt{2^2-\left(\frac{\sqrt{5}}{3}\right)^2}=\sqrt{4-\frac{5}{9}}=\frac{\sqrt{31}}{3}\,.
\]
Por construcción $\overline{MO}\parallel \overline{RS}$ entonces
$\triangle MPO\sim \triangle SPR$ y:
%
\begin{eqnarray*}
\frac{\overline{PM}}{\overline{PO}}&=&\frac{\overline{PS}}{\overline{PR}}\\
\frac{\overline{PM}}{1}&=&\frac{\sqrt{31}/3}{2}\\
\overline{PM}&=&\frac{\sqrt{31}}{6}\,.
\end{eqnarray*}
Por construcción $\overline{NT}\parallel \overline{RS}$ entonces 
$\triangle NPT\sim \triangle SPR$ y:
\begin{eqnarray*}
\frac{\overline{PN}}{\overline{PT}}&=&\frac{\overline{PS}}{\overline{PR}}\\
\frac{\overline{PN}}{5/3}&=&\frac{\sqrt{31}/3}{2}\\
\overline{PN}&=&\frac{5\sqrt{31}}{18}\\
\overline{MN}&=&\overline{PN}-\overline{PM}=\sqrt{31}\left(\frac{5}{18}-\frac{1}{6}\right) = \frac{\sqrt{31}}{9}\,.
\end{eqnarray*}
$\triangle PKR$ es un triángulo rectángulo porque $\angle PKR$ está subtendido por un diámetro. Por construcción $\overline{PK}=\overline{PM}$ y por el Teorema de Pitágoras:
\[
\overline{RK}=\sqrt{2^2-\left(\frac{\sqrt{31}}{6}\right)^2} = \frac{\sqrt{113}}{6}\,.
\]
$\triangle LPR$ es un triángulo rectángulo porque $\overline{PL}$ es una tangente por lo que $\angle LPR$ es un ángulo recto. $\overline{PL}=\overline{MN}$ por construcción y por el Teorema de Pitágoras:
\[
\overline{RL}=\sqrt{2^2+\left(\frac{\sqrt{31}}{9}\right)^2} = \frac{\sqrt{355}}{9}\,.
\]
Por construcción
$\overline{RC}=\overline{RH}=3/2$ y $\overline{CD} \parallel \overline{LK}$. Por triángulos semejantes:%
\begin{eqnarray*}
\frac{\overline{RD}}{\overline{RC}}&=&\frac{\overline{RL}}{\overline{RK}}\\
\frac{\overline{RD}}{3/2}&=&\frac{\sqrt{355}/9}{\sqrt{113}/6}\\
\overline{RD}&=&\sqrt{\frac{355}{113}}\\
\overline{RD}^2&=&\frac{355}{113}\approx 3.14159292035\approx \pi\,.
\end{eqnarray*}
En la Fig.~\ref{f.square-rama1b} los segmentos de línea se etiquetan con sus longitudes.
\end{proof}

\begin{figure}[ht]
\begin{center}
\begin{tikzpicture}[scale=.95,align=left]
\clip (-6,-5.1) rectangle +(11.5,10.2);
% Draw circle and horizontal diameter
\draw[name path=circle] (0,0)  coordinate (o) node[below] {$O$} circle[radius=5cm];

\draw (-5,0) coordinate (p) node[left] {$P$} -- (5,0) coordinate (r) node[right] {$R$};
\vertex{o};
\coordinate (h) at (-2.5,0);
\node[below] at (h) {$H$};
\vertex{h};
\coordinate (t) at (10/3,0);
\node[below] at (t) {$T$};

% Draw perpendicular TQ
\path[name path=tq] (t) -- +(0,5);
\path[name intersections={of=tq and circle,by=q}];
\draw (t) -- (q) node[above] {$Q$};


\path (p) -- node[above,xshift=10pt] {\sm{1/2}} (h) -- node[above] {\sm{1/2}} (o) -- node[above] {\sm{2/3}} (t) -- node[above] {\sm{1/3}} (r);
% Draw chord RS and line PS
\path[name path=tq] (t) -- +(0,5);
\path[name intersections={of=tq and circle,by=q}];
\path[name path=rcirc] (r) let \p1 = ($ (t) - (q) $) in circle ({veclen(\x1,\y1)});
\path[name intersections={of=rcirc and circle,by=s}];
\draw (r) -- node[left,xshift=-4pt] {\sm{\sqrt{5}/3}} (s);
\node[above right] at (s) {$S$};
\draw[name path=ps] (p) -- node[above right,xshift=-4pt,yshift=16pt] {\sm{\overline{PS}=\sqrt{31}/3}} (s);
% Draw TN
\path[name path=tn] (t) -- +($(s)-(r)$);
\path[name intersections={of=ps and tn,by=n}];
\draw (t) -- (n) node[above] {$N$};
% Draw OM
\path[name path=om] (o) -- +($(s)-(r)$);
\path[name intersections={of=ps and om,by=m}];
\draw (o) -- (m) node[above left] {$M$};
\path (p) -- node[below,xshift=32pt,yshift=12pt] {\sm{\sqrt{31}/6}} (m);
\path (m) -- node[below,xshift=8pt,yshift=0pt] {\sm{\sqrt{31}/9}} (n);
% Draw chord PK
\draw (p) -- node[right,xshift=-2pt,yshift=10pt] {\sm{\overline{PK}=\sqrt{31}/6}} +(-62.3:4.64) coordinate (k) node[below left] {$K$};
% Draw tangent PL
\draw let \p1 = ($ (m) - (n) $), \n1 = {veclen(\x1,\y1)} in (p) -- node[left] {\sm{\sqrt{31}/9}} (-5,-\n1) coordinate (l) node[left] {$L$};
% Connect L and K to R
\draw (r) -- (l) -- (k) -- cycle;
% Find point C on RK
\coordinate (c) at ($(r)!7.5cm!(k)$);
\path (r) -- node[below,xshift=12pt,yshift=0pt] {\sm{\overline{RC}=3/2}} (c);
\path (r) -- node[below,xshift=0pt,yshift=-12pt] {\sm{\overline{RK}=\sqrt{113}/6}} (k);
\node[below] at (c) {$C$};
% Draw CD
\path[name path=cd] (c) -- +($(l)-(k)$);
\path[name path=lr] (l) -- (r);
\path[name intersections={of=cd and lr,by=d}];
\draw (c) -- (d);
\node[above,xshift=2pt] at (d) {$D$};
\path (r) -- node[above,xshift=-40pt,yshift=-8pt] {\sm{\overline{RD}=\sqrt{355/113}}} (d);
\path (r) -- node[below,xshift=-28pt,yshift=-15pt] {\sm{\overline{RL}=\sqrt{355}/9}} (l);
\draw[rotate=28] (k) rectangle +(7pt,7pt);
\draw[rotate=-160] (s) rectangle +(7pt,7pt);
\draw[rotate=-160] (n) rectangle +(7pt,7pt);
\draw[rotate=-160] (m) rectangle +(7pt,7pt);
\draw[rotate=-90] (p) rectangle +(7pt,7pt);
\end{tikzpicture}
\end{center}
\caption{Primera construcción de Ramanujan con segmentos de recta etiquetados}\label{f.square-rama1b}
\end{figure}

%%%%%%%%%%%%%%%%%%%%%%%%%%%%%%%%%%%%%%%%%%%%%%%%%%%%%%%%%%%%


\section{Segunda construcción de Ramanujan}\label{s.square-ramanujan-second}
\index{Ramanujan}\index{Squaring a circle!approximation}

\textbf{Construcción  (Fig.~\ref{f.square-ramanujan-2a}):}
\begin{enumerate}

\item Construir una circunferencia unitaria centrada en $O$ de diámetro $\overline{AB}$ y que $C$ sea la intersección de la perpendicular a $\overline{AB}$ en $O$ con la circunferencia.

\item Trisecar el segmento de recta $\overline{AO}$ tal que $\overline{AT}=1/3$ y $\overline{TO}=2/3$ (Thm.~\ref{thm.trisect-constructible}).

\item Construir $\overline{BC}$ y encontrar puntos $M,N$ tales que $\overline{CM}=\overline{MN}=\overline{AT}=1/3$.

\item Construir $\overline{AM}$, $\overline{AN}$ y que $P$ sea el punto sobre $\overline{AN}$ tal que $\overline{AP}=\overline{AM}$.

\item A partir de $P$ construir una recta paralela a $\overline{MN}$ que corte a $\overline{AM}$ en $Q$.

\item Construir $\overline{OQ}$ y a partir de $T$ construir una recta paralela a $\overline{OQ}$ que corte a $\overline{AM}$ en $R$.

\item Construir $\overline{AS}$ tangente a $A$ tal que $\overline{AS}=\overline{AR}$.

\item Construir $\overline{SO}$.
\end{enumerate}

\begin{figure}[ht]
\begin{center}
\begin{tikzpicture}[scale=1]
\clip (-4.4,-4.2) rectangle +(8.8,8.6);
% Scale at 4

% Coordinates of circle
\coordinate (O) at (0,0);
\vertex{O};
\coordinate (A) at (-4,0);
\coordinate (B) at (4,0);
\coordinate (C) at (0,4);

% Draw circle and diameter
\node [draw,circle through=(A),name path=circle] at (O) {};
\draw (A) -- (B);
\draw [thick,dashed] (C) -- (O);
\draw (O) rectangle +(7pt,7pt);
\draw[rotate=-90] (A) rectangle +(7pt,7pt);

\coordinate (T) at (-2.667,0);
\path (A) -- node[below] {\sm{1/3}} (T);
\path (T) -- node[below] {\sm{2/3}} (O);
\path (O) -- node[below] {\sm{1}} (B);

\draw (C) -- node[right] {\sm{1/3}} +(-45:1.333) coordinate (M);
\draw (M) -- node[right] {\sm{1/3}} +(-45:1.333) coordinate (N);
\draw (N) -- node[near start,right] {\sm{\sqrt{2}-2/3}}(B);

\draw[name path=AM] (A) -- (M);
\draw[name path=AN] (A) -- (N);

\node [circle through=(M),name path=AMcircle] at (A) {};

\path[name intersections={of=AMcircle and AN,by=P}];

\path[name path=PQ] (P) -- +(135:2);
\path[name intersections={of=PQ and AM,by=Q}];
\draw (P) -- (Q) -- (O);

\path[name path=QT] (T) -- ($(Q)+(-2.667,0)$) -- (Q);
\path[name intersections={of=QT and AM,by=R}];
\draw (T) -- (R);

\node [circle through=(R),name path=ARcircle] at (A) {};
\path[name path=AS] (A) -- ($(A)+(0,-2.5)$);
\path[name intersections={of=ARcircle and AS,by=S}];
\draw (A) -- (S);

\draw (S) -- (O);

\node[below right] at (O) {$O$};
\node[left] at (A) {$A$};
\node[right] at (B) {$B$};
\node[above left,xshift=-8pt] at (B) {\sm{45^\circ}};
\node[above] at (C) {$C$};
\node[below] at (T) {$T$};
\node[right] at (M) {$M$};
\node[right] at (N) {$N$};
\node[below] at (P) {$P$};
\node[above left] at (Q) {$Q$};
\node[above left] at (R) {$R$};
\node[above left] at (S) {$S$};
\end{tikzpicture}
\end{center}
\caption{Segunda construcción de Ramanujan}\label{f.square-ramanujan-2a}
\end{figure}



\begin{theorem}\label{thm.ramanujan2}
$\quad 3\sqrt{\overline{SO}}=\left(9^2+\displaystyle\frac{19^2}{22}\right)^{1/4}\approx \pi$.
\end{theorem}
\begin{proof}
$\triangle COB$ es un triángulo rectángulo así que por el Teorema de Pitágoras $\overline{CB}=\sqrt{2}$ y:
\[
\overline{NB}=\sqrt{2}-2/3\,.
\]
$\triangle COB$ es isoceles por lo que $\angle NBA =\angle MBA=45^\circ$. Por la Ley de los Cosenos\index{Law of cosines}:
\begin{eqnarray*}
\overline{AN}^2&=&\overline{BA}^2 + \overline{BN}^2-2\cdot\overline{BA}\cdot\overline{BN}\cdot\cos \angle NBA\\
&=&2^2+\left(\sqrt{2}-\frac{2}{3}\right)^2-2\cdot 2 \cdot \left(\sqrt{2}-\frac{2}{3}\right)\cdot \frac{\sqrt{2}}{2}
%&=&\left(4+2+\frac{4}{9}-4\right) + \sqrt{2}\cdot \left(-\frac{4}{3}+\frac{4}{3}\right)
=\frac{22}{9}\\
\overline{AN}&=&\sqrt{\frac{22}{9}}\,.
\end{eqnarray*}
De nuevo por la Ley de los Cosenos:
\begin{eqnarray*}
\overline{AM}^2&=&\overline{BA}^2 + \overline{BM}^2-2\cdot\overline{BA}\cdot\overline{BM}\cdot\cos \angle MBA\\
&=&2^2+\left(\sqrt{2}-\frac{1}{3}\right)^2-2\cdot 2 \cdot \left(\sqrt{2}-\frac{1}{3}\right)\cdot \frac{\sqrt{2}}{2}
%&=&\left(4+2+\frac{1}{9}-4\right) + \sqrt{2}\cdot \left(-\frac{2}{3}+\frac{2}{3}\right)
=\frac{19}{9}\\
\overline{AM}&=&\sqrt{\frac{19}{9}}\,.
\end{eqnarray*}
Por construcción $\overline{QP}\parallel \overline{MN}$ entonces
$\triangle MAN\sim \triangle QAP$, y por construcción $\overline{AP}=\overline{AM}$:
\begin{eqnarray*}
\frac{\overline{AQ}}{\overline{AM}}&=&\frac{\overline{AP}}{\overline{AN}}=\frac{\overline{AM}}{\overline{AN}}\\
\overline{AQ}&=&\frac{\overline{AM}^2}{\overline{AN}}=\frac{19/9}{\sqrt{22/9}}=\frac{19}{3\sqrt{22}}\,.
\end{eqnarray*}
Por construcción $\overline{TR}\parallel \overline{OQ}$ entonces
$\triangle RAT\sim \triangle QAO$ y:
\begin{eqnarray*}
\frac{\overline{AR}}{\overline{AQ}}&=&\frac{\overline{AT}}{\overline{AO}}\\
\overline{AR}&=&\overline{AQ}\cdot\frac{\overline{AT}}{\overline{AO}}=\frac{19}{3\sqrt{22}}\cdot\frac{1/3}{1}=\frac{19}{9\sqrt{22}}\,.
\end{eqnarray*}



Por construcción $\overline{AS}=\overline{AR}$ y $\triangle OAS$ es un triángulo rectángulo porque $\overline{AS}$ es una tangente. Por el Teorema de Pitágoras:
\begin{eqnarray*}
\overline{SO}&=&\sqrt{1^2+\left(\frac{19}{9\sqrt{22}}\right)^2}\\
3\sqrt{\overline{SO}}&=&3\left(1^2+\frac{19^2}{9^2\cdot 22}\right)^{1/4}=
\left(9^2+\frac{19^2}{22}\right)^{1/4}\approx 3.14159265258\approx\pi\,.
\end{eqnarray*}

En la Fig.~\ref{f.square-ramanujan-2b} los segmentos de línea están etiquetados con sus longitudes.
\end{proof}


\begin{figure}[ht]
\begin{center}
\begin{tikzpicture}[scale=1]
\clip (-5,-4.2) rectangle +(10,8.6);
% Scale at 4

% Coordinates of circle
\coordinate (O) at (0,0);
\coordinate (A) at (-4,0);
\coordinate (B) at (4,0);
\coordinate (C) at (0,4);

% Draw circle and diameter
\node [draw,circle through=(A),name path=circle] at (O) {};
\draw (A) -- (B);
\draw [thick,dashed] (C) -- (O);
\draw (O) rectangle +(9pt,9pt);
\draw[rotate=-90] (A) rectangle +(9pt,9pt);

\coordinate (T) at (-2.667,0);
\path (A) -- node[below] {\sm{1/3}} (T);
\path (T) -- node[below] {\sm{2/3}} (O);
\path (O) -- node[below] {\sm{1}} (B);

\draw (C) -- node[right] {\sm{1/3}} +(-45:1.333) coordinate (M);
\draw (M) -- node[right] {\sm{1/3}} +(-45:1.333) coordinate (N);
\draw (N) -- node[near start,right] {\sm{\sqrt{2}-2/3}}(B);

\draw[name path=AM] (A) -- node[above,xshift=10pt,yshift=14pt] {\sm{\overline{AM}=\sqrt{19/9}}} (M);
\draw[name path=AN] (A) -- node[below,yshift=-5pt] {\sm{\overline{AN}=\sqrt{22/9}}} (N);

\node [circle through=(M),name path=AMcircle] at (A) {};

\path[name intersections={of=AMcircle and AN,by=P}];

\path[name path=PQ] (P) -- +(135:2);
\path[name intersections={of=PQ and AM,by=Q}];
\draw (P) -- (Q) -- (O);
\path (A) -- node[above,xshift=-10pt,yshift=2pt] {\sm{\overline{AQ}=19/3\sqrt{22}}} (Q);

\path[name path=QT] (T) -- ($(Q)+(-2.667,0)$) -- (Q);
\path[name intersections={of=QT and AM,by=R}];
\draw (T) -- (R);
\path (A) -- node[above,xshift=-6pt,yshift=0pt] {\sm{19/9\sqrt{22}}} (R);

\node [circle through=(R),name path=ARcircle] at (A) {};
\path[name path=AS] (A) -- ($(A)+(0,-2.5)$);
\path[name intersections={of=ARcircle and AS,by=S}];
\draw (A) -- node[xshift=5pt,yshift=4pt] {\sm{\overline{AS}=\;\;\;19/9\sqrt{22}}} (S);

\draw (S) -- node[right,xshift=-2pt,yshift=-6pt] {\sm{\overline{SO}=\sqrt{1+\left(\frac{19^2}{9^2\cdot 22}\right)}}} (O);
\node[below right] at (O) {$O$};
\node[left] at (A) {$A$};
\node[right] at (B) {$B$};
\node[above left,xshift=-8pt] at (B) {\sm{45^\circ}};
\node[above] at (C) {$C$};
\node[below] at (T) {$T$};
\node[right] at (M) {$M$};
\node[right] at (N) {$N$};
\node[below] at (P) {$P$};
\node[above left] at (Q) {$Q$};
\node[above left] at (R) {$R$};
\node[above left] at (S) {$S$};
\end{tikzpicture}
\end{center}
\caption{Segunda construcción de Ramanujan con segmentos de recta etiquetados}\label{f.square-ramanujan-2b}
\end{figure}


%%%%%%%%%%%%%%%%%%%%%%%%%%%%%%%%%%%%%%%%%%%%%%%%%%%%%%%%%%%%%



\section{Cuadratura de un círculo con una cuadratriz}\label{s.square-quad}
\index{Squaring a circle!quadratrix@with a quadratrix}
\index{Quadratrix!squaring a circle}

La cuadratriz se describe en la Secc.

Sea $t=\overline{DE}$ la distancia que se ha desplazado la recta horizontal por el eje $y$, y sea $\theta$ el ángulo correspondiente entre la recta giratoria y el eje $x$. Sea $P$ la posición de la unión de las dos rectas. El lugar geométrico de $P$ es la curva cuadratriz.

Sea $F$ la proyección de $P$ sobre el eje $x$ y sea $G$ la posición de la unión cuando ambas rectas alcanzan el eje $x$, es decir, $G$ es la intersección de la curva cuadratriz y el eje $x$ (Fig.~\ref{f.square-quad}).
\begin{figure}[ht]
\begin{center}
\begin{tikzpicture}[scale=.7,domain=.01:1.57,samples=100]
\draw (0,0) node[below left] {$A$} node [above right,xshift=6pt] {$\theta$} -- (8,0) node[below right] {$B$} -- (8,8) node[above right] {$C$} -- (0,8) node[above left] {$D$} -- cycle;
\draw[name path=horiz] (0,5) -- (8,5);
\draw[name path=slant] (0,0) -- (61:8);
\path[name intersections={of=horiz and slant,by=joint}];
\draw (joint) -- node[right] {$y$} (joint |- 0,0) coordinate (f);
\path (0,0) -- node[below] {$x$} (0,0 -| joint) node[below] {$F$};
\coordinate (e) at (0,5);
\path (e) node[left] {$E$} -- node[left] {$t$} (0,8);
\path (0,0) -- node[left] {$1-t$} (0,5);
\node[above right,xshift=4pt] at (joint) {$P$};
\node[below] at (4.28,0) {$G$};
\draw[name path=curve,thick] plot (4.3*.637*\x,{12.5*.637*\x*cot(\x r)});
\end{tikzpicture}
\end{center}
\caption{Cuadratura del círculo con una cuadratriz}\label{f.square-quad}
\end{figure}

\begin{theorem}
$\quad\overline{AG}=2/\pi$.
\end{theorem}

\begin{proof}
Sea $y=\overline{PF}=\overline{EA}=1-t$. Puesto que en una cuadratriz $\theta$ disminuye al mismo ritmo que $t$ aumenta:
\begin{eqnarray*}
\frac{1-t}{1} &=& \frac{\theta}{\pi/2}\\
&&\\
\theta &=&\frac{\pi}{2}(1-t)\,.
\end{eqnarray*}
Sea $x=\overline{AF}=\overline{EP}$. Entonces $\tan \theta = y/x$ entonces:
\begin{align}\label{eq.quad-square}
x = \frac{y}{\tan\theta}=y\cot\theta=y\cot \frac{\pi}{2}(1-t)=y\cot \frac{\pi}{2}y\,.
\end{align}
Normalmente expresamos una función como $y=f(x)$ pero también se puede expresar como $x=f(y)$. 

Para obtener $x=\overline{AG}$ no podemos simplemente enchufar $y=0$ en la Ec.~\ref{eq.quad-square}, porque $\cot 0$ no está definido, así que vamos a calcular el límite de $x$ a medida que $y$ va a $0$. 
En primer lugar realizar la sustitución $z=(\pi/2)y$ para obtener:
\[
x = y\cot \frac{\pi}{2}y = \frac{2}{\pi} \left(\frac{\pi}{2}y\cot \frac{\pi}{2}y\right)=\frac{2}{\pi}(z\cot z)\,,
\]
y luego tomar el límite:
\[
\lim_{z\rightarrow 0} x=\frac{2}{\pi}\lim_{z\rightarrow 0} (z\cot z) = \frac{2}{\pi}\lim_{z\rightarrow 0} \left(\frac{z\cos z}{\sin z}\right) = \frac{2}{\pi}\lim_{z\rightarrow 0} \left(\frac{\cos z}{(\sin z)/z}\right) = \frac{2}{\pi}\frac{\cos 0}{1} = \frac{2}{\pi}\,,
\]
donde hemos utilizado $\lim_{z\rightarrow 0} (\sin z/z)=1$ (Thm.~\ref{thm.limit-sine-over}).
\end{proof}

\subsection*{¿Cuál es la sorpresa?}

Es sorprendente que se puedan construir aproximaciones tan exactas a $\pi$. Por supuesto, uno no puede dejar de asombrarse ante las ingeniosas construcciones de Ramanujan.

\subsection*{Fuentes}

La construcción de Kocha\'{n}ski aparece en \cite{bold}. Las construcciones de Ramanujan están en \cite{ramanujan1,ramanujan2}. La cuadratura del círculo usando la cuadratriz está en \cite[pp.~48--49]{martin} y \cite{wiki:quad}.
