% !TeX root = surprises.tex

\chapter{Geometric Constructions Using Origami}\label{c.origami-constructions}

%%%%%%%%%%%%%%%%%%%%%%%%%%%%%%%%%%%%%%%%%%%%%%%%%%%%%%%%%%%%%%%

Este capítulo muestra que las construcciones con origami son más poderosas que las construcciones con regla y compás. Damos dos construcciones para triseccionar un ángulo, una por Hisashi Abe (Secc.~\ref{s.abe-trisection}) y la otra por George E. Martin (Sect.~\ref{s.martin-trisection}), dos construcciones para duplicar un cubo, una de Peter Messer (Sect.~\ref{s.messer}) y la otra de Marghareta P. Beloch (Sect.~\ref{s.cube2}), y la construcción de un nonágono, un polinomio regular de nueve lados (Sect.~\ref{s.nonagon}).

\section{Trisección de Abe de un ángulo}\label{s.abe-trisection}
\index{Origami!trisection of an angle}
\index{Trisection of an angle!origami@using origami}

\noindent\textbf{Construcción:}
Dado un ángulo agudo $\angle PQR$, construir $p$, la perpendicular a $\overline{QR}$ en $Q$. Construir $q$, la perpendicular a $p$ que corta a $\overline{PQ}$ en el punto $A$, y construir $r$, la perpendicular a $p$ en $B$ que está a mitad de camino entre $Q$ y $A$. Utilizando el axioma~6 construir el pliegue $l$ que sitúa $A$ en $A'$ sobre $\overline{PQ}$ y $Q$ en $Q'$ sobre $r$. Sea $B'$ el reflejo de $B$ alrededor de $l$. Construir rectas que pasen por $\overline{QB'}$ y $QQ'$ (Fig.~\ref{f.abe1}).
\begin{figure}[ht]
\begin{center}
\begin{tikzpicture}[scale=.8]

% Place points P, Q, R
\coordinate (P) at (60:10cm); %(5,8.67);
\coordinate (Q) at (0,0);
\coordinate (R) at (10,0);
\node[below right] at (P) {$P$};
\node[left]  at (Q) {$Q$};
\node[right] at (R) {$R$};

% Draw PQR
\draw (P)  -- (Q) -- (R);

% Draw perpendicular to QR
\draw (Q) -- node[left,very near end] {$p$} +(0,11);

% Draw parallel to QR and parallel halfway
\coordinate (A) at (0,5);
\coordinate (B) at (0,2.5);
\draw  (A) -- node[above,very near end] {$q$} +(10,0);
\draw  (B) -- node[above,very near end] {$r$} +(10,0);
\node[left] at (A) {$A$};
\node[left] at (B) {$B$};
\path (Q) -- node[left] {$a$} (B) -- node[left] {$a$} (A);
\draw (A) rectangle +(8pt,8pt);
\draw (B) rectangle +(8pt,8pt);
\draw (Q) rectangle +(8pt,8pt);

% Tangent line y = -2.75x + 10.69

% Draw fold
\coordinate (D) at (0,10.69);
\coordinate (fold-x) at (3.89,0);
\coordinate (AP) at (3.65,6.33);
\coordinate (QP) at (6.87,2.5);
\coordinate (BP) at (5.26,4.42);
\node[left] at (D) {$D$};
\node[above,yshift=6pt] at (AP) {$A'$};
\node[above,yshift=6pt] at (QP) {$Q'$};
\node[above,xshift=2pt,yshift=2pt] at (BP) {$B'$};
\draw [very thick,dashed] (D) -- node[left,near start] {$l$} ($(D)!1.03!(fold-x)$);

% Draw line of reflections
\draw (D) -- (QP);

% Draw trisecting lines
\draw (Q) -- ($(Q)!1.3!(QP)$);
\draw (Q) -- ($(Q)!1.3!(BP)$);

% Complete triangle
\draw (A) -- (QP);

\draw[->,very thick,dotted,bend left=40] ($(A)+(.1,.1)$) to ($(AP)+(-.1,0)$);
\draw[->,dotted,very thick,bend right=30] ($(Q)+(.1,-.1)$) to ($(QP)+(0,-.1)$);
\end{tikzpicture}
\end{center}
\caption{Trisección de Abe de un ángulo}\label{f.abe1}
\end{figure}

\begin{theorem} $\angle PQB'=\angle B'QQ'=\angle Q'QR=\angle PQR/3$.
\end{theorem}

\begin{proof}
\mbox{}\\
(1)
$A', B', Q'$ son reflexiones en torno a la recta $l$ de los puntos $A,B,Q$ sobre la recta $\overline{DQ}$, por lo que están sobre la recta $\overline{DQ'}$ reflejada. Por construcción $\overline{AB}=\overline{BQ}$, $\angle ABQ'=\angle QBQ'=90^\circ$ y $\overline{BQ'}$ es un lado común, por lo que $\triangle ABQ'\cong \triangle QBQ'$ por lado-ángulo-lado. Por lo tanto, $\angle AQQ'=\angle QAQ'=\alpha$ por lo que $\triangle AQ'Q$ es isoceles (Fig.~\ref{f.abe2}).

\begin{figure}[ht]
\begin{center}
\begin{tikzpicture}[scale=.8]

% Place points P, Q, R
\coordinate (P) at (60:10cm);
\coordinate (Q) at (0,0);
\coordinate (R) at (10,0);
\node[below right] at (P) {$P$};
\node[left,xshift=-4pt] at (Q) {$Q$};
\node[right] at (R) {$R$};

% Draw PQR
\draw (Q) -- (R);

% Draw perpendicular to QR
\draw (Q) -- node[left,very near end] {$p$} +(0,11);

% Draw parallel to QR and parallel halfway
\coordinate (A) at (0,5);
\coordinate (B) at (0,2.5);
\draw (A) -- node[above,very near end] {$q$} +(10,0);
\draw[name path=Br] (B) -- node[above,very near end] {$r$} +(10,0);
\node[left,xshift=-4pt] at (A) {$A$};
\node[left,xshift=-4pt] at (B) {$B$};
\path (Q) -- node[left,xshift=-4pt] {$a$} (B) -- node[left,xshift=-4pt] {$a$} (A);
\draw (A) rectangle +(8pt,8pt);
\draw (B) rectangle +(8pt,8pt);
\draw (Q) rectangle +(8pt,8pt);

% Tangent line y = -2.75x + 10.69

% Draw fold
\coordinate (D) at (0,10.69);
\coordinate (fold-x) at (3.89,0);
\coordinate (AP) at (3.65,6.33);
\coordinate (QP) at (6.87,2.5);
\coordinate (BP) at (5.26,4.42);
\node[left] at (D) {$D$};
\node[above,yshift=6pt] at (AP)  {$A'$};
\node[above,xshift=2pt,yshift=6pt] at (QP) {$Q'$};
\node[above,xshift=4pt,yshift=2pt] at (BP) {$B'$};
\draw[name path=fold,very thick,dashed] (D) -- node[left,near start] {$l$}  ($(D)!1.03!(fold-x)$);
	
% Draw line of reflections
\draw (D) -- (AP);

% Draw trisecting lines
\draw (Q) -- ($(Q)!1.3!(BP)$);

\draw [very thick,loosely dash dot,red] (Q) -- (QP);
\draw [very thick,loosely dash dot,red] (QP) -- (AP);
\draw [very thick,loosely dash dot,red] (AP) -- (Q);
\draw [very thick,loosely dash dot dot,blue] ($(Q)+(0,-4pt)$) -- ($(QP)+(0,-4pt)$);
\draw [very thick,dash dot dot,blue] ($(QP)+(0,-4pt)$) -- ($(A)+(0,-4pt)$);
\draw [very thick,dash dot dot,blue] ($(A)+(-4pt,0)$) -- ($(Q)+(-4pt,0)$);

\draw (A) -- (AP);

\node[left,xshift=-40pt,yshift=7pt] at (QP) {$\alpha$};
\node[left,xshift=-40pt,yshift=-6pt] at (QP) {$\alpha$};
\node[right,xshift=40pt,yshift=6pt] at (Q) {$\alpha$};
\node[right,xshift=40pt,yshift=28pt] at (Q) {$\alpha$};
\node[right,xshift=30pt,yshift=42pt] at (Q) {$\alpha$};

\draw (AP) -- (P);

\path[name path=Qr] (Q) -- (QP);
\path[name intersections = {of = fold and Qr, by = {U}}];
\node[above left,xshift=-2pt,yshift=-2pt] at (U) {$U$};
\draw[rotate=20] (U) rectangle +(8pt,8pt);
\path[name intersections = {of = fold and Br, by = {V}}];
\node[above left,xshift=-2pt,yshift=-2pt] at (V)  {$V$};
\end{tikzpicture}
\end{center}
\caption{Pruebas de la trisección de Abe ($U,V$ se utilizan en la prueba 2)}\label{f.abe2}
\end{figure}

Por reflexión $\triangle AQ'Q\cong \triangle A'QQ'$, por lo que $\triangle A'QQ'$ es también un triángulo isóceles.
$\overline{QB'}$, el reflejo de $\overline{Q'B}$, es la mediatriz de un triángulo isóceles, por lo que $\angle A'QB'=\angle Q'QB'=\angle QQ'B=\alpha$.
Alternando los ángulos interiores, $\angle Q'QR=\angle QQ'B=\alpha$.
En conjunto tenemos:
\[
\triangle PQB'=\angle A'QB'=\angle B'QQ'=\angle Q'QR=\alpha\,.
\]
\end{proof}

\begin{proof}
\mbox{}\\
(2)
Como $l$ es un pliegue es la mediatriz de $\overline{QQ'}$. Denotemos la intersección de $l$ con $\overline{QQ'}$ por $U$ y su intersección con $\overline{QB'}$ por $V$ (Fig.~\ref{f.abe2}). Por tanto, $\triangle VUQ\cong \triangle VUQ'$ por lado-ángulo-lado ya que $\overline{VU}$ es un lado común, los ángulos en $U$ son ángulos rectos y $\overline{QU}=\overline{Q'U}$. Por tanto, $\angle VQU=\angle VQ'U=\angle alfa$ y $\angle Q'QR=\angle VQ'U=\angle alfa$ por alternancia de ángulos interiores.

Como en la primera demostración $A', B', Q'$ son todas reflexiones alrededor de $l$, por lo que están sobre la recta $\overline{DQ'}$ y $\overline{A'B'}=\overline{AB}=\overline{BQ}=\overline{B'Q'}=a$. Entonces $\triangle A'B'Q\cong\triangle Q'B'Q$ por lado-ángulo-lado y $\angle A'QB'=\angle Q'QB'=\alpha$.
\end{proof}

\section{Trisección de un ángulo de Martin}\label{s.martin-trisection}
\index{Origami!trisection of an angle}
\index{Trisection of an angle!origami@using origami}

\noindent\textbf{Construcción:}
Dado un ángulo agudo $\angle PQR$, sea $M$ el punto medio de $\overline{PQ}$. Construir $p$ la perpendicular a $\overline{QR}$ por $M$ y construir $q$ la perpendicular a $p$ por $M$ de modo que $q\parallel\overline{QR}$. Utilizando el axioma 6 construir el pliegue $l$ que sitúe $P$ en $P'$ sobre $p$ y $Q$ en $Q'$ sobre $q$. Si más de un pliegue es posible elegir el que interseca $\overline{PM}$. Construir $\overline{PP'}$ y $\overline{QQ'}$  (Fig.~\ref{f.martin}).

\begin{theorem}
$\angle Q'QR=\angle PQR/3$.
\end{theorem}
\begin{proof}
Denotemos la intersección de $\overline{QQ'}$ con $p$ por $U$ y su intersección con $l$ por $V$. Denotemos la intersección de $\overline{PQ}$ y $\overline{P'Q'}$ con $l$ por $W$. No es inmediato que $\overline{PQ}$ y $\overline{P'Q'}$ intersecan $l$ en el mismo punto. Pero $\triangle PWP' \sim \triangle QWQ'$ por lo que las altitudes bisecan ambos ángulos verticales $\angle PWP', \angle QWQ'$ y deben estar en la misma recta.

Por lo tanto, $\triangle QMU\cong \triangle PMP'$ por ángulo-lado-ángulo ya que $\angle P'PM=\angle UQM=\beta$ por ángulos interiores alternos, $\overline{QM}=\overline{MP}=a$ porque $M$ es el punto medio de $\overline{PQ}$ y $\angle QMU=\angle PMP'=\gamma$ son ángulos verticales. Por tanto, $\overline{P'M}=\overline{MU}=b$.

$\triangle P'MQ'\cong \triangle UMQ'$ por lado-ángulo-lado, ya que $\overline{P'M}=\overline{MU}=b$, los ángulos en $M$ son ángulos rectos y $\overline{MQ'}$ es un lado común. Como la altitud del triángulo isóceles $\triangle P'Q'U$ es la bisectriz de $\angle P'Q'U$, resulta que $\angle P'Q'M=\angle UQ'M=\alpha$. Además, $\angle UQ'M=\angle Q'QR=\angle alfa$ por ángulos interiores alternos. $\triangle QWV\cong\triangle Q'WV$ por lado-ángulo-lado ya que $\overline{QV}=\overline{VQ'}=c$, los ángulos en $V$ son ángulos rectos y $\overline{VW}$ es un lado común. Por tanto:
\begin{eqnarray*}
\angle WQV&=&\beta=\angle WQ'V=2\alpha\\
\angle PQR &=& \beta + \alpha = 3\alpha\,.
\end{eqnarray*}
\end{proof}

\begin{figure}[t]
\begin{center}
\begin{tikzpicture}[scale=.8]

% Place points P, Q, R
\coordinate (P) at (60:10cm); %(5,8.67);
\coordinate (Q) at (0,0);
\coordinate (R) at (10,0);
\node[below right] at (P) {$P$};
\node[above left] at (Q) {$Q$};
\node[right] at (R) {$R$};

% Draw PQR
\draw (R)  -- (Q);
\draw [name path=pq] (Q) -- (P);

% M is the midpoint of PQ
\coordinate (M) at (2.5, 4.33);
\node[above left,xshift=2pt] at (M) {$M$};
\draw [rotate=-90] (M) rectangle +(9pt,9pt);

% Drop a perpendicular from M to QR and extend the line upwards
% This is the given line p
\coordinate (pQR) at (M |- Q);
\draw [name path=p] (pQR) --
   node[left, very near end,yshift=20pt] {$p$}
   ($(pQR)!2!(M)$);
\draw (pQR) rectangle +(9pt,9pt);

% Construct q perpendicular to p through M
\draw [name path=q] ($(M)+(-2,0)$) --
   node[above, very near start,xshift=-30pt] {$q$}
   ($(M)+(10,0)$);

% Construct the fold line t
% Its equation is y = -2.75x + 18.51, as obtained from Geogebra
\coordinate (t1) at (6.7,.085);
\coordinate (t2) at (3.5,8.89);
\draw [very thick,dashed,name path=t] (t1) --
   node[very near end,left] {$l$}
   (t2);

% Construct a perpendicular to t through P
\coordinate (perp-p) at ($(t1)!(P)!(t2)$);
\path [name path=perp-p] (P) -- ($(P)!2.5!(perp-p)$);

% Get its intersection with t denoted Pt
% and its intersection with p named PP
\path [name intersections = {of = t and perp-p, by = {Pt}}];
\path [name intersections = {of = p and perp-p, by = {PP}}];
\node[left] at (PP) {$P'$};
\draw [rotate=22] (Pt) rectangle +(9pt,9pt);

% Draw PT
\draw (P) -- (PP);

% Construct a perpendicular to t through Q
\coordinate (perp-q) at ($(t1)!(Q)!(t2)$);
\path[name path=perp-q] (Q) -- ($(Q)!2.1!(perp-q)$);

% Get its intersection with t denoted V
% and its intersection with q denoted S=Q'
\path [name intersections = {of = t and perp-q, by = {V}}];
\path [name intersections = {of = q and perp-q, by = {QP}}];
\node[above,yshift=4pt] at (QP) {$Q'$};
\node[above left,xshift=-4pt,yshift=-2pt] at (V) {$V$};
\draw [rotate=22] (V) rectangle +(9pt,9pt);

% Draw Q QP
\draw [name path=qs] (Q) -- (QP);

% Get the intersection of QS with p denoted U
\path [name intersections = {of = p and qs, by = {U}}];
\node[above left] at (U) {$U$};

% Draw PP QP
\draw [name path=ts] (PP) -- (QP);

% Get its intersection with QP denoted W
\path [name intersections = {of = ts and pq, by = {W}}];
\node[right,xshift=4pt,yshift=4pt] at (W) {$W$};

% Label line segments
\path (P) -- node[left] {$a$} (M);
\path (M) -- node[left]  {$a$} (Q);
\path (PP) -- node[left]  {$b$} (M);
\path (M) -- node[right] {$b$} (U);
\path (Q) -- node[below,near end] {$c$} (V);
\path (V) -- node[below] {$c$} (QP);

% Label angles
\node [xshift=5pt,yshift=20pt]        at (M) {$\gamma$};
\node [xshift=-5pt,yshift=-20pt]      at (M) {$\gamma$};
\node [xshift=15pt,yshift=13pt]       at (Q) {$\beta$};
\node [xshift=-10pt,yshift=-10pt]     at (P) {$\beta$};
\node [left,xshift=-30pt,yshift=7pt]  at (QP) {$\alpha$};
\node [left,xshift=-30pt,yshift=-7pt] at (QP) {$\alpha$};
\node [right,xshift=25pt,yshift=5pt]  at (Q) {$\alpha$};
\end{tikzpicture}
\end{center}
\caption{Trisección de Martin de un ángulo}\label{f.martin}
\end{figure}

\section{Duplicación de un cubo de Messer}\label{s.messer}
\index{Doubling a cube!origami@using origami}
\index{Origami!doubling a cube}

Un cubo de volumen $V$ tiene lados de longitud $\sqrt[3]{V}$. Un cubo con el doble de volumen tiene caras de longitud $\sqrt[3]{2 V}=\sqrt[3]{2}\sqrt[3]{V}$, por lo que si podemos construir $\sqrt[3]{2}$ podemos multiplicar por la longitud dada $\sqrt[3]{V}$ para duplicar el cubo.

\noindent\textbf{Construcción:}
Divide el lado de un cuadrado unitario en tercios de la siguiente manera: Dobla el cuadrado por la mitad para localizar los puntos $I=(0,1/2)$ y $J=(1,1/2)$. A continuación se construyen las rectas $\overline{AC}$ y $\overline{BJ}$ (Fig.~\ref{f.messer1}). El punto de intersección $K=(2/3,1/3)$ se obtiene resolviendo las dos ecuaciones $y=1-x$ y $y=x/2$.

Construimos $\overline{EF}$, la perpendicular a $\overline{AB}$ que pasa por $K$, y construimos la reflexión $\overline{GH}$ de $\overline{BC}$ alrededor de $\overline{EF}$. Ahora el lado del cuadrado se ha dividido en tercios.

\begin{figure}[t]
\begin{center}
\begin{tikzpicture}[scale=.55]
% Draw square
\coordinate (A) at (0,12);
\coordinate (B) at (0,0);
\coordinate (C) at (12,0);
\coordinate (D) at (12,12);

\node[left]  at (A) {$A=(0,1)$};
\node[left]  at (B) {$B=(0,0)$};
\node[right] at (C) {$C=(1,0)$};
\node[right] at (D) {$D=(1,1)$};

\draw [thick] (A)  -- (B) -- (C) -- (D) -- cycle;

% Divide a side in half

\coordinate (M)  at (0,6);
\coordinate (N) at (12,6);
\node[left] at (M) {$I=(0,1/2)$};
\node[right] at (N) {$J=(1,1/2)$};
\draw [thick,dashed] (M) -- (N);


\draw [very thick,dotted,name path=ac] (A) -- 
   node[near start,above,xshift=24pt] {$y=1-x$} (C);
\draw [very thick,dotted,name path=be2] (B) -- 
   node[near start,above,xshift=-12pt] {$y=x/2$} (N);

\path [name intersections = {of = ac and be2, by = {I}}];
\node[below,xshift=-6pt,yshift=-8pt] at (I) {$K=$};
\node[below,xshift=-6pt,yshift=-20pt] at (I) {$(2/3,1/3)$};

\coordinate (E)  at (0,4);
\coordinate (F) at (12,4);
\node[left] at (E) {$E=(0,1/3)$};
\node[right] at (F) {$F=(1,1/3)$};
\draw [thick,dashed] (E) -- (F);

\coordinate (G)  at (0,8);
\coordinate (H) at (12,8);
\node[left] at (G) {$G=(0,2/3)$};
\node[right] at (H) {$H=(1,2/3)$};
\draw (G) -- (H);
\end{tikzpicture}
\end{center}
\caption{Dividir una longitud en tercios}\label{f.messer1}
\end{figure}

Utilizando el axioma~6 situar $C$ en $C'$ de $\overline{AB}$ y $F$ en $F'$ de $\overline{GH}$.  Denotemos por $L$ el punto de intersección del pliegue con $\overline{BC}$ y por $b$ la longitud de $\overline{BL}$. Renombrar la longitud del lado del cuadrado por $a+1$ donde $a=\overline{AC'}$. La longitud de $\overline{LC}$ es $(a+1)-b$.  (Fig.~\ref{f.messer3}).

\begin{theorem}
$\overline{AC'}=\sqrt[3]{2}$.
\end{theorem}

\begin{proof}
Al realizar el pliegue el segmento de recta $\overline{LC}$ se refleja en el segmento de recta $\overline{LC'}$ y $\overline{CF}$ se pliega en el segmento de recta $\overline{C'F'}$. Por lo tanto:
\begin{align}
\overline{GC'}=a-\frac{a+1}{3}=\frac{2a-1}{3}\,.\label{eq.one-third}
\end{align}
Como $\angle FCL$ es un ángulo recto, también lo es $\angle F'C'L$.

\begin{figure}[t]
\begin{center}
\begin{tikzpicture}[scale=.65]
% Draw and label square
\coordinate (A) at (0,12);
\coordinate (B) at (0,0);
\coordinate (C) at (12,0);
\coordinate (D) at (12,12);
\node[left]  at (A) {$A$};
\node[left]  at (B) {$B$};
\node[right] at (C) {$C$};
\node[right] at (D) {$D$};
\draw (B) rectangle +(9pt,9pt);
\draw[rotate=90] (C) rectangle +(9pt,9pt);
\draw [thick] (A)  -- (B) -- (C) -- (D) -- cycle;

% Draw line one-third from botton
\coordinate (E)  at (0,4);
\coordinate (F) at (12,4);
\node[left] at (E) {$E$};
\node[right] at (F) {$F$};
\draw [name path=ef] (E) -- (F);

% Draw line two-thirds from bottom
\coordinate (G)  at (0,8);
\coordinate (H) at (12,8);
\node[left] at (G) {$G$};
\node[right] at (H) {$H$};
\draw[rotate=-90] (G) rectangle +(9pt,9pt);
\draw (G) -- (H);

% Draw reflections of C and F
\coordinate (CP) at (0,5.31);
\coordinate (FP) at (2.96,8);
\node[left] at (CP) {$C'$};
\node[above right,yshift=8pt] at (CP) {$\alpha$};
\node[below right,xshift=-2pt,yshift=-12pt] at (CP) {$\alpha'$};
\node[above] at (FP) {$F'$};
\node[below left,xshift=-8pt] at (FP) {$\alpha'$};
\draw[rotate=-50] (CP) rectangle +(9pt,9pt);
\draw (CP) -- (FP);

% Draw fold and fold arrows
% Tangent is y = 2.26x - 10.9
% Crosses x axis at (4.83,0)
\coordinate (J) at (4.83,0);
\node[below] at (J) {$L$};
\node[above left,xshift=-8pt] at (J) {$\alpha$};
\draw [very thick,dashed,name path=jd] (J) -- node[very near end,left] {$l$} (10,12);
\draw[thick,dotted,bend right=40,->] (C) to ($(CP)+(4pt,0)$);
\draw[thick,dotted,bend right=40,->] (F) to ($(FP)+(4pt,4pt)$);

% Draw hypotenuses of right triangles
\draw (CP) -- (J);
\path (J)  -- (C);

% Labels on BC and hypotenuses
\path (CP) -- node[right] {$(a+1)-b$} (J);
\path (J)  -- node[below] {$(a+1)-b$} (C);
\path (B)  -- node[below] {$b$} (J);
\path (C)  -- node[right] {$\displaystyle\frac{a+1}{3}$} (F);
\path (CP) -- node[right,xshift=10pt] {$\displaystyle\frac{a+1}{3}$} (FP);

% Labels on AB
\draw[<->] ($(A)+(-1,0)$)    --
  node[fill=white] {$a$} ($(CP)+(-1,0)$);
\draw[<->] ($(CP)+(-1,0)$)   --
  node[fill=white] {$1$} ($(B)+(-1,0)$);
\draw[<->] ($(CP)+(-2.5,0)$) --
  node[fill=white] {$\displaystyle\frac{2a-1}{3}$} ($(G)+(-2.5,0)$);
\draw[<->] ($(A)+(-2.5,0)$) --
  node[fill=white] {$\displaystyle\frac{a+1}{3}$} ($(G)+(-2.5,0)$);
\end{tikzpicture}
\end{center}

\caption{Construcción de $\sqrt[3]{2}$}\label{f.messer3}
\end{figure}

$\triangle C'BL$ es un triángulo rectángulo así que por el Teorema de Pitágoras:
\begin{subeqnarray}
1^2 + b^2 &=& ((a+1)-b)^2\\
%&=& a^2+2a+1 - 2(a+1)b + b^2\\
%a^2+2a - 2(a+1)b&=&0\\
b&=&\frac{a^2+2a}{2(a+1)}\,.\slabel{eq.value-of-b}
\end{subeqnarray}

$\angle GC'F' + \angle F'C'L + \angle LC'B = 180^\circ$ ya que forman la recta $\overline{GB}$. Denotemos $\angle GC'F'$ por $\alpha$. Entonces:
\[
\angle LC'B=180^\circ - \angle F'C'L - \angle GC'F'= 180^\circ - 90^\circ - \alpha = 90^\circ -\alpha\,,
\]
que denotamos por $\alpha'$. Los triángulos $\triangle C'BL$, $\triangle F'GC'$ son triángulos rectángulos por lo que $\angle C'LB=\alpha$ y $\angle C'F'G=\alpha'$. Por lo tanto, $\triangle C'BL\sim\triangle F'GC'$ y:
\[
\frac{\overline{BL}}{\overline{C'L}}=\frac{\overline{GC'}}{\overline{C'F'}}\,.
\]
Usando la Ec.~\ref{eq.one-third} tenemos:
\[
\frac{b}{(a+1)-b}=\frac{\displaystyle\frac{2a-1}{3}}{\displaystyle\frac{a+1}{3}}\,.
\]
Sustituyendo por $b$ usando la Ec.~\ref{eq.value-of-b} da:
\[
\displaystyle\frac{\displaystyle\frac{a^2+2a}{2(a+1)}}{(a+1)-\displaystyle\frac{a^2+2a}{2(a+1)}}=\frac{2a-1}{a+1}\,.
\]
Simplifica la ecuación para obtener $a^3=2$ y $a=\sqrt[3]{2}$.
\end{proof}

\section{La duplicación de un cubo de Beloch}\label{s.cube2}
\index{Doubling a cube!origami@using origami}
\index{Origami!doubling a cube}

Dado que el pliegue de Beloch (Axioma~6) puede resolver ecuaciones cúbicas, es razonable conjeturar que puede utilizarse para doblar un cubo. Aquí damos una construcción directa que utiliza el pliegue.

\noindent\textbf{Construcción:}
Sea $A=(-1,0)$, $B=(0,-2)$. Sea $p$ la recta $x=1$ y sea $q$ la recta $y=2$. Utilizar el pliegue de Beloch para construir el pliegue $l$ que sitúa $A$ en $A'$ sobre $p$ y $B$ en $B'$ sobre $q$. Denotemos la intersección del pliegue y el eje $y$ por $Y$ y la intersección del pliegue y el eje $x$ por $X$. (Fig.~\ref{f.beloch-doubling}).

\begin{figure}[b]
\begin{center}
\begin{tikzpicture}[scale=.8]
% Draw and label square
\coordinate (O) at (0,0);
\coordinate (A) at (-2,0);
\coordinate (B) at (0,-4);
\node[below left,xshift=-7pt] at (O) {$O$};
\node[below left,yshift=-12pt] at (O) {$(0,0)$};
\node[above left,xshift=-7pt] at (A) {$A$};
\node[below left,xshift=2pt,yshift=0pt] at (A) {$(-1,0)$};
\node[above right,xshift=10pt] at (A) {$\alpha$};
\node[left,xshift=-12pt] at (B) {$B$};
\node[left,yshift=-12pt] at (B) {$(0,-2)$};
\node[above right,yshift=12pt] at (B) {$\alpha'$};

\draw[thick] (0,-4.5) --  node[very near end,above left,yshift=12pt] {$y$-axis} +(0,10);
\draw[thick] (-5,0)   -- node[very near start,above left] {$x$-axis} +(12,0);
\draw[thick] (2,-4.5) -- node[very near start, right,yshift=-10pt] {$p\!:x=1$} +(0,10);
\draw[thick] (-5,4) -- node[very near start, above,xshift=-16pt] {$q\!: y=2$} +(12,0);

\coordinate (AP) at (2,5);
\node[above right] at (AP) {$A'$};
\coordinate (BP) at (6.34,4);
\node[above right] at (BP) {$B'$};

% Tangent y = -0.8x + 1.26

% Exchanged X and Y 
\coordinate (X) at (0,2.52);
\coordinate (Y) at (3.15,0);
\node[right,xshift=4pt,yshift=2pt] at (X) {$Y$};
\node[below right,yshift=-14pt] at (X) {$\alpha$};
\node[below left,xshift=2pt,yshift=-12pt] at (X) {$\alpha'$};
\node[above right,xshift=10pt] at (Y) {$X$};
\node[below left,xshift=-10pt] at (Y) {$\alpha$};
\node[above left,xshift=-13pt] at (Y) {$\alpha'$};
\draw [very thick,dashed] ($(X)!-1.1!(Y)$) -- node[very near end,right,xshift=8pt] {$l$} ($(X)!2!(Y)$);

\draw [very thick,dotted] (A) -- (AP);
\draw [very thick,dotted] (B) -- (BP);

\draw[thick,dotted,bend left=40,->] (A) to ($(AP)+(-4pt,0)$);
\draw[thick,dotted,bend left=40,->] (B) to ($(BP)+(-6pt,-3pt)$);

\draw[rotate=-130] (X) rectangle +(10pt,10pt);
\draw[rotate=-130] (Y) rectangle +(10pt,10pt);
\end{tikzpicture}
\end{center}
\caption{La duplicación del cubo de Beloch}\label{f.beloch-doubling}
\end{figure}

\begin{theorem}
$\overline{OY}=\sqrt[3]{2}$.
\end{theorem}
\begin{proof}
El pliegue es la mediatriz de $\overline{AA'}$ y $\overline{BB'}$ por lo que $\overline{AA'}\parallel\overline{BB'}$. Por ángulos interiores alternos $\angle YAO =\angle BXO=\alpha$. El etiquetado de los otros ángulos en la figura se sigue de las propiedades de los triángulos rectángulos.

$\triangle AOY\sim \triangle YOX \sim \triangle XOB$ y $\overline{OA}=1$, $\overline{OB}=2$ se dan así:
\[
\begin{array}{l}
\displaystyle\frac{\overline{OY}}{\overline{OA}}=\displaystyle\frac{\overline{OX}}{\overline{OY}}=\displaystyle\frac{\overline{OB}}{\overline{OX}}\\
\\
\displaystyle\frac{\overline{OY}}{1}=\displaystyle\frac{\overline{OX}}{\overline{OY}}=\displaystyle\frac{2}{\overline{OX}}\,.
\end{array}
\]
De la primera y segunda relaciones tenemos $\overline{OX}=\overline{OY}^2$ y de la primera y tercera relaciones tenemos $\overline{OY}\:\overline{OX}=2$.
Sustituyendo por $\overline{OX}$ se obtiene $\overline{OY}^3=2$ y
$\overline{OY}=\sqrt[3]{2}$.
\end{proof}

\section{Construcción de un nonágono regular}\label{s.nonagon}

Se construye un nonágono (polígono regular de nueve lados) derivando la ecuación cúbica de su ángulo central y resolviendo la ecuación mediante el método de Lill y el pliegue de Beloch. El ángulo central es $\theta=360^\circ/9=40^\circ$. Por Thm.~\ref{thm.triple-angle}:
\[
\cos 3\theta=4\cos^3 \theta -3\cos\theta\,.
\]
Sea $x=\cos 40^{\circ}$. Entonces para el nonágono la ecuación es $4x^3-3x+(1/2)=0$ ya que $\cos 3\cdot 40^\circ=\cos 120^\circ=-(1/2)$. La figura~\ref{f.nonagon2} muestra las trayectorias para la ecuación construida según el método de Lill.

\begin{figure}[ht]
\begin{center}
\begin{tikzpicture}[scale=.85]
% Draw help lines and axes
\draw[step=10mm,white!60!black] (-1,-4) grid (9,1);
\draw[thick] (-1,0) -- (9,0);
\draw[thick] (0,-4) -- (0,1);
\foreach \x in {1,...,9}
  \node at (\x-.3,.3) {\sm{\x}};
\foreach \y in {-3,...,1}
  \node at (-.3,\y-.3) {\sm{\y}};
  
% Points of first path
\coordinate (A) at (0,0);
\coordinate (B) at (4,0);
\coordinate (C) at (7,0);
\coordinate (D) at (7,-.5);
\node[above left] at (A) {$P$};
\node[below right,xshift=12pt] at (A) {$37.45^\circ$};
\node[below right] at (D) {$Q$};

% Draw first path
\draw[very thick,-{Stealth[scale=1.4,inset=2pt]}] 
  (A) -- node[below,xshift=6pt] {$a_3$} (B);
\draw[{Stealth[scale=1.4,inset=2pt,reversed]}-,very thick]
  (B) -- ($(B)+(0,.1)$);
\draw[name path=c,very thick,{Stealth[scale=1.4,inset=2pt]}-]
  (B) -- node[below] {$a_1$} (C);
\draw[very thick,-{Stealth[scale=1.4,inset=2pt]}]
  (C) -- node[right,yshift=-2pt] {$a_0$} (D);

% Draw extension of second segment of first path
\draw[thick,name path=b] 
  ($(B)+(0,-4)$) -- node[right] {$a_2$} ($(B)+(0,1)$);

% Draw second path
\path[name path=one] (A) -- +(-37.45:6cm);
\path [name intersections = {of = b and one, by = {R}}];
\node[below left] at (R) {$R$};
\draw[thick,dashed] (A) -- (R);

\path[name path=two] (R) -- +(52.5463:6cm);
\path [name intersections = {of = c and two, by = {S}}];
\node[above] at (S) {$S$};
\draw[thick,dashed] (R) -- (S);

\draw[thick,dashed] (S) -- (D);

% Draw right angle rectangles
\draw[thick,rotate=52.5463] (R) rectangle +(9pt,9pt);
\draw[thick,rotate=-127.4537] (S) rectangle +(9pt,9pt);
\end{tikzpicture}
\end{center}
\caption{Método de Lill para un nonágono}\label{f.nonagon2}
\end{figure}

La segunda trayectoria parte de $P$ con un ángulo aproximado de $-37,45^\circ$. Giros de $90^\circ$ en $R$ y luego de $-90^\circ$ en $S$ hacen que la trayectoria intersecte a la primera en su punto final $Q$. Por tanto, $x=-\tan (-37,45^\circ)\approx 0,766$ es una raíz de $$4x^3-3x+(1/2)$$.

La raíz se puede obtener utilizando el pliegue de Beloch. Construir la recta $a_2'$ paralela a $a_2$ a la misma distancia de $a_2$ que $a_2$ de $P$. Aunque la longitud de $a_2$ es cero, sigue teniendo dirección (hacia arriba), por lo que se puede construir la recta paralela. Análogamente, se construye la recta $a_1'$ paralela a $a_1$ a la misma distancia de $a_1$ que $a_1$ de $Q$. El pliegue de Beloch $\overline{RS}$ sitúa simultáneamente $P$ en $P'$ sobre $a_2'$ y $Q$ en $Q'$ sobre $a_1'$. Esto construye el ángulo $\angle SPR=-37.45^\circ$ (Fig.~\ref{f.nonagon3}).

\begin{figure}[ht]
\begin{center}
\begin{tikzpicture}[scale=.85]
% Draw help lines and axes
\draw[step=10mm,white!60!black] (-1,-7) grid (9,1);
\draw[thick] (-1,0) -- (9,0);
\draw[thick] (0,-7) -- (0,1);
\foreach \x in {1,...,9}
  \node at (\x-.3,.3) {\sm{\x}};
\foreach \y in {-6,...,1}
  \node at (-.3,\y-.3) {\sm{\y}};
  
% Points of first path
\coordinate (A) at (0,0);
\coordinate (B) at (4,0);
\coordinate (C) at (7,0);
\coordinate (D) at (7,-.5);
\node[above right] at (A) {$P$};
\node[below right] at (D) {$Q$};

% Draw first path
\draw[very thick,-{Stealth[scale=1.4,inset=2pt]}] 
  (A) -- node[below] {$a_3$} (B);
\draw[{Stealth[scale=1.4,inset=2pt,reversed]}-,very thick]
  (B) -- ($(B)+(0,.1)$);
\draw[name path=c,very thick,{Stealth[scale=1.4,inset=2pt]}-]
  (B) -- node[below] {$a_1$} (C);
\draw[very thick,-{Stealth[scale=1.4,inset=2pt]}]
  (C) -- node[right,yshift=-2pt] {$a_0$} (D);

% Draw extension of second segment of first path
\draw[very thick,loosely dotted,name path=b] 
  ($(B)+(0,-7)$) -- node[right,near end] {$a_2$} ($(B)+(0,1)$);

% Draw second path
\path[name path=one] (A) -- +(-37.45:6cm);
\path [name intersections = {of = b and one, by = {R}}];
\node[below left] at (R) {$R$};
\path[name path=two] (R) -- +(52.55:6cm);
\path [name intersections = {of = c and two, by = {S}}];
\node[above] at (S) {$S$};
\draw[very thick,dashed] (R) -- (S);

% Draw parallel lines
\draw[thick,name path=para-2] 
  (8,1) -- node[right,yshift=8pt] {$a_2'$} (8,-7);
\draw[thick,name path=para-1] 
  (-1,.5) -- node[right,xshift=44mm] {$a_1'$} (9,.5);

% Draw second segments of the folds
\path[name path=p-two] (A) -- +(-37.45:11cm);
\path [name intersections = {of = para-2 and p-two, by = {PP}}];
\node[below left] at (PP) {$P'$};
\draw[very thick,dotted] (A) -- (PP);

\path[name path=p-one] (D) -- +(142.55:2cm);
\path [name intersections = {of = para-1 and p-one, by = {QP}}];
\node[above] at (QP) {$Q'$};
\draw[very thick,dotted] (D) -- (QP);

% Draw right angle indications
\draw[thick,rotate=-37.45] (R) rectangle +(9pt,9pt);
\draw[thick,rotate=-127.4537] (S) rectangle +(9pt,9pt);
\end{tikzpicture}
\end{center}
\caption{El pliegue de Beloch para resolver la ecuación del nonágono}\label{f.nonagon3}
\end{figure}

Por el método de Lill $-\tan (-37,45^\circ)\approx 0,766$ y por tanto $\cos \theta \approx 0,766$ es una raíz de la ecuación del ángulo central $\theta$. Terminamos la construcción del nonágono construyendo $\cos^{-1} 0,766\approx 40^\circ$.

El triángulo rectángulo $\triangle ABC$ con $\angle CAB\approx 37,45^\circ$ y $\overline{AB}=1$ tiene lado opuesto $\overline{BC}\approx 0,766$ por definición de tangente (Fig.~\ref{f.nonagon5-eq}).
Doblar $\overline{CB}$ sobre la $\overline{AB}$ de modo que el reflejo de $C$ sea $D$ y $\overline{DB}=0,766$. Extender $\overline{DB}$ y construir $E$ de modo que $\overline{DE}=1$.
Doblar $\overline{DE}$ para reflejar $E$ en $F$ en la extensión de $\overline{BC}$ (Fig.~\ref{f.nonagon5-central}). Entonces:
\[
\angle BDF=\cos^{-1} \frac{0.766}{1}\approx 40^\circ\,.
\]

\begin{figure}[ht]
\begin{minipage}{.45\textwidth}
\begin{center}
\begin{tikzpicture}[scale=1]
\draw (0,0) coordinate (A) -- (4,0) coordinate (B);
\draw (B) -- node[right] {$0.766$} 
  ($(B)+(0,0.766*4)$) coordinate (C);
\draw (A) -- (C);
\draw[rotate=90] (B) rectangle +(8pt,8pt);
\node[above left] at (A) {$A$};
\node[above right] at (B) {$B$};
\node[right] at (C) {$C$};
\coordinate (D) at (0.234*4,0);
\node[below left] at (D)  {$D$};
\coordinate (E) at ($(D)+(4,0)$);
\draw (D) -- (B);
\draw (B) -- (E);
\node[above right] at (E) {$E$};
\draw[very thick,dotted,->,bend right=50] ($(C)+(-.2,0)$) to ($(A)+(.94,.4)$);
\draw[<->] ($(D)+(0,-1.2)$) -- node[fill=white] {$1$} ($(E)+(0,-1.2)$);
\draw[<->] ($(A)+(0,-.8)$) -- node[fill=white] {$1$} ($(B)+(0,-.8)$);
\node[above right,xshift=14pt] at (A) {$37.45^\circ$};
\vertex{D};
\vertex{E};
\end{tikzpicture}
\caption{La tangente que es la solución de la ecuación del nonágono}\label{f.nonagon5-eq}
\end{center}
\end{minipage}
\hfill
\begin{minipage}{.50\textwidth}
\begin{center}
\begin{tikzpicture}[scale=1]
\coordinate (B) at (4,0);
\draw (B) -- ($(B)+(0,0.766*4)$) coordinate (C);
\draw[rotate=90] (B) rectangle +(8pt,8pt);
\node[above right] at (B) {$B$};
\node[right] at (C) {$C$};
\coordinate (D) at (0.234*4,0);
\node[above left] at (D) {$D$};
\node[above right,xshift=8pt,yshift=4pt] at (D) {$40^\circ$};
\coordinate (E) at ($(D)+(4,0)$);
\draw (D) -- node[fill=white] {$0.766$} (B);
\draw (B) -- (E);
\node[above right,xshift=4pt] at (E) {$E$};
\coordinate (F) at ($(B)+(0,4)$);
\draw[very thick,dotted,->,bend right=50] ($(E)+(.1,.2)$) to ($(F)+(.2,0)$);
\draw (B) -- (F);
\node[left] at (F) {$F$};
\draw (D) -- node[fill=white] {$1$} (F);
\draw[<->] ($(D)+(0,-.8)$) -- node[fill=white] {$1$} ($(E)+(0,-.8)$);
\vertex{C};
\vertex{E};
\coordinate (A) at (0,0) node [above left] {$A$};
\draw (A) -- (D);
\vertex{A};
\vertex{D};
\end{tikzpicture}
\caption{El coseno del ángulo central del nonágono}\label{f.nonagon5-central}
\end{center}
\end{minipage}
\end{figure}

\subsection*{¿Cuál es la sorpresa?}

Vimos en los capítulos~\ref{c.trisect} y~\ref{c.square} que herramientas como el neusis pueden realizar construcciones que no se pueden hacer con una regla y un compás. Es sorprendente que la trisección de un ángulo y la duplicación de un cubo puedan construirse utilizando sólo papel plegado. Roger C. Alperin\index{Alperin, Roger C.} ha desarrollado una jerarquía de cuatro métodos de construcción cada uno más potente que el anterior.

\subsection*{Fuentes}

Este capítulo se basa en \cite{alperin,lang,martin,newton}.
