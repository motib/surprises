% !TeX root = surprises.tex

\maketitle

\foreword

\begin{flushright}
\parbox{7cm}{
\begin{footnotesize}
\begin{flushright}
If everyone were exposed to mathematics in its natural state, with all the challenging fun and surprises that that entails, I think we would see a dramatic change both in the attitude of students toward mathematics, and in our conception of what it means to be ``good at math.''\\
Paul Lockhart\mbox{}\\\mbox{}\\
I'm really hungry for surprises because each one makes us ever-so-slightly but substantially smarter.\\
Tadashi Tokieda
\end{flushright}
\end{footnotesize}
}
\end{flushright}

\medskip

Mathematics, when appropriately approached, can provide us with plentiful pleasant surprises. This is confirmed by a Google search of ``mathematical surprises,'' which, surprisingly, yields almost half a billion items. What is a surprise? The origins of the word trace back to Old French with roots in Latin: ``sur'' (over) and ``prendre'' (to take, to grasp, to seize). Literally, to surprise is to overtake. As a noun, surprise is both an unanticipated or bewildering event or circumstance, as well as the emotion caused by it.

Consider, for example, an extract from a lecture by Maxim Bruckheimer\footnote{Maxim Bruckheimer was a mathematician who was one of the founders of the Open University UK and Dean of its Faculty of Mathematics. He was Head of the Department of Science Teaching at the Weizmann Institute of Science.} on the Feuerbach circle: ``Two points lie on one and only one straight line, this is no surprise. However, three points are not necessarily on one straight line and if, during a geometrical exploration, three points `fall into' a straight line, this is a surprise and frequently we need to refer to this fact as a theorem to be proven. Any three points not on a straight line lie on one circle. However, if four points lie on the same circle, this is a surprise that should be formulated as a theorem. \ldots{} Insofar as the number of points on a straight line is larger than $3$, so is the theorem the more surprising. Likewise, insofar as the number of points lying on one circle is larger than $4$, so is the theorem the more surprising. Thus, the statement that for any triangle there are nine related points on the same circle … is very surprising. Moreover, in spite of the magnitude of the surprise, its proof is elegant and easy.''

In this book Mordechai Ben-Ari offers a rich collection of mathematical surprises, most of them less well known than the Feuerbach Circle and with sound reasons for including them. First, in spite of being absent from textbooks, the mathematical gems of this book are accessible with just a high school background (and patience, and paper and pencil, since fun does not come for free). Second, when a mathematical result challenges what we take for granted, we are indeed surprised (Chaps.~\ref{c.collapse}, \ref{c.compass}). Similarly, we are surprised by: the cleverness of an argument (Chaps.~\ref{c.trisect}, \ref{c.square}), the justification of the possibility of a geometric construction by algebraic means (Chap.~\ref{c.heptadecagon}), a proof relying on an apparently unrelated topic (Chaps.~\ref{c.five}, \ref{c.museum}), a strange proof by induction (Chap.~\ref{c.induction}), new ways of looking at a well-known result (Chap.~\ref{c.quadratic}), a seemingly minor theorem becoming the foundation of a whole field of mathematics (Chap.~\ref{c.ramsey}), unexpected sources of inspiration (Chap.~\ref{c.langford}), rich formalizations emerging from purely recreational activities such as origami (Chaps.~\ref{c.origami-axioms}--\ref{c.origami-constructions}). These are all different reasons for the inclusion of the pleasant, beautiful and memorable mathematical surprises in this lovely book.
   
So far I have addressed how the book relates to the first part of the definition of surprise, the cognitive rational reasons for the unexpected. As to the second aspect, the emotional aspect, this book is a vivid instantiation of what many mathematicians claim regarding the primary reason for doing mathematics: it is fascinating! Moreover, they claim that mathematics stimulates both our intellectual curiosity and our esthetic sensibilities, and that solving a problem or understanding a concept provides a spiritual reward, which entices us to keep working on more problems and concepts. 

It has been said that the function of a foreword tell readers why they should read the book. I have tried to accomplish this, but I believe that the fuller answer will come from you, the reader, after reading it and experiencing what the etymology of the word surprise suggests: to be overtaken by it!



\vspace{\baselineskip}
\begin{flushright}
\textit{Abraham Arcavi}
\end{flushright}


\preface

Godfried Toussaint's article on the ``collapsing compass''  \cite{toussaint} made a profound impression on me. It would never have occurred to me that the modern compass with a friction joint is not the one used in Euclid's day. In this book I present a selection of mathematical results that are not only interesting, but that surprised me when I first encountered them.

The mathematics required to read the book is secondary-school mathematics, but that does not mean that the material is simple. Some of the proofs are quite long and require that the reader be willing to persevere in studying the material. The reward is understanding of some of the most beautiful results in mathematics. The book is not a textbook, because the wide range of topics covered doesn't fit neatly into a syllabus. It is appropriate for enrichment activities for secondary-school students, for college-level seminars and for mathematics teachers.

 The chapters can be read independently. (An exception is that Chap.~\ref{c.origami-axioms} on the axioms of origami is a prerequisite for Chaps.~\ref{c.origami-cube},~\ref{c.origami-constructions}, the other chapters on origami.) Notes relevant to all chapters are given below in list labeled Style.

\subsection*{What Is a Surprise?}

There were three criteria for including a topic in the book:
\begin{itemize}
\item The theorem surprised me. Particularly surprising were the theorems on constructibility with a straightedge and compass. The extremely rich mathematics of origami was almost shocking: when a mathematics teacher proposed a project on origami, I initially turned her down because I doubted that there could be any serious mathematics associated with the art form.
Other topics were included because although I knew the results, their proofs were surprising in their elegance and accessibility, in particular, Gauss's purely algebraic proof that a regular heptadecagon can be constructed.

\item The material does not appear in secondary-school and college textbooks, and I have had to read advanced textbooks and the research literature. There are Wikipedia articles on most of the topics, but you have to know where to look and the articles are often outlines.

\item The theorems and proofs are accessible with a good knowledge of secondary-school mathematics.
\end{itemize}
Each chapter concludes with a paragraph \textit{What Is the Surprise?} which explains my choice of the topic.

\subsection*{An Overview of the Contents}

Chapter~\ref{c.collapse} presents Euclid's proof that any construction that is possible with a fixed compass is possible with a collapsing compass. Many proofs have been given, but, as Toussaint shows, most are incorrect because they depend on diagrams which do not always correctly depict the geometry. To emphasize that one must not trust diagrams, I present the famous alleged proof that every triangle is isoceles. 

Over the centuries mathematicians unsuccessfully sought to trisect an arbitrary angle (divide it into three equal parts) using only a straightedge and compass. Underwood Dudley made a comprehensive study of trisectors who find incorrect constructions; most constructions are approximations that are claimed to be accurate. Chapter~\ref{c.trisect} starts by presenting two of these constructions and developing the trigonometric formulas showing that they are only approximations. To show that trisection using just a straightedge and compass is of no practical importance, trisections using more complex tools are presented, Archimedes's \emph{neusis} and Hippias's \emph{quadratrix}. The chapter ends with a proof that it is impossible to trisect an arbitrary angle with a straightedge and compass. 

Squaring a circle (given a circle construct a square with the same area) cannot be performed using a straightedge and compass because the value of $\pi$ cannot be constructed. Chapter~\ref{c.square} presents three elegant constructions of close approximations to $\pi$, one by Kocha\'{n}ski and two by Ramanujan. The chapter concludes by showing that a quadratrix can be used to square a circle.

The four-color theorem states that it is possible to color any planar map with four colors, such that no countries with a common boundary are colored with the same color. The proof of this theorem is extremely complicated, but the proof of the five-color theorem is elementary and elegant, as shown in Chapter~\ref{c.five}. The chapter also presents Percy Heawood's demonstration that Alfred Kempe's ``proof'' of the four-color theorem is incorrect.

How many guards must be employed by an art museum so that all the walls are under constant observation by at least one guard? The proof in Chapter~\ref{c.museum} is quite clever, using graph coloring to solve what at first sight appears to be a purely geometrical problem.

Chapter~\ref{c.induction} presents some lesser-known results and their proofs by induction: theorems on Fibonacci numbers and Fermat numbers, McCarthy's $91$ function, and the Josephus problem.

Chapter~\ref{c.quadratic} presents Po-Shen Loh's method of solving quadratic equations. The method is a critical element of Gauss's algebraic proof that a heptadecagon can be constructed (Chapter~\ref{c.heptadecagon}). The chapter includes al-Khwarizmi's geometric construction for finding roots of quadratic equations and a geometric construction used by Cardano in the development of the formula for finding roots of cubic equations.

Ramsey theory is a topic in combinatorics that is an active area of research. It looks for patterns among subsets of large sets. Chapter~\ref{c.ramsey} presents simple examples of Schur triples, Pythagorean triples, Ramsey numbers and van der Waerden's problem. The proof of the theorem on Pythagorean triples was accomplished recently with the aid of a computer program based on mathematical logic. The chapter concludes with a digression on the ancient Babylonians' knowledge of Pythagorean triples.

C. Dudley Langford observed his son playing with colored blocks and noticed that he had laid them out in an interesting sequence. Chapter~\ref{c.langford} presents his theorem on the conditions for such a sequence to be possible.

Chapter~\ref{c.origami-axioms} contains the seven axioms of origami, together with the detailed calculations of the analytic geometry of the axioms, and characterizations of the folds as geometric loci.

Chapter~\ref{c.origami-cube} presents Eduard Lill's method and the origami fold proposed by Margharita P. Beloch. I introduce Lill's method as a magic trick, so I won't spoil it by giving details here.

Chapter~\ref{c.origami-constructions} shows that origami can perfom constructions not possible with a straightedge and compass: trisecting an angle, squaring a circle and constructing a nonagon (a regular polygon with nine sides).

Chapter~\ref{c.compass} presents the theorem by Georg Mohr and Lorenzo Mascheroni that any construction with a straightedge and compass can be performed using only a compass.

The corresponding claim that a straightedge only is sufficient is  incorrect, because a straightedge cannot compute lengths that are square roots. Jean-Victor Poncelet conjectured and Jakob Steiner proved that a straightedge is sufficient provided that there exists a single fixed circle somewhere in the plane (Chap.~\ref{c.straightedge}).

If two triangles have the same perimeter and the same area, must they be congruent? That seems reasonable but it turns out not to be true, although it takes quite a bit of algebra and geometry to find a non-congruent pair as shown in Chap.~\ref{c.congruent}.

Chapter~\ref{c.heptadecagon} presents Gauss's tour-de-force: a proof that a heptadecagon (a regular polygon with seventeen sides) can be constructed using a straightedge and compass. By a clever argument on the symmetry of the roots of polynomials, he obtained a formula that uses only arithmetic operators and square roots. Gauss did not give an explicit construction of a heptadecagon, so the elegant construction by James Callagy is presented. The chapter concludes with constructions of a regular pentagon based on Gauss's method for the construction of a heptadecagon.

To keep the book as self-contained as possible, Appendix~\ref{a.trig} collects proofs of theorems of geometry and trigonometry that may not be familiar to the reader.

\subsection*{Style}

\begin{itemize}
\item The reader is assumed to have a good knowledge of secondary-school mathematics, including:
\begin{itemize}
\item Algebra: polynomials and division of polynomials $(x^2-3x+2)/(x-2)=x-1$, \emph{monic} polynomials such as $x^2-3x+2$ where the coefficient of the highest power is $1$, quadratic equations, multiplication of expressions with exponents $a^m\cdot a^n=a^{m+n}$.
\item Euclidean geometry: congruent triangles $\triangle ABC \cong \triangle DEF$ and the criteria for congruence, similar triangles $\triangle ABC \sim \triangle DEF$ and the ratios of their sides, circles and their inscribed and central angles.
\item Analytic geometry: the cartesian plane, computing lengths and slopes of line segments $m=(y_2-y_1)/(x_2-x_1)$, the formula for a circle.
\item Trigonometry: the functions $\sin,\cos,\tan$ and the conversions between them, angles in the unit circle, the trigonometric functions of angles reflected around an axis such as $\cos (180^\circ-\theta)=-\cos\theta$.
\end{itemize}
\item Statements to be proved are called \emph{theorems} with no attempt to distinguish between theorems, lemmas and corollaries.
\item When a theorem follows a construction, the variables that appear in the theorem refer to labeled points, lines, angles, etc. in the figure accompanying the construction.
\item The full names of mathematicians have been given without biographical information that can be found easily in Wikipedia.
\item The book is written so that it is as self-contained as possible, but occasionally the presentation depends on advanced mathematical concepts and theorems that are given without proofs. In such cases, a summary of the material is presented in boxes which may be skipped.
\item There are no exercises, but the ambitious reader is invited to prove each theorem before reading the proof.
\item Geometric constructions can be studied using software such as Geogebra.
\item $\overline{AB}$ is used both for the name of a line segment and for the length of the segment.
\item $\triangle ABC$ is used both for the name a triangle and for the area of the triangle.
\end{itemize}


\subsection*{Acknowledgments}

This book would never have been written without the encouragement of Abraham Arcavi who welcomed me to trespass on his turf of mathematics education. He also graciously wrote the foreword. Avital Elbaum Cohen and Ronit Ben-Bassat Levy were always willing to help me (re-)learn secondary-school mathematics. Oriah Ben-Lulu introduced me to the mathematics of origami and collaborated on the proofs. I am grateful to Michael Woltermann for permission to use several sections of his reworking of Heinrich D\"{o}rrie's book. Jason Cooper, Richard Kruel, Abraham Arcavi and the anonymous reviewers provided helpful comments.

I would like to thank the team at Springer for their support and encouragement, in particular the editor Richard Kruel.

The book is published under the Open Access program and I would like to thank the Weizmann Institute of Science for funding the publication. The \LaTeX{} source files for the book (which include the Ti\textit{k}Z source for the diagrams) are available at:
\begin{center}
\url{https://github.com/motib/surprises}
\end{center}

\medskip

\begin{flushright}
\textit{Mordechai (Moti) Ben-Ari}
\end{flushright}

\tableofcontents

