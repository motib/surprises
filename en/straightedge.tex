% !TeX root = surprises.tex

\chapter{A Straightedge and One Circle is Sufficient}\label{c.straightedge}

%%%%%%%%%%%%%%%%%%%%%%%%%%%%%%%%%%%%%%%%%%%%%%%%%%%%%%%%%%%%%%%

\abstract*{Can every construction with a straightedge and compass be done with only a straightedge? The answer is no because lines represent linear equations and cannot represent quadratic equations like circles. In 1822 Jean-Victor Poncelet conjectured that a straightedge only is sufficient, provided that one circle exists in the plane. This chapter presents the proof by Jakob Steiner from 1833.}

%%%%%%%%%%%%%%%%%%%%%%%%%%%%%%%%%%%%%%%%%%%%%%%%%%%%%%%%%%%%%%%

Can every construction with a straightedge and compass be done with only a straightedge? The answer is no because lines are defined by linear equations and cannot represent circles which are defined by quadratic equations. In 1822 Jean-Victor Poncelet\index{Poncelet, Jean-Victor} conjectured that a straightedge only is sufficient provided that \emph{one circle} exists in the plane. This was proved in 1833 by Jakob Steiner.\index{Steiner, Jakob}

After explaining in Sect.~\ref{s.se-what} what is meant by performing a construction with only a straightedge and one circle, the proof is presented in stages starting with five auxiliary constructions: construction of a line parallel to a given line (Sect.~\ref{s.parallel}), construction of a perpendicular to a given line (Sect.~\ref{s.perp}), copying a line segment in a given direction (Sect.~\ref{s.copy}), construction of a line segment as a ratio of segments (Sect.~\ref{s.relative}) and construction of a square root (Sect.~\ref{s.root}). Section~\ref{s.line-circle-straight} shows how to find the intersection(s) of a line with a circle and Sect.~\ref{s.two-circles} shows how to find the intersection(s) of two circles.

\section{What Is a Construction With Only a Straightedge?}\label{s.se-what}
A construction using a straightedge and compass is a sequence of three operations:
\begin{itemize}
\item Find the point of intersection of two lines.
\item Find the point(s) of intersection of a line and a circle.
\item Find the point(s) of intersection of two circles.
\end{itemize}
The first operation can be performed with a straightedge only.

A circle is defined by a point $O$, its \emph{center}, and by a  \emph{radius} $r$, a line segment of length $r$ one of whose endpoints is the center. If we can construct the points labeled $X$ and $Y$ in Fig.~\ref{f.se-only-line-circle} we can claim to have successfully constructed the points of intersection of a given circle with a given line. Similarly, the construction of $X,Y$ in Fig.~\ref{f.se-only-two-circles} is the construction of the points of intersection of two given circles. The circles drawn with dashed lines in a diagram do not actually appear in a construction; they are just used to help understand the construction.

The single given circle used in the constructions, called the \emph{fixed circle}, can appear anywhere in the plane and can have an arbitrary radius.

\begin{figure}[t]
\subfigures
\leftfigure[c]{
\begin{tikzpicture}[scale=.8]
\coordinate (O) at (0,0);
\node[above right] at (O) {$O$};
\draw[thick,dashed,name path=circle] (0,0) circle[radius=2cm];
\draw (0,0) -- node[left] {$r$} ++(-60:2cm);
\draw[name path=line] (-3,-.5) -- ++(20:6cm);
\path [name intersections={of=circle and line,by={X,Y}}];
\node[above right,xshift=-2pt,yshift=4pt] at (X) {$X$};
\node[above left] at (Y) {$Y$};
\vertex{O};
%\vertex{X};
%\vertex{Y};
\end{tikzpicture}
}
\hfill
\rightfigure[c]{
\begin{tikzpicture}[scale=.8]
\coordinate (O1) at (0,0);
\coordinate (O2) at (3,0);
\node[above right] at (O1) {$O_1$};
\node[above right] at (O2) {$O_2$};
\draw[thick,dashed,name path=circle1] (0,0) circle[radius=2cm];
\draw[thick,dashed,name path=circle2] (3,0) circle[radius=1.7cm];
\draw (0,0) -- node[left] {$r_1$} ++(-60:2cm);
\draw (3,0) -- node[left,below] {$r_2$} ++(-20:1.7cm);
\path [name intersections={of=circle1 and circle2,by={X,Y}}];
\node[above,yshift=4pt] at (X) {$X$};
\node[below,yshift=-4pt] at (Y) {$Y$};
\vertex{O1};
\vertex{O2};
%\vertex{X};
%\vertex{Y};
\end{tikzpicture}
}
\leftcaption{$X,Y$ are the points of intersection of a line and a circle}\label{f.se-only-line-circle}
\rightcaption{$X,Y$ are the points of intersections of two circles}\label{f.se-only-two-circles}
\end{figure}

\section{Construction of a Line Parallel to a Given Line}\label{s.parallel}

\begin{theorem}\label{thm.straight-parallel}
Given a line $l$ defined by two points $A,B$ and a point $P$ not on the line, it is possible to construct a line through $P$ that is parallel to $\overline{AB}$.
\end{theorem}

\begin{proof}

There are two cases to the proof.

\textit{Case 1:}
$\overline{AB}$ is a \emph{directed line segment} if the midpoint $M$ of $\overline{AB}$ is given.  Construct a ray that extends $\overline{AP}$ and choose any point $S$ on the ray beyond $P$. Construct the lines $\overline{BP}, \overline{SM}, \overline{SB}$. The intersection of $\overline{BP}$ and $\overline{SM}$ is denoted $O$. Construct a ray that extends $\overline{AO}$ and denote by $Q$ the intersection of the ray $\overline{AO}$ with $\overline{SB}$ (Fig.~\ref{f.se-parallel-directed}).

We claim that $\overline{PQ}\parallel \overline{AB}$. 

\begin{figure}[ht]
\begin{center}
\begin{tikzpicture}
\draw[name path=pq] (-4,0) -- (4,0);
\draw (-2,-2) node[below left] {$A$} coordinate (A) -- (2,-2) node[below right] {$B$} coordinate (B);
\draw[name path=as] (A) -- ++(50:4cm) node[above] {$S$} coordinate (S);
\draw[name path=sb] (S) -- (B);
\path [name intersections={of=pq and as,by={P}}];
\path [name intersections={of=pq and sb,by={Q}}];
\node[above left] at (P) {$P$};
\node[above right] at (Q) {$Q$};
\draw[name path=pb] (P) -- (B);
\draw[name path=qa] (Q) -- (A);
\path [name intersections={of=pb and qa,by={O}}];
\node[right,xshift=2pt] at (O) {$O$};
\coordinate (M) at (0,-2);
\node[below right] at (M) {$M$};
\draw (S) -- (M);
%\vertex{O};
%\vertex{P};
%\vertex{A};
%\vertex{B};
\end{tikzpicture}
\end{center}
\caption{Construction of a parallel line in the case of a directed line}\label{f.se-parallel-directed}
\end{figure}

\newpage
The proof uses Ceva's theorem.

\textit{Ceva's theorem  (Thm.~\ref{thm.ceva}):}\index{Ceva's theorem} If the line segments from the vertices of a triangle to the opposite edges intersect in a point $O$ (as in Fig.~\ref{f.se-parallel-directed}), the lengths of the segments satisfy:
\[
\frac{\overline{AM}}{\overline{MB}}\cdot\frac{\overline{BQ}}{\overline{QS}}\cdot\frac{\overline{SP}}{\overline{PA}} = 1\,.
\]
In Fig.~\ref{f.se-parallel-directed} $M$ is the midpoint of $\overline{AB}$ so $\displaystyle\frac{\overline{AM}}{\overline{MB}}=1$ and the equation becomes:
\begin{align}
\frac{\overline{BQ}}{\overline{QS}}=\frac{\overline{PA}}{\overline{SP}}=\frac{\overline{AP}}{\overline{PS}}\,,\label{eq.ceva}
\end{align}
since the order of the endpoints of a line segment is not important.

We claim that $\triangle ABS \sim \triangle PQS$:
\begin{eqnarray*}
\frac{\overline{BS}}{\overline{QS}}&=&\frac{\overline{BQ}}{\overline{QS}}+\frac{\overline{QS}}{\overline{QS}} = \frac{\overline{BQ}}{\overline{QS}}+1\\
&&\\
\frac{\overline{AS}}{\overline{PS}} &=& \frac{\overline{AP}}{\overline{PS}} + \frac{\overline{PS}}{\overline{PS}} = \frac{\overline{AP}}{\overline{PS}} + 1\,.
\end{eqnarray*}
Using Eq.~\ref{eq.ceva}:
\[
\frac{\overline{BS}}{\overline{QS}}=\frac{\overline{BQ}}{\overline{QS}}+1=\frac{\overline{AP}}{\overline{PS}}+1=\frac{\overline{AP}}{\overline{PS}}+\frac{\overline{PS}}{\overline{PS}}=\frac{\overline{AS}}{\overline{PS}}\,,
\]
and it follows that $\triangle ABS \sim \triangle PQS$ and therefore $\overline{PQ}\parallel\overline{AB}$.

\textit{Case 2:}
$\overline{AB}$ is not necessarily a directed line segment. The fixed circle $c$ has center $O$ and radius $r$. $P$ is the point not on the line through which it is required to construct a line parallel to $l$ (Fig.~\ref{f.se-parallel-other1}).

Choose $M$, any point on $l$, and construct a ray extending $\overline{MO}$ that intersects the circle at $U,V$.
$\overline{UV}$ is a directed line segment because $O$, the center of the circle, bisects the diameter $\overline{UV}$. Choose a point $A$ on $l$ and use the construction for a directed line segment (Case 1) to construct a line through $A$ parallel to $\overline{UV}$ which intersects the circle at $X,Y$ (Fig.~\ref{f.se-parallel-other2}).

\begin{figure}[t]
\subfigures
\leftfigure[c] {
\begin{tikzpicture}[scale=.75]
\coordinate (O) at (0,0);
\node[below right] at (O) {$O$};
\draw[name path=circle] (O) circle[radius=2cm];
\draw[name path=l] (-4,-3) --
  node[above, near end,xshift=24pt] {$l$} +(7,0);
\path[name path=mo] (-2,-3) coordinate (M) -- 
  ($(-2,-3)!1.65!(O)$);
\node[below] at (M) {$M$};
\path [name intersections={of=circle and mo,by={V,U}}];
\node[below,xshift=2pt,yshift=-4pt] at (U) {$U$};
\node[right,xshift=4pt] at (V) {$V$};
\draw (M) -- (U) -- node[left] {$r$} (O) -- node[left] {$r$} (V);
\node at (-1.6,1.6) {$c$};
\coordinate (P) at (-4,1);
\node[above left] at (P) {$P$};
\vertex{O};
\vertex{P};
\end{tikzpicture}
}
\hfill
\rightfigure[c] {
\begin{tikzpicture}[scale=.75]
\coordinate (O) at (0,0);
\node[below right] at (O) {$O$};
\draw[name path=circle] (O) circle[radius=2cm];
\draw[name path=l] (-4,-3) --
  node[above,near end,xshift=24pt] {$l$} +(7,0);
\path[name path=mo] (-2,-3) coordinate (M) --
  ($(-2,-3)!1.65!(O)$);
\node[below] at (M) {$M$};
\path [name intersections={of=circle and mo,by={V,U}}];
\node[below,xshift=2pt,yshift=-4pt] at (U) {$U$};
\node[right,xshift=4pt] at (V) {$V$};
\draw (M) -- (V);
\path[name path=ax] (-3,-3) coordinate (A) --
  ($(-3,-3)!1.8!(-1,0)$);
\node[below] at (A) {$A$};
\path [name intersections={of=circle and ax,by={Y,X}}];
\node[left] at (X) {$X$};
\node[above] at (Y) {$Y$};
\node at (-1.6,1.6) {$c$};
\draw (A) -- (Y);
\coordinate (P) at (-4,1);
\node[above left] at (P) {$P$};
\vertex{O};
\vertex{P};
\end{tikzpicture}
}
\leftcaption{Construction of  a directed line}\label{f.se-parallel-other1}
\rightcaption{Construction of  a line parallel to the directed line}\label{f.se-parallel-other2}
\end{figure}

\begin{figure}[b]
\begin{center}
\begin{tikzpicture}[scale=.75]
\coordinate (O) at (0,0);
\node[below right] at (O) {$O$};
\draw[name path=circle] (O) circle[radius=2cm];
\draw[name path=l] (-5,-3) --
  node[above,near end,xshift=40pt] {$l$} +(9,0);
\path[name path=mo] (-2,-3) coordinate (M) -- 
  ($(-2,-3)!1.65!(O)$);
\node[below] at (M) {$M$};
\path [name intersections={of=circle and mo,by={V,U}}];
\node[below,xshift=2pt,yshift=-4pt] at (U) {$U$};
\node[above right] at (V) {$V$};
\draw (M) -- (V);
\path[name path=ax] (-3,-3) coordinate (A) -- 
  ($(-3,-3)!1.8!(-1,0)$);
\node[below] at (A) {$A$};
\path [name intersections={of=circle and ax,by={Y,X}}];
\node[left] at (X) {$X$};
\node[above] at (Y) {$Y$};
\node at (-1.6,1.6) {$c$};
\draw (A) -- (Y);
\coordinate (P) at (-4,1);
\node[above] at (P) {$P$};
\path[name path=xo] (X) -- ($(X)!2.2!(O)$);
\path[name intersections={of=circle and xo,by={Xp}}];
\node[above right] at (Xp) {$X'$};
\draw (X) -- (Xp);
\path[name path=yo] (Y) -- ($(Y)!2.2!(O)$);
\path[name intersections={of=circle and yo,by={y,Yp}}];
\node[below right] at (Yp) {$Y'$};
\draw (Y) -- (Yp);
\path[name path=xy] (Xp) -- ($(Xp)!1.6!(Yp)$);
\path[name intersections={of=l and xy,by={B}}];
\node[below] at (B) {$B$};
\draw (Xp) -- (B);
\draw[thick,dotted,name path=z] (-5,0) -- 
  (4,0) node[above,near end,xshift=40pt] {$l'$};
\draw[thick,dashed] (-5,1) -- +(9,0);
\path[name intersections={of=ax and z,by={Z}}];
\path[name intersections={of=xy and z,by={Zp}}];
\node[above left] at (Z) {$Z$};
\node[below right] at (Zp) {$Z'$};
\vertex{O};
\vertex{P};
\end{tikzpicture}
\end{center}
\caption{Proof that $l'$ is parallel to $l$}\label{f.se-parallel-other3}
\end{figure}

Construct a diameter from $X$ through $O$ that intersects the other side of the circle at $X'$, and similarly construct the diameter $\overline{YY'}$. Construct the ray from $X'$ through $Y'$ and denote by $B$ its intersection with $l$. We claim that $M$ is the bisector of $\overline{AB}$ so that $\overline{AB}$ is a directed line segment and therefore a line can be constructed through $P$ parallel to $l$ (Fig.~\ref{f.se-parallel-other3}).


$\overline{OX}, \overline{OX'}, \overline{OY}, \overline{OY'}$ are all radii of the circle and $\angle XOY = \angle X'OY'$ since they are vertical angles, so $\triangle XOY\cong\triangle X'OY'$ by side-angle-side. Define\footnote{Define, not construct, because we are in the middle of the proof that such a line can be constructed.} $l'$ to be a line through $O$ parallel to $l$ that intersects $\overline{XY}$ at $Z$ and $\overline{X'Y'}$ at $Z'$. $\angle XOZ=\angle X'OZ'$ are vertical angles, $\angle ZXO=\angle Z'X'O$ are alternate interior angles and $\overline{XO}=\overline{XO'}$ are radii, so $\triangle XOZ\cong\triangle X'OZ'$ by angle-side-angle and $\overline{ZO}=\overline{OZ'}$. Therefore, $\overline{AMOZ}$ and $\overline{BMOZ'}$ are parallelograms and $\overline{AM}=\overline{ZO}=\overline{OZ'}=\overline{MB}$.
\end{proof}

\begin{theorem}
Given a line segment $\overline{AB}$ and a point $P$ not on the line, it is possible to construct a line segment $\overline{PQ}$ that is parallel to $\overline{AB}$ and whose length is equal to the length of $\overline{AB}$, that is, it is possible to copy $\overline{AB}$ parallel to itself with $P$ as one of its endpoints.
\end{theorem}

\begin{proof}
We have proved that it is possible to construct a line $m$ through $P$ parallel to $\overline{AB}$ and a line $n$ through $B$ to parallel to $\overline{AP}$. The quadrilateral $\overline{ABQP}$ is a parallelogram so opposite sides are equal $\overline{AB}=\overline{PQ}$ (Fig.~\ref{f.se-parallel-other4}).
\end{proof}

\begin{figure}[t]
\begin{center}
\begin{tikzpicture}[scale=.5]
\coordinate (P) at (0,0);
\coordinate (Q) at (3,0);
\coordinate (A) at (-2,2.5);
\coordinate (B) at (1,2.5);
\draw ($(P)!-.6!(Q)$) -- node[above,near end,xshift=36pt,yshift=-5pt] {$m$} ($(P)!1.8!(Q)$);
\node[below] at (P) {$P$};
\node[below left] at (Q) {$Q$};
\draw ($(A)!-.6!(B)$) -- node[above,near end,xshift=40pt,yshift=-5pt] {$l$} ($(A)!2.5!(B)$);
\node[above left] at (A) {$A$};
\node[above right] at (B) {$B$};
\draw (A) -- (P);
\draw ($(B)!-.3!(Q)$) -- node[above,near end,xshift=18pt,yshift=-18pt] {$n$} ($(B)!1.4!(Q)$);
\end{tikzpicture}
\end{center}
\caption{Construction of  a copy of a line parallel to an existing line}\label{f.se-parallel-other4}
\end{figure}

\section{Construction of a Perpendicular to a Given Line}\label{s.perp}

\begin{theorem}\label{thm.straight-perp}
Given a line segment $l$ and a point $P$ not on $l$, it is possible to construct a perpendicular to $l$ through $P$.
\end{theorem}

\begin{proof}
By Thm.~\ref{thm.straight-parallel} construct a line $l'$ parallel to $l$ that intersects the fixed circle at $U,V$. Construct the diameter $\overline{UOU'}$ and the chord $\overline{VU'}$ (Fig.~\ref{f.se-perp}). $\angle UVU'$ is a right angle because it is subtended by a diameter. Therefore $\overline{VU'}$ is perpendicular to $\overline{UV}$ and $l$. Again by Thm.~\ref{thm.straight-parallel} construct the parallel to $\overline{VU'}$ through $P$.
\end{proof}

\begin{figure}[htb]
\begin{center}
\begin{tikzpicture}[scale=.7]
\coordinate (O) at (0,0);
\coordinate (P) at (3.5,.6);
\draw[name path=circle] (O) circle[radius=2cm];
\draw[name path=l] (-4,-3) -- node[above,near end,xshift=45pt] {$l$} ++(9,0);
\draw[name path=lp] (-3,-1) -- node[above,near end,xshift=40pt] {$l'$} ++(8,0);
\node[above left] at (O) {$O$};
\node[right] at (P) {$P$};
\path[name intersections={of=circle and lp,by={U,V}}];
\node[below left] at (U) {$U$};
\node[below right] at (V) {$V$};
\path[name path=d] (U) -- ($(U)!2.3!(O)$);
\path[name intersections={of=circle and d,by={Up}}];
\draw (U) -- (Up);
\node[above right] at (Up) {$U'$};
\draw (Up) -- (V);
\path[name path=p] (P) -- ++(0,-4);
\path[name intersections={of=p and l,by={X}}];
\draw (X) rectangle +(9pt,9pt);
\draw[rotate=90] (V) rectangle +(9pt,9pt);
\vertex{O};
\vertex{P};
\draw (P) -- ++(0,1);
\draw (P) -- (X);
\end{tikzpicture}
\end{center}
\caption{Construction of  a perpendicular line}\label{f.se-perp}
\end{figure}

\newpage

\section{Copying a Line Segment in a Given Direction}\label{s.copy}

\begin{theorem}\label{thm.straight-direction}
It is possible to construct a copy of a given line segment in the direction of another line.
\end{theorem}

The meaning of  ``direction'' is that the line defined by two points $A',H'$ is at an angle $\theta$ relative to some axis and the goal is to construct $\overline{AS}=\overline{PQ}$ such that $\overline{AS}$ will have the same angle $\theta$ relative to that axis (Fig.~\ref{f.se-copy1}).

\begin{proof}
By Thm.~\ref{thm.straight-parallel} it is possible to construct a line segment $\overline{AH}$ such that $\overline{AH}\parallel\overline{A'H'}$, and to construct a line segment $\overline{AK}$ such that $\overline{AK}\parallel\overline{PQ}$.
$\angle HAK=\theta$ so it remains to find a point $S$ on $\overline{AH}$ so that $\overline{AS}=\overline{PQ}$.

\begin{figure}[t]
\begin{center}
\begin{tikzpicture}[scale=.7]
\coordinate (A) at (0,0);
\coordinate (P) at (3cm,2);
\coordinate (Q) at (4.5cm,2);
\draw (P) -- (Q);
\node[left] at (P) {$P$};
\node[right] at (Q) {$Q$};
\coordinate (A1) at (-3,1);
\draw (A1) -- ++(60:3cm) coordinate (H1);
\draw (A1) -- ++(0:2cm);
\node[left] at (A1) {$A'$};
\node[left] at (H1) {$H'$};
\draw (A) -- ++(60:1.5cm) coordinate (S);
\node[left] at (S) {$S$};
\draw (A) -- ++(1.5,0);
\node[left] at (A) {$A$};
\node[above right,xshift=4pt] at (A1) {$\theta$};
\node[above right,xshift=4pt] at (A) {$\theta$};
\draw (A) -- ++(60:3cm) coordinate (H);
\node[left] at (H) {$H$};
\draw (A) -- ++(1.5,0) coordinate (K);
\node[right] at (K) {$K$};
\vertex{P};
\vertex{Q};
\vertex{A};
\vertex{S};
\end{tikzpicture}
\end{center}
\caption{Copying a line segment in a given direction}\label{f.se-copy1}
\end{figure}

Construct two radii $\overline{OU}, \overline{OV}$ of the fixed circle which are parallel to $\overline{AH}, \overline{AK}$, respectively, and construct a ray through $K$ parallel to $\overline{UV}$. Denote its intersection with $\overline{AH}$ by $S$ (Fig.~\ref{f.se-copy3}). By construction, $\overline{AH}\parallel\overline{OU}$ and $\overline{AK}\parallel\overline{OV}$, so $\angle SAK=\angle HAK=\angle UOV=\theta$. $\overline{SK}\parallel\overline{UV}$ and $\triangle SAK\sim\triangle UOV$ by angle-angle-angle, $\triangle UOV$ is isosceles because $\overline{OU}, \overline{OV}$ are radii of the same circle. Therefore, $\triangle SAK$ is isosceles and $\overline{AS}=\overline{AK}=\overline{PQ}$.
\end{proof}

\begin{figure}[b]
\begin{center}
\begin{tikzpicture}[scale=.7]
\coordinate (A) at (0,0);
\coordinate (P) at (3cm,2);
\coordinate (Q) at (4.5cm,2);
\draw (P) -- (Q);
\node[left] at (P) {$P$};
\node[right] at (Q) {$Q$};
\coordinate (A1) at (-3,1);
\draw (A1) -- ++(60:3cm) coordinate (H1);
\node[left] at (A1) {$A'$};
\node[left] at (H1) {$H'$};
\node[left] at (A) {$A$};
\draw (A) -- ++(60:3cm) coordinate (H);
\node[left] at (H) {$H$};
\draw (A) -- ++(1.5,0) coordinate (K);
\node[right] at (K) {$K$};
\draw (A) -- (K);
\path (A) -- ++(60:1.5cm) coordinate (S);
\node[right] at (S) {$S$};
\draw (K) -- ($(K)!1.8!(S)$);
\node[above right,xshift=4pt] at (A) {$\theta$};
\node[above right,xshift=4pt] at (A1) {$\theta$};
\draw (A1) -- ++(1.5,0);
\vertex{P};
\vertex{Q};
\begin{scope}[xshift=3cm]
\coordinate (O) at (6,1);
\draw[name path=circle] (O) circle[radius=2.5cm];
\node[above left] at (O) {$O$};
\path[name path=u] (O) -- ++(60:2.5cm);
\path[name path=v] (O) -- ++(2.5,0);
\path[name intersections={of=circle and u,by={U}}];
\path[name intersections={of=circle and v,by={V}}];
\node[above right] at (U) {$U$};
\node[right] at (V) {$V$};
\draw (O) -- (U) -- (V) -- cycle;
\node[above right,xshift=4pt] at (O) {$\theta$};
\vertex{O};
\end{scope}
\end{tikzpicture}
\end{center}
\caption{Using the fixed circle to copy the line segment}\label{f.se-copy3}
\end{figure}

\section{Construction of a Line Segment as a Ratio of Segments}\label{s.relative}

\begin{theorem}\label{thm.straight-relative}
Given line segments of lengths $n, m, s$, it is possible to construct a line segment of length:
\[x=\displaystyle\frac{n}{m}s\,.\]
\end{theorem}

\begin{proof}
Choose points $A,B,C$ not on the same line and construct rays $\overline{AB}, \overline{AC}$. By Thm.~\ref{thm.straight-direction} it is possible to construct points $M,N,S$ such that $\overline{AM}= m$, $\overline{AN} =n$, $\overline{AS}=s$. By Thm.~\ref{thm.straight-parallel} construct a line through $N$ parallel to $\overline{MS}$ which intersects $\overline{AC}$ at $X$ and label $\overline{AX}$ by $x$ (Fig.~\ref{f.se-three2}). $\triangle MAS\sim\triangle NAX$ by angle-angle-angle so $\displaystyle\frac{m}{n}=\displaystyle\frac{s}{x}$ and $x=\displaystyle\frac{n}{m}s$.
\end{proof}

\begin{figure}[t]
\begin{center}
\begin{tikzpicture}[scale=.8]
\coordinate (A) at (0,0);
\draw[name path=ac] (A) node[left] {$A$} -- ++(7,0) node[right] {$C$};
\draw (A) -- ++(40:5cm) node[right] {$B$};
\path (A) -- node[above,xshift=-2pt] {$m$} ++(40:3cm) coordinate (M) node[above left] {$M$};
\path (A) -- ++(40:4cm) coordinate (N) node[above left] {$N$};
\path[name path=ms] (M) -- ++(-50:3.5cm);
\path[name path=nx] (N) -- ++(-50:4cm);
\path[name intersections={of=ac and ms,by={S}}];
\path[name intersections={of=ac and nx,by={X}}];
\node[below] at (S) {$S$};
\node[below] at (X) {$X$};
\path (A) -- node[below] {$s$} (S);
\draw (S) -- (M);
\draw (X) -- (N);
\draw[<->] ($(A)+(0,-.8)$) -- node[fill=white] {$x$} ($(X)+(0,-.8)$);
\draw[<->] ($(A)+(-.6,.8)$) -- node[fill=white] {$n$} ++(40:3.9cm);
\end{tikzpicture}
\end{center}
\caption{Similar triangles to construct the ratio of lengths}\label{f.se-three2}
\end{figure}

\section{Construction of a Square Root}\label{s.root}

\begin{theorem}\label{thm.straight-sqrt}
Given line segments of lengths $a,b$, it is possible to construct a line segment of length $\sqrt{ab}$.
\end{theorem}

\begin{proof}
We want to express $x=\sqrt{ab}$ as $x=\displaystyle\frac{n}{m}s$ in order to use Thm.~\ref{thm.straight-relative}.
\begin{itemize}
\setlength{\itemsep}{0pt}
\item For $n$ we use $d$, the diameter of the fixed circle.
\item For $m$ we use $t=a+b$ which can be constructed from $a,b$ by Thm.~\ref{thm.straight-direction}.
\item We define $s=\sqrt{hk}$ where $h,k$ are defined as expressions on the lengths $a,b,t,d$.
\end{itemize}
Define $h=\displaystyle\frac{d}{t}a$ and $k=\displaystyle\frac{d}{t}b$ and then compute:
\begin{eqnarray*}
x&=&\sqrt{ab}=\sqrt{\frac{th}{d}\frac{tk}{d}}=\sqrt{\left(\frac{t}{d}\right)^2hk}=\frac{t}{d}\sqrt{hk}=\frac{t}{d}s\\
h+k &=& \frac{d}{t}a + \frac{d}{t}b = \frac{d(a+b)}{t} = \frac{dt}{t} = d\,.
\end{eqnarray*}

By Thm.~\ref{thm.straight-direction} construct $\overline{HA}= h$ on a diameter $\overline{HK}$ of the fixed circle. From $h+k=d$ we have $\overline{AK}=k$ (Fig.~\ref{f.se-sqrt}). By Thm.~\ref{thm.straight-perp} construct a perpendicular to $\overline{HK}$ at $A$ and denote the intersection of this line with the circle by $S$. $\overline{OS}=\overline{OK}=d/2$ and $\overline{OA}=(d/2)-k$. 
\begin{figure}[t]
\begin{center}
\begin{tikzpicture}[scale=.7]
\coordinate (O) at (0,0);
\coordinate (H) at (-3,0);
\coordinate (K) at (3,0);
\node at (-2.4,2.4) {$c$};
\draw (H) -- (K);
\draw[name path=circle] (O) circle[radius=3cm];
\node[below] at (O) {$O$};
\node[left] at (H) {$H$};
\node[right] at (K) {$K$};
\path[name path=as] (1,0) coordinate (A) -- ++(0,3.2);
\node[below] at (A) {$A$};
\path[name intersections={of=circle and as,by={S}}];
\node[above] at (S) {$S$};
\draw (A) -- node[right] {$s$} (S);
\path (H) -- node[above] {$h$} (A);
\path (A) -- node[above] {$k$} (K);
\draw (O) -- node[left,xshift=-2pt] {$\displaystyle\frac{d}{2}$} (S);
\node at (.5,-1.5) {$\displaystyle\frac{d}{2}-k$};
\draw[->] (.5, -1.2) -- ++(0,1);
\draw[rotate=90] (A) rectangle +(8pt,8pt);
\vertex{O};
\end{tikzpicture}
\end{center}
\caption{Construction of a square root}\label{f.se-sqrt}
\end{figure}

By Pythagoras's Theorem:
\begin{eqnarray*}
s^2&=& \left(\frac{d}{2}\right)^2 - \left(\frac{d}{2}-k\right)^2\\
&=& \left(\frac{d}{2}\right)^2 - \left(\frac{d}{2}\right)^2 + 2\frac{dk}{2} - k^2\\
&=& k(d-k) = kh\\
s&=&\sqrt{hk}\,.
\end{eqnarray*}

Now $x=\displaystyle\frac{t}{d}s$ can be constructed by Thm.~\ref{thm.straight-relative}.
\end{proof}

\section{Construction of the Intersection of a Line and a Circle}\label{s.line-circle-straight}

\begin{theorem}
Given a line $l$ and a circle $c(O,r)$, it is possible to construct their points of intersection (Fig.~\ref{f.se-line-circle1}).
\end{theorem}
\begin{figure}[t]
\begin{center}
\begin{tikzpicture}[scale=.7]
\coordinate (O) at (0,0);
\node[below right] at (O) {$O$};
\vertex{O};
\draw[thick,dashed,name path=circle] (O) circle[radius=3cm];
\draw (O) -- node[above] {$r$} ++(-130:3cm) coordinate (R);
\draw[name path=l] (O) ++(170:4cm) --
  node[below, near end,xshift=30pt,yshift=10pt] {$l$} ++(20:8cm);
\path[name intersections={of=circle and l,by={Y,X}}];
\node[above left] at (X) {$X$};
\node[above right] at (Y) {$Y$};
\end{tikzpicture}
\end{center}
\caption{Construction of the points of intersection of a line and a circle (1)}\label{f.se-line-circle1}
\end{figure}

\begin{proof}
By Thm.~\ref{thm.straight-perp} it is possible to construct a perpendicular from the center of the circle $O$ to the line $l$. The intersection of $l$ with the perpendicular is denoted by $M$. $\overline{OM}$ bisects the chord $\overline{XY}$, where $X, Y$ are the intersections of the line with the circle (Fig.~\ref{f.se-line-circle2}). Define $\overline{XY}=2s$ and $\overline{OM}=t$. Note that $s,X,Y$ are just definitions not entities  have been constructed.
\begin{figure}[b]
\begin{center}
\begin{tikzpicture}[scale=.7]
\coordinate (O) at (0,0);
\draw[thick,dashed,name path=circle] (O) circle[radius=3cm];
\node[below right] at (O) {$O$};
\vertex{O};
\path (O) --  ++(-130:3cm) coordinate (R);
\node[below left,yshift=2pt,xshift=2pt] at (R) {$R$};
\draw[name path=l] (O) ++(170:4cm) --
  node[below, near end,xshift=30pt,yshift=10pt] {$l$} ++(20:8cm);
\path[name intersections={of=circle and l,by={Y,X}}];
\node[above left] at (X) {$X$};
\node[above right] at (Y) {$Y$};
\draw (O) -- node[below] {$r$} (X);
\path (X) -- ($(X)!.5!(Y)$) coordinate (M);
\node[above] at (M) {$M$};
\draw (O) -- node[right] {$t$} (M);
\path (X) -- node[above] {$s$} (M);
\path (M) -- node[above] {$s$} (Y);
\draw (O) ++(170:4cm) -- ++(20:3.1cm) -- ++(-70:10pt) -- ++(20:10pt);
\draw (O) -- node[below] {$t$} +(50:2) coordinate (RTT);
\draw (O) -- node[below] {$t$} +(-130:2) coordinate (RT);
\vertex{RT};
\draw (RT) -- node[right,yshift=-2pt] {$r-t$} ($(RT)+(-130:1cm)$);
\vertex{RTT};
\draw[<->] ($(RT)+(.5cm,-1.6cm)$) -- node[fill=white] {$r+t$}+(50:5);
\end{tikzpicture}
\end{center}
\caption{Construction of the points of intersection of a line and a circle (2)}\label{f.se-line-circle2}
\end{figure}

By Pythagoras's Theorem $s^2=r^2-t^2=(r+t)(r-t)$. By Thm.~\ref{thm.straight-direction} it is possible to construct line segments of length $t$ from $O$ in the two directions $\overline{OR}$ and $\overline{RO}$. The result is two line segments of length $r+t,r-t$.

By Thm.~\ref{thm.straight-sqrt} a line segment of length $s=\sqrt{(r+t)(r-t)}$ can be constructed, and by Thm.~\ref{thm.straight-direction} line segments of length $s$ from $M$ along $l$ in both directions can be constructed. Their other endpoints are the points of intersection of $l$ and $c$.
\end{proof}

\newpage

\section{Construction of the Intersection of Two Circles}\label{s.two-circles}

\begin{theorem}
Given two circles $c(O_1,r_1), c(O_2,r_2)$, it is possible to construct their points of intersection.
\end{theorem}

\begin{proof}
Construct $\overline{O_1O_2}$ and label its length $t$ (Fig.~\ref{f.se-circle-circle1}).
Label by $A$ be the point of intersection of $\overline{O_1O_2}$ and $\overline{XY}$, and label $q=\overline{O_1A}$, $x=\overline{XA}$ (Fig.~\ref{f.se-circle-circle2}). $A$ has not yet been constructed, but if $q,x$ are constructed then by Thm.~\ref{thm.straight-direction} the point $A$ at length $q$ from $O_1$ in the direction $\overline{O_1O_2}$ can be constructed.

\begin{figure}[t]
\begin{center}
\begin{tikzpicture}[scale=.9]
\coordinate (O1) at (0,0);
\coordinate (O2) at (2.5,0);
\node[below left] at (O1) {$O_1$};
\node[below right] at (O2) {$O_2$};
\vertex{O1};
\vertex{O2};
\draw[thick,dashed,name path=circle1] (O1) circle[radius=2cm];
\draw[thick,dashed,name path=circle2] (O2) circle[radius=1.6cm];
\path [name intersections={of=circle1 and circle2,by={X,Y}}];
\node[above,yshift=4pt] at (X) {$X$};
\node[below,yshift=-4pt] at (Y) {$Y$};
\draw (O1) -- node[above] {$r_1$} ++(160:2cm);
\draw (O2) -- node[above] {$r_2$} ++(30:1.6cm);
\draw (O1) -- (O2);
\node at (-1.7,1.6) {$c_1$};
\node at (3.8,1.4) {$c_2$};
\draw[<->] (0,-.6) -- node[fill=white] {$t$} +(2.5,0);
\end{tikzpicture}
\end{center}
\caption{Construction of the intersection of two circles (1)}\label{f.se-circle-circle1}
\end{figure}

\begin{figure}[b]
\begin{center}
\begin{tikzpicture}[scale=.9]
\coordinate (O1) at (0,0);
\coordinate (O2) at (2.5,0);
\vertex{O1};
\vertex{O2};
\node[below left] at (O1) {$O_1$};
\node[below right] at (O2) {$O_2$};
\draw[thick,dashed,name path=circle1] (O1) circle[radius=2cm];
\draw[thick,dashed,name path=circle2] (O2) circle[radius=1.6cm];
\path [name intersections={of=circle1 and circle2,by={X,Y}}];
\node[above,yshift=4pt] at (X) {$X$};
\node[below,yshift=-4pt] at (Y) {$Y$};
\draw (O1) -- node[above,xshift=-4pt] {$r_1$} (X);
\draw (O2) -- node[above,xshift=4pt] {$r_2$} (X);
\draw[name path=oo] (O1) -- (O2);
\node at (-1.7,1.6) {$c_1$};
\node at (3.8,1.4) {$c_2$};
\draw[name path=xy] (X) -- (Y);
\path[name intersections={of=xy and oo,by={A}}];
\node[below left] at (A) {$A$};
\draw (A) rectangle +(6pt,6pt);
\path (O1) -- node[below,xshift=-2pt] {$q$} (A);
\path (X) -- node[left,yshift=-2pt] {$x$} (A);
\draw[<->] (0,-.6) -- node[fill=white] {$t$} +(2.5,0);
\end{tikzpicture}
\end{center}
\caption{Construction of the intersection of two circles (2)}\label{f.se-circle-circle2}
\end{figure}

Once $A$ has been constructed, by Thm.~\ref{thm.straight-perp} a perpendicular to $\overline{O_1O_2}$ at $A$ can be constructed, and by Thm.~\ref{thm.straight-direction} it is possible to construct line segments of length $x$ from $A$ in both directions along the perpendicular. Their other endpoints are  the points of intersection of the circles.

\noindent\textbf{Construction of the length $q$:} Define $d=\sqrt{r_1^2+t^2}$, the hypotenuse of a right triangle, which can be constructed from the known lengths $r_1,t$. Note that $\triangle O_1XO_2$ is not necessarily a right triangle; the right triangle can be constructed anywhere in the plane. In the right triangle $\triangle XAO_1$, $\cos\angle XO_1A=q/r_1$. By the Law of Cosines \index{Law of cosines} for $\triangle XO_1O_2$:
\begin{eqnarray*}
r_2^2 &=& t^2 + r_1^2 - 2r_1t\cos\angle XO_1O_2\\
&=& t^2 + r_1^2 - 2tq\\
2tq &=& (t^2+r_1^2) - r_2^2=d^2-r_2^2\\
q&=&\frac{(d+r_2)(d-r_2)}{2t}\,.
\end{eqnarray*}
By Thm.~\ref{thm.straight-direction} these lengths can be constructed and by Thm.~\ref{thm.straight-relative} $q$ can be constructed from $d+r_2,d-r_2,2t$.

\medskip

\noindent\textbf{Construction of the length $x$:} By Pythagoras's Theorem:
\[
x=\sqrt{r_1^2-q^2}=\sqrt{(r_1+q)(r_1-q)}\,.
\]
By Thm.~\ref{thm.straight-direction}, $h =r_1+ q,k= r_1 - q$ can be constructed, as can $x=\sqrt{hk}$ by Thm.~\ref{thm.straight-sqrt}.
\end{proof}

\vspace{-3ex}

\subsection*{What Is the Surprise?}

A compass is necessary because a straightedge can only compute the roots of linear equations and not values such as $\sqrt{2}$, the hypotenuse of an isoceles right-triangle with sides of length $1$. However, it is surprising that the existence of only one circle, regardless of the position of its center and the length of its radius, is sufficient to perform any construction that is possible with a straightedge and compass.

\vspace{-3ex}

\subsection*{Sources}

This chapter is based on problem $34$ of \cite{dorrie1} reworked by Michael Woltermann \cite{dorrie2}.
