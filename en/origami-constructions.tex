% !TeX root = surprises.tex

%%%%%%%%%%%%%%%%%%%%%%%%%%%%%%%%%%%%%%%%%%%%%%%%%%%%%%%%%%%%%%%%

\chapter{Geometric Constructions Using Origami}\label{c.origami-constructions}

%%%%%%%%%%%%%%%%%%%%%%%%%%%%%%%%%%%%%%%%%%%%%%%%%%%%%%%%%%%%%%%

\abstract*{This chapter shows that constructions with origami are more powerful than constructions with a straightedge and compass. We give origami constructions for the following problm: two constructions for trisecting an angle, one by Hisashi Abe and one by  George E. Martin, two constructions for doubling a cube, one by Peter Messer and one by Marghareta P. Beloch, and the construction of a nonagon, a regular polynomial with nine sides.}

%%%%%%%%%%%%%%%%%%%%%%%%%%%%%%%%%%%%%%%%%%%%%%%%%%%%%%%%%%%%%%%

\index{Origami!geometric constructions}
This chapter shows that constructions with origami are more powerful than constructions with a straightedge  and compass. We give two constructions for trisecting an angle, one by Hisashi Abe\index{Abe, Hisashi} (Sect.~\ref{s.abe-trisection}) and the other by  George E. Martin\index{Martin, George E.} (Sect.~\ref{s.martin-trisection}), two constructions for doubling a cube, one by Peter Messer\index{Messer, Peter} (Sect.~\ref{s.messer}) and the other by Marghareta P. Beloch\index{Beloch, Margherita P.} (Sect.~\ref{s.cube2}), and the construction of a nonagon, a regular polynomial with nine sides (Sect.~\ref{s.nonagon}).

\section{Abe's Trisection of an Angle}\label{s.abe-trisection}
\index{Origami!trisection of an angle}
\index{Trisection of an angle!origami@using origami}

\noindent\textbf{Construction:}
Given an acute angle $\angle PQR$, construct $p$, the perpendicular to $\overline{QR}$ at $Q$. Construct $q$, a perpendicular to $p$ that intersects $\overline{PQ}$ at point $A$, and construct $r$, the perpendicular to $p$ at $B$ that is halfway between $Q$ and $A$. Using Axiom~6 construct the fold $l$ that places $A$ at $A'$ on $\overline{PQ}$ and $Q$ at $Q'$ on $r$. Let $B'$ be the reflection of $B$ around $l$. Construct lines through $\overline{QB'}$ and $QQ'$ (Fig.~\ref{f.abe1}).
\begin{figure}[ht]
\begin{center}
\begin{tikzpicture}[scale=.8]

% Place points P, Q, R
\coordinate (P) at (60:10cm); %(5,8.67);
\coordinate (Q) at (0,0);
\coordinate (R) at (10,0);
\node[below right] at (P) {$P$};
\node[left]  at (Q) {$Q$};
\node[right] at (R) {$R$};

% Draw PQR
\draw (P)  -- (Q) -- (R);

% Draw perpendicular to QR
\draw (Q) -- node[left,very near end] {$p$} +(0,11);

% Draw parallel to QR and parallel halfway
\coordinate (A) at (0,5);
\coordinate (B) at (0,2.5);
\draw  (A) -- node[above,very near end] {$q$} +(10,0);
\draw  (B) -- node[above,very near end] {$r$} +(10,0);
\node[left] at (A) {$A$};
\node[left] at (B) {$B$};
\path (Q) -- node[left] {$a$} (B) -- node[left] {$a$} (A);
\draw (A) rectangle +(8pt,8pt);
\draw (B) rectangle +(8pt,8pt);
\draw (Q) rectangle +(8pt,8pt);

% Tangent line y = -2.75x + 10.69

% Draw fold
\coordinate (D) at (0,10.69);
\coordinate (fold-x) at (3.89,0);
\coordinate (AP) at (3.65,6.33);
\coordinate (QP) at (6.87,2.5);
\coordinate (BP) at (5.26,4.42);
\node[left] at (D) {$D$};
\node[above,yshift=6pt] at (AP) {$A'$};
\node[above,yshift=6pt] at (QP) {$Q'$};
\node[above,xshift=2pt,yshift=2pt] at (BP) {$B'$};
\draw [very thick,dashed] (D) -- node[left,near start] {$l$} ($(D)!1.03!(fold-x)$);

% Draw line of reflections
\draw (D) -- (QP);

% Draw trisecting lines
\draw (Q) -- ($(Q)!1.3!(QP)$);
\draw (Q) -- ($(Q)!1.3!(BP)$);

% Complete triangle
\draw (A) -- (QP);

\draw[->,very thick,dotted,bend left=40] ($(A)+(.1,.1)$) to ($(AP)+(-.1,0)$);
\draw[->,dotted,very thick,bend right=30] ($(Q)+(.1,-.1)$) to ($(QP)+(0,-.1)$);
\end{tikzpicture}
\end{center}
\caption{Abe's trisection of an angle}\label{f.abe1}
\end{figure}

\begin{theorem} $\angle PQB'=\angle B'QQ'=\angle Q'QR=\angle PQR/3$.
\end{theorem}

\begin{proof}[1]
$A', B', Q'$ are reflections around the line $l$ of the points $A,B,Q$ on the line $\overline{DQ}$, so they are on the reflected line $\overline{DQ'}$. By construction $\overline{AB}=\overline{BQ}$, $\angle ABQ'=\angle QBQ'=90^\circ$ and $\overline{BQ'}$ is a common side, so $\triangle ABQ'\cong \triangle QBQ'$ by side-angle-side. Therefore, $\angle AQQ'=\angle QAQ'=\alpha$ so $\triangle AQ'Q$ is isoceles (Fig.~\ref{f.abe2}).

\begin{figure}[ht]
\begin{center}
\begin{tikzpicture}[scale=.8]

% Place points P, Q, R
\coordinate (P) at (60:10cm);
\coordinate (Q) at (0,0);
\coordinate (R) at (10,0);
\node[below right] at (P) {$P$};
\node[left,xshift=-4pt] at (Q) {$Q$};
\node[right] at (R) {$R$};

% Draw PQR
\draw (Q) -- (R);

% Draw perpendicular to QR
\draw (Q) -- node[left,very near end] {$p$} +(0,11);

% Draw parallel to QR and parallel halfway
\coordinate (A) at (0,5);
\coordinate (B) at (0,2.5);
\draw (A) -- node[above,very near end] {$q$} +(10,0);
\draw[name path=Br] (B) -- node[above,very near end] {$r$} +(10,0);
\node[left,xshift=-4pt] at (A) {$A$};
\node[left,xshift=-4pt] at (B) {$B$};
\path (Q) -- node[left,xshift=-4pt] {$a$} (B) -- node[left,xshift=-4pt] {$a$} (A);
\draw (A) rectangle +(8pt,8pt);
\draw (B) rectangle +(8pt,8pt);
\draw (Q) rectangle +(8pt,8pt);

% Tangent line y = -2.75x + 10.69

% Draw fold
\coordinate (D) at (0,10.69);
\coordinate (fold-x) at (3.89,0);
\coordinate (AP) at (3.65,6.33);
\coordinate (QP) at (6.87,2.5);
\coordinate (BP) at (5.26,4.42);
\node[left] at (D) {$D$};
\node[above,yshift=6pt] at (AP)  {$A'$};
\node[above,xshift=2pt,yshift=6pt] at (QP) {$Q'$};
\node[above,xshift=4pt,yshift=2pt] at (BP) {$B'$};
\draw[name path=fold,very thick,dashed] (D) -- node[left,near start] {$l$}  ($(D)!1.03!(fold-x)$);
	
% Draw line of reflections
\draw (D) -- (AP);

% Draw trisecting lines
\draw (Q) -- ($(Q)!1.3!(BP)$);

\draw [very thick,loosely dash dot,red] (Q) -- (QP);
\draw [very thick,loosely dash dot,red] (QP) -- (AP);
\draw [very thick,loosely dash dot,red] (AP) -- (Q);
\draw [very thick,loosely dash dot dot,blue] ($(Q)+(0,-4pt)$) -- ($(QP)+(0,-4pt)$);
\draw [very thick,dash dot dot,blue] ($(QP)+(0,-4pt)$) -- ($(A)+(0,-4pt)$);
\draw [very thick,dash dot dot,blue] ($(A)+(-4pt,0)$) -- ($(Q)+(-4pt,0)$);

\draw (A) -- (AP);

\node[left,xshift=-40pt,yshift=7pt] at (QP) {$\alpha$};
\node[left,xshift=-40pt,yshift=-6pt] at (QP) {$\alpha$};
\node[right,xshift=40pt,yshift=6pt] at (Q) {$\alpha$};
\node[right,xshift=40pt,yshift=28pt] at (Q) {$\alpha$};
\node[right,xshift=30pt,yshift=42pt] at (Q) {$\alpha$};

\draw (AP) -- (P);

\path[name path=Qr] (Q) -- (QP);
\path[name intersections = {of = fold and Qr, by = {U}}];
\node[above left,xshift=-2pt,yshift=-2pt] at (U) {$U$};
\draw[rotate=20] (U) rectangle +(8pt,8pt);
\path[name intersections = {of = fold and Br, by = {V}}];
\node[above left,xshift=-2pt,yshift=-2pt] at (V)  {$V$};
\end{tikzpicture}
\end{center}
\caption{Proofs of Abe's trisection ($U,V$ are used in Proof 2)}\label{f.abe2}
\end{figure}

By reflection $\triangle AQ'Q\cong \triangle A'QQ'$, so $\triangle A'QQ'$ is also an isoceles triangle.
$\overline{QB'}$, the reflection of $\overline{Q'B}$, is the perpendicular bisector of an isoceles triangle, so $\angle A'QB'=\angle Q'QB'=\angle QQ'B=\alpha$.
By alternating interior angles, $\angle Q'QR=\angle QQ'B=\alpha$.
Together we have:
\[
\triangle PQB'=\angle A'QB'=\angle B'QQ'=\angle Q'QR=\alpha\,.
\]
\end{proof}

\begin{proof}[2]
Since $l$ is a fold it is the perpendicular bisector of $\overline{QQ'}$. Denote the intersection of $l$ with $\overline{QQ'}$ by $U$ and its intersection with $\overline{QB'}$ by $V$ (Fig.~\ref{f.abe2}). $\triangle VUQ\cong \triangle VUQ'$ by side-angle-side since $\overline{VU}$ is a common side,  the angles at $U$ are right angles and $\overline{QU}=\overline{Q'U}$. Therefore, $\angle VQU=\angle VQ'U=\alpha$ and $\angle Q'QR=\angle VQ'U=\alpha$ by alternating interior angles.

As in the first proof $A', B', Q'$ are all reflections around $l$, so they are on the line $\overline{DQ'}$ and $\overline{A'B'}=\overline{AB}=\overline{BQ}=\overline{B'Q'}=a$. Then $\triangle A'B'Q\cong\triangle Q'B'Q$ by side-angle-side and $\angle A'QB'=\angle Q'QB'=\alpha$.
\end{proof}

\newpage

\section{Martin's Trisection of an Angle}\label{s.martin-trisection}
\index{Origami!trisection of an angle}
\index{Trisection of an angle!origami@using origami}

\noindent\textbf{Construction:}
Given an acute angle $\angle PQR$, let $M$ be the midpoint of $\overline{PQ}$. Construct $p$ the perpendicular to $\overline{QR}$ through $M$ and construct $q$ perpendicular to $p$ through $M$ so $q\parallel\overline{QR}$. Using Axiom 6 construct the fold $l$ that places $P$ at $P'$ on $p$ and $Q$ at $Q'$ on $q$. If more than one fold is possible choose the one that intersects $\overline{PM}$. Construct $\overline{PP'}$ and $\overline{QQ'}$  (Fig.~\ref{f.martin}).

\begin{theorem}
$\angle Q'QR=\angle PQR/3$.
\end{theorem}
\begin{proof}
Denote the intersection of $\overline{QQ'}$ with $p$ by $U$ and its intersection with $l$ by $V$. Denote the intersection of $\overline{PQ}$ and $\overline{P'Q'}$ with $l$ by $W$. It is not immediate that $\overline{PQ}$ and $\overline{P'Q'}$ intersect $l$ at the same point. But $\triangle PWP' \sim \triangle QWQ'$ so the altitudes bisect both vertical angles $\angle PWP', \angle QWQ'$ and they must be on the same line.

$\triangle QMU\cong \triangle PMP'$ by angle-side-angle since  $\angle P'PM=\angle UQM=\beta$ by alternate interior angles, $\overline{QM}=\overline{MP}=a$ because $M$ is the midpoint of $\overline{PQ}$ and $\angle QMU=\angle PMP'=\gamma$ are vertical angles. Therefore, $\overline{P'M}=\overline{MU}=b$.

\newpage

$\triangle P'MQ'\cong \triangle UMQ'$ by side-angle-side, since $\overline{P'M}=\overline{MU}=b$, the angles at $M$ are right angles and $\overline{MQ'}$ is a common side. Since the altitude of the isoceles triangle $\triangle P'Q'U$ is the bisector of $\angle P'Q'U$, it follows that $\angle P'Q'M=\angle UQ'M=\alpha$. Furthermore, $\angle UQ'M=\angle Q'QR=\alpha$ by alternate interior angles. $\triangle QWV\cong\triangle Q'WV$ by side-angle-side since $\overline{QV}=\overline{VQ'}=c$, the angles at $V$ are right angles and $\overline{VW}$ is a common side. Therefore:
\begin{eqnarray*}
\angle WQV&=&\beta=\angle WQ'V=2\alpha\\
\angle PQR &=& \beta + \alpha = 3\alpha\,.
\end{eqnarray*}
\end{proof}

\begin{figure}[t]
\begin{center}
\begin{tikzpicture}[scale=.8]

% Place points P, Q, R
\coordinate (P) at (60:10cm); %(5,8.67);
\coordinate (Q) at (0,0);
\coordinate (R) at (10,0);
\node[below right] at (P) {$P$};
\node[above left] at (Q) {$Q$};
\node[right] at (R) {$R$};

% Draw PQR
\draw (R)  -- (Q);
\draw [name path=pq] (Q) -- (P);

% M is the midpoint of PQ
\coordinate (M) at (2.5, 4.33);
\node[above left,xshift=2pt] at (M) {$M$};
\draw [rotate=-90] (M) rectangle +(9pt,9pt);

% Drop a perpendicular from M to QR and extend the line upwards
% This is the given line p
\coordinate (pQR) at (M |- Q);
\draw [name path=p] (pQR) --
   node[left, very near end,yshift=20pt] {$p$}
   ($(pQR)!2!(M)$);
\draw (pQR) rectangle +(9pt,9pt);

% Construct q perpendicular to p through M
\draw [name path=q] ($(M)+(-2,0)$) --
   node[above, very near start,xshift=-30pt] {$q$}
   ($(M)+(10,0)$);

% Construct the fold line t
% Its equation is y = -2.75x + 18.51, as obtained from Geogebra
\coordinate (t1) at (6.7,.085);
\coordinate (t2) at (3.5,8.89);
\draw [very thick,dashed,name path=t] (t1) --
   node[very near end,left] {$l$}
   (t2);

% Construct a perpendicular to t through P
\coordinate (perp-p) at ($(t1)!(P)!(t2)$);
\path [name path=perp-p] (P) -- ($(P)!2.5!(perp-p)$);

% Get its intersection with t denoted Pt
% and its intersection with p named PP
\path [name intersections = {of = t and perp-p, by = {Pt}}];
\path [name intersections = {of = p and perp-p, by = {PP}}];
\node[left] at (PP) {$P'$};
\draw [rotate=22] (Pt) rectangle +(9pt,9pt);

% Draw PT
\draw (P) -- (PP);

% Construct a perpendicular to t through Q
\coordinate (perp-q) at ($(t1)!(Q)!(t2)$);
\path[name path=perp-q] (Q) -- ($(Q)!2.1!(perp-q)$);

% Get its intersection with t denoted V
% and its intersection with q denoted S=Q'
\path [name intersections = {of = t and perp-q, by = {V}}];
\path [name intersections = {of = q and perp-q, by = {QP}}];
\node[above,yshift=4pt] at (QP) {$Q'$};
\node[above left,xshift=-4pt,yshift=-2pt] at (V) {$V$};
\draw [rotate=22] (V) rectangle +(9pt,9pt);

% Draw Q QP
\draw [name path=qs] (Q) -- (QP);

% Get the intersection of QS with p denoted U
\path [name intersections = {of = p and qs, by = {U}}];
\node[above left] at (U) {$U$};

% Draw PP QP
\draw [name path=ts] (PP) -- (QP);

% Get its intersection with QP denoted W
\path [name intersections = {of = ts and pq, by = {W}}];
\node[right,xshift=4pt,yshift=4pt] at (W) {$W$};

% Label line segments
\path (P) -- node[left] {$a$} (M);
\path (M) -- node[left]  {$a$} (Q);
\path (PP) -- node[left]  {$b$} (M);
\path (M) -- node[right] {$b$} (U);
\path (Q) -- node[below,near end] {$c$} (V);
\path (V) -- node[below] {$c$} (QP);

% Label angles
\node [xshift=5pt,yshift=20pt]        at (M) {$\gamma$};
\node [xshift=-5pt,yshift=-20pt]      at (M) {$\gamma$};
\node [xshift=15pt,yshift=13pt]       at (Q) {$\beta$};
\node [xshift=-10pt,yshift=-10pt]     at (P) {$\beta$};
\node [left,xshift=-30pt,yshift=7pt]  at (QP) {$\alpha$};
\node [left,xshift=-30pt,yshift=-7pt] at (QP) {$\alpha$};
\node [right,xshift=25pt,yshift=5pt]  at (Q) {$\alpha$};
\end{tikzpicture}
\end{center}
\caption{Martin's trisection of an angle}\label{f.martin}
\end{figure}

\vspace{-3ex}

\section{Messer's Doubling of a Cube}\label{s.messer}
\index{Doubling a cube!origami@using origami}
\index{Origami!doubling a cube}
A cube of volume $V$ has sides of length $\sqrt[3]{V}$. A cube with twice the volume has sides of length $\sqrt[3]{2 V}=\sqrt[3]{2}\sqrt[3]{V}$, so if we can construct $\sqrt[3]{2}$ we can multiply by the given length $\sqrt[3]{V}$ to double the cube.

\medspace

\noindent\textbf{Construction:}
Divide the side of a unit square into thirds as follows: Fold the square in half to locate the points $I=(0,1/2)$ and $J=(1,1/2)$. Next, construct the lines $\overline{AC}$ and $\overline{BJ}$ (Fig.~\ref{f.messer1}). The point of intersection $K=(2/3,1/3)$ can be obtained by solving the two equations $y=1-x$ and $y=x/2$.

Construct $\overline{EF}$, the perpendicular to $\overline{AB}$ through $K$, and construct the reflection $\overline{GH}$ of $\overline{BC}$ around $\overline{EF}$. The side of the square has now been divided into thirds.

\begin{figure}[t]
\begin{center}
\begin{tikzpicture}[scale=.55]
% Draw square
\coordinate (A) at (0,12);
\coordinate (B) at (0,0);
\coordinate (C) at (12,0);
\coordinate (D) at (12,12);

\node[left]  at (A) {$A=(0,1)$};
\node[left]  at (B) {$B=(0,0)$};
\node[right] at (C) {$C=(1,0)$};
\node[right] at (D) {$D=(1,1)$};

\draw [thick] (A)  -- (B) -- (C) -- (D) -- cycle;

% Divide a side in half

\coordinate (M)  at (0,6);
\coordinate (N) at (12,6);
\node[left] at (M) {$I=(0,1/2)$};
\node[right] at (N) {$J=(1,1/2)$};
\draw [thick,dashed] (M) -- (N);


\draw [very thick,dotted,name path=ac] (A) -- 
   node[near start,above,xshift=24pt] {$y=1-x$} (C);
\draw [very thick,dotted,name path=be2] (B) -- 
   node[near start,above,xshift=-12pt] {$y=x/2$} (N);

\path [name intersections = {of = ac and be2, by = {I}}];
\node[below,xshift=-6pt,yshift=-8pt] at (I) {$K=$};
\node[below,xshift=-6pt,yshift=-20pt] at (I) {$(2/3,1/3)$};

\coordinate (E)  at (0,4);
\coordinate (F) at (12,4);
\node[left] at (E) {$E=(0,1/3)$};
\node[right] at (F) {$F=(1,1/3)$};
\draw [thick,dashed] (E) -- (F);

\coordinate (G)  at (0,8);
\coordinate (H) at (12,8);
\node[left] at (G) {$G=(0,2/3)$};
\node[right] at (H) {$H=(1,2/3)$};
\draw (G) -- (H);
\end{tikzpicture}
\end{center}
\caption{Dividing a length into thirds}\label{f.messer1}
\end{figure}

Using Axiom~6 place $C$ at $C'$ on $\overline{AB}$ and $F$ at $F'$ on $\overline{GH}$.  Denote by $L$ the point intersection of the fold with $\overline{BC}$ and denote by $b$ the length of $\overline{BL}$. Rename the length of the side of the square to $a+1$ where $a=\overline{AC'}$. The length of $\overline{LC}$ is $(a+1)-b$  (Fig.~\ref{f.messer3}).

\begin{theorem}
$\overline{AC'}=\sqrt[3]{2}$.
\end{theorem}

\begin{proof}
When the fold is performed the line segment $\overline{LC}$ is reflected onto the line segment $\overline{LC'}$ and $\overline{CF}$ is folded onto the line segment $\overline{C'F'}$. Therefore:
\begin{align}
\overline{GC'}=a-\frac{a+1}{3}=\frac{2a-1}{3}\,.\label{eq.one-third}
\end{align}
Since $\angle FCL$ is a right angle, so is $\angle F'C'L$.

\begin{figure}[t]
\begin{center}
\begin{tikzpicture}[scale=.65]
% Draw and label square
\coordinate (A) at (0,12);
\coordinate (B) at (0,0);
\coordinate (C) at (12,0);
\coordinate (D) at (12,12);
\node[left]  at (A) {$A$};
\node[left]  at (B) {$B$};
\node[right] at (C) {$C$};
\node[right] at (D) {$D$};
\draw (B) rectangle +(9pt,9pt);
\draw[rotate=90] (C) rectangle +(9pt,9pt);
\draw [thick] (A)  -- (B) -- (C) -- (D) -- cycle;

% Draw line one-third from botton
\coordinate (E)  at (0,4);
\coordinate (F) at (12,4);
\node[left] at (E) {$E$};
\node[right] at (F) {$F$};
\draw [name path=ef] (E) -- (F);

% Draw line two-thirds from bottom
\coordinate (G)  at (0,8);
\coordinate (H) at (12,8);
\node[left] at (G) {$G$};
\node[right] at (H) {$H$};
\draw[rotate=-90] (G) rectangle +(9pt,9pt);
\draw (G) -- (H);

% Draw reflections of C and F
\coordinate (CP) at (0,5.31);
\coordinate (FP) at (2.96,8);
\node[left] at (CP) {$C'$};
\node[above right,yshift=8pt] at (CP) {$\alpha$};
\node[below right,xshift=-2pt,yshift=-12pt] at (CP) {$\alpha'$};
\node[above] at (FP) {$F'$};
\node[below left,xshift=-8pt] at (FP) {$\alpha'$};
\draw[rotate=-50] (CP) rectangle +(9pt,9pt);
\draw (CP) -- (FP);

% Draw fold and fold arrows
% Tangent is y = 2.26x - 10.9
% Crosses x axis at (4.83,0)
\coordinate (J) at (4.83,0);
\node[below] at (J) {$L$};
\node[above left,xshift=-8pt] at (J) {$\alpha$};
\draw [very thick,dashed,name path=jd] (J) -- node[very near end,left] {$l$} (10,12);
\draw[thick,dotted,bend right=40,->] (C) to ($(CP)+(4pt,0)$);
\draw[thick,dotted,bend right=40,->] (F) to ($(FP)+(4pt,4pt)$);

% Draw hypotenuses of right triangles
\draw (CP) -- (J);
\path (J)  -- (C);

% Labels on BC and hypotenuses
\path (CP) -- node[right] {$(a+1)-b$} (J);
\path (J)  -- node[below] {$(a+1)-b$} (C);
\path (B)  -- node[below] {$b$} (J);
\path (C)  -- node[right] {$\displaystyle\frac{a+1}{3}$} (F);
\path (CP) -- node[right,xshift=10pt] {$\displaystyle\frac{a+1}{3}$} (FP);

% Labels on AB
\draw[<->] ($(A)+(-1,0)$)    --
  node[fill=white] {$a$} ($(CP)+(-1,0)$);
\draw[<->] ($(CP)+(-1,0)$)   --
  node[fill=white] {$1$} ($(B)+(-1,0)$);
\draw[<->] ($(CP)+(-2.5,0)$) --
  node[fill=white] {$\displaystyle\frac{2a-1}{3}$} ($(G)+(-2.5,0)$);
\draw[<->] ($(A)+(-2.5,0)$) --
  node[fill=white] {$\displaystyle\frac{a+1}{3}$} ($(G)+(-2.5,0)$);
\end{tikzpicture}
\end{center}
\vspace{-2ex}
\caption{Construction of $\sqrt[3]{2}$}\label{f.messer3}
\end{figure}

$\triangle C'BL$ is a right triangle so by Pythagoras's Theorem:
\begin{subeqnarray}
1^2 + b^2 &=& ((a+1)-b)^2\\
%&=& a^2+2a+1 - 2(a+1)b + b^2\\
%a^2+2a - 2(a+1)b&=&0\\
b&=&\frac{a^2+2a}{2(a+1)}\,.\slabel{eq.value-of-b}
\end{subeqnarray}

$\angle GC'F' + \angle F'C'L + \angle LC'B = 180^\circ$ since they form the straight line $\overline{GB}$. Denote $\angle GC'F'$ by $\alpha$. Then:
\[
\angle LC'B=180^\circ - \angle F'C'L - \angle GC'F'= 180^\circ - 90^\circ - \alpha = 90^\circ -\alpha\,,
\]
which we denote by $\alpha'$. The triangles $\triangle C'BL$, $\triangle F'GC'$ are right triangles so $\angle C'LB=\alpha$ and $\angle C'F'G=\alpha'$. Therefore, $\triangle C'BL\sim\triangle F'GC'$ and:
\[
\frac{\overline{BL}}{\overline{C'L}}=\frac{\overline{GC'}}{\overline{C'F'}}\,.
\]
Using Eq.~\ref{eq.one-third} we have:
\[
\frac{b}{(a+1)-b}=\frac{\displaystyle\frac{2a-1}{3}}{\displaystyle\frac{a+1}{3}}\,.
\]
Substituting for $b$ using Eq.~\ref{eq.value-of-b} gives:
\[
\displaystyle\frac{\displaystyle\frac{a^2+2a}{2(a+1)}}{(a+1)-\displaystyle\frac{a^2+2a}{2(a+1)}}=\frac{2a-1}{a+1}\,.
\]
Simplify the equation to obtain $a^3=2$ and $a=\sqrt[3]{2}$.
\end{proof}

\section{Beloch's Doubling of a Cube}\label{s.cube2}
\index{Doubling a cube!origami@using origami}
\index{Origami!doubling a cube}

Since the Beloch fold (Axiom~6) can solve cubic equations it is reasonable to conjecture that it can be used to double a cube. Here we give a direct construction that uses the fold.

\medskip

\noindent\textbf{Construction:}
Let $A=(-1,0)$, $B=(0,-2)$. Let $p$ be the line $x=1$ and let $q$ be the line $y=2$. Use the Beloch fold to construct the fold $l$ that places $A$ at $A'$ on $p$ and $B$ at $B'$ on $q$. Denote the intersection of the fold and the $y$-axis by $Y$ and the intersection of the fold and the $x$-axis by $X$ (Fig.~\ref{f.beloch-doubling}).

\begin{figure}[b]
\begin{center}
\begin{tikzpicture}[scale=.8]
% Draw and label square
\coordinate (O) at (0,0);
\coordinate (A) at (-2,0);
\coordinate (B) at (0,-4);
\node[below left,xshift=-7pt] at (O) {$O$};
\node[below left,yshift=-12pt] at (O) {$(0,0)$};
\node[above left,xshift=-7pt] at (A) {$A$};
\node[below left,xshift=2pt,yshift=0pt] at (A) {$(-1,0)$};
\node[above right,xshift=10pt] at (A) {$\alpha$};
\node[left,xshift=-12pt] at (B) {$B$};
\node[left,yshift=-12pt] at (B) {$(0,-2)$};
\node[above right,yshift=12pt] at (B) {$\alpha'$};

\draw[thick] (0,-4.5) --  node[very near end,above left,yshift=12pt] {$y$-axis} +(0,10);
\draw[thick] (-5,0)   -- node[very near start,above left] {$x$-axis} +(12,0);
\draw[thick] (2,-4.5) -- node[very near start, right,yshift=-10pt] {$p\!:x=1$} +(0,10);
\draw[thick] (-5,4) -- node[very near start, above,xshift=-16pt] {$q\!: y=2$} +(12,0);

\coordinate (AP) at (2,5);
\node[above right] at (AP) {$A'$};
\coordinate (BP) at (6.34,4);
\node[above right] at (BP) {$B'$};

% Tangent y = -0.8x + 1.26

% Exchanged X and Y 
\coordinate (X) at (0,2.52);
\coordinate (Y) at (3.15,0);
\node[right,xshift=4pt,yshift=2pt] at (X) {$Y$};
\node[below right,yshift=-14pt] at (X) {$\alpha$};
\node[below left,xshift=2pt,yshift=-12pt] at (X) {$\alpha'$};
\node[above right,xshift=10pt] at (Y) {$X$};
\node[below left,xshift=-10pt] at (Y) {$\alpha$};
\node[above left,xshift=-13pt] at (Y) {$\alpha'$};
\draw [very thick,dashed] ($(X)!-1.1!(Y)$) -- node[very near end,right,xshift=8pt] {$l$} ($(X)!2!(Y)$);

\draw [very thick,dotted] (A) -- (AP);
\draw [very thick,dotted] (B) -- (BP);

\draw[thick,dotted,bend left=40,->] (A) to ($(AP)+(-4pt,0)$);
\draw[thick,dotted,bend left=40,->] (B) to ($(BP)+(-6pt,-3pt)$);

\draw[rotate=-130] (X) rectangle +(10pt,10pt);
\draw[rotate=-130] (Y) rectangle +(10pt,10pt);
\end{tikzpicture}
\end{center}
\caption{Beloch's doubling of the cube}\label{f.beloch-doubling}
\end{figure}

\begin{theorem}
$\overline{OY}=\sqrt[3]{2}$.
\end{theorem}
\begin{proof}
The fold is the perpendicular bisector of both $\overline{AA'}$ and $\overline{BB'}$ so $\overline{AA'}\parallel\overline{BB'}$. By alternate interior angles $\angle YAO =\angle BXO=\alpha$. The labeling of the other angles in the figure follows from the properties of right triangles.

\newpage

$\triangle AOY\sim \triangle YOX \sim \triangle XOB$ and $\overline{OA}=1$, $\overline{OB}=2$ are given so:
\[
\begin{array}{l}
\displaystyle\frac{\overline{OY}}{\overline{OA}}=\displaystyle\frac{\overline{OX}}{\overline{OY}}=\displaystyle\frac{\overline{OB}}{\overline{OX}}\\
\\
\displaystyle\frac{\overline{OY}}{1}=\displaystyle\frac{\overline{OX}}{\overline{OY}}=\displaystyle\frac{2}{\overline{OX}}\,.
\end{array}
\]
From the first and second ratios we have $\overline{OX}=\overline{OY}^2$ and from the first and third ratios we have $\overline{OY}\:\overline{OX}=2$.
Substituting for $\overline{OX}$ gives $\overline{OY}^3=2$ and
$\overline{OY}=\sqrt[3]{2}$.
\end{proof}

\vspace{-5ex}

\section{Construction of a Regular Nonagon}\label{s.nonagon}

\index{Origami!construction of a nonagon}
\index{Nonagon|see {Origami, construction of a nonagon}}

A nonagon (a regular polygon with nine sides) is constructed by deriving the cubic equation for its central angle and then solving the equation using Lill's method and the Beloch fold. The central angle is $\theta=360^\circ/9=40^\circ$. By Thm.~\ref{thm.triple-angle}:
\[
\cos 3\theta=4\cos^3 \theta -3\cos\theta\,.
\]
Let $x=\cos 40^{\circ}$. Then for the nonagon the equation is $4x^3-3x+(1/2)=0$ since $\cos 3\cdot 40^\circ=\cos 120^\circ=-(1/2)$. Figure~\ref{f.nonagon2} shows the paths for the equation constructed according to Lill's method.

\begin{figure}[h]
\begin{center}
\begin{tikzpicture}[scale=.85]
% Draw help lines and axes
\draw[step=10mm,white!60!black] (-1,-4) grid (9,1);
\draw[thick] (-1,0) -- (9,0);
\draw[thick] (0,-4) -- (0,1);
\foreach \x in {1,...,9}
  \node at (\x-.3,.3) {\sm{\x}};
\foreach \y in {-3,...,1}
  \node at (-.3,\y-.3) {\sm{\y}};
  
% Points of first path
\coordinate (A) at (0,0);
\coordinate (B) at (4,0);
\coordinate (C) at (7,0);
\coordinate (D) at (7,-.5);
\node[above left] at (A) {$P$};
\node[below right,xshift=12pt] at (A) {$37.45^\circ$};
\node[below right] at (D) {$Q$};

% Draw first path
\draw[very thick,-{Stealth[scale=1.4,inset=2pt]}] 
  (A) -- node[below,xshift=6pt] {$a_3$} (B);
\draw[{Stealth[scale=1.4,inset=2pt,reversed]}-,very thick]
  (B) -- ($(B)+(0,.1)$);
\draw[name path=c,very thick,{Stealth[scale=1.4,inset=2pt]}-]
  (B) -- node[below] {$a_1$} (C);
\draw[very thick,-{Stealth[scale=1.4,inset=2pt]}]
  (C) -- node[right,yshift=-2pt] {$a_0$} (D);

% Draw extension of second segment of first path
\draw[thick,name path=b] 
  ($(B)+(0,-4)$) -- node[right] {$a_2$} ($(B)+(0,1)$);

% Draw second path
\path[name path=one] (A) -- +(-37.45:6cm);
\path [name intersections = {of = b and one, by = {R}}];
\node[below left] at (R) {$R$};
\draw[thick,dashed] (A) -- (R);

\path[name path=two] (R) -- +(52.5463:6cm);
\path [name intersections = {of = c and two, by = {S}}];
\node[above] at (S) {$S$};
\draw[thick,dashed] (R) -- (S);

\draw[thick,dashed] (S) -- (D);

% Draw right angle rectangles
\draw[thick,rotate=52.5463] (R) rectangle +(9pt,9pt);
\draw[thick,rotate=-127.4537] (S) rectangle +(9pt,9pt);
\end{tikzpicture}
\end{center}
\caption{Lill's method for a nonagon}\label{f.nonagon2}
\end{figure}


The second path starts from $P$ at an angle of approximately $-37.45^\circ$. Turns of $90^\circ$ at $R$ and then $-90^\circ$ at $S$ cause the path to intersect the first path at its endpoint $Q$. Therefore, $x=-\tan (-37.45^\circ)\approx 0.766$ is a root of $4x^3-3x+(1/2)$.

The root can be obtained using the Beloch fold. Construct the line $a_2'$ parallel to $a_2$ at the same distance from $a_2$ as $a_2$ is from $P$. Although the length of $a_2$ is zero, it still has a direction (upwards) so the parallel line can be constructed. Similarly, construct the line $a_1'$ parallel to $a_1$ at the same distance from $a_1$ as $a_1$ is from $Q$. The Beloch fold $\overline{RS}$ simultaneously places $P$ at $P'$ on $a_2'$ and $Q$ at $Q'$ on $a_1'$. This constructs the angle $\angle SPR=-37.45^\circ$ (Fig.~\ref{f.nonagon3}).

\begin{figure}[h]
\begin{center}
\begin{tikzpicture}[scale=.85]
% Draw help lines and axes
\draw[step=10mm,white!60!black] (-1,-7) grid (9,1);
\draw[thick] (-1,0) -- (9,0);
\draw[thick] (0,-7) -- (0,1);
\foreach \x in {1,...,9}
  \node at (\x-.3,.3) {\sm{\x}};
\foreach \y in {-6,...,1}
  \node at (-.3,\y-.3) {\sm{\y}};
  
% Points of first path
\coordinate (A) at (0,0);
\coordinate (B) at (4,0);
\coordinate (C) at (7,0);
\coordinate (D) at (7,-.5);
\node[above right] at (A) {$P$};
\node[below right] at (D) {$Q$};

% Draw first path
\draw[very thick,-{Stealth[scale=1.4,inset=2pt]}] 
  (A) -- node[below] {$a_3$} (B);
\draw[{Stealth[scale=1.4,inset=2pt,reversed]}-,very thick]
  (B) -- ($(B)+(0,.1)$);
\draw[name path=c,very thick,{Stealth[scale=1.4,inset=2pt]}-]
  (B) -- node[below] {$a_1$} (C);
\draw[very thick,-{Stealth[scale=1.4,inset=2pt]}]
  (C) -- node[right,yshift=-2pt] {$a_0$} (D);

% Draw extension of second segment of first path
\draw[very thick,loosely dotted,name path=b] 
  ($(B)+(0,-7)$) -- node[right,near end] {$a_2$} ($(B)+(0,1)$);

% Draw second path
\path[name path=one] (A) -- +(-37.45:6cm);
\path [name intersections = {of = b and one, by = {R}}];
\node[below left] at (R) {$R$};
\path[name path=two] (R) -- +(52.55:6cm);
\path [name intersections = {of = c and two, by = {S}}];
\node[above] at (S) {$S$};
\draw[very thick,dashed] (R) -- (S);

% Draw parallel lines
\draw[thick,name path=para-2] 
  (8,1) -- node[right,yshift=8pt] {$a_2'$} (8,-7);
\draw[thick,name path=para-1] 
  (-1,.5) -- node[right,xshift=44mm] {$a_1'$} (9,.5);

% Draw second segments of the folds
\path[name path=p-two] (A) -- +(-37.45:11cm);
\path [name intersections = {of = para-2 and p-two, by = {PP}}];
\node[below left] at (PP) {$P'$};
\draw[very thick,dotted] (A) -- (PP);

\path[name path=p-one] (D) -- +(142.55:2cm);
\path [name intersections = {of = para-1 and p-one, by = {QP}}];
\node[above] at (QP) {$Q'$};
\draw[very thick,dotted] (D) -- (QP);

% Draw right angle indications
\draw[thick,rotate=-37.45] (R) rectangle +(9pt,9pt);
\draw[thick,rotate=-127.4537] (S) rectangle +(9pt,9pt);
\end{tikzpicture}
\end{center}
\caption{The Beloch fold for solving the equation of the nonagon}\label{f.nonagon3}
\end{figure}

By Lill's method $-\tan (-37.45^\circ)\approx 0.766$ and therefore $\cos \theta \approx 0.766$ is a root of the equation for the central angle $\theta$. We finish the construction of the nonagon by constructing $\cos^{-1} 0.766\approx 40^\circ$.

The right triangle $\triangle ABC$ with $\angle CAB\approx 37.45^\circ$ and $\overline{AB}=1$ has opposite side $\overline{BC}\approx 0.766$ by definition of tangent (Fig.~\ref{f.nonagon5-eq}).
Fold $\overline{CB}$ onto the $\overline{AB}$ so that the reflection of $C$ is $D$ and $\overline{DB}=0.766$. Extend $\overline{DB}$ and construct $E$ so that $\overline{DE}=1$.
Fold $\overline{DE}$ to reflect $E$ at $F$ on the extension of $\overline{BC}$ (Fig.~\ref{f.nonagon5-central}). Then:
\[
\angle BDF=\cos^{-1} \frac{0.766}{1}\approx 40^\circ\,.
\]

\begin{figure}[ht]
\subfigures
\leftfigure[c]{
\begin{tikzpicture}[scale=1]
\draw (0,0) coordinate (A) -- (4,0) coordinate (B);
\draw (B) -- node[right] {$0.766$} 
  ($(B)+(0,0.766*4)$) coordinate (C);
\draw (A) -- (C);
\draw[rotate=90] (B) rectangle +(8pt,8pt);
\node[above left] at (A) {$A$};
\node[above right] at (B) {$B$};
\node[right] at (C) {$C$};
\coordinate (D) at (0.234*4,0);
\node[below left] at (D)  {$D$};
\coordinate (E) at ($(D)+(4,0)$);
\draw (D) -- (B);
\draw (B) -- (E);
\node[above right] at (E) {$E$};
\draw[very thick,dotted,->,bend right=50] ($(C)+(-.2,0)$) to ($(A)+(.94,.4)$);
\draw[<->] ($(D)+(0,-1.2)$) -- node[fill=white] {$1$} ($(E)+(0,-1.2)$);
\draw[<->] ($(A)+(0,-.8)$) -- node[fill=white] {$1$} ($(B)+(0,-.8)$);
\node[above right,xshift=14pt] at (A) {$37.45^\circ$};
\vertex{D};
\vertex{E};
\end{tikzpicture}
}
\hfill
\rightfigure[c]{
\begin{tikzpicture}[scale=1]
\coordinate (B) at (4,0);
\draw (B) -- ($(B)+(0,0.766*4)$) coordinate (C);
\draw[rotate=90] (B) rectangle +(8pt,8pt);
\node[above right] at (B) {$B$};
\node[right] at (C) {$C$};
\coordinate (D) at (0.234*4,0);
\node[above left] at (D) {$D$};
\node[above right,xshift=8pt,yshift=4pt] at (D) {$40^\circ$};
\coordinate (E) at ($(D)+(4,0)$);
\draw (D) -- node[fill=white] {$0.766$} (B);
\draw (B) -- (E);
\node[above right,xshift=4pt] at (E) {$E$};
\coordinate (F) at ($(B)+(0,4)$);
\draw[very thick,dotted,->,bend right=50] ($(E)+(.1,.2)$) to ($(F)+(.2,0)$);
\draw (B) -- (F);
\node[left] at (F) {$F$};
\draw (D) -- node[fill=white] {$1$} (F);
\draw[<->] ($(D)+(0,-.8)$) -- node[fill=white] {$1$} ($(E)+(0,-.8)$);
\vertex{C};
\vertex{E};
\coordinate (A) at (0,0) node [above left] {$A$};
\draw (A) -- (D);
\vertex{A};
\vertex{D};
\end{tikzpicture}
}
\leftcaption{The tangent that is the solution of the equation for the nonagon}\label{f.nonagon5-eq}
\rightcaption{The cosine of the central angle of the nonagon}\label{f.nonagon5-central}
\end{figure}

\newpage

\subsection*{What Is the Surprise?}

We saw in Chaps.~\ref{c.trisect} and~\ref{c.square} that tools such as the neusis can perform constructions that cannot be done with a straightedge and compass. It is surprising that trisecting an angle and doubling a cube can be constructed using only paper folding. Roger C. Alperin\index{Alperin, Roger C.} has developed a hierarchy of four methods of construction each more powerful than the previous one.

\subsection*{Sources}

This chapter is based on \cite{alperin,lang,martin,newton}.
